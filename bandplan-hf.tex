\begin{landscape}
\section{Frekvenser Amatörradio LF/MF/HF}
\subsection{Bandplaner LF/MF/HF}
Alla frekvenser i kHz, bandbredder i Hz.

\subsubsection{Bandplan 2.2\,km, 135,7--137,8\,kHz}
\begin{tabular}{rrrll}
\textbf{Frekvens} &  & \textbf{BW} & \textbf{Trafik} & \textbf{Noteringar} \\ \hline
135,7 & 135,8 & 200 & CQ, QRSS, Digi & OBS! Högsta effekt 1W ERP. \\ \hline
\end{tabular}

\subsubsection{Bandplan 600\,m, 472--479\,kHz}
\begin{tabular}{rrrll}
\multicolumn{2}{c}{\textbf{Frekvens}} & \textbf{BW} & \textbf{Trafik} & \textbf{Noteringar} \\ \hline
472 & 479 & 200 & CW, QRSS, Digi & OBS! Högsta utstrålad effekt 1W EIRP \\ \hline
\end{tabular}

\subsubsection{Bandplan 160\,m, 1810--2000\,kHz}
\begin{tabular}{rrrll}
\multicolumn{2}{c}{\textbf{Frekvens}} & \textbf{BW} & \textbf{Trafik} & \textbf{Noteringar} \\ \hline
1 810 & 1 838 & 200  & CW         & Exklusivt för CW. Interkontinental trafik har prio. \\ \hline
1 838 & 1 840 & 500  & Smalband   & Ej packet på 160m, PSK 1 838,150                    \\ \hline
1 840 & 1 850 & 2700 & Alla moder & Även digimode. SSB QRP 1 843 kHz                    \\ \hline
1 850 & 1 900 & 2700 & Alla moder & OBS! Max 10 W till ant.                             \\ \hline
1 900 & 1 950 & 2700 & Alla moder & OBS! Max 100 W till ant.                            \\ \hline
1 950 & 2 000 & 2700 & Alla moder & OBS! Max 10 W till ant.                             \\ \hline
\end{tabular}

\subsubsection{Bandplan 80\,m, 3500--3800\,kHz}
\begin{tabular}{rrrll}
\multicolumn{2}{c}{\textbf{Frekvens}} & \textbf{BW} & \textbf{Trafik} & \textbf{Noteringar} \\ \hline
3 500 & 3 510 & 200  & CW             & Exklusivt CW                         \\ 
      &       &      &                & Interkontinental DX-trafik har prio  \\ \hline
3 510 & 3 580 & 200  & CW             & Exklusivt CW contest 3510-–560       \\ 
      &       &      &                & CW QRS 3 555 kHz, CW QRP 3 560       \\ \hline
3 580 & 3 600 & 500  & Smalband, Digi & PSK 3 580,150                        \\
      &       &      &                & Automatiska Digimoder 3 590--600     \\ \hline
3 600 & 3 620 & 2700 & Alla moder     & Digimoder Automatiska Digimoder      \\ \hline
3 600 & 3 650 & 2700 & Alla moder     & SSB contest 3 600--650               \\
      &       &      &                & DV 3 630                             \\ \hline
3 650 & 3 700 & 2700 & Alla moder     & SSB QRP 3 690                        \\ \hline
3 700 & 3 800 & 2700 & Alla moder     & Contest 3 700-–800                   \\
      &       &      &                & Image 3 775                          \\
      &       &      &                & Region 1 nödfrekvens 3 760           \\ \hline
3 775 & 3 800 & 2700 & Alla moder     & Interkontinental DX-trafik prioritet \\ \hline
\end{tabular}

\subsubsection{Bandplan 40\,m, 7000--7200\,kHz}
\begin{tabular}{rrrll}
\multicolumn{2}{c}{\textbf{Frekvens}} & \textbf{BW} & \textbf{Trafik} & \textbf{Noteringar} \\ \hline
7\,000 & 7\,040 & 200  & CW         & Exklusivt CW.                             \\
      &       &      &            & QRP aktivitetscentrum 7\,030\,kHz           \\ \hline
7\,040 & 7\,050 & 500  & Smalband   & Digimoder Automatiska inom 7\,047–-050\,kHz \\ \hline
7\,050 & 7\,060 & 2700 & Alla moder & Digimoder Automatiska inom 7\,050–-053\,kHz \\ \hline
7\,060 & 7\,100 & 2700 & Alla moder & SSB contest i segmentet                   \\
      &       &      &            & DV 7 070 kHz, SSB QRP 7\,090 kHz           \\ \hline
7\,100 & 7\,130 & 2700 & Alla moder & Region 1 nödfrekvens 7\,110 kHz            \\ \hline
7\,130 & 7\,200 & 2700 & Alla moder & SSB contest i segmentet                   \\
      &       &      &            & Image 7\,165\,kHz                           \\ \hline
7\,175 & 7\,200 & 2700 & Alla moder & Interkontinental DX-trafik prio           \\ \hline
\end{tabular}

\subsubsection{Bandplan 30 m, 10100--10150 kHz}
\begin{tabular}{rrrll}
\multicolumn{2}{c}{\textbf{Frekvens}} & \textbf{BW} & \textbf{Trafik} & \textbf{Noteringar} \\ \hline
10\,100 & 10\,140 & 200 & CW       & CW exkl. Max 150 Watt på 30 m    \\
       &        &     &          & CW QRP 10\,116\,kHz                     \\ \hline
10\,140 & 10\,150 & 500 & Smalband & Digimoder PSK 10142,150\,kHz. Ej Packet \\ \hline
\end{tabular}

\subsubsection{Bandplan 20 m, 14000--14350 kHz}
\begin{tabular}{rrrll}
\multicolumn{2}{c}{\textbf{Frekvens}} & \textbf{BW} & \textbf{Trafik} & \textbf{Noteringar} \\ \hline
14\,000 & 14\,070 & 200  & CW         & Exklusivt CW                            \\
       &        &      &            & Conctest 14\,000-–060                     \\
       &        &      &            & CW QRS 14 055, CW QRP 14\,060            \\ \hline
14\,070 & 14\,099 & 500  & Smalband   & PSK 14 070,150                          \\
       &        &      &            & Auto Digimoder 14 089-–099              \\ \hline
14\,099 & 14\,101 & 200  & Fyrar      & Exklusivt IBP, endast fyrar             \\ \hline
14\,101 & 14 \,12 & 2700 & Alla moder & Digitala moder och obevakade Digimoder  \\ \hline
14\,112 & 14\,350 & 2700 & Alla moder & SSB Contest 14 125--300                 \\
       &        &      &            & DV 14 130, DXpedition prio 14\,195$\pm$5 \\ \hline
14\,300 & 14\,350 & 2700 & Alla moder & Image 14\,230, SSB QRP 14\,285            \\
       &        &      &            & Global nödfrekvens 14 300               \\ \hline
\end{tabular}

\subsubsection{Bandplan 17 m, 18068--18168 kHz}
\begin{tabular}{rrrll}
\multicolumn{2}{c}{\textbf{Frekvens}} & \textbf{BW} & \textbf{Trafik} & \textbf{Noteringar} \\ \hline
18 068 & 18 095 & 200  & CW         & CW exklusivt. QRP 18 086             \\ \hline
18 095 & 18 109 & 500  & Smalband   & Digimoder PSK 18 100,150             \\
       &        &      &            & Automatiska Digimoder 18 105-–18 109 \\ \hline
18 109 & 18 111 & 200  & Fyrar      & Exklusivt fyrar, IBP fyrnät          \\ \hline
18 111 & 18 168 & 2700 & Alla moder & Digi 18 111–-18 120                  \\
       &        &      &            & SSB QRP 18 130, DV 18 150            \\
       &        &      &            & Global nödfrekv. 18 160\\ \hline
\end{tabular}

\subsubsection{Bandplan 15 m, 21000--21450 kHz}
\begin{tabular}{rrrll}
\multicolumn{2}{c}{\textbf{Frekvens}} & \textbf{BW} & \textbf{Trafik} & \textbf{Noteringar} \\ \hline
21 000 & 21 070 & 200  & CW         & Exklusivt CW, QRS 21 055, CW QRP 21 060          \\ \hline
21 070 & 21 110 & 500  & Smalband   & PSK 21080.150, Automatiska Digimoder 21 090–-110 \\
21 110 & 21 120 & 2700 & Alla moder & Alla moder utom SSB!                             \\
       &        &      &            & Digimoder, och Automatiska Digimoder             \\ \hline
21 120 & 21 149 & 500  & Smalband   &                                                  \\ \hline
21 149 & 21 151 & 200  & Fyrar      & Exklusivt fyrar. IBP fyrnät                      \\ \hline
21 151 & 21 450 & 2700 & Alla moder & DV 21 180, SSB QRP 21 285, Image 21 340          \\
       &        &      &            & Global nödfrekv. 21 360                          \\ \hline
\end{tabular}

\subsubsection{Bandplan 12 m, 24890--24990 kHz}
\begin{tabular}{rrrll}
\multicolumn{2}{c}{\textbf{Frekvens}} & \textbf{BW} & \textbf{Trafik} & \textbf{Noteringar} \\ \hline
24 890 & 24 915 & 200  & CW         & Exklusivt CW, QRP 24 906                             \\ \hline
24 915 & 24 929 & 500  & Smalband   & PSK 24 920.150, Automatiska Digimoder 24 925–-24 929 \\ \hline
24 929 & 24 931 & 200  & Fyrar      & Fyrar, IBP fyrnät                                    \\ \hline
24 931 & 24 990 & 2700 & Alla moder & Auto Digimoder 24 931-–24 940                        \\
       &        &      &            & SSB QRP 24 950, DV 24 960                            \\ \hline
\end{tabular}

\subsubsection{Bandplan 10 m, 28000-29700 kHz}
\begin{tabular}{rrrll}
\multicolumn{2}{c}{\textbf{Frekvens}} & \textbf{BW} & \textbf{Trafik} & \textbf{Noteringar} \\ \hline
28 000 & 28 070 & 200  & CW         & Exklusivt CW, QRS 28 055, CW QRP 28 060                \\ \hline
28 070 & 28 190 & 500  & Smalband   & PSK 28 120.150, Auto Digimoder inom 28 120--150        \\ \hline
28 190 & 28 199 & 200  & Fyrar IBP  & Regionala fyrar med tidsdelning                        \\ \hline
28 199 & 28 201 & 200  & Fyrar IBP  & IBP fyrnät                                             \\ \hline
28 201 & 28 225 & 200  & Fyrar IBP  & kontinuerligt sändande fyrar                           \\ \hline
28 225 & 28 300 & 2700 & Alla moder & Övriga fyrar                                           \\ \hline
28 300 & 28 320 & 2700 & Alla moder & Digimoder och Automatiska Digimoder                    \\ \hline
28 320 & 29 100 & 2700 & Alla moder & DV 28 330 kHz, SSB QRP 28 360 kHz                      \\
       &        &      &            & Image 28 680 kHz                                       \\ \hline
29 100 & 29 200 & 6000 & Alla moder & FM simplex, 10 kHz kanaler                             \\
       &        &      &            & Maximalt ±2.5 kHz dev., max 2.5 kHz mod.frek.          \\ \hline
29 200 & 29 300 & 6000 & Alla moder & Digimoder och Automatiska Digimoder                    \\ \hline
29 300 & 29 510 & 6000 & Satellit   & Nerlänk fr. satellit. EJ SÄNDNING I SEGMENTET          \\ \hline
29 510 & 29 520 & 6000 & Skydd      & Skyddsfrekvens för satelliter. EJ SÄNDNING I SEGMENTET \\ \hline
29 520 & 29 590 & 6000 & Alla moder & FM Repeater in RH1--8, 100 kHz duplex, 2.5 kHz NBFM    \\ \hline
29 600 & 29 620 & 6000 & Alla moder & FM simplex, anrop 29 600                               \\
       &        &      &            & FM simplex repeater 29 610                             \\ \hline
29 620 & 29 700 & 6000 & Alla moder & FM Repeater ut RH1--8, 100 kHz duplex                  \\ \hline
\end{tabular}
\end{landscape}

\clearpage
