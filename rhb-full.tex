%%%%%%%%%%%%%%%%%%%%%%%%%%%%%%%%%%%%%%%%%%%%%%%%%%%%%%%%%%%%%%%%%%%%%%%%%%%%
% diverse saker här som används i dokumentet %%%%%%%%%%%%%%%%%%%%%%%%%%%%%%%
%%%%%%%%%%%%%%%%%%%%%%%%%%%%%%%%%%%%%%%%%%%%%%%%%%%%%%%%%%%%%%%%%%%%%%%%%%%%
\newcommand{\TitleText}{Radiohandboken}
\newcommand{\SubtitleText}{För Sändaramatörer, privatradioanvändare\\
	och andra radiointresserade}
\newcommand{\Forfattare}{Täpp-Anders Sikvall}
\newcommand{\Initialer}{SM5UEI}
\newcommand{\Epostadress}{anders@sikvall.se}
\newcommand{\DokVersion}{2.3.0}
\newcommand{\DokYear}{2025}
\newcommand{\DokumentDatum}{\today}
\newcommand{\ISBN}{---}
\newcommand{\Linkhome}{www.sikvall.se}
%%%%%%%%%%%%%%%%%%%%%%%%%%%%%%%%%%%%%%%%%%%%%%%%%%%%%%%%%%%%%%%%%%%%%%%%%%%
\documentclass[10pt,swedish,a4paper,
  twoside]{book}                       % De viktigaste inställningarna för dokumentet
\usepackage[a4paper]{geometry}         % Särskilda dokumentinställningar
\usepackage[utf8]{inputenc}            % Indata som UTF-8 är väldigt bra för t.ex. svenska
\usepackage[T1]{fontenc}               % Fontkodning enligt modern standard
\usepackage[swedish]{babel}            % Svensk avstavning och mycket mer
\usepackage[colorlinks,
  linkcolor=black,
  urlcolor=blue]{hyperref}             % Hyperref möjliggör länkar utanför dokumentet
\usepackage{listings}
\usepackage{enumitem}                  % Enumrering
\usepackage{multirow}                  % Tabeller med tabellrader som spänner över mer än en textrad
\usepackage{amsmath}                   % Utökade matematiska symboler
\usepackage{tabularx}                  % Tabulering med fler optioner än original
\usepackage{graphicx}                  % Kunna sätta bilder bättre
\usepackage[table,x11names]{xcolor}    % Referera till färger med deras X11-namn
\usepackage{fancyhdr}                  % Kunna snyggare typsätta huvud och fot på varje sida
\usepackage[yyyymmdd]{datetime}        % Mer europeiskt datumformat (ISO)
  \renewcommand{\dateseparator}{-}
\usepackage{lastpage}                  % Kunna referera till sista sidan i dokumentet
\usepackage{pdfpages}                  % Kunna lägga in PDF-kommentarer
\usepackage{icomma}                    % Intelligenta kommatteringar för stora och små tal
\usepackage{pdflscape}                 % Kunna göra vissa sidor liggande i PDF
\usepackage{wrapfig}                   % Kunna wrappa figurer
\usepackage{float}                     % Utökade optioner för floats
\usepackage[bottom]{footmisc}          % Optioner för fotnoter, sätt alltid på slutet av samma sida
\usepackage{longtable}                 % Tabeller som sidbryter
\usepackage{fbb}                       % En font inspirerad av Bembo
\newif\iftitlefoot
\raggedbottom                          % Stretcha inte sidor mer än nödvändigt
\setlist{nosep}                        % Separera inte numrerade och punktlistor mer
\usepackage{todonotes}                 % Kunna lägga in att-göra i dokumentet
\renewcommand{\arraystretch}{1.15}
\headheight=20pt
\begin{document}

\pagestyle{empty}

%%% Makron för typsättning av morse
\newcommand{\Mdotlength}{3pt}
\newcommand{\Mdahlength}{9pt}
\newcommand{\Mcharseplength}{9pt}
\newcommand{\Mwordseplength}{21pt}
\newcommand{\Mdiditlength}{6pt}
\newcommand{\Mfetvadd}{1.5pt}
\newcommand{\Mcharsep}{\hspace{9pt}}
\newcommand{\Mwordsep}{\hspace{21pt}}

\newcommand{\dit}{\raisebox{0.5ex}{$\rule{\Mdotlength}{\Mfetvadd}\hspace{\Mdotlength}$}}
\newcommand{\dah}{\raisebox{0.5ex}{$\rule{\Mdahlength}{\Mfetvadd}\hspace{\Mdotlength}$}}

\newcommand{\MAdam}{\dit\dah            \Mcharsep}
\newcommand{\MBertil}{\dah\dit\dit\dit  \Mcharsep}
\newcommand{\MCesar}{\dah\dit\dah\dit   \Mcharsep}
\newcommand{\MDavid}{\dah\dit\dit       \Mcharsep}
\newcommand{\MErik}{\dit                \Mcharsep} 
\newcommand{\MFilip}{\dit\dit\dah\dit   \Mcharsep}
\newcommand{\MGustav}{\dah\dah\dit      \Mcharsep}
\newcommand{\MHelge}{\dit\dit\dit\dit   \Mcharsep}
\newcommand{\MIvar}{\dit\dit            \Mcharsep}
\newcommand{\MJohan}{\dit\dah\dah\dah   \Mcharsep}
\newcommand{\MKalle}{\dah\dit\dah       \Mcharsep}
\newcommand{\MLudvig}{\dit\dah\dit\dit  \Mcharsep}
\newcommand{\MMartin}{\dah\dah          \Mcharsep}
\newcommand{\MNiklas}{\dah\dit          \Mcharsep}
\newcommand{\MOlof}{\dah\dah\dah        \Mcharsep}
\newcommand{\MPetter}{\dit\dah\dah\dit  \Mcharsep}
\newcommand{\MQvintus}{\dah\dah\dit\dah \Mcharsep}
\newcommand{\MRudolf}{\dit\dah\dit      \Mcharsep}
\newcommand{\MSigurd}{\dit\dit\dit      \Mcharsep}
\newcommand{\MTore}{\dah                \Mcharsep}
\newcommand{\MUrban}{\dit\dit\dah       \Mcharsep}
\newcommand{\MViktor}{\dit\dit\dit\dah  \Mcharsep}
\newcommand{\MWilhelm}{\dit\dah\dah     \Mcharsep}
\newcommand{\MXerxes}{\dah\dit\dit\dah  \Mcharsep}
\newcommand{\MYngve}{\dah\dit\dah\dah   \Mcharsep}
\newcommand{\MZata}{\dah\dah\dit\dit    \Mcharsep}
\newcommand{\MAke}{\dit\dah\dah\dit\dah \Mcharsep}
\newcommand{\MArlig}{\dit\dah\dit\dah   \Mcharsep}
\newcommand{\MOsten}{\dah\dah\dah\dit   \Mcharsep}
\newcommand{\MUbel}{\dit\dit\dah\dah    \Mcharsep} % Tyskt Ü
\newcommand{\Mch}{\dah\dah\dah\dah      \Mcharsep} % Tyskt CH

\newcommand{\MPunkt}{\dit\dah\dit\dah\dit\dah           \Mcharsep}
\newcommand{\MSTOP}{\MPunkt}
\newcommand{\MKomma}{\dah\dah\dit\dit\dah\dah           \Mcharsep}
\newcommand{\MFragetecken}{\dit\dit\dah\dah\dit\dit     \Mcharsep}
\newcommand{\MBraktecken}{\dah\dit\dit\dah\dit          \Mcharsep}
\newcommand{\MBindestreck}{\dah\dit\dit\dit\dit\dah     \Mcharsep}
\newcommand{\MSluttecken}{\dit\dah\dit\dah\dit          \Mcharsep}
\newcommand{\MSOS}{\dit\dit\dit\dah\dah\dah\dit\dit\dit \Mcharsep}

\newcommand{\MRepetera}{\dit\dit\hspace{\Mdiditlength}\dit\dit \Mcharsep}
\newcommand{\MVanta}{\dit\dah\dit\dit\dit                     \Mcharsep}
\newcommand{\MAvslutning}{\dit\dah\dah\dit\dah\dit            \Mcharsep}
\newcommand{\MAtskillnad}{\dah\dit\dit\dit\dah                \Mcharsep}
\newcommand{\MFelskrivning}{\dit\dit\dit\dit\dit\dit\dit\dit  \Mcharsep}

\newcommand{\MEtt}{\dit\dah\dah\dah\dah   \Mcharsep}
\newcommand{\MTva}{\dit\dit\dah\dah\dah	  \Mcharsep}
\newcommand{\MTre}{\dit\dit\dit\dah\dah   \Mcharsep}
\newcommand{\MFyra}{\dit\dit\dit\dit\dah  \Mcharsep}
\newcommand{\MFem}{\dit\dit\dit\dit\dit   \Mcharsep}
\newcommand{\MSex}{\dah\dit\dit\dit\dit   \Mcharsep}
\newcommand{\MSju}{\dah\dah\dit\dit\dit   \Mcharsep}
\newcommand{\MAtta}{\dah\dah\dah\dit\dit  \Mcharsep}
\newcommand{\MNio}{\dah\dah\dah\dah\dit   \Mcharsep}
\newcommand{\MNoll}{\dah\dah\dah\dah\dah  \Mcharsep}

\newcommand{\MWord}{\MWS}

\lstset{%
literate={a}{\MAdam}1
         {b}{\MBertil}1
         {c}{\MCesar}1
         {d}{\MDavid}1
         {e}{\MErik}1
         {f}{\MFilip}1
         {g}{\MGustav}1
         {h}{\MHelge}1
         {i}{\MIvar}1
         {j}{\MJohan}1
         {k}{\MKalle}1
         {l}{\MLudvig}1
         {m}{\MMartin}1
         {n}{\MNiklas}1
         {o}{\MOlof}1
         {p}{\MPetter}1
         {q}{\MQvintus}1
         {r}{\MRudolf}1
         {s}{\MSigurd}1
         {t}{\MTore}1
         {u}{\MUrban}1
         {v}{\MViktor}1
         {w}{\MWilhelm}1
         {x}{\MXerxes}1
         {y}{\MYngve}1
         {z}{\MZata}1
         {å}{\MAke}1
         {ä}{\MArlig}1
         {ö}{\MOsten}1
         {.}{\MPunkt}1
         {\ }{\Mwordsep}1
         {,}{\MKomma}1
         {0}{\MNoll}1
}
\newcommand{\morse}[1]{\lstinline{#1}}



\newgeometry{bottom=3cm,top=3cm}

\vfill
\vspace*{4cm}
\centerline{\includegraphics[width=\paperwidth]{logo/rubrikbild}}
\begin{flushright}
	\Huge{\bfseries{\TitleText}} \\[3mm]
	\Large{\bfseries{\SubtitleText}}
\end{flushright}

\newpage

\vspace*{\fill}

\noindent \TitleText

\vspace{1 em}

\noindent \Forfattare, \Initialer 

\noindent Karlholmsbruk 

\noindent Version \DokVersion

\noindent Datum \DokumentDatum

\noindent ISBN: \ISBN

\newpage

%\newgeometry{top=3cm}

\renewcommand{\headrulewidth}{0.2 pt}
\renewcommand{\footrulewidth}{0.2 pt}

\fancyhead[RE]{\small\nouppercase{\leftmark}}      % Chapter in the right on even pages
\fancyhead[RO]{\small\nouppercase{\rightmark}}     % Section in the left on odd pages
\fancyhead[LE]{\small\nouppercase{\rightmark}}      % Chapter in the right on even pages
\fancyhead[LO]{\small\nouppercase{\leftmark}}     % Section in the left on odd pages

\fancyfoot[RE]{\small Radiohandboken}
\fancyfoot[RO]{\small \DokVersion}
\fancyfoot[LE]{\small \DokVersion}
\fancyfoot[LO]{\small Radiohandboken}


\cfoot{\small \thepage}

\pagestyle{fancy}

\widowpenalty=9999
\clubpenalty=9999

%	\setlength{\headsep}{1em}2

\cleardoublepage

%%% Innehållsförteckning
\setcounter{secnumdepth}{5}
\setcounter{tocdepth}{5}
\tableofcontents

\newpage

\setlength{\parskip}{1 ex}
\setlength{\parindent}{0pt}


%%%%%%%%%%%%%%%%%%%%%%%%%%%%%%%%%%%%%%%%%%%%%%%%%%%%%%%%%%%%%%%%%%%%%%%%%%%%
% diverse saker här som används i dokumentet %%%%%%%%%%%%%%%%%%%%%%%%%%%%%%%
%%%%%%%%%%%%%%%%%%%%%%%%%%%%%%%%%%%%%%%%%%%%%%%%%%%%%%%%%%%%%%%%%%%%%%%%%%%

% !TeX encoding = UTF-8
% !TeX spellcheck = sv_SE

\section*{Förord}

Denna version av radiohandboken innehåller en stor mängt förändringar mot
tidigare versioner. Dels har jag valt att ta bort de specifika VHF- och
HF-versionerna av den och i stället bara inkludera allting i samma handbok för
enkelthets skull. Jag har lagt vikt vid att i stället försöka göra det tydligt
så den som inte är intresserad av vissa delar enkelt kan hoppa över dem.

Det har också medfört en helt ny struktur och kapitelindelning på en högre
nivå som gör att det är mer logiskt strukturerat. Alla frekvenslistor är
flyttade till ett eget kapitel som gör det enklare att skriva ut dem medan
reglemente och teknik osv fått helt egna kapitel.

Ett index har skapats med indexering av de olika begreppen i boken så att det
ska bli enklare att hitta. Denna indexering återfinns längst bak i boken med
länkar så att man kan klicka sig fram i en PDF.

De svenska morsetecknen har också fått komma med i denna utgåva, det
är inte så vanligt att folk ägnar sig åt morsesignalering i dag som
det var en gång i tiden men det hör definitivt till så det ska förstås
vara med.

Många frekvenslistor har blivit uppdaterade med nya frekvenser, information
och kanaler, felaktigheter har korrigerats och jag hoppas att det mesta nu
faktiskt är ganska korrekt!

\url{https://sikvall.se}

Och kör radio där ute. Alltid med stil.

\vspace{4mm}

Karlholm, \DokumentDatum\\
\textit{Täpp-Anders Sikvall\\
	SM5UEI}

\section*{Om författaren}

Jag heter Täpp Anders Sikvall och har haft ett radiointresse sedan
barnsben. När jag gick i gymnasiet tog jag amatörradiolicens efter dåtidens
regler och har sedan haft amatörradio som en hobby.

Jag jobbar också professionellt med radiosystem och på fritiden tycker jag om
att gå i skog och mark (och köra portabelt). Jag utbildar också nya
radioamatörer och är även en av SSA:s provförrättare.

\section*{Licens}

Denna handbok är författad av Täpp Anders Sikvall, SM5UEI. Den är
släppt under en creative commons licens som kallas för CC BY-NC-SA.

I korthet innebär det att du får distribuera och trycka upp materialet
hur du vill, dela med dig av det men inte för kommersiella ändamål. Du
får dessutom använda materialet i dina egna alster men om du gör det
ska du släppa dessa under samma licens. Mitt namn ska stå kvar i
originalet och om du använder det i ett annat verk ska det vara
tydligt att det är ett derivat.

För fullständig information kan du besöka\\
\url{https://creativecommons.org/licenses/by-nc-sa/4.0/deed.sv}

Licensen gäller alla versioner av radiohandboken, alla utgåvor och
alla utdrag ur den.

\clearpage

\section*{Nyheter i denna utgåva}

\begin{itemize}
  \item Licensvillkoren för verket är uppdaterade
  \item Om författaren tillagt
  \item Mer om kommersiell radio och myndighetsradio tillagt liksom information om
  \item undantag från tillståndsplikt
\end{itemize}
\todo{Fyll i vad som ändrats sedan sist}

\clearpage

\pagestyle{fancy}
% !TeX encoding = UTF-8
% !TeX spellcheck = sv_SE
\chapter{Reglemente}

Vi tittar på reglementen kring användning av radiofrekvenser och utrustning
inom Sverige och vilka myndigheter och lagar som styr detta. Vi går vidare med
PR-radio och amatörradio och det fortfarande ganska nya instegscertifikatet i
detta kapitel.

\clearpage

\section{Tillståndsfri radio}

I sverige regleras radioanvändningen av Post- och telestyrelsens (PTS)
bestämmelser och det gäller all radioanvändning. Delar av detta delegeras
sedan till olika myndigheter exempelvis för marin radio och flygradio. Själva
utrustningen som får användas för olika typer av radiosändning regleras i
radioutrustningslagen SFS 2916:392 och det övergripande EU-direktivet
2014/53/EU.


Det är i undantagsföreskriften du hittar alla villkor som hänger samman med
tillståndsfri radio\-sänd\-ning. Här står effekter, modulationstyper och
användningsområden upptagna för respektive frekvens samt andra villkor.

Det är också i denna skrift som användningen av Amatörradio regleras rent
juridiskt. Sedan finns inte alla rekommendationer och liknande med i denna
skrift, här får man själv hitta det reglemente som amatörradioorganisationer
mm kommit överens om hur man ska nyttja de olika bandet.

Det innebär att radioanvändning för diverse ändamål regleras i denna skrift så
som:

\begin{itemize}
 \item PR-radio på 27 MHz
 \item 69 MHz åkeriradio som också får användas för andra ändamål
 \item Personlig radio på 446 och 444 MHz
 \item Radio för jakt, fiske och jordbruk på 155 MHz
 \item Radio för jakt på 31 MHz
 \item ... och mycket mer.
\end{itemize}

En sak som är intressant är att amatörradio behandlas också i samma skrift
trots det kräver ett godkänt avlagt prov som leder till ett
amatörradiocertifikat. Tidigare reglerades amatörradio separat på Televerkets
tid i en författningssamling som kallades för Televerket B:90. Denna har dock
helt ersatts av bestämmelserna i undantagsföreskriften.

Däremot krävs det numera inget särskilt tillstånd som det gjorde tidigare för
att få använda amatörradio, annat än att den som opererar radioutrustningen
skall ha, eller stå under uppsikt av, en person med giltigt
amatörradiocertifikat.

\subsection{Undantag från tillståndsplikt}

I sverige regleras all radioanvändning med sändare som inte kräver att man
söker tillstånd för det i en skrift som ges ut av Post- och Telestyrelsen
(PTS) och som bär namnet "Untandag från tillståndsplikt". Denna skrift
uppdateras lite då och då när det finns behov av det och här regleras all
radiotrafik som inte kräver särskilt tillstånd.

\subsection{Banden som omfattas}

Detta gäller då PR-radio, 69 MHz radio, PMR-bandet på 446 MHz och KDR-bandet
på 444 MHz som ofta används på byggen eller säkerhetsarrangemang för arenor
med mera liksom Jakt- och jordbruksfrekvenserna på 31 och 155 MHz med mera.
Inget av dessa band kräver licens eller tillstånd men det krävs att man följer
reglerna i hänseende till vad bandet är tänkt att användas till.

Det finns också en rad med andra tillämpningar som går under detta dokument från
tekniska och medicinska och mycket mer. Det rekommenderas att skumma igenom
dokumentet då det faktiskt är ganska intressant.

\subsection{Endast godkända radioapparater}

Radioapparaterna måste också vara godkända och CE-märkta för att man ska
uppfylla villkoren och här blir det ofta en del problem med privatimporterade
radioapparater från Asien som ibland kan ha förskräckligt dåliga sändare. Det
är straffbart att använda icke godkänd radioutrustning.

\section{Amatörradio}

Amatörradio har en lång historia och sträcker sig tillbaka till radions
barndom. Rent formellt så var det i USA i slutet av 1890-talet som radioamatörer
började sända telegram till varandra med den teknik som fanns till buds då. Det
blev mycket populärt bland elingenjörer och andra teknikintresserade och runt
1910 började man få problem med interferens och störningar ochg beslutade sig
för att formalisera det hela. Olika restriktioner infördes men i och med detta
regelverk fick vi också vissa krav på kunskaper för att operera radiosändare.

\subsection{Amatörradiocertifikat HAREC}

Om det fulla amatörradiocertifikatet är ett certifikat som är utformat efter de
principer som anges i HAREC T/R 61-02, Harmonised Amateur Radio Examination
Certificate, Vilnius 2004, uppdaterad 2014 och 2016.

Detta är ett ganska omfattande dokument och i Sverige så har vi t.ex.
KonCEPT-boken som används för ubildning av nya radioamatörer, den kan laddas ned
som PDF från ssa.se eller beställas som papperbok i deras webshop. För att bli
radioamatör behöver man svara tillräckligt många rätt på två stycken delprov,
ett teknikprov som avhandlar elektriska kretsar, radioteknik, sändare och
mottagare, vågutbredning, eletromagnetiska fält, grundläggande matematik och
fysik som är tillämplig, viss komponentlära, förståelse för störningar och att
bli störd, filter och antenner mm. Det andra provet är ett prov över reglementet
som tillämpas, både lagar och författningar som reglerar amatörradio men även
saker som mer praktiska som exempelvis bokstaveringsalfabetet och Q-koder mm.

Ett godkänt sådant prov ger möjlighet att operera som radioamatör på
en stor mängd olika frekvensband, alla de som i Sverige är utmärkta
som amatörradioband.

\subsection{Amatörradiocertifikat insteg}
\label{sec:instegscertifikat}

Instegscertifikatet är ett förenklat teknikprov men ungefär samma prov vad
gäller reglementen osv. Det förenklade teknikprovet innebär dock en del
begränsningar eftersom instegsamatören inte kan ges riktigt samma förtroende att
sända på alla frekvenser. Exempelvis har man därmed begränsat de frekvensband
som får användas till sådana som är exklusiva för amatörradio i Sverige.

Konskapskraven i Sverige skall motsvara det som finns i ''ECC Report 89'': \\
\url{https://docdb.cept.org/download/409}.

PTS förväntas delegera examineringen till SSA och deras utsedda provförrättare
och en särskild utbildningbok för instegscertet håller på att tas fram just nu
och ska snart gå till tryck (Mars 2025).

Tyvärr fick inte SSA gehör för sin remiss att även öppna 70\,cm bandet för
instegscertifikat. Bandet är inte länge ett exklusivt amatörband men vi delar
det med lågeffektssändare för fjärrstyrning mm och det bör inte finnas hinder då
fullcertare får sända med 200\,W i bandet och kan söka högeffektstillstånd på
upp till 1\,kW.

Radioamatörer med instegscertifikat tilldelas anropssignaler i serien SH som
även tidigare använts för noviser. Vid uppgradering till fullt cert får man i
stället en signal i SA-serien är tanken.

\subsection{Amatörradioband för instegscertifikat}

\begin{table}[H]
\centering
\begin{tabular}{rrlr}
  \textbf{Frekvens} & \textbf{Våglängd} & \textbf{Notiser}               & \textbf{Max Effekt} \\ \hline
              7 MHz &          40 meter &                                &           25\,W PEP \\
             14 MHz &          20 meter & 25\,W PEP                      &                     \\
             21 MHz &          15 meter &                                &           25\,W PEP \\
             28 MHz &          20 meter & Liknande utbredning som 27 MHz &           25\,W PEP \\
             50 MHz &           6 meter &                                &           25\,W PEP \\ \hline
            144 MHz &           2 meter &                                &           25\,W ERP


\end{tabular}
\caption{Amatörradioband öppna för instegscertifikat i Ssverige}
\end{table}

50\,MHz är egentligen ett VHF-band men räknas ibland till kortvåg (HF) då de
flesta amatörradiostationer för kortvåg omfattar även 6\,m-bandet.

Myndigheterna har gett instegscertare tillträde till amatörradioband som är
exklusiva för amatörradio vilket innebär att ett antal populära band inte finns
med. Arbete fortsätter på att försöka tillföra även 70 cm bandet (432--438 MHz)
till de band som är tillåtna för instegsamatörer men det har inte kommit med i
denna utgåva av PTS undantagsföreskrift.

\subsection{Effektrestriktioner instegscertifikat}

På frekvensbanden 7, 14, 21, 28 och 15\,MHz tillåter man 25\,W PEP tillfört
antennsystemet. Detta är effekten vid en key-down på morsenyckel. Antennvinsten
är inte medräknad på dessa frekvensband.

För 144 MHz däremot är det 25\,W ERP vilket innebär att du får mata en
halvvågsdipol direkt med 25\,W men om du ska använda riktantenner med högre
antennförstärkning än 0\,dBd eller 2,15\,dBi så måste du sänka sändareffekten i
motsvarande grad.

Anledningen till denna skillnad är att det är ganska svårt att bygga antenner
med hög förstärkning, eller att bestämma dess faktiska förstärkning på
kortvågsbanden medan det för 145\,MHz och uppåt är betydligt enklare att få till
det och dessutom mäta och bestämma antennens förstärkning.

\subsection{Restriktioner på utrustning för instegscertifikat}

Som instegsamatör får man inte operera hemmabyggda sändare, modifierade sändare
eller andra sändare som saknar CE-märkning. Det innebär att man måste hålla sig
till fabriksbyggd utrustning som säljs på den europeiska marknaden.

\subsection{Instegscertifikat utomlands}

Certifikatet äger heller ingen giltighet utomlands så som det fulla,
HAREC-baserade, amatörradiocertifikatet gör. Du får därför i de flesta fall inte
ta med din radiosändare till andra länder och använda den där på
instegscertifikatet.

\section{Tillståndsgiven radioverksamhet}

Radiotrafik som kräver tillstånd regleras i andra dokument eller av respektive
myndighet som svarar för examinering och tillståndsgivning exempelvis
luftfartsverket för luftfartsradio och så vidare. Försvarsmakten svarar också
för användning av militär radio och har långt gående möjligheter att sända på
frekvenser som i fredstid används för annat.

\subsection{Föreningsdrivna radionät}

Kommersiell radio får tillstånd genom ansökan om frekvenser för basstation och
mobila stationer efter behov och betalar då en licensavgift årligen för
detta. Det finns inget som hindrar privatpersoner egentligen att söka sådana
tillstånd och det finns exempel på privatpersoner som satt upp repeatrar och där
användarna genom att vara medlem i en förening då kan använda dessa utan att
vara radioamatörer eller ha särskilda egna tillstånd.

Detta förfarande finns bland annat i dag på några orter kring Stockholm och i
Malmö och är populärt som sambandsmedel inte minst för preppers och för
radiointresserade som vill lära sig mer om radio. Företrädesvis kör dessa på
VHF-bandet i 2-metersområdet (150\,MHz) eller UHF-bandet i 75\,cm området
(400\,MHz).

Enklast finner man det gällande dokumentet genom att söka efter det på PTS
hemsida \url{https://pts.se} där det alltid finns den senaste versionen.

Det finns också andra föreningar som exempelvis scouter med flera som har
tillstånd att bedriva radiotrafik.

\subsection{Kommersiella radionät}

Det finns också exempel på kommersiella radionät, det som tidigirare var
åkeriradionätet på 69 MHz men numera får användas rent allmänt är ett sådant
exempel. Det finns också lokala aktörer som har exempelvis taxiradio,
byggföretag som har allmän kommunikationsradio på sina byggen och många andra
varianter.

Dessa behöver tillståndssökas och en licensavgift ska betalas årligen.

\subsection{Myndigheters radionät}

Myndigheters radionät består i regel av de radionät som behövs för flera olika
verksamheter så som brandbekämpning, räddningstjänst, polisens arbete samt
även försvarsmakten räknas ibland in här. Från att ha haft flera olika system
så har dessa system numera gått samman i RAKEL (radiokommunikation för
effektiv ledning) i ett gemensamt radionät för alla blåljustjänster. RAKEL
tillhandahåller också visst skydd mot avlyssning som standard men kan även
förses med stark kryptografi för att skydda särskilt känslig information, så
kallat ''end-to-end'' krypto (E2E).

RAKEL används främst av sammhällsnyttiga tjänster och myndigheter men det
finns också exempel på när mindre aktörer och även lokaltrafik har fått
använda nätet så det är väl använt. RAKEL kommer på sikt fasas ut till förmån
för ett nytt kommunikationsnät som bygger på 5G-teknik via de kommersiella
operatörernas nät i viss mån samt där det behövs egna radiobasstationer på
700~MHz-bandet. Detta nya nät kallas ibland för SWEN (swedish emergency
network) och befinner sig ännu på planeringsstadet även om prov kan komma
ganska snart.

Enligt myndigheten för samhällsskydd och beredskap (MSB) planerar man avveckla
RAKEL till 2030 men sannolikt kommer det ta lite länge än så innan SWEN är helt
redo att ersätta det.

\chapter{Trafik}

I detta kapitel tittar vi närmare på hur radiotrafik bedrivs både för amatörradioband och andra band. Här finns också bokstaveringsalfabeten, både det svenska och det internationella, Q-koder och mycket annat som är radio\-trafik\-relaterat.

\section{Signaler och anrop}

\subsection{Landsprefix}

Här är inte alla länder med utan de vanligaste som körs från Sverige.

\begin{center}
  \begin{footnotesize}
    \begin{longtable}{lll}
      \caption{Utvalda landsprefix} \\
      \textbf{Land}                 & \textbf{DXCC}  & \textbf{Prefixserier}                             \\ \hline
			Belgien                       & ON             & ONA--OTZ                                          \\
			Canada                        & VE             & CYA--CZZ, VAA--VGZ, VOA--VOZ, VXA--VYZ, XJA--XOZ  \\
			Frankrike                     & F              & FAA--FZZ, HWA--HWZ, THA--THZ, TKA--TKZ, TMA--TMZ, \\
			                              &                & TOA--TQZ, TVA--TXZ                                \\
			Frankr. särsk.                & FG FH FK       &                                                   \\
			                              & FM, FO, FP, FR &                                                   \\
			                              & FS, FT, FW, FY &                                                   \\
			Förenta Staterna              & K              & AAA--ALZ, KAA--KZZ, NAA--NZZ, WAA--WZZ            \\
			Grekland                      & SV             & J4A--J4Z, SVA--SVZ                                \\
			Italien                       & I              & IAA--IZZ                                          \\
			Japan                         & JA             & 7JA--7NZ, 8JA--8NZ, JAA--JSZ                      \\
			Kroatien                      & 9A             & 9AA--9AZ                                          \\
			Nederländerna                 & PA             & PAA--PLZ                                          \\
			Polen                         & SP             & 3ZA--3ZZ, HFA--HFZ, SNA--SRZ                      \\
			Rumänien                      & YO             & YOA--YRZ                                          \\
			Ryssland (Eur.)               & UA1 3 4 5 6 7  & RAA--RZZ, UAA-UIZ                                 \\
			Ryssland (Asi.)               & UA8 9 0        & RAA--RZZ, UAA-UIZ                                 \\
			Schweiz                       & HB             & HBA--HBZ HEA--HEZ                                 \\
			Spanien                       & EA             & AMA--AOZ, EAA--EHZ                                \\
			Storbritt. England            & G, 2E, M       & 2AA--2ZZ, GAA--GZZ, MAA--MZZ, VPA--VQZ,           \\
			                              &                & VSA--VSZ,ZBA--ZJZ, ZNA--ZOZ, ZQA--ZQZ             \\
			Storbritt. Skottland          & GM, 2M, MM     &                                                   \\
			Storbritt. Övrigt             & VP2, VP6, VP8  &                                                   \\
			                              & VP9, VQ9, ZB   &                                                   \\
			Sverige                       & SM             & 7SA--7SZ, 8SA--8SZ, SAA--SMZ                      \\
			Tyskland                      & DL             & DAA--DRZ, Y2A--Y9Z                                \\
			Ukraina                       & UT             & EMA--EOZ, URA--UZA                                \\
			Ungern                        & HA             & HAA--HAZ, HGA--HGZ                                \\
			Österrike                     & OE             & OEA--OEZ\\
		\end{longtable}
	\end{footnotesize}
\end{center}

\subsection{Svenska signaler}

Svenska signaler förekommer inom ett antal prefix. Enligt ITU disponerar Sverige
förljande signalserier: 7SA--7SZ samt 8SA--8SZ och vidare de mer kända SAA--SMZ.
Dessa har används till varierande ändamål, exempelvis har flyget signaler i
serien SE-AAA--ZZZ. Polisen har tidigare använt signaler i serien SHA plus fyra
siffror, detta är nu ersatt med nytt system i.o.m. RAKEL. Räddningstjänsten
använde SDA med fyra siffror. Signaler som 7SA + 4 siffror används för mindre
yrkesbåtar SC+4 siffror för fritidsbåtar.

Amatörradion använder ett antal signaler, de viktigaste är:

\begin{tabular}{ll}
	SM & Amatörradiosignal utdelad av PTS (nya signaler tilldelas ej i
        serien) \\ SA & Amatörradiosignal tilldelad av SSA \\ SK & Klubbsignaler
        (som regel tvåställiga efter distriktsiffran) \\ & numera tilldelas även
        klubbar SA-signaler som är tvåställiga efter distriktssiffran \\ SL &
        Militära signaler (som regel två- eller treställiga efter
        distriktsiffran)
\end{tabular}

Dessa signaler följs av en \textit{distriktsiffra} se särskilt avsnitt och sedan
2-ställiga eller 3-ställiga bokstavskombinationer som är den personliga
signalen. Exempel är SM0UEI som är min egen signal, distriktsiffran är 0 dvs
hemmavarande i Stockholms län. Ett annat exempel kan vara SK5JV tidigare
Fagersta amatörradioklubb.

Repeatrar som tillhör klubbar får ofta signal efter klubben med tillägg /R för
repeater.

Det finns numera även ett stort antal signaler som är tillfälliga eller knutna
till särskilda event, exempelvis scoutverksamhet som ibland sänder amatörradio
och särskilda forskningsfartyg, flyg- och rymdfart mm.

Som suffix används följande:

\begin{tabular}{ll}
	/M  & Mobil (rörlig) sändaramatör, även portabel \\
	/MM & Mobil till sjöss (mobil maritime)          \\
	/AM & Mobil i luften (aeromobile)                \\
	/P  & Portabel (för stunden uppsatt station)     \\
	/R  & Repeaterstation
\end{tabular}

\subsection{Svenska distrikten}

Sverige delas in i följande distrikt efter sina län:

\begin{table}[h]
	\centering
\begin{tabular}{cl}
	\textbf{Distrikt} & \textbf{Län}                                     \\ \hline %\endhead
	      0        & Stockholm                                        \\
	      1        & Gotland                                          \\
	      2        & Västerbotten, Norrbotten                         \\
	      3        & Gävleborg, Jämtland, Västernorrland              \\
	      4        & Örebro, Värmland, Dalarna                        \\
	      5        & Östergötland, Södermanland, Västmanland, Uppsala \\
	      6        & Halland, Västra götaland                         \\
	      7        & Skåne, Blekinge, Kronoberg, Jönköping, Kalmar    \\
	      8        & Speciella stationer utanför landets gränser
\end{tabular}
\caption{Distriktssiffor i Sverige}
\end{table}

\subsubsection{Karta över svenska amatörradiodistrikt}

\begin{figure}
	\centering
	\includegraphics[width=12cm]{pic/sm-distrikt-stor}
	\label{fig:sm-distrikt}
	\caption{Svenska distrikt, karta med tillstånd från
          \href{https://SSA.SE}{ssa.se}}
\end{figure}

Distrikten förekommer som siffra i utdelade anropssignaler. Radioamatörer byter
inte distriktsiffra under resa i annat distrikt, i stället används suffix
(tillägg efter ordinarie signal) som t.ex. /M för mobil. Ofta uppger man "SM0UEI
mobilt i SM3-land" (SM0UEI/3/M) ibland (SM0UEI/3M) för att påvisa att man
befinner sig utanför ordinarie distrikt.

En radioamatör kan byta sin distriktsiffra om den sänder från ett annat distrikt
än sitt hemmavarande. Man kan också göra ett tillägg med /n där n är den siffra
för det distrikt man befinner sig i. En stockholmsamatör som befinner sig i
Gävleborgs län kan alltså antingen använda SM3UEI eller SM0UEI/3 även med
tillägget M för mobil och P för portabel om man så önskar.

Det unika för en radioamatörs signal är alltså prefixet + suffixet, som exempel
är identifieraren för SM0UEI prefixet SM och suffixet UEI eftersom
distriktsiffran kan ändra sig.

\section{Terminologi och trafik}

\subsection{Bokstaveringsalfabetet (Svenska)}

\begin{table}[H]
	\centering
\begin{longtable}{cl|cl|cl }
	A & Adam   & O & Olof    & 1 & Ett        \\
	B & Bertil & P & Petter  & 2 & Tvåa       \\
	C & Cesar  & Q & Qvintus & 3 & Trea       \\
	D & David  & R & Rudolf  & 4 & Fyra       \\
	E & Erik   & S & Sigurd  & 5 & Femma      \\
	F & Filip  & T & Tore    & 6 & Sexa       \\
	G & Gustav & U & Urban   & 7 & Sju        \\
	H & Helge  & V & Viktor  & 8 & Åtta       \\
	I & Ivar   & W & Wilhelm & 9 & Nia        \\
	J & Johan  & X & Xerxes  & 0 & Nolla      \\
	K & Kalle  & Y & Yngve   & . & Punkt      \\
	L & Ludvig & Z & Zäta    & , & Komma      \\
	M & Martin & Å & Åke     & - & Minus      \\
	N & Niklas & Ä & Ärlig   & + & Plus       \\
	  &        & Ö & Östen   &   & Mellanslag \\
\end{longtable}
\caption{Svenska bokstaveringsalfabetet}
\end{table}

\subsection{Bokstaveringsalfabetet (Internationella)}
\begin{table}[H]
\centering
\begin{tabular}{cl|cl|cl}
	A & Alfa     &  P   & Papa       & 0 & Zero    \\
	B & Bravo    &  Q   & Quebec     & 1 & One     \\
	C & Charlie  &  R   & Romeo      & 2 & Two     \\
	D & Delta    &  S   & Sierra     & 3 & Tree    \\
	E & Echo     &  T   & Tango      & 4 & Fower   \\
	F & Foxtrot  &  U   & Uniform    & 5 & Fife    \\
	G & Golf     &  V   & Victor     & 6 & Six     \\
	H & Hotel    &  W   & Whiskey    & 7 & Seven   \\
	I & India    &  X   & X-ray      & 8 & Ait     \\
	J & Juliet   &  Y   & Yankee     & 9 & Niner   \\
	K & Kilo     &  Z   & Zulu       & . & Stop    \\
	L & Lima     & Å/AA & Alfa-Alfa  & , & Decimal \\
	M & Mike     & Ä/AE & Alfa-Echo  & - & Minus   \\
	N & November & Ö/OE & Oscar-Echo & + & Plus    \\
	O & Oscar    &      &            &   & Space   \\
\end{tabular}
\caption{Internationella bokstaveringsalfabetet (ITU-alfabetet)}
\end{table}

\subsection{Q-koder}
I tabellen listas några av de vanligast förekommande Q-koderna på amatörradiobanden.
Det finns förstås många fler koder men detta anses som de vanligaste.

\begin{longtable}{ll}
	\textbf{Kod} & \textbf{Fråga / Svar}                                                         \\ \hline
	\endhead
	\caption{Q-koder}\\
\endlastfoot
	QRA & Vad heter er station?                                                \\
	    & Vår station heter ...                                                \\ \hline
	QRB & Hur långt bort från min station befinner ni er?                      \\
	    & Avståndet mellan oss är ungefär ...                                  \\ \hline
	QRG & Kan ni ange min exakta frekvens?                                     \\
	    & Er exakta frekvens är ... (MHz/kHz)                                  \\ \hline
	QRH & Varierar min frekvens/våglängd?                                      \\
	    & Er frekvens/våglängd varierar.                                       \\ \hline
	QRI & Hur är min sändningston (CW)?                                        \\
	    & Er sändningston är 1--God, 2--Varierande, 3--Dålig                   \\ \hline
	QRK & Vilken uppfattbarhet har mina signaler?                              \\
	    & Uppfattbarheten hos dina signaler är:                                \\
	    & 1--Dålig, 2--Bristfällig, 3--Ganska god, 4--God, 5--Utmärkt          \\ \hline
	QRL & Är ni upptagen?                                                      \\
	    & Jag är upptagen med ... (namn/signal) stör ej.                       \\ \hline
	QRM & Är ni störd av annan station?                                        \\
	    & Störningarna är:                                                     \\
	    & 1--Obef., 2--Svaga, 3--Måttliga, 4--Starka, 5--Mycket starka         \\ \hline
	QRN & Besväras ni av atmosfäriska störningar?                              \\
	    & Störningarna är:                                                     \\
	    & 1--Obef., 2--Svaga, 3--Måttliga, 4--Starka, 5--Mycket starka         \\ \hline
	QRO & Kan jag (ska jag) öka sändareffekten?                                \\
	    & Öka sändareffekten.                                                  \\ \hline
	QRP & Kan jag (ska jag) minska sändareffekten?                             \\
	    & Minska sändareffekten.                                               \\ \hline
	QRQ & Kan jag (får jag) öka sändningshastigheten?                          \\
	    & Öka sändningshastigheten.                                            \\ \hline
	QRS & Kan jag (skall jag) sända långsammare?                               \\
	    & Sänd långsammare.                                                    \\ \hline
	QRT & Skall jag avbryta sändningen?                                        \\
	    & Avbryt sändningen                                                    \\ \hline
	QRU & Har ni något till mig?                                               \\
	    & Jag har inget till er. Se även QTC.                                  \\ \hline
	QRV & Är ni redo?                                                          \\
	    & Jag är redo.                                                         \\ \hline
	QRX & När anropar ni mig härnäst?                                          \\
	    & Jag anropar er kl ... (på ... MHz/kHz)                               \\ \hline
	QRZ & Vem anropar mig?                                                     \\
	    & Ni anropas av ... (på ... MHz/kHz).                                  \\ \hline
	QSA & Vilken styrka har mina signaler?                                     \\
	    & Era signaler är:                                                     \\
	    & 1--Ej uppf., 2--Svaga, 3--Ganska starka, 4--Starka, 5--Mycket starka \\ \hline
	QSB & Svajar styrkan på mina signaler?                                     \\
	    & Styrkan på era signaler svajar.                                      \\ \hline
	QSK & Kan du höra mig mellan dina tecken och får jag avbryta dig?          \\
	    & Jag kan höra dig mellan mina tecken och du får avbryta.              \\ \hline
	QSL & Kan ni ge mig kvittens?                                              \\
	    & Jag kvitterar.                                                       \\ \hline
	QSO & Ha ni förbindelse med ... eller ... (förmedlat)?                     \\
	    & Jag har förbindelse med ... (via ...)                                \\ \hline
	QST & Har tidigare använts som allmänt anrop men ersatts av CQ             \\ \hline
	QSY & Skall jag övergå till att sända på annan frekvens?                   \\
	    & Gå över till att sända på annan frekvens (eller ... kHz/MHz).        \\ \hline
	QTC & Hur många telegram har ni att sända?                                 \\
	    & Jag har ... telegram till dig (eller ...).                           \\ \hline
	QTH & Vilken är er geografiska position?                                   \\
	    & Min geografiska position är ...                                      \\ \hline
	QTR & Kan ni ge mig rätt tid?                                              \\
	    & Rätt tid är ...                                                      \\
\end{longtable}

\subsection{Lokator}

Lokator (Maidenhead locator) är ett praktiskt sätt att tala om sin ungefärliga
position genom att ange endast sex stycken tecken. En lokator kan t.ex. se ut
som JO89VK vilket täcker in nordvästa Järfälla. Det finns många verktyg för att
räkna på lokator där ute, det är bra att känna sin egen. Det finns appar för
detta till telefonerna som både kan räkna på bäring, distans mellan två rutor
och dessutom via telefonens GPS bestämma vilken lokator du för närvarande
befinner dig i.

Första paret dela in jorden i 18x18 fält, dvs 20 grader per fält longitud och 10
grader per fält latitud. Varje sådant fält delas sedan in i 10x10 rutor som
numreras 0-9 på vardera axeln. Dessa i sin tur delas sedan in i 24x24 smårutor
som då får storleksordningen 2.5 grader latitud och 5 grader long. vardera.

\section{Uppträdande}

När vi kör amatörradio finns det ett antal saker att tänka på som har att göra
med hur vi beter oss mot varandra på banden. Se detta som en guide till hur man
bör uppträda på banden.

En radioamatör måste vara \textbf{tolerant}. Vi delar frekvenser med många andra
personer, en del av dem kommer inte ha samma uppfattning som du själv har om
saker och ting. Här gäller det att vara tolerant, förstående och framför allt
inte bli upprörd över personer som kanske inte beter sig som du önskade att de
betedde sig.

Radioamatörer är \emph{aldrig ensamma på banden} helt oavsett om någon svarar på
ditt allmänna anrop eller ej så finns det i det närmaste \textbf{garanterat
  någon som lyssnar}.

Tänk på vad du säger och att du undviker diskutera ämnen som kan verka
\textbf{upprörande} eller \textbf{stötande}. Ämnen som bör undvikas är
\textbf{religion} och livs\-å\-skå\-d\-ni\-ng, \textbf{politisk} ideologi,
\textbf{ekonomiska} eller \textbf{sociala} frågor m.m. där motparter kan ha
starka åsikter som inte nödvändigtvis stämmer med dina egna. Radion är inte ett
agitationsrum för sådana frågor.

\textbf{Svordomar}, \textbf{könsord} och liknande undviker vi helt. Språket
skall vara vårdat men behöver inte vara strikt. Tänk på att din motpart är inte
den enda som lyssnar utan det finns \textit{andra amatörer som lyssnar},
icke-amatörer som lyssnar, myndigheter som lyssnar och så vidare.

Ha \textbf{förståelse} för att andra kanske inte har dina egna detaljkunskaper,
professionalism med mera. Agera \textbf{ödmjukt} gentemot andra människor på
banden.

Blir du ändå upprörd, undvik att \emph{agera på det} över huvud taget. Sänd inte
över annans sändning, s.k. ''gummitumme'', eller stör på annat vis för du är
upprörd. Avsluta hellre QSO:t, byt frekvens eller återkom lite senare när du
lugnat ned dig. Tänk på att \textit{de flesta konflikter orsakas av okunskap
  eller brist på förståelse}. \textbf{Agera vuxet} i sådana situationer och
jobba för att \textbf{de-eskalera} situationen.

En skicklig amatör \textit{lyssnar mycket innan sändning}. Vi anropar på ett
korrekt sätt och avslutar på ett korrekt sätt. Vi försöker uppge våra respektive
signaler på ett \emph{tydligt och läsligt sätt}, i dag finns det en tendens att
sluddra över signalerna framför allt på 2m och 70cm banden, gör inte
det. Tydlighet är en vinning i sig.

När någon ny i ringen inträder, räkna upp de deltagande signalerna så att
personen tydligt får en bild av alla som är med och vem som är på turen före och
efter hen.

Vi pratar inte \textbf{nedvärderande} om personer varesig de är andra amatörer
eller ej, eller en viss grupp av personer. Vi undviker \textbf{sexuella
  anspelningar} och vitsar ''\textbf{under bältet}'' liksom allt för
\textbf{personliga detaljer}. Amatörradion är främst
för \textbf{tekniska diskussioner} av rent \textbf{privat natur} eller
av \textbf{allmänt intresse för hobbyn}, tester och prov med mera.

Undvik väldigt \textbf{långa sändningspass}. Ibland händer det saker hos dina
motstationer som att de får ett viktigt telefonsamtal eller måste springa ut i
köket för katten har rivit ner något, ett barn ramlar eller annat som gör att
man måste kvickt lämna radion. Att \textbf{långprata} i sådana lägen gör det
svårt att tala om ''QRX --- jag måste ta hand om en sak, anropar dig igen om 5
min.''. Enstaka gånger kanske man behöver förklara något lite längre men gör det
till en vana att lämna luckor så ofta som möjligt.

\textbf{Nödtrafik har alltid prioritet} och måste respekteras på alla
frekvenser.

\section{Repeatrar}

Repeatrars syfte är främst att förlänga kommunikationen från mobila och portabla
amatörsändare. Samtal mellan fasta stationer förekommer men om ni hör varandra
på direkten, övergå gärna till en simplex-frekvens i stället för att belägga
repeatern.

Lämna luckor mellan er när ni växlar station som sänder. Gör det möjligt för
andra att ''breaka-in'' särskilt om ert QSO fortsätter under längre tid. Ta
hänsyn till att andra kanske vill använda repeatern för att nå personer som de
inte kan nå annars. Hänsyn åt båda hållen förutsätts här.

Repeatern är en begränsad resurs. Det är inte okay att lägga beslag på den under
långa perioder när andra kanske behöver den, var ödmjuk inför att någon driver
repeatern och har satt upp den i första hand för att supporta mobila stationer.

Nödtrafik har alltid prioritet.

\section{QSO}

Konsten att genomföra ett radiosamtal (QSO) i olika sammanhang. Ofta blir folk
nervösa i början för hur detta går till. Man säger sin signal och motstationens
i fel ordning eller liknande.

Man börjar alltid med motstationens signal. Det bör fallas naturligt att ropa så
och man avslutar anropet med sin egen signal så att motstationen dels vet vem
som anropar men också andra hör. Kanske vill en annan station ha ett utbyte med
dig om du inte får svar från den tilltänkta.

Ett radiosamtal består som regel av tre delar. Först sker ett anrop, när kontakt
etablerats utväxlas ett antal meddelande (dialog) och när man är klarar avslutas
samtalet. Dessa tre delar är ganska standard. Man följer detta ganska strikt
t.ex. på kortvågen där telefoni oftast innebär SSB. Anledningen är enkel, det
går inte höra när någon släpper sändtangenten eller bara är tyst och tänker.

När man kör FM över repeatrar på VHF/UHF är det inte lika vanligt att man både
öppnar och avslutar varje sändning med motparten och sin egen signal. Men man
skall regelbundet upprepa signalerna och i praktiken är det lämpligt att göra
kanske var femte minut eller oftare.

\subsection{Anropet}

Ett anrop kan se ut ungefär såhär:

\begin{tabular}{ll}
	Station & Meddelande                            \\ \hline
	SMØUEI  & SAØMAD från SMØUEI, SAØMAD kom.       \\
	SAØMAD  & SMØUEI från SAØMAD, jag lyssnar, kom.
\end{tabular}

Därefter övergår radiosamtalet i dialog eller meddelandesändning.

\subsection{Allmänt anrop}

Används när man inte ropar på någon särskild motstation utan önskar samtal med
vem som helst. På svenska använder man ofta just orden ''allmänt anrop'' medan
på engelska är det vanligare att man uttalar CQ (seek you). Ett allmänt androp
kan se ut såhär:

-- Allmänt anrop, allmänt anrop, allmänt anrop från SM0UEI SM0UEI SM0UEI kallar
allmänt anrop och lyssnar.

Eller på engelska:

-- CQ CQ CQ this is SMØUEI calling CQ CQ CQ and standing by.

\subsection{Meddelandesändning}

-- SA0MAD från SM0UEI, tack för svaret. Din signal är 59 hos mig, mitt QTH är
JO89WA och namnet är Anders. SA0MAD från SM0UEI kom.

-- SM0UEI från SA0MAD, tack för rapporten. Din signal är 57 hos mig, jag
befinner mig i JO89VK men kommer under kvällen byta QTH. Jag kommer då vara QRV
på 3663 kHz. QSL? SM0UEI från SA0MAD.

-- SA0MAD från SM0UEI, QSL på det, QRX 19.30 på frekvens 3663 kHz.

\subsection{Avslutning}

-- SA0MAD från SM0UEI, tack för rapport och vi hörs senare, 73, slut kom

-- SM0UEI från SA0MAD, 73 tillbaka, klart slut.

\subsection{Pile-up och tävling}

Ibland kan det bli väldigt många motstationer samtidigt som ropar. Nu
gäller det att spetsa öronen! Först gäller det att sålla. Rara
signaler från långtbortistan ger mer poäng i en contest som regel
eller från länder du inte kört osv beroende på regler. Försök att
sålla med ''du som sänder från Florida'' eller ''VK7 kom igen'' osv
till det är en station kvar. Kör den snabbt, ropa CQ igen och börja
sålla igen. Stationer du hör signalen på kör du direkt.

Direkt när det uppstår en pile-up är det effektivt att köra split. Dvs
du lyssnar 5--10\,kHz upp eller ned från den frekvens du sänder
på. Det gör det lättare för dig att behålla kommandot under
pile-up. Ligger du och sänder i ett frekvensområde som är särskilt
ägnat för DX är det smart att lägga Rx-frekvensen strax utanför. Det
undviker att man stökar ned i DX-bandet.

Kör du split skall du säga det efter varje sändning. "CQ CQ CQ de Sierra Mike
Zero Uniform Echo India listening 5 up" exempelvis. På CW bör en split vara
minst 2 kHz och på SSB bör den vara minst 5 kHz ännu hellre 10 kHz. Tänk på att
när du startar din split måste du kolla så att båda frekvenserna är ok. Låt inte
din pile-up sprida ut sig för mycket även om det är kanske enklare för dig så är
risken stor att den stör någon annan.

Kör korta QSO. Utbyt snabbt den information som behövs och ta sedan nästa. Ha
förståelse för att det kan bli krockar i en pile-up. När du hör en partiell
signal eller station du vill prata med håll fast vid den. Om du har svårt att
läsa den be den repetera tills ni är klara. Genom att du är auktoriteten på
frekvensen kommer pile-up:en att lugna ned sig och vänta på sin tur. Om du
''hattar omkring'' är risken att all radiodisciplin far ut genom fönstret.

Ofta är det 1 kHz upp som gäller vid CW och digitala trafiksätt.  Du vill
försöka få tag i Södra Shetlandsöarna, ett mycket ovanligt DX, som har en stor
pile up och ropar på 14.195 MHz

\begin{tabular}{lrl}
	Station & Frekvens & Meddelande                     \\ \hline
	VP8SSI  &   14.195 & QRZ VP8SSI  5 to 15 UP         \\
	SMØUEI  &   14.203 & SM0UEI                         \\
	VP8SSI  &   14.195 & SM0UEI 59                      \\
	SMØUEI  &   14.203 & 59 thank you                   \\
	VP8SSI  &   14.195 & Thanks. QRZ VP8SSI 5 to 15 up.
\end{tabular}

Du kommer troligtvis att behöva upprepa din anropssignal flera gånger, men
lyssna efter varje gång du ropat så att du hör vem han svarar. Svarar han inte
dig så får du vänta tills han ropar något som indikerar att han avslutat
kontakten. Exempelvis: Thanks, VP8SSI eller VP8SSI 5 to 15 up.

Du deltar i en tävling där man skall ange singnalrapport och löpnumret från
start av tävlingenpå den kontakt du har. Du har hitintills kontaktat 30
stationer i tävlingen. Du har hittat en ledig frekvens och ropar CQ.

\begin{tabular}{ll}
	Station & Meddelande                       \\ \hline
	SMØUEI  & CQ contest SM0UEI SM0UEI contest \\
	ON3XYZ  & ON3XYZ                           \\
	SMØUEI  & ON3XYZ you are 59 031            \\
	ON3XYZ  & Thanks 031 you are 59 044        \\
	SMØUEI  & 44 Thanks. QRZ SM0UEI
\end{tabular}

Notera att både när man jagar DX och när man deltar i tävlingar så ger man och
får man signalrapporten 59 om man kör foni och 599 telegrafi åtminstone i
internationella tävlingar för att minska risken för fel. Får du en annan
signalrapport än 59 eller 599 i en tävling så är det viktigt du anger korrekt i
din logg. I mer lokala tester som NAC och SAC förekommer det att man ger
korrekta signalrapporter, alltså efter hur väl man hörs.

Om du försöker nå en motstation med pile-up var uppmärksam på dennes sändningar
och vänta på din tur. Tala gärna om signal och var du sänder från men släpp
sedan fram andra. Tänk på hur du själv skulle vilja att en pile-up på din egen
station skulle vilja agera. Den gyllene regeln är också alltid lyssna först ---
sänd sedan!

\clearpage




% !TeX encoding = UTF-8
% !TeX spellcheck = sv_SE

\chapter{Teknik}

I teknikkapitlet kommer vi titta på en mängd begrepp som har med radioteknik
att göra. Det är effekt och brus, beräkning av radiohorisont och utbredning
och mycket mer i detta kapitel som kan vara till nytta för radiooperatören att
känna till.

\clearpage

\section{Decibel}

Decibel är ett grundläggande begrepp inom radiovärlden och är i grunden en
logaritmisk skala som bygger på tiologaritmen. Från början använde man
begreppet Bel [B] (efter Alexander Graham Bell) men det blir ofta decimaltal
så i stället så kör man decibel. Det går 10 dB på 1 B.

Definitionen på decibel är enligt följande:

\begin{equation}
  \text{dB} = 10 \cdot \log_{10} \left( \frac{P_{ut}}{P_{in}} \right)
\end{equation}

där dB är antalet decibel, $P_{up}$ är effekten som kommer ut ur något och
$P_{in}$ är effekten som går in. Detta något kan vara en förstärkare, en
dämpare, en kabel eller nästan vad som helst som radiosignalen går igenom.

Som minnesregel kan man minnas att +3 dB betyder att effekten dubblats. Om vi
matar in 10 W i en förstärkare och får ut 20 W är förstärkningen 3 dB. På
samma sätt är -3 dB en förlust av 50 \% så om vi har en kabel med förlusten 3
dB för en viss frekvens så tappar vi halva effekten i kabeln.

\begin{table}[h]
\centering
\begin{tabular}{rr|rr|rr|rr}
	\bf dB & \bf faktor & \bf dB & \bf faktor & \bf dB & \bf faktor & \bf dB & \bf faktor \\ \hline
	     1 &       1,26 &     10 &         10 &     -1 &       0,79 &    -10 &        0,1 \\
	     2 &       1,58 &     20 &        100 &     -2 &       0,63 &    -20 &       0,01 \\
	     3 &       2,00 &     30 &     1\,000 &     -3 &       0,50 &    -30 &      0,001 \\
	     5 &       3,16 &     50 &   100\,000 &     -5 &       0,32 &    -50 &  0,000\,01
\end{tabular}
\caption{Decibel och omräkningsfaktorer}
\end{table}

\subsection{Effekt i dBW och dBm}

Effekter anges i W eller i decibel relaterat till 1 mW (dBm) eller relaterat
1W (dBW). Tabell över effekt och decibelwatt nedan:

\begin{table}[H]
\centering
\begin{tabular}{rrr|rrr|rrr}
	\textbf{Effekt} & \textbf{dBW} & \textbf{dBm} & \textbf{Effekt} & \textbf{dBW} & \textbf{dBm} & \textbf{Effekt} & \textbf{dBW} & \textbf{dBm} \\ \hline
	    1 \textmu W &          -60 &          -30 &             1 W &            0 &           30 &           100 W &           20 &           50 \\
	   10 \textmu W &          -50 &          -20 &             3 W &            5 &           35 &           250 W &           24 &           54 \\
	  100 \textmu W &          -40 &          -10 &             5 W &            7 &           37 &           500 W &           27 &           57 \\
	           1 mW &          -30 &            0 &            10 W &           10 &           40 &            1 kW &           30 &           60 \\
	          10 mW &          -20 &           10 &            20 W &           13 &           43 &          1.5 kW &           32 &           62 \\
	         100 mW &          -10 &           20 &            50 W &           17 &           47 &          2.0 kW &           33 &           63
\end{tabular}
\caption{Tabell över effekt och decibelskalor}
\end{table}

\subsection{Antennvinst i dBi och dBd}

Riktantenner sägs ha en så kallad antennvinst och den berättar något om
antennens riktverkan. En riktantenn har en (eller några få) riktningar som den
fungerar mycket bra i, övriga riktningar är den sämre än en rundstrålande
antenn.

En rundstrålande antenn strålar lika mycket i alla riktningar i ett plan.
Detta gäller exempelvis en vertikalt monterad dipol eller kvartsvågsantenn på
ett jordplan.

Antennvinsten kan jämföras med en vanlig dipol och då använder man begreppet
dBd där den sista bokstaven står för "jämfört med en dipolantenn". I
professionella radiosammanhang använder man dock en tänkt teoretisk antenn som
strålar exakt likadant i alla riktningar, inte bara i planet utan även uppåt
och nedåt. En sådan antenn finns egentligen bara i teorin men gör det möjligt
att skapa ett referensmått för just riktverkan, denna teoretiska antenn kallas
för "isotrop antenn" och då använder man beteckningen dBi där i på slutet står
för just isotrop antenn.

Skillnaden i de båda måtten kan räknas om enligt följande:

\begin{align}
\text{dBi} &= \text{dBd} + 2,15\\
\text{dBd} &= \text{dBi} - 2,15
\end{align}

\section{S-värden, signalvärde, S-meter}

Signalstyrkan i amatörradio uttrycks oftast som S-värden. Dessa fås i regel
genom nivån på AGC hos mottagaren. Därför ser man sälla utslag vid riktigt
låga signaler.

Standard kalibrering för S-metern är enligt skalan i tabellen
\ref{tab:s-varden}

\begin{table}[ht]
\centering
\begin{tabular}{r|rr|rr||r|rr|rr}
      & \multicolumn{2}{c|}{\textbf{$<$ 30 MHz}} &
  \multicolumn{2}{c}{\textbf{$>$ 30 MHz}}       && \multicolumn{2}{c|}{\textbf{$<$ 30 MHz}} &
  \multicolumn{2}{c}{\textbf{$>$ 30 MHz}}\\ \textbf{S} & \textbf{dBm}
  & \textbf{\textmu V} & \textbf{dBm} & \textbf{\textmu V}&   \textbf{S} & \textbf{dBm}
  & \textbf{\textmu V} & \textbf{dBm} & \textbf{\textmu V} \\\hline

	   1 & -121 & 0.21  & -141 & 0.02 & 9+10 & -63 & 160  & -83 & 16  \\
	   2 & -115 & 0.40  & -135 & 0.04 & 9+20 & -53 & 500  & -73 & 50  \\
	   3 & -109 & 0.80  & -129 & 0.08 & 9+30 & -43 & 1600 & -63 & 160 \\
	   4 & -103 & 1.60  & -123 & 0.16 & 9+40 & -33 & 5000 & -53 & 500 \\
	   5 & -97  & 3.20  & -117 & 0.32 &      &     &      &     &     \\
	   6 & -91  & 6.30  & -111 & 0.63 &      &     &      &     &     \\
	   7 & -85  & 12.60 & -105 & 1.26 &      &     &      &     &     \\
	   8 & -79  & 25.00 & -99  & 2.50 &      &     &      &     &     \\
	   9 & -73  & 50.00 & -93  & 5.00 &      &     &      &     &     \\
\end{tabular}
\caption{Tabell över S-värden, effekt och spänning}
\label{tab:s-varden}
\end{table}

\section{Modulationer}

\subsection{Bandbredd olika modulationer}

Olika modulationer upptar olika bandbredd. Detta är mycket viktigt att förstå
när man ställer in sin radiostation. Detta gäller särskilt att beakta i
närheten av nödfrekvenser eller bandkanten. När vi talar om bandbredder här
förstås den bandbredd vari minst 98\% av signalens effekt befinner sig.

\begin{table}[H]
\centering
\begin{tabular}{lrl}
	\textbf{Modulation} & \textbf{Bandbredd} & \textbf{Kommentarer}                  \\ \hline
	CW                  &          $<$500 Hz & Smalbandigt                           \\
	AM                  &              6 kHz & Amplitudmodulering med fullt sidband  \\
	SSB*                &              <3 kHz & Amplitudmodulering med enkelt sidband \\
	NFM                 &           7-12 kHz & Smalbandig FM                         \\
	FM                  &             16 kHz & Normal FM                             \\
	WFM                 &          16-75 kHz & Bredbandig FM (t.ex. rundradio)
\end{tabular}
\caption{Normal bandbredd vid olika modulationsslag}
\end{table}

För SSB gäller att USB och LSB fungerar lite olika. När man beräknar den
högsta eller lägsta frekvensen utgår man från den inställda frekvensen $f$.
För USB gäller då att högsta frekvensen är $f+3$\,kHz. För LSB blir det
$f-3$\,kHz. Detta innebär att om du sänder på 80\,m-bandet och du får sända
telefoni från 3600--3800\,kHz och vill lägga dig i undre bandkanten och köra
LSB skall du ställa in din radio på 3603\,kHz som lägsta frekvens. Använd
gärna lite marginal och kör exempelvis 3605\,kHz i stället.

Den egentliga modulationsfrekvensen är dock lite mer komplicerad. Normalt
anges den verkliga modulationsfrekvensen som ca 2,7\,kHz och det beror på att
man i regel filtrerar bort ljudet under 300\,Hz och det över 3000\,Hz. Detta
innebär att det akustiska frekvensomfånget blir 300--3000\,Hz och därmed
upptar signalen inte mer än 2,7\,kHz.

Det är vanligt att man märker stationer som kör överdriven bandbredd. Antingen
som en följd av att man vill öka sin modulationsvinst, okunskap eller man har
skruvat i sin radio. Syftet kan var att få bättre genomslag vid långväga
förbindelser.

\subsection{Telegrafi, CW}

CW står för continuous waves och innebär en rent omodulerad bärvåg. I
mottagaren används en oscillator för att återskapa hörbar signal. Detta
används för telegrafi och modulationsslaget är oftast A1A. Ibland sänds
telegrafi som modulerad AM-bärvåg också som då moduleras med t.ex. 700\,Hz
ton. Det är dock mindre vanligt.

Bandbredden för CW är i teorin mycket smal. I praktiken blir den lite beroende
på frekvens från några Hz till något hundratal Hz beroende på frekvensband och
sändarens beskaffenhet i form av jitter och frekvensstabilitet.

Bandbredden hos CW består av fasbruset vilket normalt är så undertryckt att det
egentligen inte betyder så mycker samt stig- respektive falltiden när man
nycklar eller släpper nyckeln. Sker detta mjukt är bandbredden låg, har man
skarp in- eller urkoppling av bärvågen nyttjar man mer bandbredd.

\subsection{Amplitudmodulering, AM}

Amplitudmodulering finns i flera olika varianter. Vanlig AM består av en
bärvåg vars styrka varieras i takt med signalen som skall sändas. Denna
förändring av bärvågen producerar sidband och det är i dessa som den egentliga
informationen återfinns. Bärvågen i sig får dock lejonparten av signalen
varför det är en sändningsklass som nästan aldrig används inom
amatörradiobanden.

Bandbredden hos AM-modulerad signal kan beräknas genom att man tar två gånger
högsta modulationsfrekvensen. Detta ger t.ex. vid en modulationsfrekvens som
går från 300-3000\,Hz en bandbredd som varierar med talet från upp till
6\,kHz.

\begin{equation}
	B=2f_m
\end{equation}

Där $f_m$ är högsta modulationsfrekvensen.

\subsection{SSB/ESB -- Enkelt sidband, en AM-variant}

Enkelt sidband används av radioamatörer för att minska på bandbredden samt
lägga radioenergin där den behövs mest. Eftersom båda sidbanden innehåller
samma information kan man filtrera bort dessa samt bärvågen innan man matar
sändarens förstärkarsteg med resultatet. I mottagaren behöver man dock
återskapa en referenssignal, en så kallad beat-oscillator gör detta. När man
ställer in frekvensen så försöker man därmed matcha den ursprungliga
frekvensen. Ligger man för långt från låter det kalle anka, kommer man för
nära sidbandet låter det dovt och basigt.

Enkelt sidband förkortas ESB eller SSB (single side-band) och man kan välja
vilket sidband man vill använda sig av. På amatörradiofrekvenser under 10 MHz
använder man LSB (lägre/lower sidbandet) och på frekvenser över 10 MHz används
USB/ÖSB (upper/övre sidbandet).

Detta är mycket av tradition. Använder man fel sorts sidband hörs det inget
vettigt när man försöker lyssna. Språkrytmerna gör dock att vi uppfattar det
som att mänskligt tal förekommer. I dag händer det att amatörer bryter mot
regeln och sänder med ``fel'' sidband på fel frekvens.

Bandbredden hos SSB är halva den för normal AM egentligen. Den kan därmed
beräknas som för AM och halveras.

\begin{equation}
	B=f_m
\end{equation}

Där $f_m$ är högsta modulationsfrekvensen.

\subsection{Frekvensmodulering, FM}

Frekvensmodulering består av att man tar en bärvåg och modulerar den med talet
genom att skifta dess frekvens. Om skiftet i frekvens är mycket litet kallas
moduleringen för fasmodulation. FM-modulering indelas i lite olika klasser
beroende på hur stor deviation som används. På amatörradions VHF- och UHF-band
talar vi om FM och NFM (Narrow FM, andra namn förekommer också). Ibland talar
man om bred FM, normal FM och smal FM på svenska.

Normal FM innebär att deviationen (hur mycket signalen avviker från
grundfrekvensen) är lika stor som den högsta modulationsfrekvensen. Det är
vanligt att kommunikationsradio använder sig av 3 kHz som högsta
modulationsfrekvens och 5 kHz deviation. Deviationen är då något bredare och ger
upphov till en viss modulationsvinst. När man talar om FM-radio på UKV-bandet
för rundradio så är deviationen ca 75\,kHz och högsta modulationsfrekvens ca
16\,kHz. Där är alltså svinget betydligt bredare än modulationen och detta är
bred FM.

Nu för tiden förordas en minskning av bandbredden för FM-sändningar på
amatörbanden, främst är det väl VHF och UHF där FM-sändning är vanligast
förekommande och där vill man ha en kanalindelning om 12,5\,kHz i stället för
som tidigare 25\,kHz. Om man studerar bandbredden hos olika FM-signaler kan
man använda sig av Carsons bandbreddsbegrepp:

\begin{equation}
	B=2(f_M+f_D)
\end{equation}

Där $B$ är bandbredden $f_M$ högsta modulationsfrekvensen och $f_d$ är
FM-signalens maximala deviation (även kallat sving). Carsons bandbreddsbegrepp
säger att 98\% av energin förekommer inom den stipulerade bandbredden. Det
betyder att att grannkanalen kan få ungefär 17\,dB lägre signal under sändning
vilket fortfarande inte är enormt bra. Carson var för övrigt den som faktiskt
uppfan SSB-modulationen.

\begin{center}
\begin{tabular}{rrrr}
Deviation & Modulation & Bandbredd & Kanaldelning\\ \hline
5 kHz & 3 kHz & 16 kHz & 25 kHz\\
2.5 kHz & 3 kHz & 11 kHz & 12.5 kHz\\
\end{tabular}
\end{center}


\section{Vågutbredning}
\todo{Vågutbredning}

\subsection{Markvåg}

\subsection{Rymdvåg}

\subsection{Skipzon}



\section{Termiska brusgolvet}
När man lyssnar i radion på en frekvens där ingen nyttosignal finns hörs ett
brus. Detta brus består av olika komponenter men en av de viktigaste är det
termiska brusgolvet. Detta sätter en nedre gräns för hur svaga signaler en
mottagare kan uppfatta.

Mottagaren har i sig också ett termiskt brus, detta beskrivs vanligen med
något som kallas \textit{brusfaktor} och säger hur mycket över det termiska
brusgolvet mottagaren bidrar med eget brus.

Bruset är avhängigt temperaturen som mottagarantennen ''ser'' och vanligtvis
inomhus använder man närmevärdet 300\,K när man räknar på detta vilket
motsvarar 27\,\textdegree C. När man riktar antennerna mot rymden eller på
vintern kan man räkna med en lägre brusfaktor pga den lägre
antenntemperaturen.

Brusgolvet kan beräknas med hjälp av Boltzmanns konstant och temperaturen i
Kelvin. Detta ger oss formeln:

\begin{equation}
	P=k_BT\Delta f
\end{equation}

Där:

\begin{tabular}{lll}
	$k_B$      & Boltzmanns konstant, $1,38065\cdot 10^{-23}$ & [J/K] \\
	$T$        & Temperaturen                                 & [K]   \\
	$\Delta f$ & Bandbredden i mottagaren                     & [Hz]
\end{tabular}

Om vi vet detta kan vi beräkna det termiska brusgolvet:

$$P = 1,38065\cdot 10^{-23} \cdot 300 \cdot 1 = 4.1495\cdot 10^{-21}$$

Om vi räknar om detta i dBm genon att ta 10-logaritmen av värdet och sedan
multiplicera med 10 samt addera 30 så får vi i stället -173,8\,dBm. Detta
avrundas normalt till -174\,dBm och är brusgolvet för 1\,Hz. En mottagare som
har en mottagarbandbredd på 25\,kHz kommer därmed att se ett brus som är
20\,000 ggr större. I decibel får vi då $-174 + 44 dB = -130$\,dBm.

För att en signal skall kunna detekteras får vi lägga på mottagarens brusgolv,
kanske 3\,dB samt hur mycket signal till brus i förhållande vi behöver, för FM
ca 12\,dB. När vi gjort detta får vi mottagarens känslighet när den är helt
ostörd som bör ligga runt $-130 + 3 + 12 = -115$\,dBm.

\begin{table}[H]
\centering
\begin{tabular}{rr|rr|rr}
	\textbf{RBW} & \textbf{N$_0$} & \textbf{RBW} & \textbf{N$_0$} & \textbf{RBW} & \textbf{N$_0$} \\ \hline
	         0.5 &           -141 &         6.25 &           -136 &          100 &           -124 \\
	         1.0 &           -144 &        12.50 &           -133 &          200 &           -121 \\
	         3.0 &           -139 &        25.00 &           -130 &         5000 &           -107 \\
	         5.0 &           -137 &        50.00 &           -127 &        10\,000 &           -104
\end{tabular}
\caption{Termiska brusgolvet vid några vanliga bandbredder}
\end{table}

Tabellen ovan visar hur brusgolvet ser ut för olika mottagarbandbretter. RBW är
Receive Band Width i kHz. $N_0$ är beteckningen för det termiska brusgolvet. Som
ni ser dubblas bruseffekten om man dubblar bandbredden vilket kanske inte är så
märkligt. Det gör att smalbandig kommunikation har ett bättre läge pga lägre
bruseffekt i mottagaren.


\section{Return loss och VSWR}

Return loss och VSWR anger samma sak. VSWR är vanligare inom amatörradio medan
man i profesionella sammanhang föredrar att prata om return loss. RL är
storleken på den reflekterade signalen i förhållande till den framåtgående
signalen. Return loss mäts alltså i dB enligt formeln $10\log(P_F/P_R)$ där
$P_F$ är den framåtgående effekten (forward) och $P_R$ är den reflekterade
signalen i retur.

\begin{longtable}{rrr|rrr|rrr}
	\caption{VSWR och return loss}\\
		\textbf{RL} & \textbf{VSWR} & \textbf{\%} & \textbf{RL} & \textbf{VSWR} & \textbf{\%} & \textbf{RL} & \textbf{VSWR} & \textbf{\%} \\ \hline
	\endfirsthead
	\textbf{RL} & \textbf{VSWR} & \textbf{\%} & \textbf{RL} & \textbf{VSWR} & \textbf{\%} & \textbf{RL} & \textbf{VSWR} & \textbf{\%} \\ \hline 	\endhead
	          1 &         17,39 &       79,43 &           8 &          2,32 &       15,85 &          20 &          1,22 &        1,00 \\
	          2 &          8,72 &       63,10 &          10 &          1,92 &       10,00 &          22 &          1,17 &        0,63 \\
	          3 &          5,85 &       50,12 &          12 &          1,67 &        6,31 &          24 &          1,13 &        0,40 \\
	          4 &          4,42 &       39,81 &          14 &          1,50 &        3,98 &          25 &          1,12 &        0,32 \\
	          5 &          3,57 &       31,62 &          15 &          1,43 &        3,16 &          26 &          1,11 &        0,25 \\
	          6 &          3,01 &       25,12 &          16 &          1,38 &        2,51 &          28 &          1,08 &        0,16 \\
	          7 &          2,61 &       19,95 &          18 &          1,29 &        1,58 &          30 &          1,07 &        0,10
\end{longtable}

Acceptabelt RL är ungefär från 12\,dB, riktigt bra från 20 dB och de bästa
komponenterna ligger runt 30\,dB. Många antenntuners som går med automatik
startar avstämningen först när VSWR är 1:2 eller sämre som motsvarar ca
10\,dB\,RL.

\section{CTCSS subtoner}

Inom amatörradio används ofta pilottoner (subtoner) som
CTCSS\footnote{Contnuous Tone-Conded Squelch System} för repeatrar och
liknande. På PMR446 används subtoner för att skapa virtuella grupper och
sub-kanaler. De som används är följande toner och frekvenser:

\begin{table}[H]
\centering
\begin{tabular}{rr|rr|rr|rr|rr}
\textbf{Nr} & \textbf{Frek} & \textbf{Nr} & \textbf{Frek} &\textbf{Nr} & \textbf{Frek} &\textbf{Nr} & \textbf{Frek} &\textbf{Nr} & \textbf{Frek} \textbf{Nr} \\ \hline
	 1 &  67,0 &  2 &  69,3 &  3 &  74,4 &  4 &  77,0 &  5 &  79,7 \\ \hline
	 6 &  82,5 &  7 &  85,4 &  8 &  88,5 &  9 &  91,5 & 10 &  94,8 \\ \hline
	11 &  97,4 & 12 & 100,0 & 13 & 103,5 & 14 & 107,2 & 15 & 110,9 \\ \hline
	16 & 114,8 & 17 & 118,8 & 18 & 123,0 & 19 & 127,3 & 20 & 131,8 \\ \hline
	21 & 136,5 & 22 & 141,3 & 23 & 146,2 & 24 & 151,4 & 25 & 156,7 \\ \hline
	26 & 162,2 & 27 & 167,9 & 28 & 173,8 & 29 & 179,9 & 30 & 186,2 \\ \hline
	31 & 192,8 & 32 & 203,5 & 33 & 210,7 & 34 & 218,1 & 35 & 225,7 \\ \hline
	36 & 233,6 & 37 & 241,8 & 38 & 250,3 &    &       &    &
\end{tabular}
\caption{CTCSS-toner, nummer och frekvens}
\end{table}

\subsection{CTCSS-zoner i Sverige}

Rekommendationer för sändaramatörers repeatrar i olika distrikt och län att
använda CTCSS för att hindra att störningar uppkommer vid conds mm. Det ger
också möjligheten för sändaramatörer att öppna just den repeater man önskar om
man har flera på samma frekvens omkring sig.

Generellt för dessa är att sista siffran i CTCSS-frekvensen är samma som
distriktsiffran.

\begin{table}[H]
\centering
\begin{tabular}{lcccc}
	\textbf{Område}    & \textbf{Primär} & \textbf{Sek. 1} & \textbf{Sek. 2} & \textbf{Sek. 3} \\ \hline
	Distrikt 0         & 77,0            & 123.0           & 67.0            & 100.0           \\
	Distrikt 1         & 218.1           & 233.6           &                 &                 \\
	Distrikt 2         & 107.2           & 146.2           & 162.2           & 186.2           \\
	Distrikt 3         & 127.3           & 141.3           & 250.3           &                 \\
	D4 Värml. / Örebro & 74.4            & 151.4           &                 &                 \\
	D4 Dalarna         & 85.4            & 151.4           &                 &                 \\
	Distrikt 5         & 82.5            & 91.5            & 103.5           & 203.5           \\
	Distrikt 6         & 114.8           & 118,8           & 94.8            & 131.8           \\
	Distrikt 7         & 79.7            & 156.7           & 210.7           &
\end{tabular}
\caption{Distrikt och CTCSS-toner}
\end{table}



\section{Missanpassning}

För maximal överföring av radioenergi från en sändare till en antenn eller
från antennen till en mottagare krävs att man har en bra anpassning mellan de
olika delarna. Dessa brukar vara tre saker som behöver "matchas" med varandra
så att man har en snarlik impedans hos alla tre.

Dessa är radioapparaten, transmissionsledningen och antennen. I de flesta fall
är radioapparaten en sändtagare som kan växla mellan sändning och mottagning.
Transmissionsledningen är vanligtvis någon form av koaxialkabel som har en så
kallad "karaktäristisk impedans" som i dessa sammanhang vanligtvis är
50\,$\Omega$.

\subsection{Stående våg}

Om anpassningen inte stämmer kommer man förlora energi på vägen och man får
ett felaktigt stående vågförhållande. På engelska kallas detta för Voltage
Standing Wave Ratio (spänningens ståendevågförhållande) och betecknas VSWR.
Det finns två vanligen förekommande sätt att mäta den och det ena är att man
mäter förhållandet mellan den framåtgående effekten och den reflekterade och
man betecknar dessa som ett bråktal, vanligen som 1:1,2.

\subsection{Return loss}

Man kan även uttrycka den i så kallad "return loss" eller RL. Detta begrepp
talar i stället om hur många decibel lägre den på grund av missanpassningen
reflekterade signalen är i förhållande till den framåtgående och utrycks i
decibel, vanligen som 20\,dB\,RL. Just denna siffra betyder att den
reflekterade signalen är 20\,dB lägre än den framåtgående signalen och det är
alltså 1\,\% som reflekteras.

\subsection{Bra och dålig anpassning}

En missanpassning kan sägas vara bra om den är bättre än 20\,dB\,RL och vi kan
säga att den är dålig om den är sämre än 10\,dB\,RL. I VSWR är det ungefär
1:1,2 och 1:2 ungefär.

\subsection{Konvertera mellan VSWR och RL}

\begin{equation}
	\text{RL} = 20 \cdot \log_{10} \frac{\text{VSWR}-1}{\text{VSWR}+1}
\end{equation}

\begin{equation}
	\text{VSWR} = \frac{1+10^\frac{\text{-RL}}{20}}{1-10^\frac{\text{-RL}}{20}}
\end{equation}

Där RL är return loss i dB och VSWR är ståendevågförhållandet, exempelvis 1,2.

\subsection{Tabell över stående våg och return loss}

\begin{table}[H]
\centering
\begin{tabular}{rr|rr|rr}
	\textbf{RL} & \textbf{VSWR} & \textbf{RL} & \textbf{VSWR} & \textbf{RL} & \textbf{VSWR} \\ \hline
	         40 &        1:1,02 &          18 &        1:1,29 &           8 &        1:2,32 \\
	         35 &        1:1,04 &          16 &        1:1,38 &           6 &        1:3,01 \\
	         30 &        1:1,07 &          14 &        1:1,50 &           4 &        1:4,42 \\
	         25 &        1:1,12 &          12 &        1:1,67 &           2 &        1:8,72 \\
	         20 &        1:1,22 &          10 &        1:1,92 &           1 &        1:17,4
\end{tabular}
\caption{Stående våg och return loss}
\label{tab:vswr-rl}
\end{table}

\section{Radioberäkningar för VHF och UHF}

\subsection{Beräkning av radiohorisonten}

Radiohorisonten är den sträcka som markvågen kan nå utan särskilda hjälpmedel
och i frånvaro av andra effekter som särskilda kondisioner (tropo eller
duktning) och liknande. Avståndet kan beräknas med hjälp av en enkel formel.
Radiohorisonten gäller egentligen bara när inget annat är i vägen men kan ge
en ledning till den längsta utbredning man kan förvänta sig med markvåg givet
en viss höjd.

För skepp på havet stämmer radiohorisonten ganska väl så man hittar denna
formel ofta i utbildningsmaterial för marin VHF men då med distansen i
nautiska mil i stället för km. För att få detta byter man konstanten 3,57 till
2,2 i stället.

\begin{equation}
	r = 3,57 \left(\sqrt{h_1}+\sqrt{h_2}\right)
\end{equation}

Där $r$ är avståndet till radiohorisonten givet i kilometer, $h_1$ är den ena
stationens antennhöjd över marken givet i meter och $h_2$ är den andra
stationens antennhöjd över marken också givet i meter.

Om motstationen befinner sig i markhöjd och det inte finns terräng som
blockerar kan man räkna ut hur lång din radiohorisont är på VHF-bandet baserat
enbart på din egen höjd med denna formel:

\begin{equation}
	r = \sqrt{17 \cdot h}
\end{equation}

För optisk sikt kan man säga följande:

\begin{equation}
	r = \sqrt{13 \cdot h}
\end{equation}

I båda ovanstående är radien $r$ i kilometer och höjden $h$ i meter.


\subsection{Sträckdämpning}

Sträckdämpningen beror på flera olika faktorer, inte minst terrängen och det
som finns mellan sändaren och mottagaren. I den fria rymden följer den en
enkel geometrisk utbredning men närmare marken behöver man stoppa in en del
kompensationsfaktorer.

\begin{equation}
	PL_0 = 20 \cdot \log(f) + 20 \cdot \log(d) - 27,55
\end{equation}

Där $PL_{0}$ är sträckdämpningen i decibel(dB) (Eng: Path Loss) mellan två
sändare givet avståndet $d$ i meter och frekvensen $f$ i MHz. Om man anger $d$
i kilometer i stället adderar man 60 till konstanten och får då 32,45.

För sträckdämpning vid mark får man mäta eller skatta en utbredningsdämpning
som en konstant $k$ som man använder för att modifiera formeln med och får då
följande variant:


\begin{equation}
	PL_m = 20 \cdot \log(f) + (20+k) \cdot \log(d) - 27,55
\end{equation}

Där $PL_m$ är sträckdämpningen vid marken. Faktorn $k$ kan uppskattas enligt
följande tabell:

\begin{table}[H]
	\begin{centering}
		\begin{tabular}{r|l}
			\textbf{k} & \textbf{Beskrivning} \\ \hline
			0 & Över öppen terräng med högre frekvenser och fri sikt\\
			5 & Lättare terräng, mindre kullar, gräs och få träd \\
			10 & Tuffare terräng med mer höjdvariation, klippblock, tätare skog \\
			15 & Urban miljö, större hus, höghus \\
			20 & Extremt urband miljö (tänk Manhattan)\\
		\end{tabular}
	\end{centering}
	\label{tab:frirum-faktor}
	\caption{Tabell över korrigeringsfaktor för frirumsutbredning vid marken}
\end{table}

I vanlig svensk terräng är det nog vanligast man hamnar i storleksordningen
5--10.



\section{Frekvenser VHF--UHF}

\subsection{Frekvenser ej amatörradio}

Dessa frekvenser är avsedda för allmänhet eller för specifika
ända\-mål som anges. Det innebär att de kan brukas för de ändamål som
anges i PTS för\-fatt\-nings\-sam\-ling\-ar och sammanställning över
ej tillståndspliktiga frekvenser. Observera att du är skyldig att
själv kontrollera bestämmelserna innan en frekvens brukas.

Effekten i tabellen är ustrålad effekt PEP om inte annat anges.

\subsubsection{Jaktfrekvenser 31 MHz}
% Frekvenser uppdaterade 250428

Frekvenserna på detta band var tidigare till för enbart jakt. I dag är
de öppna för övrig landmobil trafik och kan nyttjas till
fritidskommunikation av annat slag.

Högsta effekt är 5\,W och maximal sändningscykel är 10\% vilket
betyder att under en timme får man sända maximalt 6 minuter.

I oktober 2012 utökades de gamla jaktkanalerna med ett antal nya
kanaler vilket skedde i oktober 2012. De har ingen officiell
kanalnumrering eller egentlig benämning men jag har valt att numrera
upp dem efter de traditionella numren med början på 25.

Kanal 24 har dock tidigare haft en frekvens som inte längre är i bruk,
så det vore förvirrande att använda den -- den saknas därför i listan.
Nya kanaler är markerade i listan med asterisk och har fått nummer
från kanal 25 och uppåt efter frekvens. Detta gör att listan blir en
smula oordnad.

\begin{longtable}{rll|rll}
	\caption{Jaktfrekvenser 31\,MHz tabell}\\
	\textbf{Frekvens} & \textbf{Benämning} & \textbf{Tidigare} & \textbf{Frekvens} & \textbf{Benämning} & \textbf{Tidigare} \\ \hline
		\endfirsthead
	\textbf{Frekvens} & \textbf{Benämning} & \textbf{Tidigare} & \textbf{Frekvens} & \textbf{Benämning} & \textbf{Tidigare} \\ \hline
	\endhead
	           30,930 & Jakt 1             &                   &   31,180          &   Jakt 14          &                   \\
	           30,940 & Jakt 25*           &                   &   31,190          &   Jakt 15          &                   \\
	           30,950 & Jakt 26*           &                   &   31,200          &   Jakt 16          &                   \\
	           30,960 & Jakt 27*           &                   &   31,210        &     Jakt 17        &                   \\
	           30,970 & Jakt 28*           &                   &   31,220          &   Jakt 18          &                   \\
	           31,030 & Jakt 29*           &                   &   31,230        &     Jakt 32*       &                   \\
	           31,040 & Jakt 2             &                   &   31,240         &    Jakt 33*        &                   \\
	           31,050 & Jakt 3             & Kanal 1 Eller D   &   31,250          &   Jakt 19          &  Kanal 4 eller E  \\
	           31,060 & Jakt 4             & Kanal 2 Eller A   &   31,260          &   Jakt 20         &   Kanal 5 eller C \\
	           31,070 & Jakt 5             &                   &   31,270          &   Jakt 21          &                   \\
	           31,080 & Jakt 6             &                   &   31,280          &   Jakt 34*         &                   \\
	           31,090 & Jakt 7             &                   &   31,290          &   Jakt 35*         &                   \\
	           31,100 & Jakt 8             &                   &   31,300          &   Jakt 36*         &                   \\
	           31,110 & Jakt 9             &                   &   31,310          &   Jakt 37*         &                   \\
	           31,120 & Jakt 10            &                   &   31,320          &   Jakt 22          & Kanal 6 eller F   \\
	           31,130 & Jakt 30*           &                   &   31,330          &   Jakt 23          &                   \\
	           31,140 & Jakt 11            &                   &   31,340          &   Jakt 38*         &                   \\
	           31,150 & Jakt 12            &                   &   31,350          &   Jakt 39*        &                   \\
	           31,160 & Jakt 13            & Kanal 3 Eller B   &   31,360          &   Jakt 40*         &                   \\
	           31,170 & Jakt 31*           &                   &   31,370          &                    &
\end{longtable}

\subsubsection{Åkeribandet 69 MHz öppet för PMR}

Sedan några år tillbaka finns nu ett nytt band som kan användas för privat
landmobil radio (PMR). Bandet kallas allmänt för 69\,MHz-bandet och har blivit
mycket populärt på sina ställen.

Anledningen är bland annat en stor tillgång på FM-radio för bandet från gamla
åkeriradio som säljs för billiga pengar på diverse begagnatsajter och som
därmed gör det enkelt att komma igång.

Antennstorlekarna är moderata och det är ett ypperligt band för mobilradio där
våglängden är ungefär den dubbla mot 2-metersbandet och fungerar bra i många
sammanhang.

Nackdelen som den delar med 27\,MHz är att många antenner för fordon är
förkortade vilket minskar verkningsgraden på dessa en del men trots detta
fungerar det bra. Antennerna är dock betydligt mindre skrymmande än de för
27\,MHz.

På bandet kör man FM uteslutande och det rekommenderas att man skaffar en
radio med signalstyrkemätare då man på FM inte kan höra lika väl om man är
störd, däremot syns det ju på S-metern om man har störningar. Bandet lider
något av störningar i urbana miljöer men på landsbygden brukar det vara tyst
och fint.

Användningen av bandet regleras i PTS föreskrift Undantag från Tillståndsplikt
och innebär att man får använda max 25\,W ERP (dvs för en dipolantenn), max
10\% sändningscykel (dvs 6 min/timme), en kanalbredd om 25\,kHz och det finns
8 stycken kanaler upplåtna för landmobil radio. I strikt mening är inte
kommunikation bas-bas egentligen tillåten eftersom det är landmobil trafik som
avses i PTS bestämmelser. Kanal 1 får enbart användas för mobil-mobil trafik
inom Västra Götaland och Hallands län.

\begin{table}[ht]
  \centering
\begin{tabular}{rrl}
  Kanal & Frekvens & Noteringar                         \\ \hline
  1     & 69,0125  & End. mobil i V. Götaland o Halland \\
  2     & 69,0375  &                                    \\
  3     & 69,0625  &                                    \\
  4     & 69,0875  &                                    \\
  5     & 69,1125  &                                    \\
  6     & 69,1375  &                                    \\
  7     & 69,1625  &                                    \\
  8     & 69,1875  & Anv. som anropskanal               \\
\end{tabular}
\caption{Frekvenser 69 MHz}
\end{table}

\subsubsection{Jakt och jordbruksfrekvenser 155 MHz}

Observera att kanalnumren som är traditionella och frekvenserna inte
kommer helt i ordning. Fyra kanaler är markerade med $^R$ och har
särskilda restriktioner på svenskt innanvatten och territorialvatten.

\begin{table}[H]
\centering
\begin{tabular}{rlrl}
	\textbf{Frekvens} & \textbf{Benämning} & \textbf{Effekt} & \textbf{Användningsområde}           \\ \hline
	          155,400 & Jakt K6            &             5 W & Jakt, Jordbruk, Skogsbruk$^R$        \\
	          155,425 & Jakt K1            &             5 W & Jakt, Jordbruk$^R$                   \\
	          155,450 & Jakt K7            &             5 W & Jakt, Jordbruk, Skogsbruk$^R$        \\
	          155,475 & Jakt K2            &             5 W & Jakt, Jordbruk$^R$                   \\
	          155,500 & Jakt K3 VHF-M L1   &             5 W & Jakt, Jordbruk, Skogsbruk, Marin$^M$ \\
	          155,525 & Jakt K4 VHF-M L2   &             5 W & Jakt, Jordbruk, Skogsbruk Marin$^M$  \\
	          156,000 & Jakt K5            &             5 W & Jakt, PMR, Friluftskanal$^P$
\end{tabular}
\caption{Jakt- och jordbruksfrekvenser 155\,MHz}
\end{table}

\footnotesize
\begin{itemize}
	\item[$^M$] Delas med marina VHF-bandet, kanalerna L1 och L2 för fritidsbåtar.
	\item[$^P$] PMR-kanal som kan användas till allmän privatradio.
	\item[$^R$] Dessa kanaler \textit{får ej användas} på svenskt
          territorialvatten eller svenskt inre vatten. Se
          \href{https://pts.se/globalassets/startpage/dokument/legala-dokument/foreskrifter/radio/beslutade_ptsfs-2018-3-undantagsforeskrifter.pdf}{PTSFS2018:3}
          för mer information.
\end{itemize}
\normalsize

\subsubsection{Öppna PMR-bandet på 446 MHz}

I nya författningssamlingen står det uttryckligen att
repeateranvändning är förbjuden. De exakta kanalerna har också inte
heller bestämts utan bandet är upplåtet
446,0--446,2\,MHz. Traditionellt används nedanstående kanaler. Max
effekt är 500\,mW och antennen får ej vara av löstagbar
sort. Utrustningen skall vara godkänd för ändamålet.

Sedan sist har ytterligare spektrum tillförts och bandet har nu 16
kanaler. Det medges också digital PMR på alla frekvenserna men
rekommendationen är att använda K1--K8 för analogt och K9--K16 för
digitalt eftersom äldre apparater inte kan gå på de nya kanalerna
medan alla digitala kan det.

Vi vissa numreringar numreras de digitala kanalerna med start på
kanalnummer 1 på K9. I listan står de som D1--D8 där D står för
digitalt.

Endast smalbandig modulation med FM-deviation max 2.5 kHz skall
användas för att inte störa närliggande kanaler. Kanalrastret är
12,5\,kHz så modulationen bör rymmas inom den bandbredden.

\begin{table}[H]
\centering
\begin{tabular}{rll|rll}
	\textbf{Frekvens} & \textbf{Benämning} & \textbf{Rek. Anv.}&
	\textbf{Frekvens} & \textbf{Benämning} & \textbf{Rek. Anv.}      \\ \hline
	446,00625 & PMR446 K1          & PMR                                   &          446,10625 & PMR446 K9\ \,\,/D1       & DPMR \\
	446,01875 & PMR446 K2          & PMR                                   &          446,11875 & PMR446 K10/D2      & DPMR \\
	446,03125 & PMR446 K3          & PMR                                   &          446,13125 & PMR446 K11/D3      & DPMR \\
	446,04375 & PMR446 K4          & PMR                                   &          446,14375 & PMR446 K12/D4      & DPMR \\
	446,05625 & PMR446 K5          & PMR                                   &          446,15625 & PMR446 K13/D5      & DPMR \\
	446,06875 & PMR446 K6          & PMR                                   &          446,16875 & PMR446 K14/D6      & DPMR \\
	446,08125 & PMR446 K7          & PMR                                   &          446,18125 & PMR446 K15/D7      & DPMR \\
	446,09375 & PMR446 K8          & PMR                                   &          446,19375 & PMR446 K16/D8      & DPMR
\end{tabular}
\caption{PMR-frekvenser}
\label{tab:pmr-frekvenser}
\end{table}

\subsubsection{Kortdistansradio (KDR, SRBR)}

Kallas även SRBR för Short Range Business Radio.  Den traditionella
frekvenslistan ser ut som följer. En ny variant med frekvenser för
12,5\,kHz samt 6,25\,kHz kanaler finns också ute nu och kan ses i
tabell \ref{tab:SRBR-frekvenser}.

\begin{table}[h]
	\centering
\begin{tabular}{rlrl}
\textbf{Frekvens} & \textbf{Benämning} & \textbf{Effekt} & \textbf{Användningsområde} \\ \hline
444,600 & SRBR K1            & 2 W             & Short range business radio \\
444,625 & SRBR K2            & 2 W             & Short range business radio \\
444,800 & SRBR K3            & 2 W             & Short range business radio \\
444,825 & SRBR K4            & 2 W             & Short range business radio \\
444,850 & SRBR K5            & 2 W             & Short range business radio \\
444,875 & SRBR K6            & 2 W             & Short range business radio \\
444,925 & SRBR K7            & 2 W             & Short range business radio \\
444,975 & SRBR K8            & 2 W             & Short range business radio
\end{tabular}
\caption{Frekvenser för SRBR}
\end{table}

SRBR är ett ej tillståndspliktigt frekvenssegment som används för
yrkesmässig radiotrafik.

Rekommendationen är att man skall använda CTCSS eller motsvarande för
att undvika störa och bli störd av andra stationer som delar
frekvenserna.

Från PTSFS2018:3 så har bandet fått nya bärvågsfrekvenser och det har
blivit öppet för att köra med 25, 12,5 eller 6,25\,kHz
Kanalraster. Denna frekvenstabell blir lite mer komplicerad.

% Please add the following required packages to your document preamble:
% \usepackage{multirow}
\begin{table}[h]
	\centering
	\begin{tabular}{|l|l|l|l|l|l|}
		\hline
		\textbf{25 kHz}                                & \textbf{12,5 kHz}                               & \textbf{6,25 kHz}               & \textbf{25 kHz}                               & \textbf{12,5 kHz}                               & \textbf{6,25 kHz}               \\ \hline
		\multicolumn{1}{|c|}{\multirow{4}{*}{444,600}} & \multicolumn{1}{c|}{\multirow{2}{*}{444,59375}} & \multicolumn{1}{c|}{444,590625} & \multicolumn{1}{l|}{\multirow{4}{*}{444,850}} & \multicolumn{1}{l|}{\multirow{2}{*}{444,84375}} & \multicolumn{1}{l|}{444,840625} \\ \cline{3-3}\cline{6-6}
		\multicolumn{1}{|c|}{}                         & \multicolumn{1}{c|}{}                           & \multicolumn{1}{c|}{444,596875} & \multicolumn{1}{l|}{}                         & \multicolumn{1}{l|}{}                           & \multicolumn{1}{l|}{444,846875} \\ \cline{2-3}\cline{5-6}
		\multicolumn{1}{|c|}{}                         & \multicolumn{1}{c|}{\multirow{2}{*}{444,60625}} & \multicolumn{1}{c|}{444,603125} & \multicolumn{1}{l|}{}                         & \multicolumn{1}{l|}{\multirow{2}{*}{444,85625}} & \multicolumn{1}{l|}{444,853125} \\ \cline{3-3}\cline{6-6}
		\multicolumn{1}{|c|}{}                         & \multicolumn{1}{c|}{}                           & \multicolumn{1}{c|}{444,609375} & \multicolumn{1}{l|}{}                         & \multicolumn{1}{l|}{}                           & \multicolumn{1}{l|}{444,859375} \\ \hline
		\multirow{4}{*}{444,650}                       & \multirow{2}{*}{444,64375}                      & 444,640625                      & \multicolumn{1}{l|}{\multirow{4}{*}{444,875}} & \multicolumn{1}{l|}{\multirow{2}{*}{444,86875}} & \multicolumn{1}{l|}{444,865625} \\ \cline{3-3}\cline{6-6}
		                                               &                                                 & 444,646875                      & \multicolumn{1}{l|}{}                         & \multicolumn{1}{l|}{}                           & \multicolumn{1}{l|}{444,871875} \\ \cline{2-3}\cline{5-6}
		                                               & \multirow{2}{*}{444,65625}                      & 444,653125                      & \multicolumn{1}{l|}{}                         & \multicolumn{1}{l|}{\multirow{2}{*}{444,88125}} & \multicolumn{1}{l|}{444,878125} \\ \cline{3-3}\cline{6-6}
		                                               &                                                 & 444,659375                      & \multicolumn{1}{l|}{}                         & \multicolumn{1}{l|}{}                           & \multicolumn{1}{l|}{444,884375} \\ \hline
		\multirow{4}{*}{Saknas}                        & \multirow{2}{*}{444,66875}                      & 444,665625                      & \multicolumn{1}{l|}{\multirow{4}{*}{444,925}} & \multicolumn{1}{l|}{\multirow{2}{*}{444,91875}} & \multicolumn{1}{l|}{444,915625} \\ \cline{3-3}\cline{6-6}
		                                               &                                                 & 444,671875                      & \multicolumn{1}{l|}{}                         & \multicolumn{1}{l|}{}                           & \multicolumn{1}{l|}{444,921875} \\ \cline{2-3}\cline{5-6}
		                                               & \multirow{2}{*}{444,68125}                      & 444,678125                      & \multicolumn{1}{l|}{}                         & \multicolumn{1}{l|}{\multirow{2}{*}{444,93125}} & \multicolumn{1}{l|}{444,928125} \\ \cline{3-3}\cline{6-6}
		                                               &                                                 & 444,684375                      & \multicolumn{1}{l|}{}                         & \multicolumn{1}{l|}{}                           & \multicolumn{1}{l|}{444,934375} \\ \hline
		\multirow{4}{*}{444,800}                       & \multirow{2}{*}{444,79375}                      & 444,790625                      & \multicolumn{1}{l|}{\multirow{4}{*}{444,975}} & \multicolumn{1}{l|}{\multirow{2}{*}{444,91875}} & \multicolumn{1}{l|}{444,915625} \\ \cline{3-3}\cline{6-6}
		                                               &                                                 & 444,796875                      & \multicolumn{1}{l|}{}                         & \multicolumn{1}{l|}{}                           & \multicolumn{1}{l|}{444,921875} \\ \cline{2-3}\cline{5-6}
		                                               & \multirow{2}{*}{444,80625}                      & 444,803125                      & \multicolumn{1}{l|}{}                         & \multicolumn{1}{l|}{\multirow{2}{*}{444,93125}} & \multicolumn{1}{l|}{444,928125} \\ \cline{3-3}\cline{6-6}
		                                               &                                                 & 444,809375                      & \multicolumn{1}{l|}{}                         & \multicolumn{1}{l|}{}                           & \multicolumn{1}{l|}{444,934375} \\ \hline
		\multirow{4}{*}{444,825}                       & \multirow{2}{*}{444,81875}                      & 444,815625                      & \multicolumn{3}{l}{\multirow{4}{*}{}}                                                                                             \\ \cline{3-3}
		                                               &                                                 & 444,821875                      & \multicolumn{3}{l}{}                                                                                                              \\ \cline{2-3}
		                                               & \multirow{2}{*}{444,83125}                      & 444,828125                      & \multicolumn{3}{l}{}                                                                                                              \\ \cline{3-3}
		                                               &                                                 & 444,834375                      & \multicolumn{3}{l}{}                                                                                                              \\ \cline{1-3}
	\end{tabular}
\caption{Nya frekvensindelningen på kortdistansradiobandet}
\label{tab:SRBR-frekvenser}
\end{table}

\subsection{Marina VHF-frekvenser}

Marinbandet på VHF består både av duplex- och
simplexkanaler. Simplexkanalerna används skepp-till-skepp och även
ibland mot kustradio. Duplexfrekvenserna används t.ex. vid
telefonsamtal som sätts upp av kuststation till skepp eller
liknande. På dessa arbetskanaler sänder man även ut sjörapporter,
navigationsvarningar och annan information t.ex. säkerhetsvarningar
som är viktiga för sjöfarten.

\subsubsection{Kanalnummer och frekvens marina VHF-kanaler}

\begin{table}[H]
\centering
\begin{tabular}{rrr|rrr}
\textbf{Kanal} & \textbf{Skepp} & \textbf{Kust} &
\textbf{Kanal} & \textbf{Skepp} & \textbf{Kust} \\ \hline
01 & 156,050 & 160,650 & 60 & 156,025 & 160,625 \\
02 & 156,100 & 160,700 & 61 & 156,075 & 160,675 \\
03 & 156,150 & 160,750 & 62 & 156,125 & 160,725 \\
04 & 156,200 & 160,800 & 63 & 156,175 & 160,775 \\
05 & 156,250 & 160,850 & 64 & 156,225 & 160,825 \\
06 & 156,300 &         & 65 & 156,275 & 160,875 \\
07 & 156,350 & 160,950 & 66 & 156,325 & 160,925 \\
08 & 156,400 &         & 67 & 156,375 & \\
09 & 156,450 &         & 68 & 156,425 & \\
10 & 156,500 &         & 69 & 156,475 & \\
11 & 156,550 &         & 70 & 156,525 & DSC \\
12 & 156,600 &         & 71 & 156,575 & \\
13 & 156,650 &         & 72 & 156,625 & \\
14 & 156,700 &         & 73 & 156,675 & \\
15 & 156,750 &         & 74 & 156,725 & \\
16 & 156,800 & Anrop/Nöd   & 75 & 156,775 & \\
17 & 156,850 &         & 76 & 156,825 & \\
18 & 156,900 & 161,500 & 77 & 156,875 & \\
19 & 156,950 & 161,550 & 78 & 156,925 & 161,525 \\
20 & 157,000 & 161,600 & 79 & 156,975 & 161,575 \\
21 & 157,050 & 161,650 & 80 & 157,025 & 161,625 \\
22 & 157,100 & 161,700 & 81 & 157,075 & 161,675 \\
23 & 157,150 & 161,750 & 82 & 157,125 & 161,725 \\
24 & 157,200 & 161,800 & 83 & 157,175 & 161,775 \\
25 & 157,250 & 161,850 & 84 & 157,225 & 161,825 \\
26 & 157,300 & 161,950 & 85 & 157,325 & 161,925 \\
27 & 157,350 & 161,950 & 86 & 157,325 & 161,925 \\
28 & 157,400 & 162,000 & 87 & 157,375 & \\
   &         &         & 88 & 157,425 & \\
   &         &         &    &         & \\
L1 & 155,500 & Leisure        & F1 & 155,625 &Fishing \\
L2 & 155,525 & Leisure        & F2 & 155,775 &Fishing \\
   &         &         & F3 & 155,825 & Fishing\\
\end{tabular}
\caption{Marin VHF, frekvenslista}
\end{table}

I tabellen listas de kanaler som gäller i svenska farvatten. Andra
länder kan ha andra kanaler eller för olika ändamål. Det krävs en
särskild licens från PTS för att få nyttja dessa frekvenser och
radiooperatören skall ha ett SRC-certifikat (Short Range
Communication).

Anropskanal och nödkanal är kanal 16.

Vid duplextrafik är skiftet -4,6\,MHz.

I tabellen är kanaler som saknar kustfrekvens alltså
simplexkanaler. DSC står för ''Digital Selective Call'' ett sätt att
digitalt anropa skepp eller kuststationer, kanaler vikta för DSC får
inte användas för vanliga samtal.

Kanal 16 är anropsfrekvens om man inte vet motstationen passar en
annan kanal. Den är också nödfrekvens eftersom den passas av de
flesta.

Kanalerna L1--L2 är frekvenser avsedda för fritidsbåtar (Leisure) och
frekvenserna F1--F3 osv är avsedda för yrkesfiske. L1 och L2 delas med
kanal 3 och 4 på jaktradion vilket kan vara bra att känna till.

\subsubsection{Transponderkanaler}
\begin{longtable}{rrl}
	\textbf{Kanal} & \textbf{Skepp} & \textbf{Not} \\ \hline
	   \endhead
AIS1 & 161,975 & Digital trafik, transponder\\
AIS2 & 162,025 & Digital trafik, transponder\\
\end{longtable}

\subsubsection{Stockholm radio}
% Frekvenser kontrollerade 250428
% Ett antal felaktigheter fixade

Radiohorisonten är beräknad i nautiska mil, skeppet lägger till sin egen radiohorisont för att bestämma om det går att nå kuststationen eller ej.

\textbf{Ostkusten}

\begin{longtable}{lrr|lrr}
\textbf{Kuststation} & \textbf{Kanal} & \textbf{Horisont} & \textbf{Kuststation} & \textbf{Kanal}& \textbf{Horisont}\\
\hline
\endhead
Kalix          & 60 & 39 & Luleå         & 61 & 26 \\
Skellefteå     & 23 & 44 & Umeå          & 62 & 54 \\
Örnsköldsvik   & 63 & 42 & Mjällom       & 64 & 43 \\
Kramfors       & 83 & 43 & Härnösand     & 23 & 36 \\
Sundsvall      & 60 & 36 & Hudiksvall    & 61 & 54 \\
Gävle          & 23 & 37 & Östhammar     & 62 & 44 \\
Väddö          & 82 & 32 & Nacka         & 26, 23* & 43 \\
Sv. högarna    & 83 & 15 & Södertälje    & 66 & 30 \\
Torö           & 61 & 26 & Gotska sandön & 65 & 22 \\
Norrköping     & 64 & 43 & Västervik     & 23 & 45 \\
Fårö           & 62 & 25 & Visby         & 63 & 23 \\
Hoburgen       & 61 & 25 & Kalmar        & 60 & 40 \\
Ölands s. udde & 22 & 23 & Karlskrona    & 81 & 24 \\
Karlshamn      & 62 & 48 & Kivik         & 21 & 39\\
\end{longtable}
*) Sänder ej väder, varningar eller andra listor

\clearpage
\textbf{Västkusten}

\begin{longtable}{lrr|lrr}
\textbf{Kuststation} & \textbf{Kanal} & \textbf{Horisont} &
\textbf{Kuststation} & \textbf{Kanal} & \textbf{Horisont} \\
\hline
\endhead

Strömstad   & 22 & 25 & Grebbestad & 62 & 25 \\
Kungshamn   & 23 & 23 & Uddevalla  & 61 & 47 \\
Tjörn       & 81 & 26 & Göteborg   & 60 & 43 \\
Grimeton    & 22 & 35 & Halmstad   & 62 & 52 \\
Helsingborg & 60 & 28 & Malmö      & 65 & 25 \\
\end{longtable}

\textbf{Insjöarna}

\begin{longtable}{lrr|lrr}
\textbf{Kuststation} & \textbf{Kanal} & \textbf{Horisont} &
\textbf{Kuststation} & \textbf{Kanal} & \textbf{Horisont} \\
\hline
\endhead

Västerås  & 63 & 40 & Trollhättan & 03 & 32 \\
Bäckefors & 05 & 50 & Kinnekulle  & 01 & 43 \\
Karlstad  & 65 & 36 & Jönköping   & 23 & 49 \\
Motala    & 62 & 47 & Hjälmaren   & $\dagger$   &    \\
\end{longtable}

$\dagger$) Hjälmaren är permanent tagen ur drift.

\subsection{Frekvenser amatörradio VHF--UHF}

I denna skrift försöker vi omfatta de viktigaste VHF och UHF-banden
för amatörradio vilket inkluderar 6\,m-bandet, 2\,m-bandet,
70\,cm-bandet och 23\,cm-bandet.

\subsubsection{Kanalnumrering VHF/UHF}

Denna typ av kanalnumrering är överenskommen inom IARU region 1 för
6\,m, 2\,m och 70\,cm banden på
amatörradiofrekvenser. Kanalnumreringen består av ett prefix som anger
vilket band och här används F--6\,m, V--2\,m, U--70\,cm. Därefter
används 2 siffror på 6m och 2m banden och tre siffror på 70cm bandet
för att ange kanal.

Repeaterfrekvenser anges med tillägget R före kanalnumret och innebär
då normalt duplex med det skift som normalt används för bandet. Vid
repeatrar är det repeaterns utfrekvens som anges, dvs den som
mobilstationen lyssnar på. Exempel: RV48.

\begin{tabular}{crrlll}
	\textbf{Band} & \textbf{Startfrekvens} & \textbf{Kanalraster} & \textbf{Duplex} & \textbf{Första kanal} & \textbf{Beräknas}    \\ \hline
	    6\,m      &            51.000\,MHz &            10.0\,kHz & -100\,kHz       & F00                   & $f=51+k\cdot0.01$    \\
	              &                        &                      &                 &                       & $k=(f-51)/0,01$      \\ \hline
	    2\,m      &           145.000\,MHz &            12.5\,kHz & -600\,kHz       & V00                   & $f=145+k\cdot0.0125$ \\
	              &                        &                      &                 &                       & $k=(f-145)/0,0125$   \\ \hline
	   70\,cm     &           430.000\,MHz &            12.5\,kHz & -2000\,kHz      & U000                  & $f=430+k\cdot0.0125$ \\
	              &                        &                      &                 &                       & $k=(f-430)/0,0125$   \\ \hline
\end{tabular}

Eftersom amatörradiobanden ser lite olika ut i olika länder förekommer
det kanaler i numreringen som inte är tillåtna på vissa ställen. Det
är därför viktig att kontrollera att man fortfarande följer
bandplanerna i den region man är.

\begin{itemize}
\item I 6\,m bandet finns inga FM-kanaler definierade under 51\,MHz.
\item För 2\,m-bandet är FM-kanaler endast definierade från 145\,MHz och uppåt.
\item I 70\,cm-bandet är inga kanaler definierade i intervallet
  432.000--433.000\,MHz. Observera att startfrekvensen är utanför
  70\,cm bandplanen i IARU region 1.
\end{itemize}

\textit{OBS!\\ Information om kanalnumreringen för 23\,cm-bandet tas
  tacksamt mot. Maila mig på anders@sikvall.se om du har korrekt
  information.}

\subsubsection{Införande av 12.5\,kHz kanalavstånd}

För ett antal år sedan beslutade man sig att gå mot ett smalare
kanalraster på VHF och UHF och införde härmed kanalavstånd på 12.5~kHz
i bandplanerna. Ustrustning med 25~kHz kanalraster är fortsatt
tillåten och detta är en rekommendation. Vid införandet av detta så
kom även ett nytt numreringsalternativ för kanalsystemen baserat på en
basfrekvens (som ibland på svenska band ligger utanför vårt band) och
därefter numrerade man i ordning för respektive 12.5~kHz steg och
10~kHz för kortvåg.

\begin{table}[h]
\centering
\begin{tabular}{rrrr}
Kod & Basfrekvens & Kanalavstånd & Repeaterskift \\
    & [MHz]       & [kHz]        & [kHz] \\ \hline
H & 29,500 & 10,0 & -100 \\
F & 51,000 & 10,0 & -600 \\
V & 145,000& 12,5 & -600 \\
U & 430,000& 12,5 & -2000 \\
M & 1240,000 & 25 & -6000 \\
\end{tabular}
\label{tab:kanalavstand}
\caption{Kanalavstånd och beteckning olika frekvensband}
\end{table}

\subsubsection{FM-kanaler 6m-bandet}

\begin{longtable}{rrl|rrl}
\textbf{Kanal} & \textbf{Tidigare} & \textbf{Anm}
&  \textbf{Kanal} & \textbf{Tidigare} & \textbf{Anm} \\ \hline
	51,500 &      F50 &       & 51,750 &      F75 &  \\
	51,510 &      F51 & Anrop & 51,760 &      F76 &  \\
	51,520 &      F52 &       & 51,770 &      F77 &  \\
	51,530 &      F53 &       & 51,780 &      F78 &  \\
	51,540 &      F54 &       & 51,790 &      F79 &  \\
	51,550 &      F55 &       & 51,800 &      F80 &  \\
	51,560 &      F56 &       & 51,810 &     RF81 &  \\
	51,570 &      F57 &       & 51,820 &     RF82 &  \\
	51,580 &      F58 &       & 51,830 &     RF83 &  \\
	51,590 &      F59 &       & 51,840 &     RF84 &  \\
	51,600 &      F60 &       & 51,850 &     RF85 &  \\
	51,610 &      F61 &       & 51,860 &     RF86 &  \\
	51,620 &      F62 &       & 51,870 &     RF87 &  \\
	51,630 &      F63 &       & 51,880 &     RF88 &  \\
	51,640 &      F64 &       & 51,890 &     RF89 &  \\
	51,650 &      F65 &       & 51,900 &     RF90 &  \\
	51,660 &      F66 &       & 51,910 &     RF91 &  \\
	51,670 &      F67 &       & 51,920 &     RF92 &  \\
	51,680 &      F68 &       & 51,930 &     RF93 &  \\
	51,690 &      F69 &       & 51,940 &     RF94 &  \\
	51,700 &      F70 &       & 51,950 &     RF95 &  \\
	51,710 &      F71 &       & 51,960 &     RF96 &  \\
	51,720 &      F72 &       & 51,970 &     RF97 &  \\
	51,730 &      F73 &       & 51,980 &     RF98 &  \\
	51,740 &      F74 &       & 51,990 &     RF99 &
\end{longtable}

\subsubsection{FM-kanaler 2m-bandet}

\begin{longtable}{rrl|rrl}

\textbf{Frekvens} & \textbf{Kanal} & \textbf{Anm} &
\textbf{Frekvens} & \textbf{Kanal} & \textbf{Anm} \\ \hline

145,2125 & V17 &              & 145,5000 & V40  & S20  FM Anrop \\
145,2250 & V18 & S9           & 145,5125 & V41  &               \\
145,2375 & V19 & INET GW      & 145,5250 & V42  & S21           \\
145,2500 & V20 & S10          & 145,5375 & V43  &               \\
145,2625 & V21 &              & 145,5500 & V44  & S22           \\
145,2750 & V22 & S11          & 145,5625 & V45  &               \\
145,2875 & V23 & INET GW      & 145,5750 & V46  & S23           \\
145,3000 & V24 & S12  RTTY    & 145,5875 & V47  &               \\
145,3125 & V25 &              & 145,6000 & RV48 & R0            \\
145,3250 & V26 & S13          & 145,6125 & RV49 & R0X           \\
145,3375 & V27 & INET GW      & 145,6250 & RV50 & R1            \\
145,3500 & V28 & S14          & 145,6375 & RV51 & R1X           \\
145,3625 & V29 &              & 145,6500 & RV52 & R2            \\
145,3750 & V30 & S15 DV Anrop & 145,6625 & RV53 & R2X           \\
145,3875 & V31 &              & 145,6750 & RV54 & R3            \\
145,4000 & V32 & S16          & 145,6875 & RV55 & R3X           \\
145,4125 & V33 &              & 145,7000 & RV56 & R4            \\
145,4250 & V34 & S17 Scout    & 145,7125 & RV57 & R4X           \\
145,4375 & V35 &              & 145,7250 & RV58 & R5            \\
145,4500 & V36 & S18          & 145,7375 & RV59 & R5X           \\
145,4625 & V37 &              & 145,7500 & RV60 & R6            \\
145,4750 & V38 & S19          & 145,7625 & RV61 & R6X           \\
145,4875 & V39 &              & 145,7750 & RV62 & R7            \\
         &     &              & 145,7875 & RV63 & R7X
\end{longtable}

X-kanalerna uppstod när man fick platsbrist och man övergick till en
12.5\,kHz kanaldelning för repeatrar. Först senare övergick man även
till samma kanaldelning på övriga FM-kanaler. De gamla
simplexkanalerna hade inte så stor spridning i Sverige men förekom
rikligt t.ex. i Tyskland med S20 som anropsfrekvens (eller
aktivitetscenter som det numera kallas).


\subsubsection{FM-kanaler 70cm-bandet}

\begin{longtable}{rrl|rrl}
\textbf{Frekvens} & \textbf{Kanal} & \textbf{Anm} &
\textbf{Frekvens} & \textbf{Kanal} & \textbf{Anm} \\ \hline

433,4000 & U272 & SSTV    & 433,7125 & U297 &      \\
433,4125 & U273 &         & 433,7250 & U298 &      \\
433,4250 & U274 &         & 433,7375 & U299 &      \\
433,4375 & U275 &         & 433,7500 & U300 &      \\
433,4500 & U276 & Digital & 433,7625 & U301 &      \\
433,4625 & U277 &         & 433,7750 & U302 &      \\
433,4750 & U278 &         & 433,7875 & U303 &      \\
433,4875 & U279 &         & 433,8000 & U304 & APRS \\
433,5000 & U280 & Anrop   & 433,8125 & U305 &      \\
433,5125 & U281 &         & 433,8250 & U306 &      \\
433,5250 & U282 &         & 433,8375 & U307 &      \\
433,5375 & U283 &         & 433,8500 & U308 &      \\
433,5500 & U284 &         & 433,8625 & U309 &      \\
433,5625 & U285 &         & 433,8750 & U310 &      \\
433,5750 & U286 &         & 433,8875 & U311 &      \\
433,5875 & U287 &         & 433,9000 & U312 &      \\
433,6000 & U288 & RTTY    & 433,9125 & U313 &      \\
433,6125 & U289 &         & 433,9250 & U314 &      \\
433,6250 & U290 &         & 433,9375 & U315 &      \\
433,6375 & U291 &         & 433,9500 & U316 &      \\
433,6500 & U292 &         & 433,9625 & U317 &      \\
433,6625 & U293 &         & 433,9750 & U318 &      \\
433,6750 & U294 &         & 433,9875 & U319 &      \\
433,6875 & U295 &         & 434,0000 & U320 &      \\
433,7000 & U296 & FAX     &          &      &      \\

\end{longtable}

\begin{longtable}{rrl|rrl}
\textbf{Frekvens} & \textbf{Kanal} & \textbf{Anm}
&  \textbf{Frekvens} & \textbf{Kanal} & \textbf{Anm} \\ \hline

434,6000 & RU368 & RU0  & 434,8000 & RU384 & RU8   \\
434,6125 & RU369 & RU0X & 434,8125 & RU385 & RU8X  \\
434,6250 & RU370 & RU1  & 434,8250 & RU386 & RU9   \\
434,6375 & RU371 & RU1X & 434,8375 & RU387 & RU9X  \\
434,6500 & RU372 & RU2  & 434,8500 & RU388 & RU10  \\
434,6625 & RU373 & RU2X & 434,8625 & RU389 & RU10X \\
434,6750 & RU374 & RU3  & 434,8750 & RU390 & RU11  \\
434,6875 & RU375 & RU3X & 434,8875 & RU391 & RU11X \\
434,7000 & RU376 & RU4  & 434,9000 & RU392 & RU12  \\
434,7125 & RU377 & RU4X & 434,9125 & RU393 & RU12X \\
434,7250 & RU378 & RU5  & 434,9250 & RU394 & RU13  \\
434,7375 & RU379 & RU5X & 434,9375 & RU395 & RU13X \\
434,7500 & RU380 & RU6  & 434,9500 & RU396 & RU14  \\
434,7625 & RU381 & RU6X & 434,9625 & RU397 & RU14X \\
434,7750 & RU382 & RU7  & 434,9750 & RU398 & RU15  \\
434,7875 & RU383 & RU7X & 434,9875 & RU399 & RU15X \\
         &       &      & 435,0000 & RU400 &       \\

\end{longtable}

RU0X osv är här en efterkonstruktion. Egentligen så användes sällan
``X-frekvenserna'' på 70cm eftersom man dels hade nästan dubbla
antalet frekvenser för repeatrar och sedan gammalt ville man
egentligen inte ha ett smalare kanalraster, i tidernas begynnelse
körde många amatörer 70cm genom frekvenstrippling från 2m. $144,000
\cdot 3 = 432,000$\,MHz och $144,025 \cdot 3 = 432,075$\,MHz varför
man till och med hade bredare kanalraster de-facto.


\clearpage %%% Layout

\subsection{Scouters frekvenser, JOTA}

Scouter finns ofta QRV under vissa helger, \textit{Jamboree On The Air, JOTA},
förekommer några gånger per år. Här är en sammanställning av de
standardfrekvenser scouter nyttjar om de inte kör repeatrar eller leta upp
motstationer själva. Scouter kan antingen ha egna signaler, köra under
tillfälliga signaler eller vara second operator åt med någon klubbsignal.

\subsection{Nordiska scoutfrekvenser VHF}

\begin{center}
\begin{tabular}{lrr}
	\textbf{Mode} & \textbf{Frekvens} & \textbf{Kanal} \\ \hline
	FM            &      145.425  MHz &   V34 \\
	SSB           &      144.240  MHz &  \\
	CW            &      144.050  MHz &
\end{tabular}
\end{center}

Jotan hålls alltid den 3:e hela (lördag och söndag) helgen i oktober varje år.
Jotan startar officiellt vid invigningen på lördag förmiddag och slutar natten
till måndagen klockan 00:00. Många börjar redan på fredagskvällen och avslutar
på söndagseftermiddagen.

Sändningar under denna tid förekommer från ocertifierade scouter som lånar
klubbsignal, har en tillfällig signal utdelad, ibland lånar enskilda
sändaramatörer ut sina signaler.

Sändningarna skall dock alltid ske under direkt överinseende av en radioamatör
men var beredd på att det kommer vara en viss ovana och ske en del misstag.
Strunta i det och ge scouterna en kul radioupplevelse.

\scriptsize
\subsection{Repeatrar, länkar och fyrar VHF/UHF}
\subsubsection{Svenska fyrar VHF/UHF}
\begin{longtable}{llrlrrrlrll}
	Signal   & Placering           &   Frekvens & Loc    &    P & MASL & MAGL & Dir     &  Band & Mode   & Dist \\ \hline
	SKØCT/B  & Stockholm           &  5760.9030 & JO99JX &   80 &   60 &   30 & Omni    &   6cm & CW     & 0    \\
	SKØEN/B  & Väddö               & 10368.8470 & JO99JX & 1000 &   60 &   30 & Omni    &  23cm & CW     & 0    \\
	SKØEN/B  & Väddö               &  1296.8350 & JO99JX &    4 &   70 &   40 & Omni    &  23cm & CW     & 0    \\
	SKØCT/B  & Stockholm           & 10368.8400 & JO89XJ &  0.1 &   50 &   20 & Omni    &   3cm & CW     & 0    \\
	SK1UHF   & Klintehamn          &   432.4050 & JO97CJ &   30 &   65 &   60 & Omni    &  70cm & CW     & 1    \\
	SK1VHF   & Klintehamn          &   144.4470 & JO97CJ &   10 &   65 &   60 & Omni    &    2m & CW     & 1    \\
	SK1UHG   & Klintehamn          &  1296.9500 & JO97CJ &   30 &   65 &   60 & Omni    &  23cm & CW     & 1    \\
	SK1SHH   & Klintehamn          & 10368.8500 & JO97CJ &    3 &   52 &   52 & Omni    &   3cm & CW     & 1    \\
	SK2VHF   & Vindeln/Buberget    &   144.4570 & JP94TF &   80 &  300 &   10 & N+SV    &    2m & CW     & 2    \\
	SK2CP/B  & Kiruna/Esrange      &    50.0520 & KP07MU &   30 &  630 &      & Omni    &    6m & CW     & 2    \\
	SK2SHF   & Vännäs/Granl.b.     &  1296.9850 & JP93VU &   10 &  250 &   50 &         &  23cm & CW     & 2    \\
	SK2SHF   & Vännäs/Granl.b.     &  2320.9850 & JP93VU &   10 &  250 &   50 &         &  13cm & CW     & 2    \\
	SK2DR/B  & Råneå               &  1296.9370 & KP15EU &   14 &      &      & South   &  23cm & CW     & 2    \\
	SK2DR/B  & Råneå               & 10368.8200 & KP15EU &    4 &      &      & South   &   3cm & CW     & 2    \\
	SK3UHH   & Nordingrå/Rävsön    &  2320.9000 & JP92FW &      &  200 &    5 & 220°    &  13cm & CW     & 3    \\
	SK3UHF   & Nordingrå/Rävsön    &   432.4550 & JP92FW &   50 &  200 &    8 & Omni    &  70cm & CW     & 3    \\
	SK3UHG   & Nordingrå/Rävsön    &  1296.8550 & JP92FW &   30 &  200 &   10 & Omni    &  23cm & CW     & 3    \\
	SK3SIX   & Östersund           &    50.4680 & JP73HC &   15 &  480 &    7 & Omni    &    6m & CW     & 3    \\
	SK3VHF   & Östersund           &   144.4210 & JP73HC &   50 &  480 &    7 & 180°    &    2m & CW     & 3    \\
	SM3KDR   & Krokom/Aspås        &    28.2860 & JP73GI &    1 &  380 &    5 & E-W     &   10m & CW     & 3    \\
	SK4BX/B  & Garphyttan/Ånnaboda & 10368.9600 & JO79LI &      &  270 &   10 &         &   3cm & CW     & 4    \\
	SK4MPI   & Borlänge            &   144.4120 & JP70PI &  200 &  380 &   20 & NV+NO   &   2cm & PI4/CW & 4    \\
	SK4BX/B  & Garphyttan/Storst.  &   432.4600 & JO79LH &   50 &  270 &   10 & N E S W &  70cm & CW     & 4    \\
	SK4BX/B  & Garphyttan/Ånnab.   &  1296.9600 & JO79LI &      &  270 &   10 &         &  23cm & CW     & 4    \\
	SK6YH/B  & Göteborg            & 10368.8080 & JO57XQ & 1000 &  135 &   40 & 184°    &   3cm & CW     & 6    \\
	SK6MHI   & Hönö                &  1296.8000 & JO57TQ &   30 &   40 &   30 & Omni    &  23cm & CW     & 6    \\
	SK6MHI   & Göteborg            &  5760.8000 & JO57XQ &   10 &  135 &   40 & Omni    &   6cm & CW     & 6    \\
	SK6UHF   & Varberg/Veddige     &   432.4120 & JO67EH &   10 &  175 &   25 & Omni    &  70cm & CW     & 6    \\
	SK6SHG   & Tjörn Island        & 24048.8830 & JO57TX & 2x1W &  118 &    8 & N/S     & 1.5cm & CW     & 6    \\
	SK6MHI   & Göteborg            & 24048.8000 & JO57XQ &   10 &  135 &   40 & Omni    & 1.5cm & CW     & 6    \\
	SK6UHI   & Tjörn Island        &  1296.8050 & JO57TX &   30 &  128 &   18 & Omni    &  23cm & CW     & 6    \\
	SK6VHF   & Tjörn Island        &   144.4060 & JO57TX &   10 &  122 &   12 & Omni    &    2m & CW     & 6    \\
	SK6WW/B  & Karlsborg/Vaberget  & 10368.8350 & JO78FM &    7 &  240 &   20 & Omni    &   3cm & CW     & 6    \\
	SK6EI/B  & Skövde              &    50.4600 & JO68VJ &   10 &  300 &   30 & South   &    6m & CW     & 6    \\
	SM7DTE/B & Gärsnäs             &  5760.8410 & JO75DN &   40 &   86 &    8 & Omni    &   6cm & CW     & 7    \\
	SM7DTE/B & Gärsnäs             & 10368.8410 & JO75DN &   40 &   86 &    8 & Omni    &   3cm & CW     & 7    \\
	SM7DTE/B & Gärsnäs             & 24048.8430 & JO75DN &   70 &   86 &    8 & Omni    & 1.5cm & CW     & 7    \\
	SK7GH/B  & Värnamo             &    28.2980 & JO77BF &    5 &  230 &   10 & Omni    &   10m & CW     & 7    \\
	SK7VHF   & Sjöbo               &   144.4610 & JO65UQ &   10 &   25 &   25 & Omni    &    2m & CW     & 7    \\
	SK7GH/B  & Värnamo             &  1296.8250 & JO77AE &   10 &  230 &   10 & Omni    &  23cm & CW     & 7
\end{longtable}

\subsubsection{Repeatrar distrikt 0}
\begin{longtable}{llllrrl}
	Typ      & Modulation & Signal   & Ort             & Utfrekvens &   Duplex & Loc    \\ \hline
	Hotspot  & D-Star     & SKØAI-B  & Stockholm       &   433.4625 &  Simplex & JO89XG \\
	Hotspot  & D-Star     & SEØYOS-C & M/Y Erika       &   434.4500 & Duplex 0 & JO99AH \\
	Link     & FM         & SKØMM    & Sandhamn        &   434.3750 &  Simplex & JO99KG \\
	Link     & FM         & SKØMM/L  & Ingarö          &   145.2250 &  Simplex & JO99GG \\
	Link     & FM         & SMØUAO   & Kopparmora      &   434.4875 &  Simplex & JO99HI \\
	Link     & FM         & SKØRVF   & Hagsätra        &   434.4250 &  Simplex & JO99AG \\
	Repeater & FM         & SKØNN/R  & Haninge         &   434.7750 &   -2.000 & JO99BE \\
	Repeater & FM         & SKØCT/R  & Kista           &  1297.0250 &   -6.000 & JO89XJ \\
	Repeater & FM         & SLØZS/R  & Västberga       &   145.6000 &   -0.600 & JO89XH \\
	Repeater & FM         & SLØZS/R  & Västberga       &   434.9000 &   -2.000 & JO89XH \\
	Repeater & FM         & SKØPQ/R  & Kista           &   145.6750 &   -0.600 & JO89XJ \\
	Repeater & FM         & SMØOFV/R & Solna           &   145.7625 &   -0.600 & JO89XI \\
	Repeater & FM         & SKØZA/R  & Solna           &   434.8500 &   -2.000 & JO89XI \\
	Repeater & FM         & SKØRDZ   & Brottby         &   145.6500 &   -0.600 & JO99DN \\
	Repeater & FM         & SAØAZT/R & Brottby         &   434.8000 &   -2.000 & JO99BM \\
	Repeater & FM         & SM5DWC/R & Södertälje      &   434.8250 &   -2.000 & JO89TE \\
	Repeater & FM         & SMØMMO/R & Tullinge        &   145.6625 &   -0.600 & JO89XF \\
	Repeater & FM         & SKØCT/R  & Kista           &   434.6250 &   -2.000 & JO89XJ \\
	Repeater & FM         & SMØYIX/R & Söder           &   434.7250 &   -2.000 & JO99BH \\
	Repeater & FM         & SKØYZ/R  & Vallentuna      &   434.8625 &   -2.000 & JO99BM \\
	Repeater & FM         & SKØCT/R  & Kista           &   434.6625 &   -2.000 & JO89XJ \\
	Repeater & FM         & SKØQO/R  & Haninge         &   145.6875 &   -0.600 & JO99BE \\
	Repeater & FM         & SKØQO/R  & Haninge         &   434.7500 &   -2.000 & JO99BE \\
	Repeater & FM         & SKØRMT   & Täby            &   434.7375 &   -2.000 & JO99AK \\
	Repeater & DMR        & SKØRMT   & Täby            &   434.7375 &   -2.000 & JO99AK \\
	Repeater & C4FM       & SKØRMT   & Täby            &   434.7375 &   -2.000 & JO99AK \\
	Repeater & D-Star     & SKØRMT   & Täby            &   434.7375 &   -2.000 & JO99AK \\
	Repeater & DMR        & SKØRMT   & Täby            &   434.7375 &   -2.000 & JO99AK \\
	Repeater & C4FM       & SKØRMT   & Täby            &   434.7375 &   -2.000 & JO99AK \\
	Repeater & D-Star     & SKØRMT   & Täby            &   434.7375 &   -2.000 & JO99AK \\
	Repeater & DMR        & SKØRYG   & Kista           &   434.9500 &   -2.000 & JO89XJ \\
	Repeater & DMR        & SKØRYG   & Sthlm city      &   434.9625 &   -2.000 & JO99AI \\
	Repeater & DMR        & SMØWIU/R & Nynäshamn       &   434.6125 &   -2.000 & JO88XV \\
	Repeater & DMR        & SMØWIU/R & Botkyrka        &   434.8750 &   -2.000 & JO89WG \\
	Repeater & C4FM       & SKØNN    & Haninge         &   434.5375 &   -2.000 & JO99CF \\
	Repeater & DMR        & SKØSX    & Kista           &   434.9875 &   -2.000 & JO89XJ \\
	Repeater & DMR        & SKØRMQ   & Tyresö          &   434.5125 &   -2.000 & JO99CH \\
	Repeater & DMR        & SMØWIU-4 & Högdalen        &   145.5750 &   -0.600 & JO99AF \\
	Repeater & FM         & SKØMG/R  & Skarpnäck       &   145.7000 &   -0.600 & JO89TE \\
	Repeater & FM/DMR     & SKØRIX   & Sthlm city      &   145.6250 &   -0.600 & JO99AH \\
	Repeater & DMR        & SGØRPF   & Rimbo           &   434.7875 &   -2.000 & JO99BT \\
	Repeater & DMR        & SKØRYG   & Upplands Väsby  &   434.7625 &   -2.000 & JO89XM \\
	Repeater & FM/DMR     & SKØRPF   & Sigtuna         &   434.8875 &   -2.000 & JO89VP \\
	Repeater & C4FM       & SKØQO    & Bagarmossen     &   434.5750 &   -2.000 & JO99BG \\
	Repeater & DMR        & SKØNN/1  & Johanneshov     &   434.9250 &   -2.000 & JO99AH \\
	Repeater & DMR        & SKØVR    & Djurö           &   434.5875 &   -2.000 & JO99IH \\
	Repeater & DMR        & SMØWIU/R & Dalarö          &   434.8375 &   -2.000 & JO99ED \\
	Repeater & FM         & SKØRYG   & Stockholm Norr  &   145.7875 &   -0.600 & JO99DL \\
	Repeater & FM         & SKØRYG   & Upplands Väsby  &   434.6750 &   -2.000 & JO89XM \\
	Repeater & DMR        & SKØEN    & Älmsta          &   434.6000 &   -2.000 & JO99JX \\
	Repeater & FM         & SKØBJ/R  & Nynäshamn       &   145.7125 &   -0.600 & JO88XV \\
	Repeater & C4FM       & SKØMG    & Haninge/Gålö    &   434.6875 &   -2.000 & JO99CC \\
	Repeater & DMR        & SKØQO    & Haninge/Brandb. &   434.5625 &   -2.000 & JO99BE \\
	Repeater & FM/DMR     & SKØVR    & Värmdö          &   434.9750 &   -2.000 & JO99FH \\
	Repeater & FM/DMR     & SKØEN    & Älmsta          &   145.7375 &   -0.600 & JO99JX \\
	Repeater & FM/DMR     & SAØAZT   & Norrtälje       &   434.8125 &   -2.000 & JO99IS \\
	Repeater & FM         & SKØMM/R  & Ingarö          &   145.7750 &   -0.600 & JO99GG \\
	Repeater & DMR        & SKØMG    & Södertälje      &   434.7875 &   -2.000 & JO89TE \\
	Repeater & FM         & SKØBJ    & Nynäshamn       &   145.7375 &   -0.600 & JO88WT \\
	Repeater & FM         & SKØBJ/R  & Nynäshamn       &   434.7125 &   -2.000 & JO88XV \\
	Repeater & FM         & SKØBJ/R  & Nynäshamn       &   434.6500 &   -2.000 & JO89XF \\
	Repeater & FM         & SKØBJ    & Nynäshamn       &   434.9125 &   -2.000 & JO88WT \\
	Repeater & FM/DMR     & SAØAZT   & Vallentuna      &   434.5500 &   -2.000 & JO99EO \\
	Repeater & FM         & SKØBJ/R  & Huddinge        &   434.6000 &   -2.000 & JO89XF \\
	Repeater & DMR        & SMØWIU-2 & Södertälje      &   434.8750 &   -2.000 & JO89TE \\
	Repeater & FM         & SKØMT/R  & Vallentuna      &   434.7000 &   -2.000 & JO99BM \\
	Repeater & C4FM       & SKØMG/R  & Sthlm/Söderort  &   434.6375 &   -2.000 & JO99AH
\end{longtable}

\subsubsection{Repeatrar distrikt 1}

\begin{longtable}{llllrrlcl}
	Typ      & Modulation & Signal   & Ort   & Utfrekvens &  Duplex & Loc    &  \\ \hline
	Repeater & FM         & SL1ZXK/R & Slite &   434.6000 &  -2.000 & JO97JR &     &  \\
	Repeater & FM/C4FM    & SK1RGU   & Endre &   145.7750 &  -0.600 & JO97FO &     &  \\
	Repeater & FM/C4FM    & SK1BL/R  & Endre &   145.7750 & -600kHz & 1750   & QRV & JO97FO
\end{longtable}

\subsubsection{Repeatrar distrikt 2}

\begin{longtable}{llllrrlcl}
	Typ      & Modulation         & Signal    & Ort                     & Utfrekvens &  Duplex & Loc    &  &  \\ \hline
	Link     & FM                 & SM2YUW    & Kiruna                  &   434.4000 & Simplex & KP07DU &  &  \\
	Repeater & FM                 & SK2AU/R   & Arjeplog/Galtispouda    &   145.7000 &  -0.600 & JP86XC &  &  \\
	Repeater & FM                 & SK2AU/R   & Skellefteå              &   145.7000 &  -0.600 & KP04LS &  &  \\
	Repeater & FM                 & SK2RIU    & Vännäs/Granlundsberget  &   145.7250 &  -0.600 & JP93VU &  &  \\
	Repeater & FM                 & SK2RIU    & Vännäs/Granlundsberget  &   434.7250 &  -2.000 & JP93VU &  &  \\
	Repeater & FM                 & SK2RLF    & Tärnaby                 &   145.6250 &  -0.600 & JP75PR &  &  \\
	Repeater & FM                 & SK2RLJ    & Umeå/Rödberget          &   145.6500 &  -0.600 & KP03CU &  &  \\
	Repeater & FM                 & SK2RMD    & Sorsele                 &   145.6000 &  -0.600 & JP85SM &  &  \\
	Repeater & FM                 & SK2RMR    & Storuman                &   145.7250 &  -0.600 & JP85NC &  &  \\
	Repeater & FM                 & SK2RYI    & Vindeln/Åsträsk         &   145.6250 &  -0.600 & KP04DP &  &  \\
	Repeater & FM                 & SK2AU/R   & Jörn/Storklinta         &   145.7500 &  -0.600 & KP05BD &  &  \\
	Repeater & FM                 & SK2LY/R   & Lycksele                &   145.7750 &  -0.600 & JP94IO &  &  \\
	Repeater & FM                 & SM2KOT/R  & Kristineberg/Viterliden &   145.6750 &  -0.600 & JP95HB &  &  \\
	Repeater & FM                 & SK2RFR    & Kiruna                  &   145.6250 &  -0.600 & KP07DU &  &  \\
	Repeater & FM                 & SK2RFR    & Kiruna C                &   434.8250 &  -2.000 & KP07DU &  &  \\
	Repeater & FM                 & SK2DR/R   & Luleå                   &   145.6500 &  -0.600 & KP15CO &  &  \\
	Repeater & FM                 & SK2AZ/R   & Piteå                   &   145.6000 &  -0.600 & KP05PH &  &  \\
	Repeater & FM                 & SK2RWJ    & Älvsbyn                 &   145.6750 &  -0.600 & KP05LQ &  &  \\
	Repeater & FM                 & SK2HG/R   & Kalix/Raggdynan         &    51.9500 &  -0.600 & KP15KW &  &  \\
	Repeater & FM                 & SM2KXX    & Lycksele                &   434.7750 &  -1.600 & JP94HO &  &  \\
	Repeater & FM                 & SK2RMR    & Storuman                &   434.7500 &  -2.000 & JP85NC &  &  \\
	Repeater & FM                 & SK2RME    & Piteå                   &   434.6000 &  -2.000 & KP05RH &  &  \\
	Repeater & DMR                & SK2RGJ    & Kiruna                  &   434.5125 &  -2.000 & KP07CT &  &  \\
	Repeater & DMR/D-Star         & SK2DR     & Luleå                   &   434.9000 &  -2.000 & KP15CO &  &  \\
	Repeater & DMR/D-Star         & SK2RJH    & Kalix/Raggdynan         &   434.7500 &  -2.000 & KP15KW &  &  \\
	Repeater & FM/DMR             & SK2HG/R3  & Seskarö                 &   145.6750 &  -0.600 & KP15UR &  &  \\
	Repeater & FM/DMR             & SK2HG/R5  & Kalix/Raggdynan         &   145.7250 &  -0.600 & KP15KW &  &  \\
	Repeater & FM/DMR             & SK2HG/RU5 & Kalix-Vattentorn        &   434.7250 &  -2.000 & KP15NU &  &  \\
	Repeater & DMR                & SK2AT     & Vännäs                  &   434.9750 &  -2.000 & JP93XX &  &  \\
	Repeater & FM                 & SK2CI     & Boden                   &   145.6250 &  -0.600 & KP05SS &  &  \\
	Repeater & DMR                & SK2AZ     & Piteå                   &   434.8500 &  -2.000 & KP05PH &  &  \\
	Repeater & DMR                & SK2CI     & Boden                   &   434.8000 &  -2.000 & KP05TT &  &  \\
	Repeater & DMR                & SK2HG-2   & Kalix                   &   434.9875 &  -2.000 & KP15OU &  &  \\
	Repeater & FM/DMR/D-Star/C4FM & SK2AU/R   & Skellefteå              &   145.5875 &  -0.600 & KP04LS &  &  \\
	Repeater & FM                 & SJ2W/R    & Skellefteå              &   434.6750 &  -2.000 & KP04LS &  &  \\
	Repeater & FM                 & SJ2W      & Burträsk                &   434.9500 &  -2.000 & KP04HM &  &
\end{longtable}

\subsubsection{Repeatrar distrikt 3}

\begin{longtable}{llllrrlcl}
	Typ      & Modulation      & Signal   & Ort                    & Utfrekvens &   Duplex & Loc    &  &  \\ \hline
	Hotspot  & D-Star          & SK3GA-B  & Hudiksvall             &   434.4750 & Duplex 0 & JP81NR &  &  \\
	Link     & FM              & SM3KDR   & Krokom/Aspås           &   434.9750 &  Simplex & JP73GI &  &  \\
	Repeater & FM              & SK3EK/R  & Sollefteå              &   434.6500 &   -1.600 & JP83DE &  &  \\
	Repeater & FM              & SK3MF/R  & Nordingrå/Rävsön       &   145.6250 &   -0.600 & JP92FW &  &  \\
	Repeater & FM              & SK3MF/R  & Nordingrå/Rävsön       &   434.8500 &   -2.000 & JP92FW &  &  \\
	Repeater & FM              & SK3RFG   & Sundsvall              &   145.7250 &   -0.600 & JP82RJ &  &  \\
	Repeater & FM              & SK3RIA   & Östersund              &   434.7500 &   -2.000 & JP73JE &  &  \\
	Repeater & FM              & SK3RIN   & Borgsjö                &   145.7000 &   -0.600 & JP72WN &  &  \\
	Repeater & FM              & SK3RKL   & Örnsköldsvik/Rutberget &   145.7750 &   -0.600 & JP93GJ &  &  \\
	Repeater & FM              & SK3RMG   & Bergsjö                &  1297.1000 &   -6.000 & JP81MX &  &  \\
	Repeater & FM              & SK3RMX   & Hoting/Kyrktåsjö       &   145.6000 &   -0.600 & JP74XF &  &  \\
	Repeater & FM              & SK3RYK   & Söderhamn              &   145.7500 &   -0.600 & JP81NH &  &  \\
	Repeater & FM              & SK3RYK   & Söderhamn              &   434.7500 &   -1.600 & JP81NH &  &  \\
	Repeater & FM              & SK3WH    & Högakustenbron         &  1297.2750 &   -6.000 & JP82XT &  &  \\
	Repeater & FM              & SK3LH/R  & Örnsköldsvik           &   434.8750 &   -2.000 & JP93IH &  &  \\
	Repeater & FM              & SK3RNJ   & Åre/Åreskutan          &   145.7250 &   -0.600 & JP63NK &  &  \\
	Repeater & FM              & SM3XRJ   & Kramfors               &   434.6000 &   -2.000 & JP82VW &  &  \\
	Repeater & D-Star          & SK3LH-B  & Örnsköldsvik/Malmön    &   434.5750 &   -2.000 & JP93LF &  &  \\
	Repeater & FM              & SL3ZB    & Härnösand              &   434.7250 &   -2.000 & JP82XP &  &  \\
	Repeater & FM              & SK3EK/R  & Sollefteå              &   145.6500 &   -0.600 & JP83PD &  &  \\
	Repeater & D-Star          & SK3RFG-C & Sundsvall/Klissberget  &   145.5875 &   -0.600 & JP82OJ &  &  \\
	Repeater & FM/C4FM         & SK3JR/R  & Östersund              &   145.7500 &   -0.600 & JP73JE &  &  \\
	Repeater & FM              & SK3GK/R  & Sandviken/Kungsberget  &   145.7000 &   -0.600 & JP80FS &  &  \\
	Repeater & FM              & SM3VAC/R & Nyland                 &   145.7500 &   -0.600 & JP83UA &  &  \\
	Repeater & FM              & SM3VAC/R & Nyland                 &   434.9500 &   -1.600 & JP83UA &  &  \\
	Repeater & FM              & SK3RQE   & Forsa/Storberget       &   434.6750 &   -2.000 & JP81KQ &  &  \\
	Repeater & FM              & SA3EJX/R & Forsa/Storberget       &   145.6750 &   -0.600 & JP81KQ &  &  \\
	Repeater & FM              & SK3GW    & Gävle                  &   434.8750 &   -2.000 & JP80NP &  &  \\
	Repeater & FM              & SK3GK    & Sandviken              &   434.8250 &   -2.000 & JP80FS &  &  \\
	Repeater & FM              & SK3RQC   & Vemdalen               &   145.6250 &   -0.600 & JP62WK &  &  \\
	Repeater & FM              & SM3LEI/R & Årsunda                &   434.6500 &   +1.600 & JP80IM &  &  \\
	Repeater & DMR             & SK3WH    & Örnsköldsvik           &   145.5750 &   -0.600 & JP93IH &  &  \\
	Repeater & DMR             & SK3GK    & Gävle                  &   434.7000 &   -2.000 & JP80NP &  &  \\
	Repeater & DMR/D-Star      & SK3RFG   & Sundsvall/Klissberget  &   434.8000 &   -2.000 & JP82OJ &  &  \\
	Repeater & DMR/D-Star/C4FM & SM3YFX   & Föllinge               &   434.5250 &   -2.000 & JP73HQ &  &  \\
	Repeater & FM              & SK3GA/R  & Hudiksvall             &   145.7750 &   -0.600 & JP81NR &  &  \\
	Repeater & FM/DMR          & SK3RHU   & Hudiksvall             &   145.7125 &   -0.600 & JP81NR &  &  \\
	Repeater & DMR             & SK3RHU   & Hudiksvall             &   434.5750 &   -2.000 & JP81NR &  &  \\
	Repeater & FM/C4FM         & SK3JR/R2 & Östersund/Brattåsen    &   145.7875 &   -0.600 & JP73HC &  &  \\
	Repeater & DMR/D-Star/C4FM & SG9NN    & Sundsvall              &   434.5375 &   -2.000 & JP82OJ &  &  \\
	Repeater & FM              & SK3RET   & Bollnäs/Arbrå          &   145.6500 &   -0.600 & JP81CL &  &  \\
	Repeater & DMR             & SK3JR    & Östersund/Brattåsen    &   434.5625 &   -2.000 & JP73HC &  &  \\
	Repeater & DMR             & SK3RFG   & Sundsvall/Nolby        &   434.9875 &   -2.000 & JP82QH &  &  \\
	Repeater & FM              & SK3YZ/R  & Forsa                  &   145.6125 &   -0.600 & JP81KQ &  &  \\
	Repeater & FM              & SK3PH/R  & Delsbo                 &    29.6900 &   -0.100 & JP81GT &  &  \\
	Repeater & FM              & SK3EK/R  & Sollefteå              &   434.9250 &   -2.000 & JP83DE &  &  \\
	Repeater & FM              & SK3RQE   &                        &   145.6000 &   -0.600 & JP81NV &  &  \\
	Repeater & FM              & SK3W     & Österfärnebo           &   434.8500 &   -2.000 & JP80JH &  &
\end{longtable}

\subsubsection{Repeatrar distrikt 4}

\begin{longtable}{llllrrlcl}
	Typ      & Modulation & Signal   & Ort                        & Utfrekvens &   Duplex & Loc    &  &  \\ \hline
	Hotspot  & D-Star     & SG4UOF-C & Glanshammar                &   145.3375 & Duplex 0 & JO79RI &  &  \\
	Hotspot  & D-Star     & SG4UZM-B & Borlänge                   &   434.5500 & Duplex 0 & JP70RM &  &  \\
	Hotspot  & DMR        & SG4AXV   & Ekshärad                   &   433.2000 &  Simplex & JP60RE &  &  \\
	Hotspot  & DMR/D-Star & SG4AXQ   & Sunne                      &   432.5000 & Duplex 0 & JO69NU &  &  \\
	Hotspot  & DMR        & SA4ATZ   & Malung                     &   144.8375 &  Simplex & JP60UQ &  &  \\
	Link     & FM         & SK4AV/R  & Filipstad/Klockarhöjden    &   145.2000 &  Simplex & JO79CR &  &  \\
	Link     & FM         &          & Nyhammar                   &   145.3250 &  Simplex & JP70LG &  &  \\
	Link     & FM         &          & Grängesberg                &   145.3500 &  Simplex & JP70MB &  &  \\
	Link     & FM         & SK4RJJ   & Torsby/Hovfjället          &   145.2875 &  Simplex & JO69LH &  &  \\
	Link     & FM         & SA4THA   & Älvdalen                   &   434.5000 &  Simplex & JP71AF &  &  \\
	Link     & FM         & SM4FBD   & Nybble                     &   145.3000 &  Simplex & JO79BC &  &  \\
	Link     & FM         & SK4EA-L  & Lindesberg                 &   145.3000 &  Simplex & JO79OO &  &  \\
	Link     & FM         & SM4MXN   & Orsa                       &   145.2750 &  Simplex & JP71HC &  &  \\
	Repeater & FM         & SK4DM/R  & Ludvika                    &   145.7250 &   -0.600 & JP70NC &  &  \\
	Repeater & FM         & SK4DM/R  & Ludvika                    &   434.7250 &   -1.600 & JP70NC &  &  \\
	Repeater & FM         & SK4RGO   & Orsa/Grönklitt             &   434.7500 &   -1.600 & JP71GF &  &  \\
	Repeater & FM         & SK4RPK   & Torsby/Valberget           &   434.6250 &   -2.000 & JP60LC &  &  \\
	Repeater & FM         & SK4RQF   & Årjäng                     &   145.7250 &   -0.600 & JO69BJ &  &  \\
	Repeater & FM         & SM4JDP   & Mora                       &   434.7000 &   -2.000 & JP71GA &  &  \\
	Repeater & D-Star     & SG4TYA   & Mora                       &   145.5750 &   -0.600 & JP71GE &  &  \\
	Repeater & FM         & SK4IL/R  & Grums                      &   434.7250 &   -2.000 & JO69NI &  &  \\
	Repeater & FM         & SK4WV    & Vansbro                    &   145.6500 &   -0.600 & JP70AM &  &  \\
	Repeater & FM         & SK4WV    & Vansbro                    &   434.6500 &   -1.600 & JP70AM &  &  \\
	Repeater & FM         & SK4TL/R  & Örebro/Suttarboda          &   145.7125 &   -0.600 & JO79KH &  &  \\
	Repeater & FM         & SK4RGO   & Orsa/Grönklitt             &   145.7500 &   -0.600 & JP71GF &  &  \\
	Repeater & D-Star     & SK4BW-B  & Borlänge                   &   434.9000 &   -2.000 & JP70RJ &  &  \\
	Repeater & FM/C4FM    & SK4RVN   & Borlänge                   &   434.8000 &   -2.000 & JP70RJ &  &  \\
	Repeater & FM         & SK4HV/R  & Hagfors/Värmullsåsen       &   145.6750 &   -0.600 & JP60VA &  &  \\
	Repeater & FM         & SK4EA/R  & Lindesberg                 &   145.6875 &   -0.600 & JO79NP &  &  \\
	Repeater & FM         & SK4RWQ   & Arvika/Valfjället          &   434.7750 &   -2.000 & JO69CT &  &  \\
	Repeater & FM         & SK4RJJ   & Sunne/Blåbärskullen        &   145.7750 &   -0.600 & JO69KU &  &  \\
	Repeater & FM         & SK4BX/R  & Garphyttan/Storstenshöjden &   145.6500 &   -0.600 & JO79LH &  &  \\
	Repeater & FM         & SK4RUV   & Leksand                    &   145.7750 &   -0.600 & JP70MQ &  &  \\
	Repeater & DMR        & SK4BW    & Borlänge                   &   434.8500 &   -2.000 & JP70RJ &  &  \\
	Repeater & DMR        & SK4WV    & Vansbro                    &   434.6625 &   -2.000 & JP70AM &  &  \\
	Repeater & FM         & SK4EA/R  & Kopparberg                 &   145.6000 &   -0.600 & JO79MW &  &  \\
	Repeater & DMR        & SA4BNA   & Arvika                     &   434.9750 &   -2.000 & JO69GN &  &  \\
	Repeater & FM/DMR     & SK4KR    & Karlskoga                  &   434.8000 &   -2.000 & JO79FH &  &  \\
	Repeater & DMR        & SK4RGL   & Falun                      &   434.6250 &   -2.000 & JP70UP &  &  \\
	Repeater & FM         & SK4RGL   & Falun                      &   145.6250 &   -0.600 & JP70UP &  &  \\
	Repeater & FM         & SK4TL/R  & Örebro/Suttarboda          &    51.9500 &   -0.600 & JO79KH &  &  \\
	Repeater & FM/DMR     & SK4RKD   & Karlskoga                  &   145.7500 &   -0.600 & JO79FJ &  &  \\
	Repeater & DMR        & SK4KO    & Nusnäs                     &   434.9250 &   -2.000 & JP70HW &  &  \\
	Repeater & FM/DMR     & SM4WIU-3 & Leksand                    &   434.6125 &   -2.000 & JP70MR &  &  \\
	Repeater & DMR        & SK4TL    & Örebro                     &   434.7250 &   -2.000 & JO79OG &  &  \\
	Repeater & D-Star     & SG4AXV   & Ekshärad                   &   145.6000 &   -0.600 & JP60RE &  &  \\
	Repeater & FM         & SK4KO    & Sälen/Lindvallen           &   145.6000 &   -0.600 & JP61OD &  &  \\
	Repeater & DMR        & SA4BHE-R & Smedjebacken               &   434.6375 &   -2.000 & JP70GD &  &
\end{longtable}

\subsubsection{Repeatrar distrikt 5}

\begin{longtable}{llllrrlcl}
	Typ      & Modulation & Signal   & Ort                    & Utfrekvens &   Duplex & Loc    &  &  \\ \hline
	Hotspot  & D-Star     & SC5SLU-C & Uppsala                &   145.3250 & Duplex 0 & JO89QW &  &  \\
	Hotspot  & D-Star     & SM5EZN-B & Uppsala                &   433.4875 & Duplex 0 & JO89QW &  &  \\
	Hotspot  & D-Star     & SG5TAH-C & Flen/Orrhammar         &   145.3375 & Duplex 0 & JO89GB &  &  \\
	Hotspot  & DMR        & SA5HAV   & Uppsala                &   434.3750 &  Simplex & JO89VW &  &  \\
	Link     & FM         & SM5RVH   & Nyköping               &   145.4750 &  Simplex & JO88LQ &  &  \\
	Link     & FM         & SM5RVH   & Nyköping               &    51.4700 &  Simplex & JO88LQ &  &  \\
	Link     & FM         & SM5RVH   & Nyköping               &    29.1700 &  Simplex & JO88LQ &  &  \\
	Link     & FM         & SM5RVH   & Nyköping               &  1297.5000 &  Simplex & JO88LQ &  &  \\
	Link     & FM         & SM5GXQ-L & Norrköping             &   145.2375 &  Simplex & JO88CO &  &  \\
	Link     & DMR        & SA5KBE   & Stigtomta              &   145.2875 &  Simplex & JO88JT &  &  \\
	Link     & FM         & SA5BJM   & Uppsala/Fjuckby        &   144.5750 &  Simplex & JO89TX &  &  \\
	Link     & FM         & SA5BJM   & Uppsala/Fjuckby        &   433.4500 &  Simplex & JO89TX &  &  \\
	Repeater & FM         & SK5AS/R  & Linköping              &   145.7250 &   -0.600 & JO78SJ &  &  \\
	Repeater & FM         & SK5BN/R  & Finspång               &   434.9250 &   -2.000 & JO78VR &  &  \\
	Repeater & FM/D-Star  & SK5RHQ   & Västerås               &   434.7000 &   -2.000 & JO89GO &  &  \\
	Repeater & FM/C4FM    & SK5RCQ   & Kisa                   &   145.7000 &   -0.600 & JO77TX &  &  \\
	Repeater & FM         & SK5LW/R  & Eskilstuna/Hällby      &   434.8500 &   -2.000 & JO89FJ &  &  \\
	Repeater & FM         & SA5BTT   & Trosa                  &   434.8875 &   -2.000 & JO88TV &  &  \\
	Repeater & FM         & SK5BN/R  & Norrköping/Kolmården   &   145.6000 &   -0.600 & JO88FQ &  &  \\
	Repeater & FM         & SK5BN/R  & Norrköping/Östra Eneby &   434.6000 &   -2.000 & JO88BO &  &  \\
	Repeater & FM         & SK5LF/R  & Linköping/Majelden     &   434.8250 &   -2.000 & JO78TJ &  &  \\
	Repeater & DMR        & SA5BJM   & Uppsala/Fjuckby        &   434.5125 &   -2.000 & JO89TX &  &  \\
	Repeater & FM         & SK5DB/R  & Uppsala                &   145.7500 &   -0.600 & JO89VU &  &  \\
	Repeater & FM         & SK5DB/R  & Uppsala                &   434.7500 &   -2.000 & JO89VU &  &  \\
	Repeater & FM         & SK5RHQ   & Västerås               &   145.7750 &   -0.600 & JO89GO &  &  \\
	Repeater & FM         & SK5RHQ   & Västerås               &   434.7750 &   -2.000 & JO89GO &  &  \\
	Repeater & ATV        & SK5BN/R  & Norrköping/Kolmården   &  1282.0000 &  -30.000 & JO88FQ &  &  \\
	Repeater & FM         & SK5AS/R  & Linköping              &   145.7875 &   -0.600 & JO78SN &  &  \\
	Repeater & FM         & SM5RYI/R & Sala                   &   145.7125 &   -0.600 & JO89HW &  &  \\
	Repeater & DMR        & SK5RYG   & Linköping              &   434.5125 &   -2.000 & JO78SN &  &  \\
	Repeater & FM         & SK5RYG   & Linköping              &   145.6250 &   -0.600 & JO78SN &  &  \\
	Repeater & FM/DMR     & SL5ZYT/R & Norrköping             &   434.9500 &   -2.000 & JO88DQ &  &  \\
	Repeater & FM/DMR     & SG5BCG/R & Knivsta                &   434.5250 &   -2.000 & JO89VR &  &  \\
	Repeater & FM/DMR     & SM5DWC/R & Linköping              &   434.8750 &   -2.000 & JO78SM &  &  \\
	Repeater & FM         & SK5BB/R  & Arboga/Kolsva          &   434.8750 &   -2.000 & JP79WO &  &  \\
	Repeater & FM         & SK5BB/R  & Arboga/Kolsva          &   145.6750 &   -0.600 & JP79WO &  &  \\
	Repeater & D-Star     & SK5BN-C  & Norrköping             &   145.5750 &   -0.600 & JO88BR &  &  \\
	Repeater & FM/DMR     & SG5DV    & Uppsala                &   434.5875 &   -2.000 & JO89TU &  &  \\
	Repeater & FM         & SG5DV    & Uppsala                &   145.5875 &   -0.600 & JO89TU &  &  \\
	Repeater & DMR/D-Star & SK5LW/R  & Eskilstuna/Ärla        &   145.5875 &   -0.600 & JO89FJ &  &  \\
	Repeater & FM         & SK5LW/R  & Eskilstuna             &    51.8500 &   -0.600 & JO89FJ &  &  \\
	Repeater & FM         & SK5VM/R  & Eskilstuna             &   434.9750 &   -2.000 & JO89GI &  &  \\
	Repeater & FM         & SK5LW/R  & Eskilstuna/Slytan      &   145.6125 &   -0.600 & JO89HF &  &  \\
	Repeater & D-Star     & SK5UM-B  & Flen                   &   434.5500 &   -2.000 & JO89HB &  &  \\
	Repeater & FM         & SK5UM/R  & Flen/Öja               &   434.7500 &   -2.000 & JO89HB &  &  \\
	Repeater & DMR        & SK5UM/R  & Flen                   &   145.6375 &   -0.600 & JO89HB &  &  \\
	Repeater & FM         & SM5YMS   & Åtvidaberg             &   145.6625 &   -0.600 & JO78XE &  &  \\
	Repeater & FM         & SM5YMS/R & Linköping              &   434.8000 &   -2.000 & JO78SM &  &  \\
	Repeater & DMR        & SA5HAV/R & Uppsala/Rasbo          &   434.6375 &   -2.000 & JO89VW &  &  \\
	Repeater & DMR        & SL5ZO    & Finspång               &   434.8125 &   -2.000 & JO78VQ &  &  \\
	Repeater & DMR        & SA5UTR   & Nyköping               &   434.6375 &   -2.000 & JO88MS &  &  \\
	Repeater & FM/C4FM    & SA5OHR/R & Norrköping             &   434.6625 &   -2.000 & JO88BO &  &  \\
	Repeater & FM         & SK5RHT   & Linköping              &    51.9900 &   -0.600 & JO78SN &  &  \\
	Repeater & FM         & SK5UM/R  & Flen                   &   145.7625 &   -0.600 & JO89HB &  &  \\
	Repeater & FM         & SK5WR/R  & Motala                 &   145.7375 &   -0.600 & JO78NM &  &  \\
	Repeater & FM         & SK5RHT   & Linköping              &    29.6600 &   -0.100 & JO78XH &  &
\end{longtable}

\subsubsection{Repeatrar distrikt 6}

\begin{longtable}{llllrrlcl}
	Typ      & Modulation      & Signal   & Ort                   & Utfrekvens &   Duplex & Loc    &  &  \\ \hline
	Hotspot  & D-Star          & SK6GB-D  & Mölndal               &   433.7250 &  Simplex & JO67AQ &  &  \\
	Hotspot  & D-Star          & SK6GB-D  & Mölndal               &   144.8250 &  Simplex & JO67AQ &  &  \\
	Hotspot  & D-Star          & SK6MA-C  & Hjo                   &   145.2125 & Duplex 0 & JO78DH &  &  \\
	Hotspot  & D-Star          & SG6JWU-B & Halmstad              &   433.4750 & Duplex 0 & JO66LP &  &  \\
	Hotspot  & DMR/D-Star/C4FM & SK6BA-B  & Skene                 &   433.5625 & Duplex 0 & JO67HL &  &  \\
	Hotspot  & D-Star          & SG6YOW   & Alingsås              &   144.8500 &  Simplex & JO67GW &  &  \\
	Link     & FM              & SA6RP    & Floda                 &   433.4750 &  Simplex & JO67ET &  &  \\
	Link     & FM              & SM6FZG   & Skårsjön              &   144.5500 &  Simplex & JO67AN &  &  \\
	Link     & FM              & SM6FZG   & Kortedala             &   144.6000 &  Simplex & JO67AS &  &  \\
	Link     & FM              & SM6FZG   & Långedrag             &   144.5250 &  Simplex & JO57WQ &  &  \\
	Link     & FM              & SM6FZG   & Hönö                  &   144.6250 &  Simplex & JO57TQ &  &  \\
	Link     & FM              & SK6AG    & Guldheden             &   144.5750 &  Simplex & JO57XQ &  &  \\
	Link     & FM              & SM6FZG   & Mölnlycke             &   144.5875 &  Simplex & JO67BP &  &  \\
	Link     & FM              & SM6FZG   & Borås                 &   144.5125 &  Simplex & JO67MR &  &  \\
	Link     & FM              & SM6YRB   & Lidköping/Kållandsö   &   145.3000 &  Simplex & JO68NP &  &  \\
	Link     & FM              & SM6FZG   & Kungsbacka            &   144.6500 &  Simplex & JO67AL &  &  \\
	Link     & FM              & SM6FZG   & Myggenäs              &   144.6625 &  Simplex & JO58UB &  &  \\
	Link     & FM              & SM6FZG   & Guldheden             &   144.6750 &  Simplex & JO57XQ &  &  \\
	Link     & FM              & SM6FZG   & Guldheden             &    51.5500 &  Simplex & JO57XQ &  &  \\
	Link     & FM              & SM6VAG   & Hjo                   &   145.2375 &  Simplex & JO78AG &  &  \\
	Link     & FM              & SA6EAL   & Hajom                 &   145.4000 &  Simplex & JO67GM &  &  \\
	Link     & FM              & SA6GDS   & Istorp                &   145.2875 &  Simplex & JO67FI &  &  \\
	Link     & FM              & SM6TZL   & Örby                  &   145.2375 &  Simplex & JO67IL &  &  \\
	Repeater & FM              & SA6AR/R  & Angered               &   434.9250 &   -2.000 & JO67AT &  &  \\
	Repeater & FM              & SK6QW/R  & Mariestad/Katrinefors &   434.9000 &   -2.000 & JO68VQ &  &  \\
	Repeater & FM              & SK6DK/R  & Varberg/Veddige       &   434.7000 &   -1.600 & JO67EH &  &  \\
	Repeater & FM              & SK6DK/R  & Varberg/Veddige       &   145.7000 &   -0.600 & JO67EH &  &  \\
	Repeater & FM              & SA6BSN/R & Åmål                  &   434.6000 &   -2.000 & JO69IB &  &  \\
	Repeater & D-Star          & SK6DW-B  & Trollhättan           &   434.5250 &   -2.000 & JO68DG &  &  \\
	Repeater & FM              & SA6BXG/R & Kungälv/Romelanda     &   434.7375 &   -2.000 & JO67AX &  &  \\
	Repeater & FM              & SK6RPE   & Kungälv               &   145.6125 &   -0.600 & JO57XU &  &  \\
	Repeater & FM              & SM6CYJ/R & Kinnekulle            &   434.9500 &   -2.000 & JO68QO &  &  \\
	Repeater & FM              & SK6DQ/R  & Älvängen              &   434.7500 &   -2.000 & JO67BW &  &  \\
	Repeater & FM              & SK6MA/R  & Tidaholm/Hökensås     &   145.6375 &   -0.600 & JO78AD &  &  \\
	Repeater & FM              & SM6UXW/R & Ulricehamn            &   434.6750 &   -2.000 & JO67RT &  &  \\
	Repeater & D-Star          & SK6SA-B  & Guldheden             &   434.5125 &   -2.000 & JO57XQ &  &  \\
	Repeater & FM/C4FM/D-Star  & SK6RKG   & Halmstad              &   434.9250 &   -2.000 & JO66MS &  &  \\
	Repeater & FM              & SK6RPE   & Kungälv               &   434.9000 &   -2.000 & JO57XU &  &  \\
	Repeater & FM              & SM6VBT/R & Mölndal               &   145.7000 &   -0.600 & JO67AP &  &  \\
	Repeater & FM              & SM6VBT/R & Mölndal               &   434.7000 &   -2.000 & JO67AP &  &  \\
	Repeater & FM/C4FM         & SK6EI/R  & Skövde                &   434.8250 &   -2.000 & JO68VK &  &  \\
	Repeater & FM/C4FM         & SK6LK/R  & Borås                 &   434.8000 &   -2.000 & JO67MR &  &  \\
	Repeater & FM/C4FM         & SM6THE/R & Skövde                &   145.6875 &   -0.600 & JO68XJ &  &  \\
	Repeater & FM/C4FM         & SM6UXW/R & Ulricehamn            &   145.6750 &   -0.600 & JO67ST &  &  \\
	Repeater & FM/DMR          & SK6DW/R  & Trollhättan           &   145.7625 &   -0.600 & JO68DG &  &  \\
	Repeater & FM/C4FM         & SK6AG    & Guldheden             &   434.6750 &   -2.000 & JO57XQ &  &  \\
	Repeater & FM              & SL6BH/R  & Halmstad              &   434.7500 &   -2.000 & JO66KQ &  &  \\
	Repeater & FM              & SK6GO/R  & Lunden                &   145.7875 &   -0.600 & JO67AR &  &  \\
	Repeater & FM              & SK6RDG   & Guldheden             &   434.9750 &   -2.000 & JO57XQ &  &  \\
	Repeater & FM              & SK6ROY   & Kinnekulle            &   145.6000 &   -0.600 & JO68QO &  &  \\
	Repeater & FM              & SK6LK/R  & Borås                 &   145.7750 &   -0.600 & JO67MR &  &  \\
	Repeater & FM              & SK6RIC   & Alingsås              &   145.6250 &   -0.600 & JO67GW &  &  \\
	Repeater & FM              & SK6RIC   & Alingsås              &   434.6250 &   -2.000 & JO67GW &  &  \\
	Repeater & FM              & SK6RFQ   & Guldheden             &    51.8700 &   -0.600 & JO57XQ &  &  \\
	Repeater & FM              & SK6RJW   & Kungsbacka            &   145.7250 &   -0.600 & JO67AL &  &  \\
	Repeater & FM              & SK6RFQ   & Guldheden             &    29.6800 &   -0.100 & JO57XQ &  &  \\
	Repeater & FM              & SM6VBT/R & Mölndal               &    29.6900 &   -0.100 & JO67AP &  &  \\
	Repeater & FM/DMR          & SK6RFP   & Bengtsfors            &   145.7000 &   -0.600 & JO69CA &  &  \\
	Repeater & FM/DMR          & SL6ZYW/R & Bengtsfors            &   434.6875 &   -2.000 & JO69CA &  &  \\
	Repeater & FM              & SK6RKI   & Guldheden             &  1297.1500 &   -6.000 & JO57XQ &  &  \\
	Repeater & FM              & SK6IF/R  & Bokenäs               &   145.6000 &   -0.600 & JO58TH &  &  \\
	Repeater & FM              & SK6IF/R  & Lysekil               &   434.8000 &   -2.000 & JO58RG &  &  \\
	Repeater & FM/DMR/D-Star   & SA6APY   & Skara                 &   434.9875 &   -2.000 & JO68RJ &  &  \\
	Repeater & DMR             & SM6TKT/R & Borås                 &   434.5500 &   -2.000 & JO67MR &  &  \\
	Repeater & DMR             & SK6DG    & Alingsås              &   434.5375 &   -2.000 & JO67GV &  &  \\
	Repeater & DMR             & SK6AG    & Guldheden             &   434.7875 &   -2.000 & JO57XQ &  &  \\
	Repeater & FM/DMR          & SA6RP/R  & Floda                 &   434.8250 &   -2.000 & JO67ET &  &  \\
	Repeater & FM/DMR          & SK6IF    & Tanumshede            &   145.5750 &   -0.600 & JO58PR &  &  \\
	Repeater & FM              & SK6RKG   & Halmstad              &   145.6750 &   -0.600 & JO66MS &  &  \\
	Repeater & FM              & SK6JX/R  & Falkenberg            &   145.6250 &   -0.600 & JO66FV &  &  \\
	Repeater & FM              & SK6BA/R  & Skene                 &   145.6000 &   -0.600 & JO67HM &  &  \\
	Repeater & FM              & SK6BA/R  & Skene                 &   434.9500 &   -2.000 & JO67HM &  &  \\
	Repeater & DMR             & SK6RKI   & Kortedala             &   145.5875 &   -0.600 & JO67AS &  &  \\
	Repeater & FM              & SK6RJW   & Kungsbacka            &   434.7250 &   -2.000 & JO67AL &  &  \\
	Repeater & FM/DMR          & SK6QA/R  & Stenungsund           &   145.7125 &   -0.600 & JO58XB &  &  \\
	Repeater & FM/DMR          & SK6DW/R  & Trollhättan           &   434.8750 &   -2.000 & JO68DG &  &  \\
	Repeater & FM              & SK6RFQ   & Guldheden             &   434.6500 &   -2.000 & JO57XQ &  &  \\
	Repeater & FM              & SK6RFQ   & Guldheden             &   145.6500 &   -0.600 & JO57XQ &  &  \\
	Repeater & FM/DMR          & SK6IF    & Kungshamn             &   145.6750 &   -0.600 & JO58PI &  &  \\
	Repeater & DMR             & SK6RKI   & Öckerö                &   434.8500 &   -2.000 & JO57TR &  &  \\
	Repeater & FM              & SK6RKI   & Öckerö                &   145.7500 &   -0.600 & JO57TR &  &  \\
	Repeater & FM/DMR          & SK6QA/R  & Stenungsund           &   434.5625 &   -2.000 & JO58UB &  &  \\
	Repeater & FM              & SG6WAL   & Ytterby               &   145.7875 &   -0.600 & JO57WU &  &  \\
	Repeater & FM              & SM6UDU/R & Uddevalla/Bokenäs     &   434.7750 &   -2.000 & JO58UI &  &  \\
	Repeater & FM/C4FM         & SK6EE/R  & Skara                 &   145.7250 &   -0.600 & JO68RH &  &  \\
	Repeater & FM              & SM6WSC   & Trollhättan           &   434.7250 &   -2.000 & JO68EF &  &  \\
	Repeater & FM/C4FM         & SK6EE/R  & Skara                 &   434.5625 &   -2.000 & JO68RH &  &  \\
	Repeater & FM              & SM6SXJ   & Torup/Galtabo         &   434.8875 &   -2.000 & JO67LA &  &  \\
	Repeater & FM              &          &                       &   434.8625 &   -2.000 & JO67JS &  &  \\
	Repeater & FM              & SK6RIC   & Alingsås              &  1297.0250 &   -6.000 & JO67GV &  &  \\
	Repeater & FM/DMR          & SL6ZAQ   & Uddevalla             &   145.7375 &   -0.600 & JO58WH &  &  \\
	Repeater & FM/C4FM         & SK6WW/R  & Karlsborg             &   145.7625 &   -0.600 & JO78FM &  &
\end{longtable}

\subsubsection{Repeatrar distrikt 7}

\begin{longtable}{llllrrlcl}
	Typ      & Modulation      & Signal   & Ort                     & Utfrekvens &   Duplex & Loc    &  &  \\ \hline
	Hotspot  & D-Star          & SG7WDL-C & Eneryda                 &   145.2125 & Duplex 0 & JO76EQ &  \\
	Hotspot  & D-Star          & SG7HTP-C & Sölvesborg              &   145.2375 &  Simplex & JO76GB &  \\
	Hotspot  & D-Star          & SK7RRV-C & Lönsboda                &   144.8875 & Duplex 0 & JO76DJ &  \\
	Hotspot  & DMR             & SG7WSE   & Ekenässjön              &   144.8500 &  Simplex & JO77ML &  \\
	Link     & FM              & SM7KUY/R & Sölvesborg              &   434.4000 &  Simplex & JO76HB &  \\
	Link     & FM              & SA7AUX   & Linneryd                &   145.4000 &  Simplex & JO76NP &  \\
	Link     & FM              & SM7FLD   & Everöd                  &   145.2375 &  Simplex & JO75BV &  \\
	Link     & FM              & SM5GXQ   & Färjestaden             &   145.2375 &  Simplex & JO86FP &  \\
	Repeater & FM              & SM7GYT/R & Eslöv                   &   434.8125 &   -2.000 & JO65PU &  \\
	Repeater & DMR             & SA7CCO   & Sjöbo                   &   434.9250 &   -2.000 & JO65UP &  \\
	Repeater & D-Star          & SM7XAA   & Malmö                   &   434.5250 &   -2.000 & JO65MN &  \\
	Repeater & FM              & SA7BVQ/R & Eslöv                   &   434.7000 &   -2.000 & JO65PU &  \\
	Repeater & FM              & SK7REP   & Lund/Harderberga        &   145.7750 &   -0.600 & JO65PQ &  \\
	Repeater & FM              & SK7RNQ   & Vitaby                  &   145.6125 &   -0.600 & JO75BQ &  \\
	Repeater & FM              & SK7ROQ   & Gladsax                 &   434.8875 &   -2.000 & JO75DN &  \\
	Repeater & FM              & SK7REE   & Söderåsen/Stenestad     &   145.6500 &   -0.600 & JO66NB &  \\
	Repeater & FM              & SK7REE   & Söderåsen/Stenestad     &    51.8500 &   -0.600 & JO66NB &  \\
	Repeater & FM              & SK7RN/R  & Borgholm                &   145.6625 &   -0.600 & JO86HU &  \\
	Repeater & FM              & SK7RN/R  & Mörbylånga              &   145.6250 &   -0.600 & JO86FM &  \\
	Repeater & FM              & SK7RN/R  & Böda                    &   145.7500 &   -0.600 & JO87MG &  \\
	Repeater & FM              & SK7RFJ   & Karlskrona              &   145.7500 &   -0.600 & JO76TE &  \\
	Repeater & FM              & SK7FK/R  & Karlskrona              &   434.7500 &   -2.000 & JO76TE &  \\
	Repeater & DMR             & SK7HW    & Växjö                   &   434.7000 &   -2.000 & JO76KU &  \\
	Repeater & D-Star          & SK7RGM-B & Asarum                  &   434.7125 &   -2.000 & JO76KF &  \\
	Repeater & DMR/D-Star      & SK7RNQ   & Gladsax                 &   145.5750 &   -0.600 & JO75DN &  \\
	Repeater & FM/C4FM         & SK7BQ/R  & Kristianstad            &   145.7375 &   -0.600 & JO76AA &  \\
	Repeater & FM/C4FM         & SK7REZ   & Blentarp/Romeleåsen     &   145.6750 &   -0.600 & JO65TM &  \\
	Repeater & FM/C4FM         & SK7EM/R  & Blentarp/Romeleåsen     &   434.8500 &   -2.000 & JO65SN &  \\
	Repeater & FM/C4FM         & SK7RGM   & Olofström/Boafallsbacke &   145.7000 &   -0.600 & JO76FF &  \\
	Repeater & DMR/D-Star/C4FM & SK7RQX   & Hallandsås              &   145.7875 &   -0.600 & JO66LI &  \\
	Repeater & FM              & SK7CY    & Helsingborg             &  1297.2000 &   -6.000 & JO66IB &  \\
	Repeater & FM              & SK7IJ/R  & Vetlanda                &   434.6250 &   -2.000 & JO77OL &  \\
	Repeater & FM              & SK7MO/R  & Ljungby                 &   145.7250 &   -0.600 & JO66XV &  \\
	Repeater & FM              & SK7RFH   & Nässjö                  &   434.8500 &   -2.000 & JO77IP &  \\
	Repeater & FM              & SK7RIH   & Oskarshamn              &   145.7250 &   -0.600 & JO87FG &  \\
	Repeater & FM              & SK7RIH/R & Oskarshamn              &   434.7250 &   -2.000 & JO87EG &  \\
	Repeater & FM              & SK7RIH   & Oskarshamn              &    51.9100 &   -0.600 & JO87EG &  \\
	Repeater & FM              & SK7RJL/R & Lund                    &   434.7250 &   -2.000 & JO65OR &  \\
	Repeater & FM              & SK5CN/R  & Hultsfred/Gåskullen     &   145.7625 &   -0.600 & JO77WL &  \\
	Repeater & FM              & SK7RRV   & Lönsboda                &   434.9000 &   -1.600 & JO76DJ &  \\
	Repeater & FM              & SK7RYR   & Gnosjö                  &   145.6875 &   -0.600 & JO67UI &  \\
	Repeater & FM              & SK7UO/R  & Emmaboda                &   145.7750 &   -0.600 & JO76SP &  \\
	Repeater & FM              & SL7ZXW/R & Nybro                   &   145.6875 &   -0.600 & JO76VQ &  \\
	Repeater & FM              & SM7LNT/R & Mörrum                  &   434.8250 &   -2.000 & JO76IE &  \\
	Repeater & FM              & SK7HW/R  & Växjö/Hollstorp         &   145.6750 &   -0.600 & JO76KU &  \\
	Repeater & FM              & SK7IJ/R  & Vetlanda                &   145.6250 &   -0.600 & JO77OL &  \\
	Repeater & FM              & SK7RGI   & Huskvarna               &   434.7500 &   -2.000 & JO77DT &  \\
	Repeater & FM              & SK7RGI   & Jönköping/Taberg        &   145.7500 &   -0.600 & JO77AQ &  \\
	Repeater & FM              & SK7RBK   & Hässleholm/Bjärnum      &   145.7625 &   -0.600 & JO66UG &  \\
	Repeater & FM              & SM7NTJ/R & Aneby                   &   434.7250 &   -2.000 & JO77HU &  \\
	Repeater & FM              & SK7RGI   & Huskvarna               &    29.6800 &   -0.100 & JO77DT &  \\
	Repeater & FM              & SK7RFL   & Algutsrum/Öland         &   434.6000 &   -2.000 & JO86GQ &  \\
	Repeater & FM              & SK7RFH   & Nässjö                  &   145.6500 &   -0.600 & JO77IP &  \\
	Repeater & DMR             & SK7RJL   & Lund                    &   434.5875 &   -2.000 & JO65OR &  \\
	Repeater & DMR             & SG7RFH   & Nässjö                  &   434.9000 &   -2.000 & JO77IP &  \\
	Repeater & DMR             & SG7BNT   & Bruzaholm               &   434.6000 &   -2.000 & JO77PP &  \\
	Repeater & DMR             & SG7RFH   & Nässjö                  &   145.5875 &   -0.600 & JO77IP &  \\
	Repeater & FM/DMR          & SK7REE   & Söderåsen/Stenestad     &   434.6500 &   -2.000 & JO66NB &  \\
	Repeater & FM/DMR          & SK7REE   & Örkelljunga             &   434.9750 &   -2.000 & JO66PG &  \\
	Repeater & FM/D-Star       & SK7JL-B  & Spjutsbygd              &   434.8750 &   -2.000 & JO76TH &  \\
	Repeater & FM              & SK7GH/R  & Värnamo                 &   434.6000 &   -2.000 & JO77AF &  \\
	Repeater & FM              & SM7JPI/R & Svängsta                &   434.9250 &   -2.000 & JO76JE &  \\
	Repeater & DMR             & SK7BQ    & Kristianstad            &   434.5250 &   -2.000 & JO76AA &  \\
	Repeater & DMR             & SA7BIK   & Höör                    &   434.9125 &   -2.000 & JO65SW &  \\
	Repeater & FM              & SM7NTJ/R & Aneby                   &   145.7750 &   -0.600 & JO77HU &  \\
	Repeater & DMR             & SK7REE   & Helsingborg             &   434.6000 &   -2.000 & JO66IA &  \\
	Repeater & DMR             & SK7AF    & Eksjö                   &   434.5625 &   -2.000 & JO77MP &  \\
	Repeater & FM/DMR/D-star   & SK7RBK   & Bjärnum                 &   434.9500 &   -2.000 & JO66UG &  \\
	Repeater & FM/C4FM         & SK7JD/R  & Västervik               &   145.6750 &   -0.600 & JO87HS &  \\
	Repeater & DMR             & SK7RJL   & Malmö                   &   434.7750 &   -2.000 & JO65LO &  \\
	Repeater & FM              & SK7RFL   & Algutsrum/Öland         &   145.6000 &   -0.600 & JO86GQ &  \\
	Repeater & DMR             & SK7RGI   & Jönköping               &   434.9750 &   -2.000 & JO77CS &  \\
	Repeater & DMR             & SK7HR    & Sävsjö                  &   434.5250 &   -2.000 & JO77HJ &  \\
	Repeater & DMR             & SM7NTJ/R & Aneby                   &   434.9250 &   -2.000 & JO77HU &  \\
	Repeater & DMR/D-Star/C4FM & SK7RFL   & Algutsrum/Öland         &   434.5500 &   -2.000 & JO86GQ &  \\
	Repeater & FM              & SK7GH/R  & Värnamo                 &   145.6000 &   -0.600 & JO77AE &  \\
	Repeater & DMR             & SA7BJF/R & Södra Vi                &   434.6625 &   -2.000 & JO77VR &  \\
	Repeater & DMR             & SK7JD    & Västervik               &   434.6750 &   -2.000 & JO87HS &  \\
	Repeater & FM/DMR          & SG7WSE   & Ekenässjön              &   145.7125 &   -0.600 & JO77ML &  \\
	Repeater & FM/DMR          & SA7KSI/R & Tomelilla               &   434.6375 &   -2.000 & JO65XN &  \\
	Repeater & FM/DMR          & SK7DL    & Emmaboda                &   434.7875 &   -2.000 & JO76SP &  \\
	Repeater & FM              & SK7JL    & Spjutsbygd              &   145.7250 &   -0.600 & JO76TH &  \\
	Repeater & D-Star          & SK7RDS   & Malmö                   &   145.5625 &   -0.600 & JO65LO &  \\
	Repeater & D-Star          & SK7DS    & Malmö                   &   434.5125 &   -2.000 & JO65LO &  \\
	Repeater & DMR/D-Star      & SK7RMQ   & Linderöd                &   145.5875 &   -0.600 & JO65VW &  \\
	Repeater & FM              & SM7HZK/R & Moheda                  &   145.6375 &   -0.600 & JO76HX &  \\
	Repeater & DMR/D-Star      & SK7RPQ   & Malmö                   &   434.6125 &   -2.000 & JO65MN &  \\
	Repeater & FM              & SK7RN/R  & Borgholm                &   434.7750 &   -2.000 & JO86HU &
\end{longtable}

\normalsize

\section{Begrepp i bandplanerna}

\begin{itemize}
\item QRP: Aktivitetscentrum för låg effekt ($<$5W), svaga signaler
      förekommer, visa hänsyn.
\item QRS: Aktivitetscenter för långsam CW.
\item QRSS: Extremt långsam CW med dator.
\item DV: Digital Voice.
\item Image: Bildmoder exempelvis SSTV och Fax som ryms inom den specificerade
	  maximala bandbredden.
\end{itemize}

\section{Trafikregler och tumregler}

\begin{itemize}
\item Vid SSB-telefoni används LSB på frekvenser under 10 MHz och USB
      på frekvenser över 10 MHz.
\item Lägsta acceptabla inställda frekvens för LSB är 3 kHz över
      under bandkant!
\item Högsta acceptable inställda frekvens för USB är 3 kHz under
      övre bandkant!
\item IBP är International Beacon Project. Fyrarna sänder med 3 min
      intervaller och används för att studera utbredningen av
      radiosignaler globalt. Fyrarna sänder anrop och fyra 1 s toner.
      Anropet och första tonen sänds med 100W, därefter sänds tonerna
      med 10W, 1W samt 100mW.
\item Vid AM (A3J) skall hänsyn tas så att störningar på annan trafik ej fö\-re\-kom\-mer
      med de sidband som då uppstår, det gäller då både övre och undre
      sidbandet.
\item Ingen som helst sändning är tillåtet inom fyrsegmenten. Detta skall respekteras.
      Lyssna gärna på nödfrekvenserna men används dem icke, om det
      inte är du som svarar på ett nödsamtal! Undvik QSO allt för nära
      dessa också.
\item Var särskilt uppmärksam på satelliters nerlänksfrekvenser på 10\,m-bandet.
      I detta segment skall endast lyssning ske. Ingen sändning är
      tillåten här eller i skyddssegmentet strax ovanför
      satellitsegmentet. Tänk på att satelliters frekvens kan
      dopplerskiftas uppåt en hel del när de rör sig mot mottagaren.
\end{itemize}


\section{Bandplaner}

\scriptsize
\subsection{Bandplaner VHF--UHF}
\subsubsection{Bandplan 6m 50--52 MHz}
\begin{tabular}{rrrll}

\textbf{Frekvens} &  & \textbf{BW} & \textbf{Trafik} & \textbf{Noteringar} \\ \hline

50.000 & 50.100 & 500 Hz  & CW          & \textbf{CW anrp. 50.050 och 50.090 (interkont.)}             \\ \hline
50.100 & 50.130 & 2.7 kHz & CW, SSB     & Interkontinental DX-trafik. Ej QSO inom Europa               \\ \hline
50.100 & 50.200 & 2.7 kHz & CW,SSB      & DX 50.110--50.130, \textbf{50.110 50.150 anrop (interkont.)} \\ \hline
50.200 & 50.300 & 2.7 kHz & CW,SSB      & Generell användning. 50.285 för crossband                    \\ \hline
50.300 & 50.400 & 2.7 kHz & CW, MGM     & PSK 50.305, EME 50.310 – 50.320                              \\
       &        &         &             & MS 50.350 – 50.380                                           \\ \hline
50.400 & 50.500 & 1 kHz   & CW, MGM     & Endast fyrar, 50.401 ±500 Hz WSPR-fyrar                      \\ \hline
51.210 & 51.390 & 12 kHz  & FM          & Repeater Repeater in, 20/10 kHz kanalavstånd                 \\
       &        &         &             & RF81 – RF99                                                  \\ \hline
50.500 & 52.000 & 12 kHz  & Alla moder  & SSTV 50.510, RTTY 50.600, FM 51.510                          \\ \hline
51.810 & 51.990 & 12 kHz  & FM Repeater & Repeater ut, 20/10 kHz kanalavstånd                          \\
       &        &         &             & RF81 – RF99                                                  \\ \hline
\end{tabular}

\subsubsection{Bandplan 2m 144--146 MHz}
\begin{tabular}{rrrll}

\textbf{Frekvens} &  & \textbf{BW} & \textbf{Trafik} & \textbf{Noteringar} \\ \hline

144.0000 & 144.1100  & 500 Hz  & CW, EME      & \textbf{CW anrop 144.050}               \\
         &           &         &              & MS random 144.100                       \\ \hline
144.1100 & 144.1500  & 500 Hz  & CW, MGM      & EME MGM 144.120--144.160                \\
         &           &         &              & PSK31 cent. 144.138                     \\ \hline
144.1500 & 144.1800  & 2.7 kHz & CW, SSB, MGM & EME 144.150--144.160                    \\
         &           &         &              & MGM 144.160--144.180 anrop 144.170      \\ \hline
144.1800 & 144.3600  & 2.7 kHz & CW, SSB, MGM & MS SSB random 144.195--144.205          \\
         &           &         &              & \textbf{SSB anrop 144.300}              \\ \hline
144.3600 & 144.3990  & 2.7 kHz & CW, SSB, MGM & MS MGM random anrop 144.370             \\ \hline
144.4000 & 144.4900  & 500 Hz  & Fyr          & Exklusivt segment fyrar, ej QSO         \\ \hline
144.5000 & 144.7940  & 20 kHz  & All mode     & SSTV, RTTY, FAX, ATV                    \\
         &           &         &              & Linjära transpondrar                    \\ \hline
144.7940 & 144.9625  & 12 kHz  & MGM          & APRS 144.800                            \\ \hline
144.9750 & 145.19350 & 12 kHz  & FM, DV       & Rpt in 144.975--145.1935                \\
         &           &         &              & RV46–-RV63, 12.5 kHz, 600 kHz skift     \\ \hline
145.1940 & 145.2060  & 12 kHz  & FM rymd      & 145.200 för kom. m. bem. rymdfark.      \\ \hline
145.2060 & 145.5625  & 12 kHz  & FM, DV       & FM 145.2125-–145.5875  V17–V47          \\
         &           &         &              & \textbf{FM anrop 145.500}, RTTY 145.300 \\
         &           &         &              & FM simpl. INET GW 145.2375, 2875, 3375  \\
         &           &         &              & DV anrop 145.375                        \\ \hline
145.5750 & 145.7935  & 12 kHz  & FM, DV       & Rpt ut 145.575--145.7875                \\
         &           &         &              & RV46–RV63, 12.5 kHz kanalavstånd        \\ \hline
145.794  & 145.806   & 12 kHz  & FM Rymd      & 145.800, 145.200 dplx m. bem. rymdfark. \\ \hline
145.806  & 146.000   & 12 kHz  & All mode     & Exklusivt satellit                      \\ \hline
\end{tabular}

\subsubsection{Bandplan 70cm 432--438 MHz}
\begin{tabular}{rrrll}
	\textbf{Frekvens} &          & \textbf{BW} & \textbf{Trafik} & \textbf{Anmärkning}                               \\ \hline

432.0000 & 432.0250 & 500 Hz  & CW           & EME exklusivt.                                    \\ \hline
432.0250 & 432.1000 & 500 Hz  & CW, PSK31    & CW mellan 432.000--085, \textbf{CW anrop 432.050} \\
         &          &         &              & PSK31 432.088                                     \\ \hline
432.1000 & 432.3990 & 2.7 kHz & CW, SSB, MGM & \textbf{SSB anrop 432.200}                        \\
         &          &         &              & Mikrovåg talkback 432.350, FSK441 432.370         \\ \hline
432.4000 & 432.4900 & 500 Hz  & Fyr          & Exklusivt segment för fyrar                       \\ \hline
432.5000 & 432.5940 & 12 kHz  & All mode     & Linjära transpondrar IN 432.500--600              \\ \hline
432.5000 & 432.5750 & 12 kHz  & All mode     & NRAU Digital rep. in 432.500--575 2 MHz skift     \\ \hline
432.5940 & 432.9940 & 12 kHz  & All mode     & Linjära transpondrar ut 432.600--800              \\ \hline
432.5940 & 432.9940 & 12 kHz  & FM           & Rep. in 432.600--975 RU368--398 2 MHz skift       \\ \hline
432.9940 & 433.3810 & 12 kHz  & FM           & Rep. in 433.000--375 RU368--398 1.6 MHz skift     \\ \hline
433.3940 & 433.5810 & 12 kHz  & FM           & SSTV (FM/AFSK) 433.400                            \\
         &          &         &              & FM simplex U272--286 \textbf{anrop 433.500}       \\ \hline
433.6000 & 434.0000 & 20 kHz  & All mode     & RTTY (FM/AFSK) 433.600                            \\
         &          &         &              & FAX 433.700, APRS 433.800                         \\ \hline
434.0000 & 434.4940 & 20 kHz  & All mode     & NRAU Dig. kanaler 433.450, 434.475                \\ \hline
434.5000 & 434.5940 & 20 kHz  & All mode     & NRAU Dig. rep. ut 434.500--575, 2 MHz skift       \\ \hline
434.5940 & 434.9810 & 12 kHz  & FM           & NRAU Rep. ut 434.600--975 RU 368--RU398           \\
         &          &         &              & 12,5 kHz med 2 MHz skift                          \\ \hline
435.000  & 438.000  & 20 kHz  & All mode     & Exklusivt satellit\\
\end{tabular}

\subsubsection{Bandplan 23cm 1240--1300 MHz}
\begin{tabular}{rrrll}
	\textbf{Frekvens}         &               & \textbf{BW} & \textbf{Trafik} & \textbf{Anmärkning}                                          \\ \hline
	         1240.000         & 1243.250      & 20 kHz      & Alla moder      & 1240.000 - 1241.000 Digital kommunikation                    \\ \hline
	         1243.250         & 1260.000      & 20 kHz      & ATV och Data    & Repeater ut 1258.150-1259.350, R20--68                       \\ \hline
	         1260.000         & 1270.000      & 12 kHz      & Satellit        & Endast för satelliter alla moder                             \\ \hline
	         1270.000         & 1272.000      & 20 kHz      & Alla moder      & Repeater in, 1270.025-1270.700, RS1--28                      \\
                                  &               &             &                 & Packet RS29--50                                              \\ \hline
	         1272.000         & 1290.994      & 20 kHz      & ATV och Data    & Amatörtelevision ATV                                         \\ \hline
	         1290.994         & 1291.481      & 20 kHz      & FM och DV       & Repeater in Repeat. in 1291.000--1291.475                    \\
                                  &               &             &                 & RM0 – RM19, 25 kHz, 6 MHz skift                              \\ \hline
	         1291.494         & 1296.000      & 12 kHz      & Alla moder      &                                                              \\ \hline
	         1296.000         & 1296.150      & 500 Hz      & CW,  MGM        & EME 1296.000--025, \textbf{CW anrop 1296.050}                \\
                                  &               &             &                 & PSK31 1296.138 MHz                                           \\ \hline
	         1296.150         & 1296.400      & 2.7 kHz     & CW, SSB, MGM    & \textbf{SSB anrop 1296.200}                                  \\
                                  &               &             &                 & \textbf{FSK441 MS anrop 1296.370}                            \\ \hline
	         1296.400         & 1296.600      & 2.7 kHz     & CW, SSB, MGM    & Linjära transpondrar infrekvens                              \\ \hline
	         1296.600         & 1296.800      & 2.7 kHz     & CW, SSB, MGM    & SSTV/FAX 1296.500, MGM/RTTY 1296.600                         \\ \hline
	         1296.600         & 1296.800      & 2.7 kHz     & CW, SSB, MGM    & Linjära transpondrar utfrekvens                              \\
                                  &               &             &                 & 1296.750-.800 lokala fyrar max 10 W                          \\ \hline
	         1296.800         & 1296.994      & 500 Hz      & Fyrar           & Exklusivt segment för fyrar                                  \\ \hline
	         1296.994         & 1297.481      & 20 kHz      & FM              & Repeater ut Repeater ut 1297.000--1297.475                   \\
                                  &               &             &                 & RM0 – RM19, 25 kHz, 6 MHz skift                              \\ \hline
	         1297.494         & 1297.981      & 20 kHz      & FM simplex      & Simplex 25 kHz kanaler SM20--39                              \\
                                  &               &             &                 & \textbf{FM anrop 1297.500 SM20}                              \\ \hline
	         1299.000         & 1299.750      & 150 kHz     & Alla moder      & 5 st 150 kHz kanaler för DD,                                 \\
                                  &               &             &                 & 1299.075, 225, 375, 525, och 675 $\pm$75 kHz                 \\ \hline
	         1299.750         & 1300.000      & 20 kHz      & Alla moder      & 8 st FM/DV 25 kHz kan. 1299.775--1299.975
\end{tabular}
\normalsize
\clearpage

\section{Frekvenser HF}

\subsection{PR-bandet 27\,MHz}

Detta är det enda bandet som allmänheten kan använda på HF-bandet. Det delar
många egenskaper med 31\,MHz jaktradiobandet men är ett band som är äldre och
mer etablerat.

Maximal uteffekt på bandet är 4W RMS ERP dvs antennvinst överstigande en
1/2-vågs dipol (0 dBd, 2.12 dBi) måste inräknas i effekten efter avdrag för
matningsförlust. Modulationsslag AM, FM och SSB (primärt används USB) är
tillåtet på alla kanaler i dag. Traditionellt används kanal 24 för USB men i dag
får vilken kanal som helst användas.

Kanalerna med A efter är upplåtna för radiostyrning och inte för telefoni.
Undvik därför att använda dessa om du har en sändare som kan använda dessa
frekvenser. De är med i tabellen för den skall vara komplett.

\clearpage

\begin{longtable}{rrl|rrl}
	\textbf{Frekvens}& \textbf{Kanalnr}& \textbf{Övrigt}
	& \textbf{Frekvens} & \textbf{Kanalnr} & \textbf{Övrigt}  \\
	\hline \endhead
	  26,965 &       1 &                &   27,215 &      21 &           \\
	  26,975 &       2 &                &   27,225 &      22 &           \\
	  26,985 &       3 &                &   27,255 &      23 &           \\
	  26,995 &      3A & Radiostyrning  &   27,235 &      24 & SSB (USB) \\
	  27,005 &       4 &                &   27,245 &      25 &           \\
	  27,015 &       5 &                &   27,265 &      26 &           \\
	  27,025 &       6 &                &   27,275 &      27 &           \\
	  27,035 &       7 &                &   27,285 &      28 &           \\
	  27,045 &      7A & Radiostyrning  &   27,295 &      29 &           \\
	  27,055 &       8 &                &   27,305 &      30 &           \\
	  27,065 &       9 &                &   27,315 &      31 &           \\
	  27,075 &      10 &                &   27,325 &      32 &           \\
	  27,085 &      11 &                &   27,335 &      33 &           \\
	  27,095 &     11A & f.d. nödfrekv. &   27,345 &      34 &           \\
	  27,105 &      12 &                &   27,355 &      35 &           \\
	  27,115 &      13 &                &   27,365 &      36 &           \\
	  27,125 &      14 &                &   27,375 &      37 &           \\
	  27,135 &      15 &                &   27,385 &      38 &           \\
	  27,155 &      16 &                &   27,395 &      39 &           \\
	  27,165 &      17 &                &   27,405 &      40 &           \\
	  27,175 &      18 &                &          &         &           \\
	  27,185 &      19 &                &          &         &           \\
	  27,195 &     19A & Radiostyrning  &          &         &           \\
	  27,205 &      20 &                &          &         &
\end{longtable}

Många apparater är endast FM i dag men det finns de som också har SSB. Äldre
apparater hade oftast AM och FM och ibland även SSB. Telegrafi körs i princip
inte på PR-bandet, troligen för att det aldrig varit några krav på det och de
som kör heller inte haft möjlighet förr i tiden att DX-a på bandet.

Innan Televerket släppte upp bestämmelserna var det väldigt hårda bestämmelser
på bandet, i princip var det bara kommunikation inom familjen som tilläts. I dag
kan bandet användas som man vill och det är på sina håll god aktivitet.

Kom ihåg att inte överskrida effektbegränsningarna bara.

\subsection{Amatörradiofrekvenser HF}

Alla frekvenser i kHz, bandbredder i Hz.

\subsubsection{Bandplan 2.2\,km, 135,7--137,8\,kHz}

\begin{tabular}{rrrll}
\textbf{Frekvens} &  & \textbf{BW} & \textbf{Trafik} & \textbf{Noteringar} \\ \hline
135,7 & 135,8 & 200 & CQ, QRSS, Digi & OBS! Högsta effekt 1W ERP. \\ \hline
\end{tabular}

\subsubsection{Bandplan 600\,m, 472--479\,kHz}
\begin{tabular}{rrrll}
\multicolumn{2}{c}{\textbf{Frekvens}} & \textbf{BW} & \textbf{Trafik} & \textbf{Noteringar} \\ \hline
472 & 479 & 200 & CW, QRSS, Digi & OBS! Högsta utstrålad effekt 1W EIRP \\ \hline
\end{tabular}

\subsubsection{Bandplan 160\,m, 1810--2000\,kHz}
\begin{tabular}{rrrll}
\multicolumn{2}{c}{\textbf{Frekvens}} & \textbf{BW} & \textbf{Trafik} & \textbf{Noteringar} \\ \hline
1810 & 1838 & 200  & CW         & Exklusivt för CW. Interkontinental trafik har prio. \\ \hline
1838 & 1840 & 500  & Smalband   & Ej packet på 160m, PSK 1838,150                    \\ \hline
1840 & 1850 & 2700 & Alla moder & Även digimode. SSB QRP 1843 kHz                    \\ \hline
1850 & 1900 & 2700 & Alla moder & OBS! Max 10 W till ant.                             \\ \hline
1900 & 1950 & 2700 & Alla moder & OBS! Max 100 W till ant.                            \\ \hline
1950 & 2000 & 2700 & Alla moder & OBS! Max 10 W till ant.                             \\ \hline
\end{tabular}

\subsubsection{Bandplan 80\,m, 3500--3800\,kHz}
\begin{tabular}{rrrll}
\multicolumn{2}{c}{\textbf{Frekvens}} & \textbf{BW} & \textbf{Trafik} & \textbf{Noteringar} \\ \hline
3500 & 3510 & 200  & CW             & Exklusivt CW                         \\
      &       &      &                & Interkontinental DX-trafik har prio  \\ \hline
3510 & 3580 & 200  & CW             & Exklusivt CW contest 3510-–560       \\
      &       &      &                & CW QRS 3 555 kHz, CW QRP 3 560       \\ \hline
3580 & 3600 & 500  & Smalband, Digi & PSK 3580,150                        \\
      &       &      &                & Automatiska Digimoder 3590--600     \\ \hline
3600 & 3620 & 2700 & Alla moder     & Digimoder Automatiska Digimoder      \\ \hline
3600 & 3650 & 2700 & Alla moder     & SSB contest 3600--650               \\
      &       &      &                & DV 3630                             \\ \hline
3650 & 3700 & 2700 & Alla moder     & SSB QRP 3690                        \\ \hline
3700 & 3800 & 2700 & Alla moder     & Contest 3700-–800                   \\
      &       &      &                & Image 3775                          \\
      &       &      &                & Region 1 nödfrekvens 3760           \\ \hline
3775 & 3800 & 2700 & Alla moder     & Interkontinental DX-trafik prioritet \\ \hline
\end{tabular}


\subsubsection{Bandplan 60\,m, 5351\,5--5366,5\,kHz}
\begin{tabular}{rrrll}
\multicolumn{2}{c}{\textbf{Frekvens}} & \textbf{BW} & \textbf{Trafik} & \textbf{Noteringar} \\ \hline
5351\,5 & 5354\,0 &  200 & CW, Digimoder      & OBS! Högsta utstrålad effekt 15 W EIRP\\
        &         &      &                    & i 60\,m-bandet\\ \hline
5354\,0 & 5366\,0 & 2700 & Alla sändningsslag & USB rekommenderas för SSB\\ \hline
5366\,0 & 5366\,5 &   20 & Smalbandsmoder     & För extrema smalbandsmoder,\\
        &         &      &                    & max 20 Hz bandbredd\\ \hline
\end{tabular}

\subsubsection{Bandplan 40\,m, 7000--7200\,kHz}
\begin{tabular}{rrrll}
\multicolumn{2}{c}{\textbf{Frekvens}} & \textbf{BW} & \textbf{Trafik} & \textbf{Noteringar} \\ \hline
7000 & 7040 & 200  & CW         & Exklusivt CW.                             \\
     &      &      &            & QRP aktivitetscentrum 7030\,kHz           \\ \hline
7040 & 7050 & 500  & Smalband   & Digimoder Automatiska inom 7047–-050\,kHz \\ \hline
7050 & 7060 & 2700 & Alla moder & Digimoder Automatiska inom 7050–-053\,kHz \\ \hline
7060 & 7100 & 2700 & Alla moder & SSB contest i segmentet                   \\
     &      &      &            & DV 7 070 kHz, SSB QRP 7090 kHz            \\ \hline
7100 & 7130 & 2700 & Alla moder & Region 1 nödfrekvens 7110 kHz             \\ \hline
7130 & 7200 & 2700 & Alla moder & SSB contest i segmentet                   \\
     &      &      &            & Image 7165\,kHz                           \\ \hline
7175 & 7200 & 2700 & Alla moder & Interkontinental DX-trafik prio           \\ \hline
\end{tabular}

\subsubsection{Bandplan 30\,m, 10\,100--10\,150 kHz}
\begin{tabular}{rrrll}
\multicolumn{2}{c}{\textbf{Frekvens}} & \textbf{BW} & \textbf{Trafik} & \textbf{Noteringar} \\ \hline
10\,100 & 10\,140 & 200 & CW       & CW exkl. Max 150 Watt på 30 m           \\
        &         &     &          & CW QRP 10\,116\,kHz                     \\ \hline
10\,140 & 10\,150 & 500 & Smalband & Digimoder PSK 10142,150\,kHz. Ej Packet \\ \hline
\end{tabular}

\subsubsection{Bandplan 20\,m, 14\,000--14\,350 kHz}
\begin{tabular}{rrrll}
\multicolumn{2}{c}{\textbf{Frekvens}} & \textbf{BW} & \textbf{Trafik} & \textbf{Noteringar} \\ \hline
14\,000 & 14\,070 & 200  & CW         & Exklusivt CW                             \\
        &         &      &            & Conctest 14\,000-–060                    \\
        &         &      &            & CW QRS 14 055, CW QRP 14\,060            \\ \hline
14\,070 & 14\,099 & 500  & Smalband   & PSK 14 070,150                           \\
        &         &      &            & Auto Digimoder 14 089-–099               \\ \hline
14\,099 & 14\,101 & 200  & Fyrar      & Exklusivt IBP, endast fyrar              \\ \hline
14\,101 & 14 \,12 & 2700 & Alla moder & Digitala moder och obevakade Digimoder   \\ \hline
14\,112 & 14\,350 & 2700 & Alla moder & SSB Contest 14 125--300                  \\
        &         &      &            & DV 14 130, DXpedition prio 14\,195$\pm$5 \\ \hline
14\,300 & 14\,350 & 2700 & Alla moder & Image 14\,230, SSB QRP 14\,285           \\
        &         &      &            & Global nödfrekvens 14 300                \\ \hline
\end{tabular}

\subsubsection{Bandplan 17 m, 18\,068--18\,168 kHz}
\begin{tabular}{rrrll}
\multicolumn{2}{c}{\textbf{Frekvens}} & \textbf{BW} & \textbf{Trafik} & \textbf{Noteringar} \\ \hline

18\,068 & 18\,095 & 200  & CW         & CW exklusivt. QRP 18\,086             \\ \hline
18\,095 & 18\,109 & 500  & Smalband   & Digimoder PSK 18\,100,150             \\
        &         &      &            & Automatiska Digimoder 18\,105-–109 \\ \hline
18\,109 & 18\,111 & 200  & Fyrar      & Exklusivt fyrar, IBP fyrnät           \\ \hline
18\,111 & 18\,168 & 2700 & Alla moder & Digi 18\,111–-120                   \\
        &         &      &            & SSB QRP 18\,130, DV 18\,150           \\
        &         &      &            & Global nödfrekv. 18\,160              \\ \hline
\end{tabular}

\subsubsection{Bandplan 15\,m, 21\,000--21\,450 kHz}
\begin{tabular}{rrrll}
\multicolumn{2}{c}{\textbf{Frekvens}} & \textbf{BW} & \textbf{Trafik} & \textbf{Noteringar} \\ \hline

21\,000 & 21\,070 & 200  & CW         & Exklusivt CW, QRS 21\,055, CW QRP 21\,060           \\ \hline
21\,070 & 21\,110 & 500  & Smalband   & PSK 21\,080.150, Automatiska Digimoder 21\,090–-110 \\
21\,110 & 21\,120 & 2700 & Alla moder & Alla moder utom SSB!                                \\
        &         &      &            & Digimoder, och Automatiska Digimoder                \\ \hline
21\,120 & 21\,149 & 500  & Smalband   &                                                     \\ \hline
21\,149 & 21\,151 & 200  & Fyrar      & Exklusivt fyrar. IBP fyrnät                         \\ \hline
21\,151 & 21\,450 & 2700 & Alla moder & DV 21\,180, SSB QRP 21\,285, Image 21\,340          \\
        &         &      &            & Global nödfrekv. 21\,360                            \\ \hline
\end{tabular}

\subsubsection{Bandplan 12\,m, 24\,890--24\,990 kHz}
\begin{tabular}{rrrll}
\multicolumn{2}{c}{\textbf{Frekvens}} & \textbf{BW} & \textbf{Trafik} & \textbf{Noteringar} \\ \hline
24\,890 & 24\,915 & 200  & CW         & Exklusivt CW, QRP 24\,906                             \\ \hline
24\,915 & 24\,929 & 500  & Smalband   & PSK 24\,920,150, Automatiska Digimoder 24\,925–-929 \\ \hline
24\,929 & 24\,931 & 200  & Fyrar      & Fyrar, IBP fyrnät                                    \\ \hline
24\,931 & 24\,990 & 2700 & Alla moder & Auto Digimoder 24\,931-–940                        \\
       &        &      &            & SSB QRP 24\,950, DV 24,960                            \\ \hline
\end{tabular}

\small
\subsubsection{Bandplan 10\,m, 28\,000-29\,700 kHz}
\begin{tabular}{rrrll}
\multicolumn{2}{c}{\textbf{Frekvens}} & \textbf{BW} & \textbf{Trafik} & \textbf{Noteringar} \\ \hline
28\,000 & 28\,070 & 200  & CW         & Exklusivt CW, QRS 28\,055, CW QRP 28\,060              \\ \hline
28\,070 & 28\,190 & 500  & Smalband   & PSK 28 120,150, Auto Digimoder inom 28\,120--150       \\ \hline
28\,190 & 28\,199 & 200  & Fyrar IBP  & Regionala fyrar med tidsdelning                        \\ \hline
28\,199 & 28\,201 & 200  & Fyrar IBP  & IBP fyrnät                                             \\ \hline
28\,201 & 28\,225 & 200  & Fyrar IBP  & kontinuerligt sändande fyrar                           \\ \hline
28\,225 & 28\,300 & 2700 & Alla moder & Övriga fyrar                                           \\ \hline
28\,300 & 28\,320 & 2700 & Alla moder & Digimoder och Automatiska Digimoder                    \\ \hline
28\,320 & 29\,100 & 2700 & Alla moder & DV 28\,330 kHz, SSB QRP 28\,360 kHz                    \\
        &         &      &            & Image 28\,680 kHz                                      \\ \hline
29\,100 & 29\,200 & 6000 & Alla moder & FM simplex, 10\,kHz kanaler                            \\
        &         &      &            & Maximalt ±2,5 kHz dev., max 2,5\,kHz mod.frek.         \\ \hline
29\,200 & 29\,300 & 6000 & Alla moder & Digimoder och Automatiska Digimoder                    \\ \hline
29\,300 & 29\,510 & 6000 & Satellit   & Nerlänk fr. satellit. EJ SÄNDNING I SEGMENTET          \\ \hline
29\,510 & 29\,520 & 6000 & Skydd      & Skyddsfrekvens för satelliter. EJ SÄNDNING I SEGMENTET \\ \hline
29\,520 & 29\,590 & 6000 & Alla moder & FM Repeater in RH1--8, 100\,kHz duplex, 2.5\,kHz NBFM  \\ \hline
29\,600 & 29\,620 & 6000 & Alla moder & FM simplex, anrop 29\,600                              \\
        &         &      &            & FM simplex repeater 29\,610                            \\ \hline
29\,620 & 29\,700 & 6000 & Alla moder & FM Repeater ut RH1--8, 100\,kHz duplex                 \\ \hline
\end{tabular}
%\end{landscape}
\normalsize

\subsection{JOTA---Jamboree on the air, scoutfrekvenser}

Scouterna har frekvenser på HF likväl som VHF/UHF som de aktiverar vid särskilda
tillfällen ofta i tillsammans med en lokal amatörradioklubb eller vanliga
amtörradioeldsjälar som inte sällan också är scouter. Här kommer en lista på
frekvenser som är vanligt förekommande i scoutsammanhang.

\begin{table}[H]
\centering
\begin{tabular}{rrll}
	\textbf{Band} & \textbf{Frekvens} & \textbf{Trafik} & \textbf{Not} \\ \hline

               80 & 3 570  & CW  &             \\
	              & 3 940  & SSB & Ej region 2 \\
	              & 3 690  & SSB &             \\ \hline
	           40 & 7 030  & CW  &             \\
	              & 7 190  & SSB &             \\
	              & 7 090  & SSB &             \\ \hline
	           20 & 14 060 & CW  &             \\
	              & 14 290 & SSB &             \\ \hline
	           17 & 18 080 & CW  &             \\
	              & 18 140 & SSB &             \\ \hline
	           15 & 21 140 & CW  &             \\
	              & 21 360 & SSB &             \\ \hline
	           12 & 24 910 & CW  &             \\
	              & 24 960 & SSB &             \\ \hline
	           10 & 28 180 & CW  &             \\
	              & 28 390 & SSB &             \\ \hline
	            6 & 10 160 & CW  &             \\
	              & 50 160 & SSB &             \\ \hline
\end{tabular}
\caption{Scouters JOTA-frekvenser på HF}
\end{table}

Normalt aktiveras dessa frekvenser tredje veckoslutet i oktober varje
år, fredag till söndag. Då kan det vara många klubbar som finns på
frekvenserna och det är också vanligt att man hör dem på helt andra
frekvenser. De frekvenser som listas här är inte på något vis de enda
frekvenser som scouter använder.

\clearpage

\subsection{Marina MF/HF-frekvenser}

De marina HF-banden är uppdelade på ett antal band. Det finns en
generell kanalindelning med 3 kHz per kanal och SSB som
modulationssätt på respektive band. Marina HF-kanaler finns på banden
4, 6, 8, 12, 16, 18, 22 och 25 MHz.

MF även benämnd gränsvåg i marina sammanahang är inte lika ofta
används som den var en gång i tiden. Nedan listas de frekvenser som
används i Sverige.

\subsubsection{Svenska MF-kanaler}

\begin{longtable}{llrr}
\textbf{Kanal} & \textbf{Placering} & \textbf{Skepp} & \textbf{Kust}  \\ \hline
\endhead

MF1 & Gotland       & 2\,099 & 1\,6874 \\
MF2 & ---           & ---  & ---   \\
MF3 & Gislövshammar & 2\,060 & 1\,797  \\
MF4 & Härnösand     & 2\,216 & 2\,733  \\
MF5 & Bjuröklubb    & 2\,123 & 1\,779  \\
MF6 & Grimeton      & 2\,135 & 1\,710
\end{longtable}

\subsubsection{Nödfrekvenser}

\begin{longtable}{lrr}
\textbf{Band} & \textbf{Frekvens} & \textbf{DSC Frekvens}\\ \hline \endhead

MF   & 2 182  & 2 187.5  \\
HF4  & 4 125  & 4 207.5  \\
HF6  & 6 215  & 6 312.0  \\
HF8  & 8 291  & 8 414.5  \\
HF12 & 12 290 & 12 577.0 \\
HF16 & 16 429 & 16 804.5 \\
\end{longtable}

\subsubsection{Primära HF skepp-till-skepp}

\begin{longtable}{lrrrrrrrr}
\textbf{Kanal} & \textbf{HF4} & \textbf{HF6} & \textbf{HF8} &
               \textbf{HF12} & \textbf{HF16} & \textbf{HF18} &
               \textbf{HF22} & \textbf{HF25} \\
\hline
\endhead

A & 4 146 & 6 224 & 8 294 & 12 353 & 16 528 & 18 825 & 22 159 & 25 100 \\
B & 4 149 & 6 227 & 8 297 & 12 356 & 16 531 & 18 828 & 22 162 & 25 103 \\
C &       & 6 230 &       & 12 359 & 16 534 & 18 831 & 22 165 & 25 106 \\
D &       &       &       & 12 362 & 16 537 & 18 834 & 22 168 & 25 109 \\
E &       &       &       & 12 365 & 16 540 & 18 837 & 22 171 & 25 112 \\
F &       &       &       &        & 16 543 & 18 840 & 22 174 & 25 115 \\
G &       &       &       &        & 16 546 & 18 843 & 22 177 & 25 118 \\
\end{longtable}

\clearpage

\subsection{Fyrar}

\subsubsection{IBP -- International Beacon Project}

Det finns flera olika typer av fyrar men för HF är IBP (International
Beacon Project) intressant eftersom det ger operatören möjlighet att
utröna hur utbredningen ser ut för stunden genom att lyssna efter
fyrar. Fyrarna har gemensam hårdvara och synkroniseras mot
tidsreferens. Fyrar kan vara offline av olika skäl, kontrollera mot
IBP:s hemsida om du inte hör en fyr du brukar höra.

Tabellen nedan visar anropssignaler och första sändningsslotten som
fyren sänder, dvs SCHED för olika fyrar och frekvenser.

\subsubsection{Lista över IBP-fyrar}

\begin{table}[H]
\centering
\begin{tabular}{llrrrrr}
\textbf{Signal} & \textbf{QTH} & \textbf{14 100} & \textbf{18 110} &
                \textbf{21 150} & \textbf{24 930} & \textbf{28 200} \\ \hline

4U1UN  & United Nations & 00:00  & 00:10  & 00:20  & 00:30  & 00:40  \\
VE8AT  & Canada         & 00:10  & 00:20  & 00:30  & 00:40  & 00:50  \\
W6WX   & United States  & 00:20  & 00:30  & 00:40  & 00:50  & 01:00  \\
KH6RS  & Hawaii         & 00:30  & 00:40  & 00:50  & 01:00  & 01:10  \\
ZL6B   & New Zealand    & 00:40  & 00:50  & 01:00  & 01:10  & 01:20  \\
VK6RBP & Australia      & 00:50  & 01:00  & 01:10  & 01:20  & 01:30  \\
JA2IGY & Japan          & 01:00  & 01:10  & 01:20  & 01:30  & 01:40  \\
RR9O   & Russia         & 01:10  & 01:20  & 01:30  & 01:40  & 01:50  \\
VR2B   & Hong Kong      & 01:20  & 01:30  & 01:40  & 01:50  & 02:00  \\
4S7B   & Sri Lanka      & 01:30  & 01:40  & 01:50  & 02:00  & 02:10  \\
ZS6DN  & South Africa   & 01:40  & 01:50  & 02:00  & 02:10  & 02:20  \\
5Z4B   & Kenya          & 01:50  & 02:00  & 02:10  & 02:20  & 02:30  \\
4X6TU  & Israel         & 02:00  & 02:10  & 02:20  & 02:30  & 02:40  \\
OH2B   & Finland        & 02:10  & 02:20  & 02:30  & 02:40  & 02:50  \\
CS3B   & Madeira        & 02:20  & 02:30  & 02:40  & 02:50  & 00:00  \\
LU4AA  & Argentina      & 02:30  & 02:40  & 02:50  & 00:00  & 00:10  \\
OA4B   & Peru           & 02:40  & 02:50  & 00:00  & 00:10  & 00:20  \\
YV5B   & Venezuela      & 02:50  & 00:00  & 00:10  & 00:20  & 00:30  \\
\end{tabular}
\caption{IBP-fyrar}
\end{table}

\normalsize

\clearpage



\end{document}

