% !TeX encoding = UTF-8
% !TeX spellcheck = sv_SE

\section*{Förord}

Radiohandboken fyller 10 år i år! Den första utgåvan, eller version
1.0.0 publicerades på nätet 1:a november 2015. Det är fasen inte så
illa pinkat ändå!

Denna version av radiohandboken innehåller en stor mängt förändringar mot
tidigare versioner. Dels har jag valt att ta bort de specifika VHF- och
HF-versionerna av den och i stället bara inkludera allting i samma handbok för
enkelthets skull. Jag har lagt vikt vid att i stället försöka göra det tydligt
så den som inte är intresserad av vissa delar enkelt kan hoppa över dem.

Det har också medfört en helt ny struktur och kapitelindelning på en högre
nivå som gör att det är mer logiskt strukturerat. Alla frekvenslistor är
flyttade till ett eget kapitel som gör det enklare att skriva ut dem medan
reglemente och teknik osv fått helt egna kapitel.

Ett index har skapats med indexering av de olika begreppen i boken så att det
ska bli enklare att hitta. Denna indexering återfinns längst bak i boken med
länkar så att man kan klicka sig fram i en PDF.

De svenska morsetecknen har också fått komma med i denna utgåva, det
är inte så vanligt att folk ägnar sig åt morsesignalering i dag som
det var en gång i tiden men det hör definitivt till så det ska förstås
vara med.

Många frekvenslistor har blivit uppdaterade med nya frekvenser, information
och kanaler, felaktigheter har korrigerats och jag hoppas att det mesta nu
faktiskt är ganska korrekt!

Till sist ett stort tack till alla som hjälpt till med förslag och korrektur
och annat. Ovärderligt. Fortsätt med det!

\url{https://sikvall.se}

Och kör radio där ute. Alltid med stil.

\vspace{4mm}

\noindent Karlholm, \DokumentDatum\\
Täpp-Anders Sikvall\\
	SM5UEI

\vspace{\fill}

\section*{Licens}

Denna handbok är författad av Täpp-Anders Sikvall, SM5UEI. Den är
släppt under en creative commons licens som kallas för CC BY-NC-SA 4.0.

I korthet innebär det att du får distribuera och trycka upp materialet
hur du vill, dela med dig av det men inte för kommersiella ändamål. Du
får dessutom använda materialet i dina egna alster men om du gör det
ska du släppa dessa under samma licens. Mitt namn ska stå kvar i
originalet och om du använder det i ett annat verk ska det vara
tydligt att det är ett derivat.

För fullständig information kan du besöka\\
\url{https://creativecommons.org/licenses/by-nc-sa/4.0/deed.sv}

Licensen gäller alla versioner av radiohandboken, alla utgåvor och
alla utdrag ur den.
