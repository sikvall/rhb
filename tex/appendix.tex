% !TeX encoding = UTF-8
% !TeX spellcheck = sv_SE

\appendix

\chapter{Prepping och radio}
\index{prepping}
\index{krislåda}
\index{SHTF}

Det finns egentligen otroligt mycket man kan säga om prepping och
radioanvändning. Det finns väldigt många olika åsikter men det finns också en
del erfarenheter över vad som faktiskt fungerar. I denna bilaga ska vi titta
lite på några goda råd angående just vad du ska ha för radio i krislådan och
vad man ska ha det till.

Det här gäller också mest de som inte är radioamatörer, vi som är det har nog
redan god koll på olika frekvensband och även


\section{Radio till krislådan}
\index{krislåda}

I krislådan kan man tänka sig en massa olika radioapparater, de flesta tänker
förmodligen på en handapparat av något slag.

\subsection{För de kortare avstånden}

Är man inte radioamatör skulle jag skulle rekommendera en dual-band med VHF
och UHF så att man kan använda sig av de fria kanalerna på respektive band,
exempelvis jaktkanalerna på 155 MHz (se \ref{155-MHz})  samt KDR-kanalerna på
444 MHz (se \ref{444-MHz}) är en riktigt bra början.

Om du är radioamatör är ju förstås 2-metersbandet och 70 cm banden givna att
ha eftersom om du kan nå en repeater som har kraft är de ruggigt bra. Vissa
radioapparater kan förstås också programmeras att använda sig av
jaktfrekvenserna på 155 MHz eller KDR-kanalerna på 444 MHz.

\subsection{För lite längre avstånd}

För lite längre avstånd är en basradio med antenn på taket för 27 MHz
och/eller 69 MHz förmodligen ett riktigt gott val om man inte är radioamatör
och har tillgång till fler kortvågsband då man nog kan tänka sig att
80-metersbandet och 40-metersbanden kan vara väldigt användbara för att nå
större delen av skandinavien beroende på tidpunkt och när på året det är.

\subsection{Fordonsmonterad radio}

Något av det mest smidiga som finns är fordonsmonterad radio som kan ställas
upp där den behövs med antenn på taket eller externt riggade antenner efter
behov. Genom att stå på lämplig höjd finns det möjlighet att nå riktigt lång
med enkla medel.

\section{Lyssna och sänd enligt schema}
\index{sched}
\index{schema}
\index{3-3-3}


Principen ''3-3-3'' har framlagts som en princip vi kan följa lite oavsett vilken
kanalplan vi använder. I princip betyder det att vi lyssnar tre minuter vid
var tredje timme på kanal 3 (eller en annan given anropskanal, för VHF marin
är det förstås kanal 16 som gäller och för 2-metersbandet är det
anropsfrekvensen 145,500 MHz som är lämpligast osv).

Men om alla sänder och lyssnar enligt detta schema så blir det lättare att få
tag i varandra under en kris. Var tredje timme betyder alltså att du lyssnar
minst 3 minuter med klockan 00:00, 03:00, 06:00, 09:00, 12:00, 15:00, 18:00
samt 21:00.

Naturligtvis är det bara om det passar och är lämpligt att passa alla
tidpunkter och det kan vara viktigare att lyssna exempelvis på degtiderna
exempelvis 09:00, 12:00, 15:00 och 18:00 än på natten. Sannolikheten att folk
behöver vila är högre 03:00.

\subsection{En bra tidsreferens}
\index{prepping!tidsreferens}

Det är bra att ha en exakt klocka även om mobilnätet slutar att fungera.
Mobiler med GNSS\footnote{Global Navigation Satellite System, exempelvis GPS,
Galileo, Beidou och Glonass}-mottagare har som regel möjlighet att via appar
visa exakt tid. Vanliga kvartsur är utmärkta tidhållare och behöver i regel
bara justeras några sekunder på en månad.

Klockor som baseras på den tyska radiosignalen DCF77\footnote{Sänder
tidsreferens över långvåg från Mainflingen i Tyskland. DCF står för
Deutschland, C för långvågssignal och F för frankfurt. 77 är frekvensen (eg.
77,5 kHz).} är också ett utmärkt val, dessa fungerar ungefär upp till höjd med
Gävle men inte i den norra änden av landet då vi hamnar utanför
radiohorisonten.

Har du möjlighet att lyssna på utländska radiostationer på AM-banden
exempelvis över kortvåg så identifierar de sig normalt varje timme. Det finns
dock inte så många AM-stationer kvar som vi hör längre men på natten kan vi
höra långväga sådana på kortvågsbanden.

\section{Vad kan vi göra?}

Färutom att själva söka kontakt med andra och förstå vad som sker, om vi ska
ta oss till ett bättre ställe, hålla samband inom familjen et cetera så kan vi
även bistå allmänheten.

Vid exempelvis ett längre avbrott på telefoni kan radio vara ett sätt att nå
utanför området där avbrottet finns till någon annan som kan ta mot
meddelanden och ringa in dem till släkt och vänner etc och få fram meddelanden
trots att det inte går att ringa.

Här har radioamatörerna en fördel i att de kan välja frekvensband som är
rimliga för att nå ett grannland dag- eller nattetid och söka samband med
andra radioamatörer som då kan förmedla meddlenanden.

Vi kan hjälpa och bistå kommun och andra med teknisk kompetens och kunskap men
även utrustning och möjligheter att kommunicera i en kris.


\section{Vikten av övning}
\index{prepping!övning}

Det kan inte understrykas nog hur viktigt övningar är för att upprätta
radiosamband från olika platser och att prova och se till att man är redo
innan en kris uppstår. Saker som generatorer och ackumulatorbatterier som kan
laddas (exempelvis med startkablar från bilens generator) kommer man mycket
långt med!

\section{Frekvensband}
\index{prepping!frekvensband}

För de som inte har amatörradiocertifikat är egentligen följande frekvensband de mest lämpliga att försöka använda:

\begin{table}[H]
\resizebox{\textwidth}{!}{%
\begin{tabular}{rl}
	  \bf Band & \bf Användningsområde                                                                             \\ \hline
	    27 MHz & Vid gynnsamma förhållanden lång utbredning, men också risk för störningar                         \\
	    69 MHz & Ett robust band men endast tillåtet i Sverige, bra markvågsutbredning                             \\
	Flygbandet & Rekommenderas inte, AM-sändning dessutom.                                                         \\
	 VHF-Marin & För nödsamtal nära kusten, till sjöss, eller någon av de stora insjöarna.                         \\
	   155 MHz & Jaktkanalerna, bra för lite kortare förbindelser och handapparater.                               \\
	   444 MHz & Kortdistansradiokanalerna, begränsat med effekt men bra i urban miljö och mellan fordon.          \\
	   446 MHz & Räckvidden är från några hundra meter till en km ungefär i terräng beroende på. Familjemedl.
\end{tabular}}
\caption{Lämpliga frekvensband landmobil radio.}
\end{table}

\begin{table}[H]
\resizebox{\textwidth}{!}{%
\begin{tabular}{rl}
	\bf Band & \bf Användningsområde                                                                                             \\ \hline
	   160 m & Mer sällan använt och blir stora antenner men dagtid på sommaren kan det vara riktigt bra för NVIS.               \\
	    80 m & Dagtid utmärkt för kontakter runt hela skandinavien med NVIS. Nattetid ofta störningar österifrån.                \\
	    60 m & Ett relativt nytt band men också få användare, utmärkt utbredning dag som natt.                                   \\
	    40 m & Har många egenskaper från 80 meter, mer utpräglad skipzon NVIS.                                                   \\
	    20 m & Fungerar ofta globalt med bra antenner. Stor skipzon.                                                             \\
	    17 m & Liknande egenskaper som 20 m men betydligt färre användningar och ingen contest.                                  \\
	    15 m & Stort och bra band, uttalad skipzon, vid perioder med låg solaktivitet kan det vara sämre rymdvåg                 \\
	 12/10 m & Utmärkt vid hög solfläcksaktivitet, global utbredning med små medel men ofta osäkert.          \\
	     6 m & Användbart för relativt god markvågsutbredning men få användare                                                   \\
	     2 m & Repeatrar och direktkontakt, relativt enkelt bygga riktantenner. God utbredning i landsbygd. \\
	   70 cm & Utmärkt i stad och bebyggelse, som kommunikation mellan fordon etc. Också många repeatrar
\end{tabular}}
\caption{Lämpliga frekvensband amatörradio.}
\end{table}
