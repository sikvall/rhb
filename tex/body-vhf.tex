\section{Frekvenser VHF--UHF}

\subsection{Frekvenser ej amatörradio}

Dessa frekvenser är avsedda för allmänhet eller för specifika
ända\-mål som anges. Det innebär att de kan brukas för de ändamål som
anges i PTS för\-fatt\-nings\-sam\-ling\-ar och sammanställning över
ej tillståndspliktiga frekvenser. Observera att du är skyldig att
själv kontrollera bestämmelserna innan en frekvens brukas.

Effekten i tabellen är ustrålad effekt PEP om inte annat anges.

\subsubsection{Jaktfrekvenser 31 MHz}

Frekvenserna på detta band var tidigare till för enbart jakt. I dag är
de öppna för övrig landmobil trafik och kan nyttjas till
fritidskommunikation av annat slag.

Högsta effekt är 5\,W och maximal sändningscykel är 10\% vilket
betyder att under en timme får man sända maximalt 6 minuter.

I oktober 2012 utökades de gamla jaktkanalerna med ett antal nya
kanaler vilket skedde i oktober 2012. De har ingen officiell
kanalnumrering eller egentlig benämning men jag har valt att numrera
upp dem efter de traditionella numren med början på 25.

Kanal 24 har dock tidigare haft en frekvens som inte längre är i bruk,
så det vore förvirrande att använda den -- den saknas därför i listan.
Nya kanaler är markerade i listan med asterisk och har fått nummer
från kanal 25 och uppåt efter frekvens. Detta gör att listan blir en
smula oordnad.

\begin{longtable}{rll|rll}
	\caption{Jaktfrekvenser 31\,MHz tabell}\\
	\textbf{Frekvens} & \textbf{Benämning} & \textbf{Tidigare} & \textbf{Frekvens} & \textbf{Benämning} & \textbf{Tidigare} \\ \hline
		\endfirsthead
	\textbf{Frekvens} & \textbf{Benämning} & \textbf{Tidigare} & \textbf{Frekvens} & \textbf{Benämning} & \textbf{Tidigare} \\ \hline
	\endhead
	           30,930 & Jakt 1             &                   &   31,180          &   Jakt 14          &                   \\
	           30,940 & Jakt 25*           &                   &   31,190          &   Jakt 15          &                   \\
	           30,950 & Jakt 26*           &                   &   31,200          &   Jakt 16          &                   \\
	           30,960 & Jakt 27*           &                   &   31,210        &     Jakt 17        &                   \\
	           30,970 & Jakt 28*           &                   &   31,220          &   Jakt 18          &                   \\
	           31,030 & Jakt 29*           &                   &   31,230        &     Jakt 32*       &                   \\
	           31,040 & Jakt 2             &                   &   31,240         &    Jakt 33*        &                   \\
	           31,050 & Jakt 3             & Kanal 1 Eller D   &   31,250          &   Jakt 19          &  Kanal 4 eller E  \\
	           31,060 & Jakt 4             & Kanal 2 Eller A   &   31,260          &   Jakt 20         &   Kanal 5 eller C \\
	           31,070 & Jakt 5             &                   &   31,270          &   Jakt 21          &                   \\
	           31,080 & Jakt 6             &                   &   31,280          &   Jakt 34*         &                   \\
	           31,090 & Jakt 7             &                   &   31,290          &   Jakt 35*         &                   \\
	           31,100 & Jakt 8             &                   &   31,300          &   Jakt 36*         &                   \\
	           31,110 & Jakt 9             &                   &   31,310          &   Jakt 37*         &                   \\
	           31,120 & Jakt 10            &                   &   31,320          &   Jakt 22          & Kanal 6 eller F   \\
	           31,130 & Jakt 30*           &                   &   31,330          &   Jakt 23          &                   \\
	           31,140 & Jakt 11            &                   &   31,340          &   Jakt 38*         &                   \\
	           31,150 & Jakt 12            &                   &   31,350          &   Jakt 39*        &                   \\
	           31,160 & Jakt 13            & Kanal 3 Eller B   &   31,360          &   Jakt 40*         &                   \\
	           31,170 & Jakt 31*           &                   &   31,370          &                    &
\end{longtable}

\subsubsection{Åkeribandet 69 MHz öppet för PMR}

Sedan några år tillbaka finns nu ett nytt band som kan användas för privat
landmobil radio (PMR). Bandet kallas allmänt för 69\,MHz-bandet och har blivit
mycket populärt på sina ställen.

Anledningen är bland annat en stor tillgång på FM-radio för bandet från gamla
åkeriradio som säljs för billiga pengar på diverse begagnatsajter och som
därmed gör det enkelt att komma igång.

Antennstorlekarna är moderata och det är ett ypperligt band för mobilradio där
våglängden är ungefär den dubbla mot 2-metersbandet och fungerar bra i många
sammanhang.

Nackdelen som den delar med 27\,MHz är att många antenner för fordon är
förkortade vilket minskar verkningsgraden på dessa en del men trots detta
fungerar det bra. Antennerna är dock betydligt mindre skrymmande än de för
27\,MHz.

På bandet kör man FM uteslutande och det rekommenderas att man skaffar en
radio med signalstyrkemätare då man på FM inte kan höra lika väl om man är
störd, däremot syns det ju på S-metern om man har störningar. Bandet lider
något av störningar i urbana miljöer men på landsbygden brukar det vara tyst
och fint.

Användningen av bandet regleras i PTS föreskrift Undantag från Tillståndsplikt
och innebär att man får använda max 25\,W ERP (dvs för en dipolantenn), max
10\% sändningscykel (dvs 6 min/timme), en kanalbredd om 25\,kHz och det finns
8 stycken kanaler upplåtna för landmobil radio. I strikt mening är inte
kommunikation bas-bas egentligen tillåten eftersom det är landmobil trafik som
avses i PTS bestämmelser. Kanal 1 får enbart användas för mobil-mobil trafik
inom Västra Götaland och Hallands län.

\begin{table}[ht]
  \centering
\begin{tabular}{rrl}
  Kanal & Frekvens & Noteringar                         \\ \hline
  1     & 69,0125  & End. mobil i V. Götaland o Halland \\
  2     & 69,0375  &                                    \\
  3     & 69,0625  &                                    \\
  4     & 69,0875  &                                    \\
  5     & 69,1125  &                                    \\
  6     & 69,1375  &                                    \\
  7     & 69,1625  &                                    \\
  8     & 69,1875  & Anv. som anropskanal               \\
\end{tabular}
\caption{Frekvenser 69 MHz}
\end{table}

\subsubsection{Jakt och jordbruksfrekvenser 155 MHz}

Observera att kanalnumren som är traditionella och frekvenserna inte
kommer helt i ordning. Fyra kanaler är markerade med $^R$ och har
särskilda restriktioner på svenskt innanvatten och territorialvatten.

\begin{table}[H]
\centering
\begin{tabular}{rlrl}
	\textbf{Frekvens} & \textbf{Benämning} & \textbf{Effekt} & \textbf{Användningsområde}           \\ \hline
	          155,400 & Jakt K6            &             5 W & Jakt, Jordbruk, Skogsbruk$^R$        \\
	          155,425 & Jakt K1            &             5 W & Jakt, Jordbruk$^R$                   \\
	          155,450 & Jakt K7            &             5 W & Jakt, Jordbruk, Skogsbruk$^R$        \\
	          155,475 & Jakt K2            &             5 W & Jakt, Jordbruk$^R$                   \\
	          155,500 & Jakt K3 VHF-M L1   &             5 W & Jakt, Jordbruk, Skogsbruk, Marin$^M$ \\
	          155,525 & Jakt K4 VHF-M L2   &             5 W & Jakt, Jordbruk, Skogsbruk Marin$^M$  \\
	          156,000 & Jakt K5            &             5 W & Jakt, PMR, Friluftskanal$^P$
\end{tabular}
\caption{Jakt- och jordbruksfrekvenser 155\,MHz}
\end{table}

\footnotesize
\begin{itemize}
	\item[$^M$] Delas med marina VHF-bandet, kanalerna L1 och L2 för fritidsbåtar.
	\item[$^P$] PMR-kanal som kan användas till allmän privatradio.
	\item[$^R$] Dessa kanaler \textit{får ej användas} på svenskt
          territorialvatten eller svenskt inre vatten. Se
          \href{https://pts.se/globalassets/startpage/dokument/legala-dokument/foreskrifter/radio/beslutade_ptsfs-2018-3-undantagsforeskrifter.pdf}{PTSFS2018:3}
          för mer information.
\end{itemize}
\normalsize

\subsubsection{Öppna PMR-bandet på 446 MHz}

I nya författningssamlingen står det uttryckligen att
repeateranvändning är förbjuden. De exakta kanalerna har också inte
heller bestämts utan bandet är upplåtet
446,0--446,2\,MHz. Traditionellt används nedanstående kanaler. Max
effekt är 500\,mW och antennen får ej vara av löstagbar
sort. Utrustningen skall vara godkänd för ändamålet.

Sedan sist har ytterligare spektrum tillförts och bandet har nu 16
kanaler. Det medges också digital PMR på alla frekvenserna men
rekommendationen är att använda K1--K8 för analogt och K9--K16 för
digitalt eftersom äldre apparater inte kan gå på de nya kanalerna
medan alla digitala kan det.

Vi vissa numreringar numreras de digitala kanalerna med start på
kanalnummer 1 på K9. I listan står de som D1--D8 där D står för
digitalt.

Endast smalbandig modulation med FM-deviation max 2.5 kHz skall
användas för att inte störa närliggande kanaler. Kanalrastret är
12,5\,kHz så modulationen bör rymmas inom den bandbredden.

\begin{table}[H]
\centering
\begin{tabular}{rll|rll}
	\textbf{Frekvens} & \textbf{Benämning} & \textbf{Rek. Anv.}&
	\textbf{Frekvens} & \textbf{Benämning} & \textbf{Rek. Anv.}      \\ \hline
	446,00625 & PMR446 K1          & PMR                                   &          446,10625 & PMR446 K9\ \,\,/D1       & DPMR \\
	446,01875 & PMR446 K2          & PMR                                   &          446,11875 & PMR446 K10/D2      & DPMR \\
	446,03125 & PMR446 K3          & PMR                                   &          446,13125 & PMR446 K11/D3      & DPMR \\
	446,04375 & PMR446 K4          & PMR                                   &          446,14375 & PMR446 K12/D4      & DPMR \\
	446,05625 & PMR446 K5          & PMR                                   &          446,15625 & PMR446 K13/D5      & DPMR \\
	446,06875 & PMR446 K6          & PMR                                   &          446,16875 & PMR446 K14/D6      & DPMR \\
	446,08125 & PMR446 K7          & PMR                                   &          446,18125 & PMR446 K15/D7      & DPMR \\
	446,09375 & PMR446 K8          & PMR                                   &          446,19375 & PMR446 K16/D8      & DPMR
\end{tabular}
\caption{PMR-frekvenser}
\label{tab:pmr-frekvenser}
\end{table}

\subsubsection{Kortdistansradio (KDR, SRBR)}

Kallas även SRBR för Short Range Business Radio.  Den traditionella
frekvenslistan ser ut som följer. En ny variant med frekvenser för
12,5\,kHz samt 6,25\,kHz kanaler finns också ute nu och kan ses i
tabell \ref{tab:SRBR-frekvenser}.

\begin{table}[h]
	\centering
\begin{tabular}{rlrl}
\textbf{Frekvens} & \textbf{Benämning} & \textbf{Effekt} & \textbf{Användningsområde} \\ \hline
444,600 & SRBR K1            & 2 W             & Short range business radio \\
444,625 & SRBR K2            & 2 W             & Short range business radio \\
444,800 & SRBR K3            & 2 W             & Short range business radio \\
444,825 & SRBR K4            & 2 W             & Short range business radio \\
444,850 & SRBR K5            & 2 W             & Short range business radio \\
444,875 & SRBR K6            & 2 W             & Short range business radio \\
444,925 & SRBR K7            & 2 W             & Short range business radio \\
444,975 & SRBR K8            & 2 W             & Short range business radio
\end{tabular}
\caption{Frekvenser för SRBR}
\end{table}

SRBR är ett ej tillståndspliktigt frekvenssegment som används för
yrkesmässig radiotrafik.

Rekommendationen är att man skall använda CTCSS eller motsvarande för
att undvika störa och bli störd av andra stationer som delar
frekvenserna.

Från PTSFS2018:3 så har bandet fått nya bärvågsfrekvenser och det har
blivit öppet för att köra med 25, 12,5 eller 6,25\,kHz
Kanalraster. Denna frekvenstabell blir lite mer komplicerad.

% Please add the following required packages to your document preamble:
% \usepackage{multirow}
\begin{table}[h]
	\centering
	\begin{tabular}{|l|l|l|l|l|l|}
		\hline
		\textbf{25 kHz}                                & \textbf{12,5 kHz}                               & \textbf{6,25 kHz}               & \textbf{25 kHz}                               & \textbf{12,5 kHz}                               & \textbf{6,25 kHz}               \\ \hline
		\multicolumn{1}{|c|}{\multirow{4}{*}{444,600}} & \multicolumn{1}{c|}{\multirow{2}{*}{444,59375}} & \multicolumn{1}{c|}{444,590625} & \multicolumn{1}{l|}{\multirow{4}{*}{444,850}} & \multicolumn{1}{l|}{\multirow{2}{*}{444,84375}} & \multicolumn{1}{l|}{444,840625} \\ \cline{3-3}\cline{6-6}
		\multicolumn{1}{|c|}{}                         & \multicolumn{1}{c|}{}                           & \multicolumn{1}{c|}{444,596875} & \multicolumn{1}{l|}{}                         & \multicolumn{1}{l|}{}                           & \multicolumn{1}{l|}{444,846875} \\ \cline{2-3}\cline{5-6}
		\multicolumn{1}{|c|}{}                         & \multicolumn{1}{c|}{\multirow{2}{*}{444,60625}} & \multicolumn{1}{c|}{444,603125} & \multicolumn{1}{l|}{}                         & \multicolumn{1}{l|}{\multirow{2}{*}{444,85625}} & \multicolumn{1}{l|}{444,853125} \\ \cline{3-3}\cline{6-6}
		\multicolumn{1}{|c|}{}                         & \multicolumn{1}{c|}{}                           & \multicolumn{1}{c|}{444,609375} & \multicolumn{1}{l|}{}                         & \multicolumn{1}{l|}{}                           & \multicolumn{1}{l|}{444,859375} \\ \hline
		\multirow{4}{*}{444,650}                       & \multirow{2}{*}{444,64375}                      & 444,640625                      & \multicolumn{1}{l|}{\multirow{4}{*}{444,875}} & \multicolumn{1}{l|}{\multirow{2}{*}{444,86875}} & \multicolumn{1}{l|}{444,865625} \\ \cline{3-3}\cline{6-6}
		                                               &                                                 & 444,646875                      & \multicolumn{1}{l|}{}                         & \multicolumn{1}{l|}{}                           & \multicolumn{1}{l|}{444,871875} \\ \cline{2-3}\cline{5-6}
		                                               & \multirow{2}{*}{444,65625}                      & 444,653125                      & \multicolumn{1}{l|}{}                         & \multicolumn{1}{l|}{\multirow{2}{*}{444,88125}} & \multicolumn{1}{l|}{444,878125} \\ \cline{3-3}\cline{6-6}
		                                               &                                                 & 444,659375                      & \multicolumn{1}{l|}{}                         & \multicolumn{1}{l|}{}                           & \multicolumn{1}{l|}{444,884375} \\ \hline
		\multirow{4}{*}{Saknas}                        & \multirow{2}{*}{444,66875}                      & 444,665625                      & \multicolumn{1}{l|}{\multirow{4}{*}{444,925}} & \multicolumn{1}{l|}{\multirow{2}{*}{444,91875}} & \multicolumn{1}{l|}{444,915625} \\ \cline{3-3}\cline{6-6}
		                                               &                                                 & 444,671875                      & \multicolumn{1}{l|}{}                         & \multicolumn{1}{l|}{}                           & \multicolumn{1}{l|}{444,921875} \\ \cline{2-3}\cline{5-6}
		                                               & \multirow{2}{*}{444,68125}                      & 444,678125                      & \multicolumn{1}{l|}{}                         & \multicolumn{1}{l|}{\multirow{2}{*}{444,93125}} & \multicolumn{1}{l|}{444,928125} \\ \cline{3-3}\cline{6-6}
		                                               &                                                 & 444,684375                      & \multicolumn{1}{l|}{}                         & \multicolumn{1}{l|}{}                           & \multicolumn{1}{l|}{444,934375} \\ \hline
		\multirow{4}{*}{444,800}                       & \multirow{2}{*}{444,79375}                      & 444,790625                      & \multicolumn{1}{l|}{\multirow{4}{*}{444,975}} & \multicolumn{1}{l|}{\multirow{2}{*}{444,91875}} & \multicolumn{1}{l|}{444,915625} \\ \cline{3-3}\cline{6-6}
		                                               &                                                 & 444,796875                      & \multicolumn{1}{l|}{}                         & \multicolumn{1}{l|}{}                           & \multicolumn{1}{l|}{444,921875} \\ \cline{2-3}\cline{5-6}
		                                               & \multirow{2}{*}{444,80625}                      & 444,803125                      & \multicolumn{1}{l|}{}                         & \multicolumn{1}{l|}{\multirow{2}{*}{444,93125}} & \multicolumn{1}{l|}{444,928125} \\ \cline{3-3}\cline{6-6}
		                                               &                                                 & 444,809375                      & \multicolumn{1}{l|}{}                         & \multicolumn{1}{l|}{}                           & \multicolumn{1}{l|}{444,934375} \\ \hline
		\multirow{4}{*}{444,825}                       & \multirow{2}{*}{444,81875}                      & 444,815625                      & \multicolumn{3}{l}{\multirow{4}{*}{}}                                                                                             \\ \cline{3-3}
		                                               &                                                 & 444,821875                      & \multicolumn{3}{l}{}                                                                                                              \\ \cline{2-3}
		                                               & \multirow{2}{*}{444,83125}                      & 444,828125                      & \multicolumn{3}{l}{}                                                                                                              \\ \cline{3-3}
		                                               &                                                 & 444,834375                      & \multicolumn{3}{l}{}                                                                                                              \\ \cline{1-3}
	\end{tabular}
\caption{Nya frekvensindelningen på kortdistansradiobandet}
\label{tab:SRBR-frekvenser}
\end{table}

\clearpage

\subsection{Maritima VHF-frekvenser}

Marinbandet på VHF består både av duplex- och
simplexkanaler. Simplexkanalerna används skepp-till-skepp och även
ibland mot kustradio. Duplexfrekvenserna används t.ex. vid
telefonsamtal som sätts upp av kuststation till skepp eller
liknande. På dessa arbetskanaler sänder man även ut sjörapporter,
navigationsvarningar och annan information t.ex. säkerhetsvarningar
som är viktiga för sjöfarten.

\subsubsection{Kanalnummer och frekvens maritima kanaler}

\begin{table}[H]
\centering
\begin{tabular}{rrr|rrr}
\textbf{Kanal} & \textbf{Skepp} & \textbf{Kust} &
\textbf{Kanal} & \textbf{Skepp} & \textbf{Kust} \\ \hline
01 & 156,050 & 160,650 & 60 & 156,025 & 160,625 \\
02 & 156,100 & 160,700 & 61 & 156,075 & 160,675 \\
03 & 156,150 & 160,750 & 62 & 156,125 & 160,725 \\
04 & 156,200 & 160,800 & 63 & 156,175 & 160,775 \\
05 & 156,250 & 160,850 & 64 & 156,225 & 160,825 \\
06 & 156,300 &         & 65 & 156,275 & 160,875 \\
07 & 156,350 & 160,950 & 66 & 156,325 & 160,925 \\
08 & 156,400 &         & 67 & 156,375 & \\
09 & 156,450 &         & 68 & 156,425 & \\
10 & 156,500 &         & 69 & 156,475 & \\
11 & 156,550 &         & 70 & 156,525 & DSC \\
12 & 156,600 &         & 71 & 156,575 & \\
13 & 156,650 &         & 72 & 156,625 & \\
14 & 156,700 &         & 73 & 156,675 & \\
15 & 156,750 &         & 74 & 156,725 & \\
16 & 156,800 & Anrop/Nöd   & 75 & 156,775 & \\
17 & 156,850 &         & 76 & 156,825 & \\
18 & 156,900 & 161,500 & 77 & 156,875 & \\
19 & 156,950 & 161,550 & 78 & 156,925 & 161,525 \\
20 & 157,000 & 161,600 & 79 & 156,975 & 161,575 \\
21 & 157,050 & 161,650 & 80 & 157,025 & 161,625 \\
22 & 157,100 & 161,700 & 81 & 157,075 & 161,675 \\
23 & 157,150 & 161,750 & 82 & 157,125 & 161,725 \\
24 & 157,200 & 161,800 & 83 & 157,175 & 161,775 \\
25 & 157,250 & 161,850 & 84 & 157,225 & 161,825 \\
26 & 157,300 & 161,950 & 85 & 157,325 & 161,925 \\
27 & 157,350 & 161,950 & 86 & 157,325 & 161,925 \\
28 & 157,400 & 162,000 & 87 & 157,375 & \\
   &         &         & 88 & 157,425 & \\
   &         &         &    &         & \\
L1 & 155,500 & Leisure        & F1 & 155,625 &Fishing \\
L2 & 155,525 & Leisure        & F2 & 155,775 &Fishing \\
   &         &         & F3 & 155,825 & Fishing\\
\end{tabular}
\caption{Marin VHF, frekvenslista}
\end{table}

I tabellen listas de kanaler som gäller i svenska farvatten. Andra
länder kan ha andra kanaler eller för olika ändamål. Det krävs en
särskild licens från PTS för att få nyttja dessa frekvenser och
radiooperatören skall ha ett SRC-certifikat (Short Range
Communication).

Anropskanal och nödkanal är kanal 16.

Vid duplextrafik är skiftet -4,6\,MHz.

I tabellen är kanaler som saknar kustfrekvens alltså
simplexkanaler. DSC står för ''Digital Selective Call'' ett sätt att
digitalt anropa skepp eller kuststationer, kanaler vikta för DSC får
inte användas för vanliga samtal.

Kanal 16 är anropsfrekvens om man inte vet motstationen passar en
annan kanal. Den är också nödfrekvens eftersom den passas av de
flesta.

Kanalerna L1--L2 är frekvenser avsedda för fritidsbåtar (Leisure) och
frekvenserna F1--F3 osv är avsedda för yrkesfiske. L1 och L2 delas med
kanal 3 och 4 på jaktradion vilket kan vara bra att känna till.

\subsubsection{Transponderkanaler}
\begin{longtable}{rrl}
	\textbf{Kanal} & \textbf{Skepp} & \textbf{Not} \\ \hline
	   \endhead
AIS1 & 161,975 & Digital trafik, transponder\\
AIS2 & 162,025 & Digital trafik, transponder\\
\end{longtable}

\subsubsection{Stockholm radio}

Radiohorisonten är beräknad i nautiska mil, skeppet lägger till sin egen radiohorisont för att bestämma om det går att nå kuststationen eller ej.

\textbf{Ostkusten}

\begin{longtable}{lrr|lrr}
\textbf{Kuststation} & \textbf{Kanal} & \textbf{Horisont} & \textbf{Kuststation} & \textbf{Kanal}& \textbf{Horisont}\\
\hline
\endhead
Kalix          & 25 & 39 & Luleå         & 24 & 26 \\
Skellefteå     & 23 & 44 & Umeå          & 26 & 54 \\
Örnsköldsvik   & 28 & 42 & Mjällom       & 64 & 43 \\
Kramfors       & 84 & 43 & Härnösand     & 23 & 36 \\
Sundsvall      & 24 & 36 & Hudiksvall    & 25 & 54 \\
Gävle          & 23 & 37 & Östhammar     & 24 & 44 \\
Väddö          & 78 & 32 & Nacka         & 26, 23* & 43 \\
Sv. högarna    & 84 & 15 & Södertälje    & 66 & 30 \\
Torö           & 24 & 26 & Gotska sandön & 65 & 22 \\
Norrköping     & 64 & 43 & Västervik     & 23 & 45 \\
Fårö           & 28 & 25 & Visby         & 25 & 23 \\
Hoburgen       & 24 & 25 & Kalmar        & 26 & 40 \\
Ölands s. udde & 78 & 23 & Karlskrona    & 81 & 24 \\
Karlshamn      & 25 & 48 & Kivik         & 21 & 39\\
\end{longtable}
*) Sänder ej väder, varningar eller andra listor

\clearpage
\textbf{Västkusten}

\begin{longtable}{lrr|lrr}
\textbf{Kuststation} & \textbf{Kanal} & \textbf{Horisont} &
\textbf{Kuststation} & \textbf{Kanal} & \textbf{Horisont} \\
\hline
\endhead

Strömstad   & 22 & 25 & Grebbestad & 26 & 25 \\
Kungshamn   & 23 & 23 & Uddevalla  & 84 & 47 \\
Tjörn       & 81 & 26 & Göteborg   & 24 & 43 \\
Grimeton    & 22 & 35 & Halmstad   & 62 & 52 \\
Helsingborg & 24 & 28 & Malmö      & 27 & 25 \\
\end{longtable}

\textbf{Insjöarna}

\begin{longtable}{lrr|lrr}
\textbf{Kuststation} & \textbf{Kanal} & \textbf{Horisont} &
\textbf{Kuststation} & \textbf{Kanal} & \textbf{Horisont} \\
\hline
\endhead

Västerås  & 25 & 40 & Trollhättan & 25 & 32 \\
Bäckefors & 78 & 50 & Kinnekulle  & 01 & 43 \\
Karlstad  & 65 & 36 & Jönköping   & 23 & 49 \\
Motala    & 26 & 47 &             &    &    \\
\end{longtable}

\subsection{Frekvenser amatörradio VHF--UHF}

I denna skrift försöker vi omfatta de viktigaste VHF och UHF-banden
för amatörradio vilket inkluderar 6\,m-bandet, 2\,m-bandet,
70\,cm-bandet och 23\,cm-bandet.

\subsubsection{Kanalnumrering VHF/UHF}

Denna typ av kanalnumrering är överenskommen inom IARU region 1 för
6\,m, 2\,m och 70\,cm banden på
amatörradiofrekvenser. Kanalnumreringen består av ett prefix som anger
vilket band och här används F--6\,m, V--2\,m, U--70\,cm. Därefter
används 2 siffror på 6m och 2m banden och tre siffror på 70cm bandet
för att ange kanal.

Repeaterfrekvenser anges med tillägget R före kanalnumret och innebär
då normalt duplex med det skift som normalt används för bandet. Vid
repeatrar är det repeaterns utfrekvens som anges, dvs den som
mobilstationen lyssnar på. Exempel: RV48.

\begin{tabular}{crrlll}
	\textbf{Band} & \textbf{Startfrekvens} & \textbf{Kanalraster} & \textbf{Duplex} & \textbf{Första kanal} & \textbf{Beräknas}    \\ \hline
	    6\,m      &            51.000\,MHz &            10.0\,kHz & -100\,kHz       & F00                   & $f=51+k\cdot0.01$    \\
	              &                        &                      &                 &                       & $k=(f-51)/0,01$      \\ \hline
	    2\,m      &           145.000\,MHz &            12.5\,kHz & -600\,kHz       & V00                   & $f=145+k\cdot0.0125$ \\
	              &                        &                      &                 &                       & $k=(f-145)/0,0125$   \\ \hline
	   70\,cm     &           430.000\,MHz &            12.5\,kHz & -2000\,kHz      & U000                  & $f=430+k\cdot0.0125$ \\
	              &                        &                      &                 &                       & $k=(f-430)/0,0125$   \\ \hline
\end{tabular}

Eftersom amatörradiobanden ser lite olika ut i olika länder förekommer
det kanaler i numreringen som inte är tillåtna på vissa ställen. Det
är därför viktig att kontrollera att man fortfarande följer
bandplanerna i den region man är.

\begin{itemize}
\item I 6\,m bandet finns inga FM-kanaler definierade under 51\,MHz.
\item För 2\,m-bandet är FM-kanaler endast definierade från 145\,MHz och uppåt.
\item I 70\,cm-bandet är inga kanaler definierade i intervallet
  432.000--433.000\,MHz. Observera att startfrekvensen är utanför
  70\,cm bandplanen i IARU region 1.
\end{itemize}

\textit{OBS!\\ Information om kanalnumreringen för 23\,cm-bandet tas
  tacksamt mot. Maila mig på anders@sikvall.se om du har korrekt
  information.}

\subsubsection{Införande av 12.5\,kHz kanalavstånd}

För ett antal år sedan beslutade man sig att gå mot ett smalare
kanalraster på VHF och UHF och införde härmed kanalavstånd på 12.5~kHz
i bandplanerna. Ustrustning med 25~kHz kanalraster är fortsatt
tillåten och detta är en rekommendation. Vid införandet av detta så
kom även ett nytt numreringsalternativ för kanalsystemen baserat på en
basfrekvens (som ibland på svenska band ligger utanför vårt band) och
därefter numrerade man i ordning för respektive 12.5~kHz steg och
10~kHz för kortvåg.

\begin{table}[h]
\centering
\begin{tabular}{rrrr}
Kod & Basfrekvens & Kanalavstånd & Repeaterskift \\
    & [MHz]       & [kHz]        & [kHz] \\ \hline
H & 29,500 & 10,0 & -100 \\
F & 51,000 & 10,0 & -600 \\
V & 145,000& 12,5 & -600 \\
U & 430,000& 12,5 & -2000 \\
M & 1240,000 & 25 & -6000 \\
\end{tabular}
\label{tab:kanalavstand}
\caption{Kanalavstånd och beteckning olika frekvensband}
\end{table}

\subsubsection{FM-kanaler 6m-bandet}

\begin{longtable}{rrl|rrl}
\textbf{Kanal} & \textbf{Tidigare} & \textbf{Anm}
&  \textbf{Kanal} & \textbf{Tidigare} & \textbf{Anm} \\ \hline
	51,500 &      F50 &       & 51,750 &      F75 &  \\
	51,510 &      F51 & Anrop & 51,760 &      F76 &  \\
	51,520 &      F52 &       & 51,770 &      F77 &  \\
	51,530 &      F53 &       & 51,780 &      F78 &  \\
	51,540 &      F54 &       & 51,790 &      F79 &  \\
	51,550 &      F55 &       & 51,800 &      F80 &  \\
	51,560 &      F56 &       & 51,810 &     RF81 &  \\
	51,570 &      F57 &       & 51,820 &     RF82 &  \\
	51,580 &      F58 &       & 51,830 &     RF83 &  \\
	51,590 &      F59 &       & 51,840 &     RF84 &  \\
	51,600 &      F60 &       & 51,850 &     RF85 &  \\
	51,610 &      F61 &       & 51,860 &     RF86 &  \\
	51,620 &      F62 &       & 51,870 &     RF87 &  \\
	51,630 &      F63 &       & 51,880 &     RF88 &  \\
	51,640 &      F64 &       & 51,890 &     RF89 &  \\
	51,650 &      F65 &       & 51,900 &     RF90 &  \\
	51,660 &      F66 &       & 51,910 &     RF91 &  \\
	51,670 &      F67 &       & 51,920 &     RF92 &  \\
	51,680 &      F68 &       & 51,930 &     RF93 &  \\
	51,690 &      F69 &       & 51,940 &     RF94 &  \\
	51,700 &      F70 &       & 51,950 &     RF95 &  \\
	51,710 &      F71 &       & 51,960 &     RF96 &  \\
	51,720 &      F72 &       & 51,970 &     RF97 &  \\
	51,730 &      F73 &       & 51,980 &     RF98 &  \\
	51,740 &      F74 &       & 51,990 &     RF99 &
\end{longtable}

\clearpage
\subsubsection{FM-kanaler 2m-bandet}

\begin{longtable}{rrl|rrl}

\textbf{Frekvens} & \textbf{Kanal} & \textbf{Anm} &
\textbf{Frekvens} & \textbf{Kanal} & \textbf{Anm} \\ \hline

145,2125 & V17 &              & 145,5000 & V40  & S20  FM Anrop \\
145,2250 & V18 & S9           & 145,5125 & V41  &               \\
145,2375 & V19 & INET GW      & 145,5250 & V42  & S21           \\
145,2500 & V20 & S10          & 145,5375 & V43  &               \\
145,2625 & V21 &              & 145,5500 & V44  & S22           \\
145,2750 & V22 & S11          & 145,5625 & V45  &               \\
145,2875 & V23 & INET GW      & 145,5750 & V46  & S23           \\
145,3000 & V24 & S12  RTTY    & 145,5875 & V47  &               \\
145,3125 & V25 &              & 145,6000 & RV48 & R0            \\
145,3250 & V26 & S13          & 145,6125 & RV49 & R0X           \\
145,3375 & V27 & INET GW      & 145,6250 & RV50 & R1            \\
145,3500 & V28 & S14          & 145,6375 & RV51 & R1X           \\
145,3625 & V29 &              & 145,6500 & RV52 & R2            \\
145,3750 & V30 & S15 DV Anrop & 145,6625 & RV53 & R2X           \\
145,3875 & V31 &              & 145,6750 & RV54 & R3            \\
145,4000 & V32 & S16          & 145,6875 & RV55 & R3X           \\
145,4125 & V33 &              & 145,7000 & RV56 & R4            \\
145,4250 & V34 & S17 Scout    & 145,7125 & RV57 & R4X           \\
145,4375 & V35 &              & 145,7250 & RV58 & R5            \\
145,4500 & V36 & S18          & 145,7375 & RV59 & R5X           \\
145,4625 & V37 &              & 145,7500 & RV60 & R6            \\
145,4750 & V38 & S19          & 145,7625 & RV61 & R6X           \\
145,4875 & V39 &              & 145,7750 & RV62 & R7            \\
         &     &              & 145,7875 & RV63 & R7X
\end{longtable}

X-kanalerna uppstod när man fick platsbrist och man övergick till en
12.5\,kHz kanaldelning för repeatrar. Först senare övergick man även
till samma kanaldelning på övriga FM-kanaler. De gamla
simplexkanalerna hade inte så stor spridning i Sverige men förekom
rikligt t.ex. i Tyskland med S20 som anropsfrekvens (eller
aktivitetscenter som det numera kallas).

\clearpage
\subsubsection{FM-kanaler 70cm-bandet}

\begin{longtable}{rrl|rrl}
\textbf{Frekvens} & \textbf{Kanal} & \textbf{Anm} &
\textbf{Frekvens} & \textbf{Kanal} & \textbf{Anm} \\ \hline

433,4000 & U272 & SSTV    & 433,7125 & U297 &      \\
433,4125 & U273 &         & 433,7250 & U298 &      \\
433,4250 & U274 &         & 433,7375 & U299 &      \\
433,4375 & U275 &         & 433,7500 & U300 &      \\
433,4500 & U276 & Digital & 433,7625 & U301 &      \\
433,4625 & U277 &         & 433,7750 & U302 &      \\
433,4750 & U278 &         & 433,7875 & U303 &      \\
433,4875 & U279 &         & 433,8000 & U304 & APRS \\
433,5000 & U280 & Anrop   & 433,8125 & U305 &      \\
433,5125 & U281 &         & 433,8250 & U306 &      \\
433,5250 & U282 &         & 433,8375 & U307 &      \\
433,5375 & U283 &         & 433,8500 & U308 &      \\
433,5500 & U284 &         & 433,8625 & U309 &      \\
433,5625 & U285 &         & 433,8750 & U310 &      \\
433,5750 & U286 &         & 433,8875 & U311 &      \\
433,5875 & U287 &         & 433,9000 & U312 &      \\
433,6000 & U288 & RTTY    & 433,9125 & U313 &      \\
433,6125 & U289 &         & 433,9250 & U314 &      \\
433,6250 & U290 &         & 433,9375 & U315 &      \\
433,6375 & U291 &         & 433,9500 & U316 &      \\
433,6500 & U292 &         & 433,9625 & U317 &      \\
433,6625 & U293 &         & 433,9750 & U318 &      \\
433,6750 & U294 &         & 433,9875 & U319 &      \\
433,6875 & U295 &         & 434,0000 & U320 &      \\
433,7000 & U296 & FAX     &          &      &      \\

\end{longtable}

\clearpage
\begin{longtable}{rrl|rrl}
\textbf{Frekvens} & \textbf{Kanal} & \textbf{Anm}
&  \textbf{Frekvens} & \textbf{Kanal} & \textbf{Anm} \\ \hline

434,6000 & RU368 & RU0  & 434,8000 & RU384 & RU8   \\
434,6125 & RU369 & RU0X & 434,8125 & RU385 & RU8X  \\
434,6250 & RU370 & RU1  & 434,8250 & RU386 & RU9   \\
434,6375 & RU371 & RU1X & 434,8375 & RU387 & RU9X  \\
434,6500 & RU372 & RU2  & 434,8500 & RU388 & RU10  \\
434,6625 & RU373 & RU2X & 434,8625 & RU389 & RU10X \\
434,6750 & RU374 & RU3  & 434,8750 & RU390 & RU11  \\
434,6875 & RU375 & RU3X & 434,8875 & RU391 & RU11X \\
434,7000 & RU376 & RU4  & 434,9000 & RU392 & RU12  \\
434,7125 & RU377 & RU4X & 434,9125 & RU393 & RU12X \\
434,7250 & RU378 & RU5  & 434,9250 & RU394 & RU13  \\
434,7375 & RU379 & RU5X & 434,9375 & RU395 & RU13X \\
434,7500 & RU380 & RU6  & 434,9500 & RU396 & RU14  \\
434,7625 & RU381 & RU6X & 434,9625 & RU397 & RU14X \\
434,7750 & RU382 & RU7  & 434,9750 & RU398 & RU15  \\
434,7875 & RU383 & RU7X & 434,9875 & RU399 & RU15X \\
         &       &      & 435,0000 & RU400 &       \\

\end{longtable}

RU0X osv är här en efterkonstruktion. Egentligen så användes sällan
``X-frekvenserna'' på 70cm eftersom man dels hade nästan dubbla
antalet frekvenser för repeatrar och sedan gammalt ville man
egentligen inte ha ett smalare kanalraster, i tidernas begynnelse
körde många amatörer 70cm genom frekvenstrippling från 2m. $144,000
\cdot 3 = 432,000$\,MHz och $144,025 \cdot 3 = 432,075$\,MHz varför
man till och med hade bredare kanalraster de-facto.


\subsection{Scouters frekvenser, JOTA}

Scouter finns ofta QRV under vissa helger, \textit{Jamboree On The
  Air, JOTA}, förekommer några gånger per år. Här är en
sammanställning av de standardfrekvenser scouter nyttjar om de inte
kör repeatrar eller leta upp motstationer själva. Scouter kan antingen
ha egna signaler, köra under tillfälliga signaler eller vara second
operator åt med någon klubbsignal.

\subsection{Nordiska scoutfrekvenser VHF}

\begin{center}
\begin{tabular}{lrr}
	\textbf{Mode} & \textbf{Frekvens} & \textbf{Kanal} \\ \hline
	FM            &      145.425  MHz &   V34 \\
	SSB           &      144.240  MHz &  \\
	CW            &      144.050  MHz &
\end{tabular}
\end{center}

Jotan hålls alltid den 3:e hela (lördag och söndag) helgen i oktober
varje år. Jotan startar officiellt vid invigningen på lördag förmiddag
och slutar natten till måndagen klockan 00:00. Många börjar redan på
fredagskvällen och avslutar på söndagseftermiddagen.

Sändningar under denna tid förekommer från ocertifierade scouter som
lånar klubbsignal, har en tillfällig signal utdelad, ibland lånar
enskilda sändaramatörer ut sina signaler.

Sändningarna skall dock alltid ske under direkt överinseende av en
radioamatör men var beredd på att det kommer vara en viss ovana och
ske en del misstag.  Strunta i det och ge scouterna en kul
radioupplevelse.

% Radioberäkningar för VHF och UHF

\subsection{Radioberäkningar för VHF och UHF}

\subsubsection{Beräkning av radiohorisonten}

Radiohorisonten är den sträcka som markvågen kan nå utan särskilda hjälpmedel och i frånvaro av andra effekter som särskilda kondisioner (tropo eller duktning) och liknande. Avståndet kan beräknas med hjälp av en enkel formel. Radiohorisonten gäller egentligen bara när inget annat är i vägen men kan ge en ledning till den längsta utbredning man kan förvänta sig med markvåg givet en viss höjd.

För skepp på havet stämmer radiohorisonten ganska väl så man hittar denna formel ofta i utbildningsmaterial för marin VHF men då med distansen i nautiska mil i stället för km. För att få detta byter man konstanten 3,57 till 2,2 i stället.

\begin{equation*}
	r = 3,57 \left(\sqrt{h_1}+\sqrt{h_2}\right)
\end{equation*}

Där $r$ är avståndet till radiohorisonten givet i kilometer, $h_1$ är den ena stationens antennhöjd över marken givet i meter och $h_2$ är den andra stationens antennhöjd över marken också givet i meter.

\subsubsection{Sträckdämpning}

Sträckdämpningen beror på flera olika faktorer, inte minst terrängen och det som finns mellan sändaren och mottagaren. I den fria rymden följer den en enkel geometrisk utbredning men närmare marken behöver man stoppa in en del kompensationsfaktorer.

\begin{equation*}
	PL_0 = 20 \cdot \log(f) + 20 \cdot \log(d) - 27,55
\end{equation*}

Där $PL_{0}$ är sträckdämpningen i decibel(dB) (Eng: Path Loss) mellan två sändare givet avståndet $d$ i meter och frekvensen $f$ i MHz. Om man anger $d$ i kilometer i stället adderar man 60 till konstanten och får då 32,45.

För sträckdämpning vid mark får man mäta eller skatta en utbredningsdämpning som en konstant $k$ som man använder för att modifiera formeln med och får då följande variant:


\begin{equation*}
	PL_m = 20 \cdot \log(f) + (20+k) \cdot \log(d) - 27,55
\end{equation*}

Där $PL_m$ är sträckdämpningen vid marken. Faktorn $k$ kan uppskattas enligt följande tabell:

\begin{table}[h]
	\begin{centering}
		\begin{tabular}{r|l}
			\textbf{k} & \textbf{Beskrivning} \\ \hline
			0 & Över öppen terräng med högre frekvenser och fri sikt\\
			5 & Lättare terräng, mindre kullar, gräs och få träd \\
			10 & Tuffare terräng med mer höjdvariation, klippblock, tätare skog \\
			15 & Urban miljö, större hus, höghus \\
			20 & Extremt urband miljö (tänk Manhattan)\\
		\end{tabular}
	\end{centering}
	\label{tab:frirum-faktor}
	\caption{Tabell över korrigeringsfaktor för frirumsutbredning vid marken}
\end{table}

I vanlig svensk terräng är det nog vanligast man hamnar i storleksordningen 5--10.



\scriptsize
\subsection{Repeatrar, länkar och fyrar VHF/UHF}
\subsubsection{Svenska fyrar VHF/UHF}
\begin{longtable}{llrlrrrlrll}
	Signal   & Placering           &   Frekvens & Loc    &    P & MASL & MAGL & Dir     &  Band & Mode   & Dist \\ \hline
	SKØCT/B  & Stockholm           &  5760.9030 & JO99JX &   80 &   60 &   30 & Omni    &   6cm & CW     & 0    \\
	SKØEN/B  & Väddö               & 10368.8470 & JO99JX & 1000 &   60 &   30 & Omni    &  23cm & CW     & 0    \\
	SKØEN/B  & Väddö               &  1296.8350 & JO99JX &    4 &   70 &   40 & Omni    &  23cm & CW     & 0    \\
	SKØCT/B  & Stockholm           & 10368.8400 & JO89XJ &  0.1 &   50 &   20 & Omni    &   3cm & CW     & 0    \\
	SK1UHF   & Klintehamn          &   432.4050 & JO97CJ &   30 &   65 &   60 & Omni    &  70cm & CW     & 1    \\
	SK1VHF   & Klintehamn          &   144.4470 & JO97CJ &   10 &   65 &   60 & Omni    &    2m & CW     & 1    \\
	SK1UHG   & Klintehamn          &  1296.9500 & JO97CJ &   30 &   65 &   60 & Omni    &  23cm & CW     & 1    \\
	SK1SHH   & Klintehamn          & 10368.8500 & JO97CJ &    3 &   52 &   52 & Omni    &   3cm & CW     & 1    \\
	SK2VHF   & Vindeln/Buberget    &   144.4570 & JP94TF &   80 &  300 &   10 & N+SV    &    2m & CW     & 2    \\
	SK2CP/B  & Kiruna/Esrange      &    50.0520 & KP07MU &   30 &  630 &      & Omni    &    6m & CW     & 2    \\
	SK2SHF   & Vännäs/Granl.b.     &  1296.9850 & JP93VU &   10 &  250 &   50 &         &  23cm & CW     & 2    \\
	SK2SHF   & Vännäs/Granl.b.     &  2320.9850 & JP93VU &   10 &  250 &   50 &         &  13cm & CW     & 2    \\
	SK2DR/B  & Råneå               &  1296.9370 & KP15EU &   14 &      &      & South   &  23cm & CW     & 2    \\
	SK2DR/B  & Råneå               & 10368.8200 & KP15EU &    4 &      &      & South   &   3cm & CW     & 2    \\
	SK3UHH   & Nordingrå/Rävsön    &  2320.9000 & JP92FW &      &  200 &    5 & 220°    &  13cm & CW     & 3    \\
	SK3UHF   & Nordingrå/Rävsön    &   432.4550 & JP92FW &   50 &  200 &    8 & Omni    &  70cm & CW     & 3    \\
	SK3UHG   & Nordingrå/Rävsön    &  1296.8550 & JP92FW &   30 &  200 &   10 & Omni    &  23cm & CW     & 3    \\
	SK3SIX   & Östersund           &    50.4680 & JP73HC &   15 &  480 &    7 & Omni    &    6m & CW     & 3    \\
	SK3VHF   & Östersund           &   144.4210 & JP73HC &   50 &  480 &    7 & 180°    &    2m & CW     & 3    \\
	SM3KDR   & Krokom/Aspås        &    28.2860 & JP73GI &    1 &  380 &    5 & E-W     &   10m & CW     & 3    \\
	SK4BX/B  & Garphyttan/Ånnaboda & 10368.9600 & JO79LI &      &  270 &   10 &         &   3cm & CW     & 4    \\
	SK4MPI   & Borlänge            &   144.4120 & JP70PI &  200 &  380 &   20 & NV+NO   &   2cm & PI4/CW & 4    \\
	SK4BX/B  & Garphyttan/Storst.  &   432.4600 & JO79LH &   50 &  270 &   10 & N E S W &  70cm & CW     & 4    \\
	SK4BX/B  & Garphyttan/Ånnab.   &  1296.9600 & JO79LI &      &  270 &   10 &         &  23cm & CW     & 4    \\
	SK6YH/B  & Göteborg            & 10368.8080 & JO57XQ & 1000 &  135 &   40 & 184°    &   3cm & CW     & 6    \\
	SK6MHI   & Hönö                &  1296.8000 & JO57TQ &   30 &   40 &   30 & Omni    &  23cm & CW     & 6    \\
	SK6MHI   & Göteborg            &  5760.8000 & JO57XQ &   10 &  135 &   40 & Omni    &   6cm & CW     & 6    \\
	SK6UHF   & Varberg/Veddige     &   432.4120 & JO67EH &   10 &  175 &   25 & Omni    &  70cm & CW     & 6    \\
	SK6SHG   & Tjörn Island        & 24048.8830 & JO57TX & 2x1W &  118 &    8 & N/S     & 1.5cm & CW     & 6    \\
	SK6MHI   & Göteborg            & 24048.8000 & JO57XQ &   10 &  135 &   40 & Omni    & 1.5cm & CW     & 6    \\
	SK6UHI   & Tjörn Island        &  1296.8050 & JO57TX &   30 &  128 &   18 & Omni    &  23cm & CW     & 6    \\
	SK6VHF   & Tjörn Island        &   144.4060 & JO57TX &   10 &  122 &   12 & Omni    &    2m & CW     & 6    \\
	SK6WW/B  & Karlsborg/Vaberget  & 10368.8350 & JO78FM &    7 &  240 &   20 & Omni    &   3cm & CW     & 6    \\
	SK6EI/B  & Skövde              &    50.4600 & JO68VJ &   10 &  300 &   30 & South   &    6m & CW     & 6    \\
	SM7DTE/B & Gärsnäs             &  5760.8410 & JO75DN &   40 &   86 &    8 & Omni    &   6cm & CW     & 7    \\
	SM7DTE/B & Gärsnäs             & 10368.8410 & JO75DN &   40 &   86 &    8 & Omni    &   3cm & CW     & 7    \\
	SM7DTE/B & Gärsnäs             & 24048.8430 & JO75DN &   70 &   86 &    8 & Omni    & 1.5cm & CW     & 7    \\
	SK7GH/B  & Värnamo             &    28.2980 & JO77BF &    5 &  230 &   10 & Omni    &   10m & CW     & 7    \\
	SK7VHF   & Sjöbo               &   144.4610 & JO65UQ &   10 &   25 &   25 & Omni    &    2m & CW     & 7    \\
	SK7GH/B  & Värnamo             &  1296.8250 & JO77AE &   10 &  230 &   10 & Omni    &  23cm & CW     & 7
\end{longtable}

\subsubsection{Repeatrar distrikt 0}
\begin{longtable}{llllrrl}
	Typ      & Modulation & Signal   & Ort             & Utfrekvens &   Duplex & Loc    \\ \hline
	Hotspot  & D-Star     & SKØAI-B  & Stockholm       &   433.4625 &  Simplex & JO89XG \\
	Hotspot  & D-Star     & SEØYOS-C & M/Y Erika       &   434.4500 & Duplex 0 & JO99AH \\
	Link     & FM         & SKØMM    & Sandhamn        &   434.3750 &  Simplex & JO99KG \\
	Link     & FM         & SKØMM/L  & Ingarö          &   145.2250 &  Simplex & JO99GG \\
	Link     & FM         & SMØUAO   & Kopparmora      &   434.4875 &  Simplex & JO99HI \\
	Link     & FM         & SKØRVF   & Hagsätra        &   434.4250 &  Simplex & JO99AG \\
	Repeater & FM         & SKØNN/R  & Haninge         &   434.7750 &   -2.000 & JO99BE \\
	Repeater & FM         & SKØCT/R  & Kista           &  1297.0250 &   -6.000 & JO89XJ \\
	Repeater & FM         & SLØZS/R  & Västberga       &   145.6000 &   -0.600 & JO89XH \\
	Repeater & FM         & SLØZS/R  & Västberga       &   434.9000 &   -2.000 & JO89XH \\
	Repeater & FM         & SKØPQ/R  & Kista           &   145.6750 &   -0.600 & JO89XJ \\
	Repeater & FM         & SMØOFV/R & Solna           &   145.7625 &   -0.600 & JO89XI \\
	Repeater & FM         & SKØZA/R  & Solna           &   434.8500 &   -2.000 & JO89XI \\
	Repeater & FM         & SKØRDZ   & Brottby         &   145.6500 &   -0.600 & JO99DN \\
	Repeater & FM         & SAØAZT/R & Brottby         &   434.8000 &   -2.000 & JO99BM \\
	Repeater & FM         & SM5DWC/R & Södertälje      &   434.8250 &   -2.000 & JO89TE \\
	Repeater & FM         & SMØMMO/R & Tullinge        &   145.6625 &   -0.600 & JO89XF \\
	Repeater & FM         & SKØCT/R  & Kista           &   434.6250 &   -2.000 & JO89XJ \\
	Repeater & FM         & SMØYIX/R & Söder           &   434.7250 &   -2.000 & JO99BH \\
	Repeater & FM         & SKØYZ/R  & Vallentuna      &   434.8625 &   -2.000 & JO99BM \\
	Repeater & FM         & SKØCT/R  & Kista           &   434.6625 &   -2.000 & JO89XJ \\
	Repeater & FM         & SKØQO/R  & Haninge         &   145.6875 &   -0.600 & JO99BE \\
	Repeater & FM         & SKØQO/R  & Haninge         &   434.7500 &   -2.000 & JO99BE \\
	Repeater & FM         & SKØRMT   & Täby            &   434.7375 &   -2.000 & JO99AK \\
	Repeater & DMR        & SKØRMT   & Täby            &   434.7375 &   -2.000 & JO99AK \\
	Repeater & C4FM       & SKØRMT   & Täby            &   434.7375 &   -2.000 & JO99AK \\
	Repeater & D-Star     & SKØRMT   & Täby            &   434.7375 &   -2.000 & JO99AK \\
	Repeater & DMR        & SKØRMT   & Täby            &   434.7375 &   -2.000 & JO99AK \\
	Repeater & C4FM       & SKØRMT   & Täby            &   434.7375 &   -2.000 & JO99AK \\
	Repeater & D-Star     & SKØRMT   & Täby            &   434.7375 &   -2.000 & JO99AK \\
	Repeater & DMR        & SKØRYG   & Kista           &   434.9500 &   -2.000 & JO89XJ \\
	Repeater & DMR        & SKØRYG   & Sthlm city      &   434.9625 &   -2.000 & JO99AI \\
	Repeater & DMR        & SMØWIU/R & Nynäshamn       &   434.6125 &   -2.000 & JO88XV \\
	Repeater & DMR        & SMØWIU/R & Botkyrka        &   434.8750 &   -2.000 & JO89WG \\
	Repeater & C4FM       & SKØNN    & Haninge         &   434.5375 &   -2.000 & JO99CF \\
	Repeater & DMR        & SKØSX    & Kista           &   434.9875 &   -2.000 & JO89XJ \\
	Repeater & DMR        & SKØRMQ   & Tyresö          &   434.5125 &   -2.000 & JO99CH \\
	Repeater & DMR        & SMØWIU-4 & Högdalen        &   145.5750 &   -0.600 & JO99AF \\
	Repeater & FM         & SKØMG/R  & Skarpnäck       &   145.7000 &   -0.600 & JO89TE \\
	Repeater & FM/DMR     & SKØRIX   & Sthlm city      &   145.6250 &   -0.600 & JO99AH \\
	Repeater & DMR        & SGØRPF   & Rimbo           &   434.7875 &   -2.000 & JO99BT \\
	Repeater & DMR        & SKØRYG   & Upplands Väsby  &   434.7625 &   -2.000 & JO89XM \\
	Repeater & FM/DMR     & SKØRPF   & Sigtuna         &   434.8875 &   -2.000 & JO89VP \\
	Repeater & C4FM       & SKØQO    & Bagarmossen     &   434.5750 &   -2.000 & JO99BG \\
	Repeater & DMR        & SKØNN/1  & Johanneshov     &   434.9250 &   -2.000 & JO99AH \\
	Repeater & DMR        & SKØVR    & Djurö           &   434.5875 &   -2.000 & JO99IH \\
	Repeater & DMR        & SMØWIU/R & Dalarö          &   434.8375 &   -2.000 & JO99ED \\
	Repeater & FM         & SKØRYG   & Stockholm Norr  &   145.7875 &   -0.600 & JO99DL \\
	Repeater & FM         & SKØRYG   & Upplands Väsby  &   434.6750 &   -2.000 & JO89XM \\
	Repeater & DMR        & SKØEN    & Älmsta          &   434.6000 &   -2.000 & JO99JX \\
	Repeater & FM         & SKØBJ/R  & Nynäshamn       &   145.7125 &   -0.600 & JO88XV \\
	Repeater & C4FM       & SKØMG    & Haninge/Gålö    &   434.6875 &   -2.000 & JO99CC \\
	Repeater & DMR        & SKØQO    & Haninge/Brandb. &   434.5625 &   -2.000 & JO99BE \\
	Repeater & FM/DMR     & SKØVR    & Värmdö          &   434.9750 &   -2.000 & JO99FH \\
	Repeater & FM/DMR     & SKØEN    & Älmsta          &   145.7375 &   -0.600 & JO99JX \\
	Repeater & FM/DMR     & SAØAZT   & Norrtälje       &   434.8125 &   -2.000 & JO99IS \\
	Repeater & FM         & SKØMM/R  & Ingarö          &   145.7750 &   -0.600 & JO99GG \\
	Repeater & DMR        & SKØMG    & Södertälje      &   434.7875 &   -2.000 & JO89TE \\
	Repeater & FM         & SKØBJ    & Nynäshamn       &   145.7375 &   -0.600 & JO88WT \\
	Repeater & FM         & SKØBJ/R  & Nynäshamn       &   434.7125 &   -2.000 & JO88XV \\
	Repeater & FM         & SKØBJ/R  & Nynäshamn       &   434.6500 &   -2.000 & JO89XF \\
	Repeater & FM         & SKØBJ    & Nynäshamn       &   434.9125 &   -2.000 & JO88WT \\
	Repeater & FM/DMR     & SAØAZT   & Vallentuna      &   434.5500 &   -2.000 & JO99EO \\
	Repeater & FM         & SKØBJ/R  & Huddinge        &   434.6000 &   -2.000 & JO89XF \\
	Repeater & DMR        & SMØWIU-2 & Södertälje      &   434.8750 &   -2.000 & JO89TE \\
	Repeater & FM         & SKØMT/R  & Vallentuna      &   434.7000 &   -2.000 & JO99BM \\
	Repeater & C4FM       & SKØMG/R  & Sthlm/Söderort  &   434.6375 &   -2.000 & JO99AH
\end{longtable}

\subsubsection{Repeatrar distrikt 1}

\begin{longtable}{llllrrlcl}
	Typ      & Modulation & Signal   & Ort   & Utfrekvens &  Duplex & Loc    &  \\ \hline
	Repeater & FM         & SL1ZXK/R & Slite &   434.6000 &  -2.000 & JO97JR &     &  \\
	Repeater & FM/C4FM    & SK1RGU   & Endre &   145.7750 &  -0.600 & JO97FO &     &  \\
	Repeater & FM/C4FM    & SK1BL/R  & Endre &   145.7750 & -600kHz & 1750   & QRV & JO97FO
\end{longtable}

\subsubsection{Repeatrar distrikt 2}

\begin{longtable}{llllrrlcl}
	Typ      & Modulation         & Signal    & Ort                     & Utfrekvens &  Duplex & Loc    &  &  \\ \hline
	Link     & FM                 & SM2YUW    & Kiruna                  &   434.4000 & Simplex & KP07DU &  &  \\
	Repeater & FM                 & SK2AU/R   & Arjeplog/Galtispouda    &   145.7000 &  -0.600 & JP86XC &  &  \\
	Repeater & FM                 & SK2AU/R   & Skellefteå              &   145.7000 &  -0.600 & KP04LS &  &  \\
	Repeater & FM                 & SK2RIU    & Vännäs/Granlundsberget  &   145.7250 &  -0.600 & JP93VU &  &  \\
	Repeater & FM                 & SK2RIU    & Vännäs/Granlundsberget  &   434.7250 &  -2.000 & JP93VU &  &  \\
	Repeater & FM                 & SK2RLF    & Tärnaby                 &   145.6250 &  -0.600 & JP75PR &  &  \\
	Repeater & FM                 & SK2RLJ    & Umeå/Rödberget          &   145.6500 &  -0.600 & KP03CU &  &  \\
	Repeater & FM                 & SK2RMD    & Sorsele                 &   145.6000 &  -0.600 & JP85SM &  &  \\
	Repeater & FM                 & SK2RMR    & Storuman                &   145.7250 &  -0.600 & JP85NC &  &  \\
	Repeater & FM                 & SK2RYI    & Vindeln/Åsträsk         &   145.6250 &  -0.600 & KP04DP &  &  \\
	Repeater & FM                 & SK2AU/R   & Jörn/Storklinta         &   145.7500 &  -0.600 & KP05BD &  &  \\
	Repeater & FM                 & SK2LY/R   & Lycksele                &   145.7750 &  -0.600 & JP94IO &  &  \\
	Repeater & FM                 & SM2KOT/R  & Kristineberg/Viterliden &   145.6750 &  -0.600 & JP95HB &  &  \\
	Repeater & FM                 & SK2RFR    & Kiruna                  &   145.6250 &  -0.600 & KP07DU &  &  \\
	Repeater & FM                 & SK2RFR    & Kiruna C                &   434.8250 &  -2.000 & KP07DU &  &  \\
	Repeater & FM                 & SK2DR/R   & Luleå                   &   145.6500 &  -0.600 & KP15CO &  &  \\
	Repeater & FM                 & SK2AZ/R   & Piteå                   &   145.6000 &  -0.600 & KP05PH &  &  \\
	Repeater & FM                 & SK2RWJ    & Älvsbyn                 &   145.6750 &  -0.600 & KP05LQ &  &  \\
	Repeater & FM                 & SK2HG/R   & Kalix/Raggdynan         &    51.9500 &  -0.600 & KP15KW &  &  \\
	Repeater & FM                 & SM2KXX    & Lycksele                &   434.7750 &  -1.600 & JP94HO &  &  \\
	Repeater & FM                 & SK2RMR    & Storuman                &   434.7500 &  -2.000 & JP85NC &  &  \\
	Repeater & FM                 & SK2RME    & Piteå                   &   434.6000 &  -2.000 & KP05RH &  &  \\
	Repeater & DMR                & SK2RGJ    & Kiruna                  &   434.5125 &  -2.000 & KP07CT &  &  \\
	Repeater & DMR/D-Star         & SK2DR     & Luleå                   &   434.9000 &  -2.000 & KP15CO &  &  \\
	Repeater & DMR/D-Star         & SK2RJH    & Kalix/Raggdynan         &   434.7500 &  -2.000 & KP15KW &  &  \\
	Repeater & FM/DMR             & SK2HG/R3  & Seskarö                 &   145.6750 &  -0.600 & KP15UR &  &  \\
	Repeater & FM/DMR             & SK2HG/R5  & Kalix/Raggdynan         &   145.7250 &  -0.600 & KP15KW &  &  \\
	Repeater & FM/DMR             & SK2HG/RU5 & Kalix-Vattentorn        &   434.7250 &  -2.000 & KP15NU &  &  \\
	Repeater & DMR                & SK2AT     & Vännäs                  &   434.9750 &  -2.000 & JP93XX &  &  \\
	Repeater & FM                 & SK2CI     & Boden                   &   145.6250 &  -0.600 & KP05SS &  &  \\
	Repeater & DMR                & SK2AZ     & Piteå                   &   434.8500 &  -2.000 & KP05PH &  &  \\
	Repeater & DMR                & SK2CI     & Boden                   &   434.8000 &  -2.000 & KP05TT &  &  \\
	Repeater & DMR                & SK2HG-2   & Kalix                   &   434.9875 &  -2.000 & KP15OU &  &  \\
	Repeater & FM/DMR/D-Star/C4FM & SK2AU/R   & Skellefteå              &   145.5875 &  -0.600 & KP04LS &  &  \\
	Repeater & FM                 & SJ2W/R    & Skellefteå              &   434.6750 &  -2.000 & KP04LS &  &  \\
	Repeater & FM                 & SJ2W      & Burträsk                &   434.9500 &  -2.000 & KP04HM &  &
\end{longtable}

\subsubsection{Repeatrar distrikt 3}

\begin{longtable}{llllrrlcl}
	Typ      & Modulation      & Signal   & Ort                    & Utfrekvens &   Duplex & Loc    &  &  \\ \hline
	Hotspot  & D-Star          & SK3GA-B  & Hudiksvall             &   434.4750 & Duplex 0 & JP81NR &  &  \\
	Link     & FM              & SM3KDR   & Krokom/Aspås           &   434.9750 &  Simplex & JP73GI &  &  \\
	Repeater & FM              & SK3EK/R  & Sollefteå              &   434.6500 &   -1.600 & JP83DE &  &  \\
	Repeater & FM              & SK3MF/R  & Nordingrå/Rävsön       &   145.6250 &   -0.600 & JP92FW &  &  \\
	Repeater & FM              & SK3MF/R  & Nordingrå/Rävsön       &   434.8500 &   -2.000 & JP92FW &  &  \\
	Repeater & FM              & SK3RFG   & Sundsvall              &   145.7250 &   -0.600 & JP82RJ &  &  \\
	Repeater & FM              & SK3RIA   & Östersund              &   434.7500 &   -2.000 & JP73JE &  &  \\
	Repeater & FM              & SK3RIN   & Borgsjö                &   145.7000 &   -0.600 & JP72WN &  &  \\
	Repeater & FM              & SK3RKL   & Örnsköldsvik/Rutberget &   145.7750 &   -0.600 & JP93GJ &  &  \\
	Repeater & FM              & SK3RMG   & Bergsjö                &  1297.1000 &   -6.000 & JP81MX &  &  \\
	Repeater & FM              & SK3RMX   & Hoting/Kyrktåsjö       &   145.6000 &   -0.600 & JP74XF &  &  \\
	Repeater & FM              & SK3RYK   & Söderhamn              &   145.7500 &   -0.600 & JP81NH &  &  \\
	Repeater & FM              & SK3RYK   & Söderhamn              &   434.7500 &   -1.600 & JP81NH &  &  \\
	Repeater & FM              & SK3WH    & Högakustenbron         &  1297.2750 &   -6.000 & JP82XT &  &  \\
	Repeater & FM              & SK3LH/R  & Örnsköldsvik           &   434.8750 &   -2.000 & JP93IH &  &  \\
	Repeater & FM              & SK3RNJ   & Åre/Åreskutan          &   145.7250 &   -0.600 & JP63NK &  &  \\
	Repeater & FM              & SM3XRJ   & Kramfors               &   434.6000 &   -2.000 & JP82VW &  &  \\
	Repeater & D-Star          & SK3LH-B  & Örnsköldsvik/Malmön    &   434.5750 &   -2.000 & JP93LF &  &  \\
	Repeater & FM              & SL3ZB    & Härnösand              &   434.7250 &   -2.000 & JP82XP &  &  \\
	Repeater & FM              & SK3EK/R  & Sollefteå              &   145.6500 &   -0.600 & JP83PD &  &  \\
	Repeater & D-Star          & SK3RFG-C & Sundsvall/Klissberget  &   145.5875 &   -0.600 & JP82OJ &  &  \\
	Repeater & FM/C4FM         & SK3JR/R  & Östersund              &   145.7500 &   -0.600 & JP73JE &  &  \\
	Repeater & FM              & SK3GK/R  & Sandviken/Kungsberget  &   145.7000 &   -0.600 & JP80FS &  &  \\
	Repeater & FM              & SM3VAC/R & Nyland                 &   145.7500 &   -0.600 & JP83UA &  &  \\
	Repeater & FM              & SM3VAC/R & Nyland                 &   434.9500 &   -1.600 & JP83UA &  &  \\
	Repeater & FM              & SK3RQE   & Forsa/Storberget       &   434.6750 &   -2.000 & JP81KQ &  &  \\
	Repeater & FM              & SA3EJX/R & Forsa/Storberget       &   145.6750 &   -0.600 & JP81KQ &  &  \\
	Repeater & FM              & SK3GW    & Gävle                  &   434.8750 &   -2.000 & JP80NP &  &  \\
	Repeater & FM              & SK3GK    & Sandviken              &   434.8250 &   -2.000 & JP80FS &  &  \\
	Repeater & FM              & SK3RQC   & Vemdalen               &   145.6250 &   -0.600 & JP62WK &  &  \\
	Repeater & FM              & SM3LEI/R & Årsunda                &   434.6500 &   +1.600 & JP80IM &  &  \\
	Repeater & DMR             & SK3WH    & Örnsköldsvik           &   145.5750 &   -0.600 & JP93IH &  &  \\
	Repeater & DMR             & SK3GK    & Gävle                  &   434.7000 &   -2.000 & JP80NP &  &  \\
	Repeater & DMR/D-Star      & SK3RFG   & Sundsvall/Klissberget  &   434.8000 &   -2.000 & JP82OJ &  &  \\
	Repeater & DMR/D-Star/C4FM & SM3YFX   & Föllinge               &   434.5250 &   -2.000 & JP73HQ &  &  \\
	Repeater & FM              & SK3GA/R  & Hudiksvall             &   145.7750 &   -0.600 & JP81NR &  &  \\
	Repeater & FM/DMR          & SK3RHU   & Hudiksvall             &   145.7125 &   -0.600 & JP81NR &  &  \\
	Repeater & DMR             & SK3RHU   & Hudiksvall             &   434.5750 &   -2.000 & JP81NR &  &  \\
	Repeater & FM/C4FM         & SK3JR/R2 & Östersund/Brattåsen    &   145.7875 &   -0.600 & JP73HC &  &  \\
	Repeater & DMR/D-Star/C4FM & SG9NN    & Sundsvall              &   434.5375 &   -2.000 & JP82OJ &  &  \\
	Repeater & FM              & SK3RET   & Bollnäs/Arbrå          &   145.6500 &   -0.600 & JP81CL &  &  \\
	Repeater & DMR             & SK3JR    & Östersund/Brattåsen    &   434.5625 &   -2.000 & JP73HC &  &  \\
	Repeater & DMR             & SK3RFG   & Sundsvall/Nolby        &   434.9875 &   -2.000 & JP82QH &  &  \\
	Repeater & FM              & SK3YZ/R  & Forsa                  &   145.6125 &   -0.600 & JP81KQ &  &  \\
	Repeater & FM              & SK3PH/R  & Delsbo                 &    29.6900 &   -0.100 & JP81GT &  &  \\
	Repeater & FM              & SK3EK/R  & Sollefteå              &   434.9250 &   -2.000 & JP83DE &  &  \\
	Repeater & FM              & SK3RQE   &                        &   145.6000 &   -0.600 & JP81NV &  &  \\
	Repeater & FM              & SK3W     & Österfärnebo           &   434.8500 &   -2.000 & JP80JH &  &
\end{longtable}

\subsubsection{Repeatrar distrikt 4}

\begin{longtable}{llllrrlcl}
	Typ      & Modulation & Signal   & Ort                        & Utfrekvens &   Duplex & Loc    &  &  \\ \hline
	Hotspot  & D-Star     & SG4UOF-C & Glanshammar                &   145.3375 & Duplex 0 & JO79RI &  &  \\
	Hotspot  & D-Star     & SG4UZM-B & Borlänge                   &   434.5500 & Duplex 0 & JP70RM &  &  \\
	Hotspot  & DMR        & SG4AXV   & Ekshärad                   &   433.2000 &  Simplex & JP60RE &  &  \\
	Hotspot  & DMR/D-Star & SG4AXQ   & Sunne                      &   432.5000 & Duplex 0 & JO69NU &  &  \\
	Hotspot  & DMR        & SA4ATZ   & Malung                     &   144.8375 &  Simplex & JP60UQ &  &  \\
	Link     & FM         & SK4AV/R  & Filipstad/Klockarhöjden    &   145.2000 &  Simplex & JO79CR &  &  \\
	Link     & FM         &          & Nyhammar                   &   145.3250 &  Simplex & JP70LG &  &  \\
	Link     & FM         &          & Grängesberg                &   145.3500 &  Simplex & JP70MB &  &  \\
	Link     & FM         & SK4RJJ   & Torsby/Hovfjället          &   145.2875 &  Simplex & JO69LH &  &  \\
	Link     & FM         & SA4THA   & Älvdalen                   &   434.5000 &  Simplex & JP71AF &  &  \\
	Link     & FM         & SM4FBD   & Nybble                     &   145.3000 &  Simplex & JO79BC &  &  \\
	Link     & FM         & SK4EA-L  & Lindesberg                 &   145.3000 &  Simplex & JO79OO &  &  \\
	Link     & FM         & SM4MXN   & Orsa                       &   145.2750 &  Simplex & JP71HC &  &  \\
	Repeater & FM         & SK4DM/R  & Ludvika                    &   145.7250 &   -0.600 & JP70NC &  &  \\
	Repeater & FM         & SK4DM/R  & Ludvika                    &   434.7250 &   -1.600 & JP70NC &  &  \\
	Repeater & FM         & SK4RGO   & Orsa/Grönklitt             &   434.7500 &   -1.600 & JP71GF &  &  \\
	Repeater & FM         & SK4RPK   & Torsby/Valberget           &   434.6250 &   -2.000 & JP60LC &  &  \\
	Repeater & FM         & SK4RQF   & Årjäng                     &   145.7250 &   -0.600 & JO69BJ &  &  \\
	Repeater & FM         & SM4JDP   & Mora                       &   434.7000 &   -2.000 & JP71GA &  &  \\
	Repeater & D-Star     & SG4TYA   & Mora                       &   145.5750 &   -0.600 & JP71GE &  &  \\
	Repeater & FM         & SK4IL/R  & Grums                      &   434.7250 &   -2.000 & JO69NI &  &  \\
	Repeater & FM         & SK4WV    & Vansbro                    &   145.6500 &   -0.600 & JP70AM &  &  \\
	Repeater & FM         & SK4WV    & Vansbro                    &   434.6500 &   -1.600 & JP70AM &  &  \\
	Repeater & FM         & SK4TL/R  & Örebro/Suttarboda          &   145.7125 &   -0.600 & JO79KH &  &  \\
	Repeater & FM         & SK4RGO   & Orsa/Grönklitt             &   145.7500 &   -0.600 & JP71GF &  &  \\
	Repeater & D-Star     & SK4BW-B  & Borlänge                   &   434.9000 &   -2.000 & JP70RJ &  &  \\
	Repeater & FM/C4FM    & SK4RVN   & Borlänge                   &   434.8000 &   -2.000 & JP70RJ &  &  \\
	Repeater & FM         & SK4HV/R  & Hagfors/Värmullsåsen       &   145.6750 &   -0.600 & JP60VA &  &  \\
	Repeater & FM         & SK4EA/R  & Lindesberg                 &   145.6875 &   -0.600 & JO79NP &  &  \\
	Repeater & FM         & SK4RWQ   & Arvika/Valfjället          &   434.7750 &   -2.000 & JO69CT &  &  \\
	Repeater & FM         & SK4RJJ   & Sunne/Blåbärskullen        &   145.7750 &   -0.600 & JO69KU &  &  \\
	Repeater & FM         & SK4BX/R  & Garphyttan/Storstenshöjden &   145.6500 &   -0.600 & JO79LH &  &  \\
	Repeater & FM         & SK4RUV   & Leksand                    &   145.7750 &   -0.600 & JP70MQ &  &  \\
	Repeater & DMR        & SK4BW    & Borlänge                   &   434.8500 &   -2.000 & JP70RJ &  &  \\
	Repeater & DMR        & SK4WV    & Vansbro                    &   434.6625 &   -2.000 & JP70AM &  &  \\
	Repeater & FM         & SK4EA/R  & Kopparberg                 &   145.6000 &   -0.600 & JO79MW &  &  \\
	Repeater & DMR        & SA4BNA   & Arvika                     &   434.9750 &   -2.000 & JO69GN &  &  \\
	Repeater & FM/DMR     & SK4KR    & Karlskoga                  &   434.8000 &   -2.000 & JO79FH &  &  \\
	Repeater & DMR        & SK4RGL   & Falun                      &   434.6250 &   -2.000 & JP70UP &  &  \\
	Repeater & FM         & SK4RGL   & Falun                      &   145.6250 &   -0.600 & JP70UP &  &  \\
	Repeater & FM         & SK4TL/R  & Örebro/Suttarboda          &    51.9500 &   -0.600 & JO79KH &  &  \\
	Repeater & FM/DMR     & SK4RKD   & Karlskoga                  &   145.7500 &   -0.600 & JO79FJ &  &  \\
	Repeater & DMR        & SK4KO    & Nusnäs                     &   434.9250 &   -2.000 & JP70HW &  &  \\
	Repeater & FM/DMR     & SM4WIU-3 & Leksand                    &   434.6125 &   -2.000 & JP70MR &  &  \\
	Repeater & DMR        & SK4TL    & Örebro                     &   434.7250 &   -2.000 & JO79OG &  &  \\
	Repeater & D-Star     & SG4AXV   & Ekshärad                   &   145.6000 &   -0.600 & JP60RE &  &  \\
	Repeater & FM         & SK4KO    & Sälen/Lindvallen           &   145.6000 &   -0.600 & JP61OD &  &  \\
	Repeater & DMR        & SA4BHE-R & Smedjebacken               &   434.6375 &   -2.000 & JP70GD &  &
\end{longtable}

\subsubsection{Repeatrar distrikt 5}

\begin{longtable}{llllrrlcl}
	Typ      & Modulation & Signal   & Ort                    & Utfrekvens &   Duplex & Loc    &  &  \\ \hline
	Hotspot  & D-Star     & SC5SLU-C & Uppsala                &   145.3250 & Duplex 0 & JO89QW &  &  \\
	Hotspot  & D-Star     & SM5EZN-B & Uppsala                &   433.4875 & Duplex 0 & JO89QW &  &  \\
	Hotspot  & D-Star     & SG5TAH-C & Flen/Orrhammar         &   145.3375 & Duplex 0 & JO89GB &  &  \\
	Hotspot  & DMR        & SA5HAV   & Uppsala                &   434.3750 &  Simplex & JO89VW &  &  \\
	Link     & FM         & SM5RVH   & Nyköping               &   145.4750 &  Simplex & JO88LQ &  &  \\
	Link     & FM         & SM5RVH   & Nyköping               &    51.4700 &  Simplex & JO88LQ &  &  \\
	Link     & FM         & SM5RVH   & Nyköping               &    29.1700 &  Simplex & JO88LQ &  &  \\
	Link     & FM         & SM5RVH   & Nyköping               &  1297.5000 &  Simplex & JO88LQ &  &  \\
	Link     & FM         & SM5GXQ-L & Norrköping             &   145.2375 &  Simplex & JO88CO &  &  \\
	Link     & DMR        & SA5KBE   & Stigtomta              &   145.2875 &  Simplex & JO88JT &  &  \\
	Link     & FM         & SA5BJM   & Uppsala/Fjuckby        &   144.5750 &  Simplex & JO89TX &  &  \\
	Link     & FM         & SA5BJM   & Uppsala/Fjuckby        &   433.4500 &  Simplex & JO89TX &  &  \\
	Repeater & FM         & SK5AS/R  & Linköping              &   145.7250 &   -0.600 & JO78SJ &  &  \\
	Repeater & FM         & SK5BN/R  & Finspång               &   434.9250 &   -2.000 & JO78VR &  &  \\
	Repeater & FM/D-Star  & SK5RHQ   & Västerås               &   434.7000 &   -2.000 & JO89GO &  &  \\
	Repeater & FM/C4FM    & SK5RCQ   & Kisa                   &   145.7000 &   -0.600 & JO77TX &  &  \\
	Repeater & FM         & SK5LW/R  & Eskilstuna/Hällby      &   434.8500 &   -2.000 & JO89FJ &  &  \\
	Repeater & FM         & SA5BTT   & Trosa                  &   434.8875 &   -2.000 & JO88TV &  &  \\
	Repeater & FM         & SK5BN/R  & Norrköping/Kolmården   &   145.6000 &   -0.600 & JO88FQ &  &  \\
	Repeater & FM         & SK5BN/R  & Norrköping/Östra Eneby &   434.6000 &   -2.000 & JO88BO &  &  \\
	Repeater & FM         & SK5LF/R  & Linköping/Majelden     &   434.8250 &   -2.000 & JO78TJ &  &  \\
	Repeater & DMR        & SA5BJM   & Uppsala/Fjuckby        &   434.5125 &   -2.000 & JO89TX &  &  \\
	Repeater & FM         & SK5DB/R  & Uppsala                &   145.7500 &   -0.600 & JO89VU &  &  \\
	Repeater & FM         & SK5DB/R  & Uppsala                &   434.7500 &   -2.000 & JO89VU &  &  \\
	Repeater & FM         & SK5RHQ   & Västerås               &   145.7750 &   -0.600 & JO89GO &  &  \\
	Repeater & FM         & SK5RHQ   & Västerås               &   434.7750 &   -2.000 & JO89GO &  &  \\
	Repeater & ATV        & SK5BN/R  & Norrköping/Kolmården   &  1282.0000 &  -30.000 & JO88FQ &  &  \\
	Repeater & FM         & SK5AS/R  & Linköping              &   145.7875 &   -0.600 & JO78SN &  &  \\
	Repeater & FM         & SM5RYI/R & Sala                   &   145.7125 &   -0.600 & JO89HW &  &  \\
	Repeater & DMR        & SK5RYG   & Linköping              &   434.5125 &   -2.000 & JO78SN &  &  \\
	Repeater & FM         & SK5RYG   & Linköping              &   145.6250 &   -0.600 & JO78SN &  &  \\
	Repeater & FM/DMR     & SL5ZYT/R & Norrköping             &   434.9500 &   -2.000 & JO88DQ &  &  \\
	Repeater & FM/DMR     & SG5BCG/R & Knivsta                &   434.5250 &   -2.000 & JO89VR &  &  \\
	Repeater & FM/DMR     & SM5DWC/R & Linköping              &   434.8750 &   -2.000 & JO78SM &  &  \\
	Repeater & FM         & SK5BB/R  & Arboga/Kolsva          &   434.8750 &   -2.000 & JP79WO &  &  \\
	Repeater & FM         & SK5BB/R  & Arboga/Kolsva          &   145.6750 &   -0.600 & JP79WO &  &  \\
	Repeater & D-Star     & SK5BN-C  & Norrköping             &   145.5750 &   -0.600 & JO88BR &  &  \\
	Repeater & FM/DMR     & SG5DV    & Uppsala                &   434.5875 &   -2.000 & JO89TU &  &  \\
	Repeater & FM         & SG5DV    & Uppsala                &   145.5875 &   -0.600 & JO89TU &  &  \\
	Repeater & DMR/D-Star & SK5LW/R  & Eskilstuna/Ärla        &   145.5875 &   -0.600 & JO89FJ &  &  \\
	Repeater & FM         & SK5LW/R  & Eskilstuna             &    51.8500 &   -0.600 & JO89FJ &  &  \\
	Repeater & FM         & SK5VM/R  & Eskilstuna             &   434.9750 &   -2.000 & JO89GI &  &  \\
	Repeater & FM         & SK5LW/R  & Eskilstuna/Slytan      &   145.6125 &   -0.600 & JO89HF &  &  \\
	Repeater & D-Star     & SK5UM-B  & Flen                   &   434.5500 &   -2.000 & JO89HB &  &  \\
	Repeater & FM         & SK5UM/R  & Flen/Öja               &   434.7500 &   -2.000 & JO89HB &  &  \\
	Repeater & DMR        & SK5UM/R  & Flen                   &   145.6375 &   -0.600 & JO89HB &  &  \\
	Repeater & FM         & SM5YMS   & Åtvidaberg             &   145.6625 &   -0.600 & JO78XE &  &  \\
	Repeater & FM         & SM5YMS/R & Linköping              &   434.8000 &   -2.000 & JO78SM &  &  \\
	Repeater & DMR        & SA5HAV/R & Uppsala/Rasbo          &   434.6375 &   -2.000 & JO89VW &  &  \\
	Repeater & DMR        & SL5ZO    & Finspång               &   434.8125 &   -2.000 & JO78VQ &  &  \\
	Repeater & DMR        & SA5UTR   & Nyköping               &   434.6375 &   -2.000 & JO88MS &  &  \\
	Repeater & FM/C4FM    & SA5OHR/R & Norrköping             &   434.6625 &   -2.000 & JO88BO &  &  \\
	Repeater & FM         & SK5RHT   & Linköping              &    51.9900 &   -0.600 & JO78SN &  &  \\
	Repeater & FM         & SK5UM/R  & Flen                   &   145.7625 &   -0.600 & JO89HB &  &  \\
	Repeater & FM         & SK5WR/R  & Motala                 &   145.7375 &   -0.600 & JO78NM &  &  \\
	Repeater & FM         & SK5RHT   & Linköping              &    29.6600 &   -0.100 & JO78XH &  &
\end{longtable}

\subsubsection{Repeatrar distrikt 6}

\begin{longtable}{llllrrlcl}
	Typ      & Modulation      & Signal   & Ort                   & Utfrekvens &   Duplex & Loc    &  &  \\ \hline
	Hotspot  & D-Star          & SK6GB-D  & Mölndal               &   433.7250 &  Simplex & JO67AQ &  &  \\
	Hotspot  & D-Star          & SK6GB-D  & Mölndal               &   144.8250 &  Simplex & JO67AQ &  &  \\
	Hotspot  & D-Star          & SK6MA-C  & Hjo                   &   145.2125 & Duplex 0 & JO78DH &  &  \\
	Hotspot  & D-Star          & SG6JWU-B & Halmstad              &   433.4750 & Duplex 0 & JO66LP &  &  \\
	Hotspot  & DMR/D-Star/C4FM & SK6BA-B  & Skene                 &   433.5625 & Duplex 0 & JO67HL &  &  \\
	Hotspot  & D-Star          & SG6YOW   & Alingsås              &   144.8500 &  Simplex & JO67GW &  &  \\
	Link     & FM              & SA6RP    & Floda                 &   433.4750 &  Simplex & JO67ET &  &  \\
	Link     & FM              & SM6FZG   & Skårsjön              &   144.5500 &  Simplex & JO67AN &  &  \\
	Link     & FM              & SM6FZG   & Kortedala             &   144.6000 &  Simplex & JO67AS &  &  \\
	Link     & FM              & SM6FZG   & Långedrag             &   144.5250 &  Simplex & JO57WQ &  &  \\
	Link     & FM              & SM6FZG   & Hönö                  &   144.6250 &  Simplex & JO57TQ &  &  \\
	Link     & FM              & SK6AG    & Guldheden             &   144.5750 &  Simplex & JO57XQ &  &  \\
	Link     & FM              & SM6FZG   & Mölnlycke             &   144.5875 &  Simplex & JO67BP &  &  \\
	Link     & FM              & SM6FZG   & Borås                 &   144.5125 &  Simplex & JO67MR &  &  \\
	Link     & FM              & SM6YRB   & Lidköping/Kållandsö   &   145.3000 &  Simplex & JO68NP &  &  \\
	Link     & FM              & SM6FZG   & Kungsbacka            &   144.6500 &  Simplex & JO67AL &  &  \\
	Link     & FM              & SM6FZG   & Myggenäs              &   144.6625 &  Simplex & JO58UB &  &  \\
	Link     & FM              & SM6FZG   & Guldheden             &   144.6750 &  Simplex & JO57XQ &  &  \\
	Link     & FM              & SM6FZG   & Guldheden             &    51.5500 &  Simplex & JO57XQ &  &  \\
	Link     & FM              & SM6VAG   & Hjo                   &   145.2375 &  Simplex & JO78AG &  &  \\
	Link     & FM              & SA6EAL   & Hajom                 &   145.4000 &  Simplex & JO67GM &  &  \\
	Link     & FM              & SA6GDS   & Istorp                &   145.2875 &  Simplex & JO67FI &  &  \\
	Link     & FM              & SM6TZL   & Örby                  &   145.2375 &  Simplex & JO67IL &  &  \\
	Repeater & FM              & SA6AR/R  & Angered               &   434.9250 &   -2.000 & JO67AT &  &  \\
	Repeater & FM              & SK6QW/R  & Mariestad/Katrinefors &   434.9000 &   -2.000 & JO68VQ &  &  \\
	Repeater & FM              & SK6DK/R  & Varberg/Veddige       &   434.7000 &   -1.600 & JO67EH &  &  \\
	Repeater & FM              & SK6DK/R  & Varberg/Veddige       &   145.7000 &   -0.600 & JO67EH &  &  \\
	Repeater & FM              & SA6BSN/R & Åmål                  &   434.6000 &   -2.000 & JO69IB &  &  \\
	Repeater & D-Star          & SK6DW-B  & Trollhättan           &   434.5250 &   -2.000 & JO68DG &  &  \\
	Repeater & FM              & SA6BXG/R & Kungälv/Romelanda     &   434.7375 &   -2.000 & JO67AX &  &  \\
	Repeater & FM              & SK6RPE   & Kungälv               &   145.6125 &   -0.600 & JO57XU &  &  \\
	Repeater & FM              & SM6CYJ/R & Kinnekulle            &   434.9500 &   -2.000 & JO68QO &  &  \\
	Repeater & FM              & SK6DQ/R  & Älvängen              &   434.7500 &   -2.000 & JO67BW &  &  \\
	Repeater & FM              & SK6MA/R  & Tidaholm/Hökensås     &   145.6375 &   -0.600 & JO78AD &  &  \\
	Repeater & FM              & SM6UXW/R & Ulricehamn            &   434.6750 &   -2.000 & JO67RT &  &  \\
	Repeater & D-Star          & SK6SA-B  & Guldheden             &   434.5125 &   -2.000 & JO57XQ &  &  \\
	Repeater & FM/C4FM/D-Star  & SK6RKG   & Halmstad              &   434.9250 &   -2.000 & JO66MS &  &  \\
	Repeater & FM              & SK6RPE   & Kungälv               &   434.9000 &   -2.000 & JO57XU &  &  \\
	Repeater & FM              & SM6VBT/R & Mölndal               &   145.7000 &   -0.600 & JO67AP &  &  \\
	Repeater & FM              & SM6VBT/R & Mölndal               &   434.7000 &   -2.000 & JO67AP &  &  \\
	Repeater & FM/C4FM         & SK6EI/R  & Skövde                &   434.8250 &   -2.000 & JO68VK &  &  \\
	Repeater & FM/C4FM         & SK6LK/R  & Borås                 &   434.8000 &   -2.000 & JO67MR &  &  \\
	Repeater & FM/C4FM         & SM6THE/R & Skövde                &   145.6875 &   -0.600 & JO68XJ &  &  \\
	Repeater & FM/C4FM         & SM6UXW/R & Ulricehamn            &   145.6750 &   -0.600 & JO67ST &  &  \\
	Repeater & FM/DMR          & SK6DW/R  & Trollhättan           &   145.7625 &   -0.600 & JO68DG &  &  \\
	Repeater & FM/C4FM         & SK6AG    & Guldheden             &   434.6750 &   -2.000 & JO57XQ &  &  \\
	Repeater & FM              & SL6BH/R  & Halmstad              &   434.7500 &   -2.000 & JO66KQ &  &  \\
	Repeater & FM              & SK6GO/R  & Lunden                &   145.7875 &   -0.600 & JO67AR &  &  \\
	Repeater & FM              & SK6RDG   & Guldheden             &   434.9750 &   -2.000 & JO57XQ &  &  \\
	Repeater & FM              & SK6ROY   & Kinnekulle            &   145.6000 &   -0.600 & JO68QO &  &  \\
	Repeater & FM              & SK6LK/R  & Borås                 &   145.7750 &   -0.600 & JO67MR &  &  \\
	Repeater & FM              & SK6RIC   & Alingsås              &   145.6250 &   -0.600 & JO67GW &  &  \\
	Repeater & FM              & SK6RIC   & Alingsås              &   434.6250 &   -2.000 & JO67GW &  &  \\
	Repeater & FM              & SK6RFQ   & Guldheden             &    51.8700 &   -0.600 & JO57XQ &  &  \\
	Repeater & FM              & SK6RJW   & Kungsbacka            &   145.7250 &   -0.600 & JO67AL &  &  \\
	Repeater & FM              & SK6RFQ   & Guldheden             &    29.6800 &   -0.100 & JO57XQ &  &  \\
	Repeater & FM              & SM6VBT/R & Mölndal               &    29.6900 &   -0.100 & JO67AP &  &  \\
	Repeater & FM/DMR          & SK6RFP   & Bengtsfors            &   145.7000 &   -0.600 & JO69CA &  &  \\
	Repeater & FM/DMR          & SL6ZYW/R & Bengtsfors            &   434.6875 &   -2.000 & JO69CA &  &  \\
	Repeater & FM              & SK6RKI   & Guldheden             &  1297.1500 &   -6.000 & JO57XQ &  &  \\
	Repeater & FM              & SK6IF/R  & Bokenäs               &   145.6000 &   -0.600 & JO58TH &  &  \\
	Repeater & FM              & SK6IF/R  & Lysekil               &   434.8000 &   -2.000 & JO58RG &  &  \\
	Repeater & FM/DMR/D-Star   & SA6APY   & Skara                 &   434.9875 &   -2.000 & JO68RJ &  &  \\
	Repeater & DMR             & SM6TKT/R & Borås                 &   434.5500 &   -2.000 & JO67MR &  &  \\
	Repeater & DMR             & SK6DG    & Alingsås              &   434.5375 &   -2.000 & JO67GV &  &  \\
	Repeater & DMR             & SK6AG    & Guldheden             &   434.7875 &   -2.000 & JO57XQ &  &  \\
	Repeater & FM/DMR          & SA6RP/R  & Floda                 &   434.8250 &   -2.000 & JO67ET &  &  \\
	Repeater & FM/DMR          & SK6IF    & Tanumshede            &   145.5750 &   -0.600 & JO58PR &  &  \\
	Repeater & FM              & SK6RKG   & Halmstad              &   145.6750 &   -0.600 & JO66MS &  &  \\
	Repeater & FM              & SK6JX/R  & Falkenberg            &   145.6250 &   -0.600 & JO66FV &  &  \\
	Repeater & FM              & SK6BA/R  & Skene                 &   145.6000 &   -0.600 & JO67HM &  &  \\
	Repeater & FM              & SK6BA/R  & Skene                 &   434.9500 &   -2.000 & JO67HM &  &  \\
	Repeater & DMR             & SK6RKI   & Kortedala             &   145.5875 &   -0.600 & JO67AS &  &  \\
	Repeater & FM              & SK6RJW   & Kungsbacka            &   434.7250 &   -2.000 & JO67AL &  &  \\
	Repeater & FM/DMR          & SK6QA/R  & Stenungsund           &   145.7125 &   -0.600 & JO58XB &  &  \\
	Repeater & FM/DMR          & SK6DW/R  & Trollhättan           &   434.8750 &   -2.000 & JO68DG &  &  \\
	Repeater & FM              & SK6RFQ   & Guldheden             &   434.6500 &   -2.000 & JO57XQ &  &  \\
	Repeater & FM              & SK6RFQ   & Guldheden             &   145.6500 &   -0.600 & JO57XQ &  &  \\
	Repeater & FM/DMR          & SK6IF    & Kungshamn             &   145.6750 &   -0.600 & JO58PI &  &  \\
	Repeater & DMR             & SK6RKI   & Öckerö                &   434.8500 &   -2.000 & JO57TR &  &  \\
	Repeater & FM              & SK6RKI   & Öckerö                &   145.7500 &   -0.600 & JO57TR &  &  \\
	Repeater & FM/DMR          & SK6QA/R  & Stenungsund           &   434.5625 &   -2.000 & JO58UB &  &  \\
	Repeater & FM              & SG6WAL   & Ytterby               &   145.7875 &   -0.600 & JO57WU &  &  \\
	Repeater & FM              & SM6UDU/R & Uddevalla/Bokenäs     &   434.7750 &   -2.000 & JO58UI &  &  \\
	Repeater & FM/C4FM         & SK6EE/R  & Skara                 &   145.7250 &   -0.600 & JO68RH &  &  \\
	Repeater & FM              & SM6WSC   & Trollhättan           &   434.7250 &   -2.000 & JO68EF &  &  \\
	Repeater & FM/C4FM         & SK6EE/R  & Skara                 &   434.5625 &   -2.000 & JO68RH &  &  \\
	Repeater & FM              & SM6SXJ   & Torup/Galtabo         &   434.8875 &   -2.000 & JO67LA &  &  \\
	Repeater & FM              &          &                       &   434.8625 &   -2.000 & JO67JS &  &  \\
	Repeater & FM              & SK6RIC   & Alingsås              &  1297.0250 &   -6.000 & JO67GV &  &  \\
	Repeater & FM/DMR          & SL6ZAQ   & Uddevalla             &   145.7375 &   -0.600 & JO58WH &  &  \\
	Repeater & FM/C4FM         & SK6WW/R  & Karlsborg             &   145.7625 &   -0.600 & JO78FM &  &
\end{longtable}

\subsubsection{Repeatrar distrikt 7}

\begin{longtable}{llllrrlcl}
	Typ      & Modulation      & Signal   & Ort                     & Utfrekvens &   Duplex & Loc    &  &  \\ \hline
	Hotspot  & D-Star          & SG7WDL-C & Eneryda                 &   145.2125 & Duplex 0 & JO76EQ &  \\
	Hotspot  & D-Star          & SG7HTP-C & Sölvesborg              &   145.2375 &  Simplex & JO76GB &  \\
	Hotspot  & D-Star          & SK7RRV-C & Lönsboda                &   144.8875 & Duplex 0 & JO76DJ &  \\
	Hotspot  & DMR             & SG7WSE   & Ekenässjön              &   144.8500 &  Simplex & JO77ML &  \\
	Link     & FM              & SM7KUY/R & Sölvesborg              &   434.4000 &  Simplex & JO76HB &  \\
	Link     & FM              & SA7AUX   & Linneryd                &   145.4000 &  Simplex & JO76NP &  \\
	Link     & FM              & SM7FLD   & Everöd                  &   145.2375 &  Simplex & JO75BV &  \\
	Link     & FM              & SM5GXQ   & Färjestaden             &   145.2375 &  Simplex & JO86FP &  \\
	Repeater & FM              & SM7GYT/R & Eslöv                   &   434.8125 &   -2.000 & JO65PU &  \\
	Repeater & DMR             & SA7CCO   & Sjöbo                   &   434.9250 &   -2.000 & JO65UP &  \\
	Repeater & D-Star          & SM7XAA   & Malmö                   &   434.5250 &   -2.000 & JO65MN &  \\
	Repeater & FM              & SA7BVQ/R & Eslöv                   &   434.7000 &   -2.000 & JO65PU &  \\
	Repeater & FM              & SK7REP   & Lund/Harderberga        &   145.7750 &   -0.600 & JO65PQ &  \\
	Repeater & FM              & SK7RNQ   & Vitaby                  &   145.6125 &   -0.600 & JO75BQ &  \\
	Repeater & FM              & SK7ROQ   & Gladsax                 &   434.8875 &   -2.000 & JO75DN &  \\
	Repeater & FM              & SK7REE   & Söderåsen/Stenestad     &   145.6500 &   -0.600 & JO66NB &  \\
	Repeater & FM              & SK7REE   & Söderåsen/Stenestad     &    51.8500 &   -0.600 & JO66NB &  \\
	Repeater & FM              & SK7RN/R  & Borgholm                &   145.6625 &   -0.600 & JO86HU &  \\
	Repeater & FM              & SK7RN/R  & Mörbylånga              &   145.6250 &   -0.600 & JO86FM &  \\
	Repeater & FM              & SK7RN/R  & Böda                    &   145.7500 &   -0.600 & JO87MG &  \\
	Repeater & FM              & SK7RFJ   & Karlskrona              &   145.7500 &   -0.600 & JO76TE &  \\
	Repeater & FM              & SK7FK/R  & Karlskrona              &   434.7500 &   -2.000 & JO76TE &  \\
	Repeater & DMR             & SK7HW    & Växjö                   &   434.7000 &   -2.000 & JO76KU &  \\
	Repeater & D-Star          & SK7RGM-B & Asarum                  &   434.7125 &   -2.000 & JO76KF &  \\
	Repeater & DMR/D-Star      & SK7RNQ   & Gladsax                 &   145.5750 &   -0.600 & JO75DN &  \\
	Repeater & FM/C4FM         & SK7BQ/R  & Kristianstad            &   145.7375 &   -0.600 & JO76AA &  \\
	Repeater & FM/C4FM         & SK7REZ   & Blentarp/Romeleåsen     &   145.6750 &   -0.600 & JO65TM &  \\
	Repeater & FM/C4FM         & SK7EM/R  & Blentarp/Romeleåsen     &   434.8500 &   -2.000 & JO65SN &  \\
	Repeater & FM/C4FM         & SK7RGM   & Olofström/Boafallsbacke &   145.7000 &   -0.600 & JO76FF &  \\
	Repeater & DMR/D-Star/C4FM & SK7RQX   & Hallandsås              &   145.7875 &   -0.600 & JO66LI &  \\
	Repeater & FM              & SK7CY    & Helsingborg             &  1297.2000 &   -6.000 & JO66IB &  \\
	Repeater & FM              & SK7IJ/R  & Vetlanda                &   434.6250 &   -2.000 & JO77OL &  \\
	Repeater & FM              & SK7MO/R  & Ljungby                 &   145.7250 &   -0.600 & JO66XV &  \\
	Repeater & FM              & SK7RFH   & Nässjö                  &   434.8500 &   -2.000 & JO77IP &  \\
	Repeater & FM              & SK7RIH   & Oskarshamn              &   145.7250 &   -0.600 & JO87FG &  \\
	Repeater & FM              & SK7RIH/R & Oskarshamn              &   434.7250 &   -2.000 & JO87EG &  \\
	Repeater & FM              & SK7RIH   & Oskarshamn              &    51.9100 &   -0.600 & JO87EG &  \\
	Repeater & FM              & SK7RJL/R & Lund                    &   434.7250 &   -2.000 & JO65OR &  \\
	Repeater & FM              & SK5CN/R  & Hultsfred/Gåskullen     &   145.7625 &   -0.600 & JO77WL &  \\
	Repeater & FM              & SK7RRV   & Lönsboda                &   434.9000 &   -1.600 & JO76DJ &  \\
	Repeater & FM              & SK7RYR   & Gnosjö                  &   145.6875 &   -0.600 & JO67UI &  \\
	Repeater & FM              & SK7UO/R  & Emmaboda                &   145.7750 &   -0.600 & JO76SP &  \\
	Repeater & FM              & SL7ZXW/R & Nybro                   &   145.6875 &   -0.600 & JO76VQ &  \\
	Repeater & FM              & SM7LNT/R & Mörrum                  &   434.8250 &   -2.000 & JO76IE &  \\
	Repeater & FM              & SK7HW/R  & Växjö/Hollstorp         &   145.6750 &   -0.600 & JO76KU &  \\
	Repeater & FM              & SK7IJ/R  & Vetlanda                &   145.6250 &   -0.600 & JO77OL &  \\
	Repeater & FM              & SK7RGI   & Huskvarna               &   434.7500 &   -2.000 & JO77DT &  \\
	Repeater & FM              & SK7RGI   & Jönköping/Taberg        &   145.7500 &   -0.600 & JO77AQ &  \\
	Repeater & FM              & SK7RBK   & Hässleholm/Bjärnum      &   145.7625 &   -0.600 & JO66UG &  \\
	Repeater & FM              & SM7NTJ/R & Aneby                   &   434.7250 &   -2.000 & JO77HU &  \\
	Repeater & FM              & SK7RGI   & Huskvarna               &    29.6800 &   -0.100 & JO77DT &  \\
	Repeater & FM              & SK7RFL   & Algutsrum/Öland         &   434.6000 &   -2.000 & JO86GQ &  \\
	Repeater & FM              & SK7RFH   & Nässjö                  &   145.6500 &   -0.600 & JO77IP &  \\
	Repeater & DMR             & SK7RJL   & Lund                    &   434.5875 &   -2.000 & JO65OR &  \\
	Repeater & DMR             & SG7RFH   & Nässjö                  &   434.9000 &   -2.000 & JO77IP &  \\
	Repeater & DMR             & SG7BNT   & Bruzaholm               &   434.6000 &   -2.000 & JO77PP &  \\
	Repeater & DMR             & SG7RFH   & Nässjö                  &   145.5875 &   -0.600 & JO77IP &  \\
	Repeater & FM/DMR          & SK7REE   & Söderåsen/Stenestad     &   434.6500 &   -2.000 & JO66NB &  \\
	Repeater & FM/DMR          & SK7REE   & Örkelljunga             &   434.9750 &   -2.000 & JO66PG &  \\
	Repeater & FM/D-Star       & SK7JL-B  & Spjutsbygd              &   434.8750 &   -2.000 & JO76TH &  \\
	Repeater & FM              & SK7GH/R  & Värnamo                 &   434.6000 &   -2.000 & JO77AF &  \\
	Repeater & FM              & SM7JPI/R & Svängsta                &   434.9250 &   -2.000 & JO76JE &  \\
	Repeater & DMR             & SK7BQ    & Kristianstad            &   434.5250 &   -2.000 & JO76AA &  \\
	Repeater & DMR             & SA7BIK   & Höör                    &   434.9125 &   -2.000 & JO65SW &  \\
	Repeater & FM              & SM7NTJ/R & Aneby                   &   145.7750 &   -0.600 & JO77HU &  \\
	Repeater & DMR             & SK7REE   & Helsingborg             &   434.6000 &   -2.000 & JO66IA &  \\
	Repeater & DMR             & SK7AF    & Eksjö                   &   434.5625 &   -2.000 & JO77MP &  \\
	Repeater & FM/DMR/D-star   & SK7RBK   & Bjärnum                 &   434.9500 &   -2.000 & JO66UG &  \\
	Repeater & FM/C4FM         & SK7JD/R  & Västervik               &   145.6750 &   -0.600 & JO87HS &  \\
	Repeater & DMR             & SK7RJL   & Malmö                   &   434.7750 &   -2.000 & JO65LO &  \\
	Repeater & FM              & SK7RFL   & Algutsrum/Öland         &   145.6000 &   -0.600 & JO86GQ &  \\
	Repeater & DMR             & SK7RGI   & Jönköping               &   434.9750 &   -2.000 & JO77CS &  \\
	Repeater & DMR             & SK7HR    & Sävsjö                  &   434.5250 &   -2.000 & JO77HJ &  \\
	Repeater & DMR             & SM7NTJ/R & Aneby                   &   434.9250 &   -2.000 & JO77HU &  \\
	Repeater & DMR/D-Star/C4FM & SK7RFL   & Algutsrum/Öland         &   434.5500 &   -2.000 & JO86GQ &  \\
	Repeater & FM              & SK7GH/R  & Värnamo                 &   145.6000 &   -0.600 & JO77AE &  \\
	Repeater & DMR             & SA7BJF/R & Södra Vi                &   434.6625 &   -2.000 & JO77VR &  \\
	Repeater & DMR             & SK7JD    & Västervik               &   434.6750 &   -2.000 & JO87HS &  \\
	Repeater & FM/DMR          & SG7WSE   & Ekenässjön              &   145.7125 &   -0.600 & JO77ML &  \\
	Repeater & FM/DMR          & SA7KSI/R & Tomelilla               &   434.6375 &   -2.000 & JO65XN &  \\
	Repeater & FM/DMR          & SK7DL    & Emmaboda                &   434.7875 &   -2.000 & JO76SP &  \\
	Repeater & FM              & SK7JL    & Spjutsbygd              &   145.7250 &   -0.600 & JO76TH &  \\
	Repeater & D-Star          & SK7RDS   & Malmö                   &   145.5625 &   -0.600 & JO65LO &  \\
	Repeater & D-Star          & SK7DS    & Malmö                   &   434.5125 &   -2.000 & JO65LO &  \\
	Repeater & DMR/D-Star      & SK7RMQ   & Linderöd                &   145.5875 &   -0.600 & JO65VW &  \\
	Repeater & FM              & SM7HZK/R & Moheda                  &   145.6375 &   -0.600 & JO76HX &  \\
	Repeater & DMR/D-Star      & SK7RPQ   & Malmö                   &   434.6125 &   -2.000 & JO65MN &  \\
	Repeater & FM              & SK7RN/R  & Borgholm                &   434.7750 &   -2.000 & JO86HU &
\end{longtable}

\normalsize


\input{tex/bandplan-vhfuhf}
