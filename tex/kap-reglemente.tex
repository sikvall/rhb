% !TeX encoding = UTF-8
% !TeX spellcheck = sv_SE
\chapter{Reglemente}

Vi tittar på reglementen kring användning av radiofrekvenser och
utrustning inom Sverige och vilka myndigheter och lagar som styr
detta. Vi går vidare med PR-radio och amatörradio och det fortfarande
ganska nya instegscertifikatet i detta kapitel.

\clearpage

\section{Tillståndsfri radio}

I sverige regleras radioanvändningen av Post- och telestyrelsens (PTS)
bestämmelser och det gäller all radioanvändning. Delar av detta
delegeras sedan till olika myndigheter exempelvis för marin radio och
flygradio. Själva utrustningen som får användas för olika typer av
radiosändning regleras i radioutrustningslagen SFS 2916:392 och det
övergripande EU-direktivet 2014/53/EU.


Det är i undantagsföreskriften du hittar alla villkor som hänger samman med
tillståndsfri radiosändning. Här står effekter, modulationstyper och
användningsområden upptagna för respektive frekvens samt andra villkor.

Det är också i denna skrift som användningen av Amatörradio regleras rent
juridiskt. Sedan finns inte alla rekommendationer och liknande med i denna
skrift, här får man själv hitta det reglemente som amatörradioorganisationer mm
kommit överens om hur man ska nyttja de olika bandet.

Det innebär att radioanvändning för diverse ändamål regleras i denna skrift så
som:

\begin{itemize}
 \item PR-radio på 27 MHz
 \item 69 MHz åkeriradio som också får användas för andra ändamål
 \item Personlig radio på 446 och 444 MHz
 \item Radio för jakt, fiske och jordbruk på 155 MHz
 \item Radio för jakt på 31 MHz
 \item ... och mycket mer.
\end{itemize}

En sak som är intressant är att amatörradio behandlas också i samma skrift
trots det kräver ett godkänt avlagt prov som leder till ett
amatörradiocertifikat. Tidigare reglerades amatörradio separat på Televerkets
tid i en författningssamling som kallades för Televerket B:90. Denna har dock
helt ersatts av bestämmelserna i undantagsföreskriften.

Däremot krävs det numera inget särskilt tillstånd som det gjorde tidigare för
att få använda amatörradio, annat än att den som opererar radioutrustningen
skall ha, eller stå under uppsikt av, en person med giltigt
amatörradiocertifikat.

\subsection{Undantag från tillståndsplikt}

I sverige regleras all radioanvändning med sändare som inte kräver att man
söker tillstånd för det i en skrift som ges ut av Post- och Telestyrelsen
(PTS) och som bär namnet "Untandag från tillståndsplikt". Denna skrift
uppdateras lite då och då när det finns behov av det och här regleras all
radiotrafik som inte kräver särskilt tillstånd.

\subsection{Banden som omfattas}

Detta gäller då PR-radio, 69 MHz radio, PMR-bandet på 446 MHz och KDR-bandet
på 444 MHz som ofta används på byggen eller säkerhetsarrangemang för arenor
med mera liksom Jakt- och jordbruksfrekvenserna på 31 och 155 MHz med mera.
Inget av dessa band kräver licens eller tillstånd men det krävs att man följer
reglerna i hänseende till vad bandet är tänkt att användas till.

Det finns också en rad med andra tillämpningar som går under detta dokument från
tekniska och medicinska och mycket mer. Det rekommenderas att skumma igenom
dokumentet då det faktiskt är ganska intressant.

\subsection{Endast godkända radioapparater}

Radioapparaterna måste också vara godkända och CE-märkta för att man ska
uppfylla villkoren och här blir det ofta en del problem med privatimporterade
radioapparater från Asien som ibland kan ha förskräckligt dåliga sändare. Det
är straffbart att använda icke godkänd radioutrustning.

\section{Amatörradio}

Amatörradio har en lång historia och sträcker sig tillbaka till radions
barndom. Rent formellt så var det i USA i slutet av 1890-talet som radioamatörer
började sända telegram till varandra med den teknik som fanns till buds då. Det
blev mycket populärt bland elingenjörer och andra teknikintresserade och runt
1910 började man få problem med interferens och störningar ochg beslutade sig
för att formalisera det hela. Olika restriktioner infördes men i och med detta
regelverk fick vi också vissa krav på kunskaper för att operera radiosändare.

\subsection{Amatörradiocertifikat HAREC}

Om det fulla amatörradiocertifikatet är ett certifikat som är utformat efter de
principer som anges i HAREC T/R 61-02, Harmonised Amateur Radio Examination
Certificate, Vilnius 2004, uppdaterad 2014 och 2016.

Detta är ett ganska omfattande dokument och i Sverige så har vi t.ex.
KonCEPT-boken som används för ubildning av nya radioamatörer, den kan laddas ned
som PDF från ssa.se eller beställas som papperbok i deras webshop. För att bli
radioamatör behöver man svara tillräckligt många rätt på två stycken delprov,
ett teknikprov som avhandlar elektriska kretsar, radioteknik, sändare och
mottagare, vågutbredning, eletromagnetiska fält, grundläggande matematik och
fysik som är tillämplig, viss komponentlära, förståelse för störningar och att
bli störd, filter och antenner mm. Det andra provet är ett prov över reglementet
som tillämpas, både lagar och författningar som reglerar amatörradio men även
saker som mer praktiska som exempelvis bokstaveringsalfabetet och Q-koder mm.

Ett godkänt sådant prov ger möjlighet att operera som radioamatör på
en stor mängd olika frekvensband, alla de som i Sverige är utmärkta
som amatörradioband.

\subsection{Amatörradiocertifikat insteg}
\label{sec:instegscertifikat}

Instegscertifikatet är ett förenklat teknikprov men ungefär samma prov vad
gäller reglementen osv. Det förenklade teknikprovet innebär dock en del
begränsningar eftersom instegsamatören inte kan ges riktigt samma förtroende att
sända på alla frekvenser. Exempelvis har man därmed begränsat de frekvensband
som får användas till sådana som är exklusiva för amatörradio i Sverige.

Konskapskraven i Sverige skall motsvara det som finns i ''ECC Report 89'': \\
\url{https://docdb.cept.org/download/409}.

PTS förväntas delegera examineringen till SSA och deras utsedda provförrättare
och en särskild utbildningbok för instegscertet håller på att tas fram just nu
och ska snart gå till tryck (Mars 2025).

Tyvärr fick inte SSA gehör för sin remiss att även öppna 70\,cm bandet för
instegscertifikat. Bandet är inte länge ett exklusivt amatörband men vi delar
det med lågeffektssändare för fjärrstyrning mm och det bör inte finnas hinder då
fullcertare får sända med 200\,W i bandet och kan söka högeffektstillstånd på
upp till 1\,kW.

Radioamatörer med instegscertifikat tilldelas anropssignaler i serien SH som
även tidigare använts för noviser. Vid uppgradering till fullt cert får man i
stället en signal i SA-serien är tanken.

\subsection{Amatörradioband för instegscertifikat}

\begin{table}[H]
\centering
\begin{tabular}{rrlr}
  \textbf{Frekvens} & \textbf{Våglängd} & \textbf{Notiser}               & \textbf{Max Effekt} \\ \hline
              7 MHz &          40 meter &                                &           25\,W PEP \\
             14 MHz &          20 meter & 25\,W PEP                      &                     \\
             21 MHz &          15 meter &                                &           25\,W PEP \\
             28 MHz &          20 meter & Liknande utbredning som 27 MHz &           25\,W PEP \\
             50 MHz &           6 meter &                                &           25\,W PEP \\ \hline
            144 MHz &           2 meter &                                &           25\,W ERP


\end{tabular}
\caption{Amatörradioband öppna för instegscertifikat i Ssverige}
\end{table}

50\,MHz är egentligen ett VHF-band men räknas ofta till kortvåg (HF) då de
flesta amatörradiostationer för kortvåg omfattar även 6\,m-bandet.

Myndigheterna har gett instegscertare tillträde till amatörradioband som är
exklusiva för amatörradio vilket innebär att ett antal populära band inte finns
med. Arbete fortsätter på att försöka tillföra även 70 cm bandet (432--438 MHz)
till de band som är tillåtna för instegsamatörer men det har inte kommit med i
denna utgåva av PTS undantagsföreskrift.

\subsection{Effektrestriktioner instegscertifikat}

På frekvensbanden 7, 14, 21, 28 och 15\,MHz tillåter man 25\,W PEP tillfört
antennsystemet. Detta är effekten vid en key-down på morsenyckel. Antennvinsten
är inte medräknad på dessa frekvensband.

För 144 MHz däremot är det 25\,W ERP vilket innebär att du får mata en
halvvågsdipol direkt med 25\,W men om du ska använda riktantenner med högre
antennförstärkning än 0\,dBd eller 2,15\,dBi så måste du sänka sändareffekten i
motsvarande grad.

Anledningen till denna skillnad är att det är ganska svårt att bygga antenner
med hög förstärkning, eller att bestämma dess faktiska förstärkning på
kortvågsbanden medan det för 145\,MHz och uppåt är betydligt enklare att få till
det och dessutom mäta och bestämma antennens förstärkning.

\subsection{Restriktioner på utrustning för instegscertifikat}

Som instegsamatör får man inte operera hemmabyggda sändare, modifierade sändare
eller andra sändare som saknar CE-märkning. Det innebär att man måste hålla sig
till fabriksbyggd utrustning som säljs på den europeiska marknaden.

\subsection{Instegscertifikat utomlands}

Certifikatet äger heller ingen giltighet utomlands så som det fulla,
HAREC-baserade, amatörradiocertifikatet gör. Du får därför i de flesta fall inte
ta med din radiosändare till andra länder och använda den där på
instegscertifikatet.

\section{Tillståndsgiven radioverksamhet}

Radiotrafik som kräver tillstånd regleras i andra dokument eller av respektive
myndighet som svarar för examinering och tillståndsgivning exempelvis
luftfartsverket för luftfartsradio och så vidare. Försvarsmakten svarar också
för användning av militär radio och har långt gående möjligheter att sända på
frekvenser som i fredstid används för annat.

\subsection{Föreningsdrivna radionät}

Kommersiell radio får tillstånd genom ansökan om frekvenser för basstation och
mobila stationer efter behov och betalar då en licensavgift årligen för
detta. Det finns inget som hindrar privatpersoner egentligen att söka sådana
tillstånd och det finns exempel på privatpersoner som satt upp repeatrar och där
användarna genom att vara medlem i en förening då kan använda dessa utan att
vara radioamatörer eller ha särskilda egna tillstånd.

Detta förfarande finns bland annat i dag på några orter kring Stockholm och i
Malmö och är populärt som sambandsmedel inte minst för preppers och för
radiointresserade som vill lära sig mer om radio. Företrädesvis kör dessa på
VHF-bandet i 2-metersområdet (150\,MHz) eller UHF-bandet i 75\,cm området
(400\,MHz).

Enklast finner man det gällande dokumentet genom att söka efter det på PTS
hemsida \url{https://pts.se} där det alltid finns den senaste versionen.

Det finns också andra föreningar som exempelvis scouter med flera som har
tillstånd att bedriva radiotrafik.

\subsection{Kommersiella radionät}

Det finns också exempel på kommersiella radionät, det som tidigirare var
åkeriradionätet på 69 MHz men numera får användas rent allmänt är ett sådant
exempel. Det finns också lokala aktörer som har exempelvis taxiradio,
byggföretag som har allmän kommunikationsradio på sina byggen och många andra
varianter.

Dessa behöver tillståndssökas och en licensavgift ska betalas årligen.

\subsection{Myndigheters radionät}

Myndigheters radionät består i regel av de radionät som behövs för flera olika
verksamheter så som brandbekämpning, räddningstjänst, polisens arbete samt
även försvarsmakten räknas ibland in här. Från att ha haft flera olika system
så har dessa system numera gått samman i RAKEL (radiokommunikation för
effektiv ledning) i ett gemensamt radionät för alla blåljustjänster. RAKEL
tillhandahåller också visst skydd mot avlyssning som standard men kan även
förses med stark kryptografi för att skydda särskilt känslig information, så
kallat ''end-to-end'' krypto (E2E).

RAKEL används främst av sammhällsnyttiga tjänster och myndigheter men det
finns också exempel på när mindre aktörer och även lokaltrafik har fått
använda nätet så det är väl använt. RAKEL kommer på sikt fasas ut till förmån
för ett nytt kommunikationsnät som bygger på 5G-teknik via de kommersiella
operatörernas nät i viss mån samt där det behövs egna radiobasstationer på
700~MHz-bandet. Detta nya nät kallas ibland för SWEN (swedish emergency
network) och befinner sig ännu på planeringsstadet även om prov kan komma
ganska snart.

Enligt myndigheten för samhällsskydd och beredskap (MSB) planerar man avveckla
RAKEL till 2030 men sannolikt kommer det ta lite länge än så innan SWEN är helt
redo att ersätta det.
