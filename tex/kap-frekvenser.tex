% !TeX encoding = UTF-8
% !TeX spellcheck = sv_SE

\chapter{Frekvenser}

Här hittar du frekvenslistor för ett stort gäng olika radioanvändare från
Marin VHF och MF/HF till kanaler för jakt och kortdistansradio,
amatörradiobandplaner, amatörradiofyrar, med mera.

Uppdelningen är gjort mellan så kallade "allmänna frekvenser" som inte kräver
certifikat eller andra tillstånd samt "amatörradio" som kräver
amatörradiolicens. Vissa frekvenser listas dock trots de kräver tillstånd som
exempel marin VHF och MF/HF då dessa kan tänkas ses som av stort
allmänintresse.

\clearpage

\section{Övergripande frekvensplan}

\subsection{Indelning efter frekvens och våglängd}
\label{frekvens-vaglangd}
\begin{table}[H]
\centering
\begin{tabular}{llrlrl}
	\textbf{Förk.} & \textbf{Benämning}   & \textbf{Frekvens} &     & \textbf{Våglängd} &  \\ \hline
	ELF            & Extremt låg frekvens &             3--30 & Hz  &           10--100 & Mm \\
	SLF            & Superlåg frekvens    &           30--300 & Hz  &             1--10 & Mm \\
	ULF            & Ultralåg frekvens    &         300--3000 & Hz  &         100--1000 & km \\
	VLF            & Väldigt låg frekvens &             3--30 & kHz &           10--100 & km \\
	LF (LV)        & Låg frekvens         &           30--300 & kHz &             1--10 & km \\
	MF (MV)        & Mellanfrekvens       &         300--3000 & kHz &         100--1000 & m  \\
	HF (KV)        & Högfrekvens          &             3--30 & MHz &             10--100 & m  \\
	VHF (UKV)      & Väldigt hög frekvens &           30--300 & MHz &             1--10 & m  \\
	UHF            & Ultrahög frekvens    &         300--3000 & MHz &         100--1000 & mm \\
	SHF            & Superhög frekvens    &             3--30 & GHz &           10--100 & mm \\
	EHF            & Extremt hög frekvens &           30--300 & GHz &             1--10 & mm \\
	THF            & Terahertsfrekvens    &          300-3000 & GHz &         100--1000 & µm
\end{tabular}
\caption{Frekvens och våglängd övergripande}
\end{table}

Benämningarna HF, MF och LF har också andra betydelser. Exempelvis an\-vänd\-s
HF som beteckning av den signal en antenn tar mot eller sänder oavsett
frekvensband, MF kan vara mellansignalen oavsett frekvens efter omvandling i
en superheterodynmottagare och LF, ibland benämnt AF (audiofrekvens) är det
hörbara ljudet, dvs den modulation som används på signalen.

På engelska används i stället be\-nämn\-ing\-ar\-na RF för radio frequency, IF
för intermediate frequency and AF för audio frequency vilket rekommenderas då
sammanblandningsrisk med ITU-be\-nämn\-ing\-ar\-na på spektrum inte
föreligger.

Amatörradioband finns inom de flesta av dessa frekvensband utom de högsta och
lägsta frekvenserna.

\subsection{Radarband och benämningar enligt ITU}

\begin{table}[H]
\centering

\begin{tabular}{lrl}
	\textbf{Band} & \textbf{Frekvens} & \textbf{Benämning}          \\ \hline
	HF            &   0.003--0.03 GHz & High frequency              \\
	VHF           &     0.03--0.3 GHz & Very hgh frequency          \\
	UHF           &        0.3--1 GHz & Ultra high frequency        \\
	L             &          1--2 GHz & Long wave                   \\
	S             &          2--4 GHz & Short wave                  \\
	C             &          4--8 GHz &  \\
	X             &         8--12 GHz & Anv. under 2:a världskriget \\
	Ku            &        12--18 GHz & ''Kurz under''              \\
	K             &        18--27 GHz & Tyska ''Kurz'' (kortvåg)    \\
	Ka            &        27--40 GHz & Kurz-above (över)           \\
	V             &        40--75 GHz &  \\
	W             &       75--110 GHz &  \\
	mm            &      110--300 GHz & Millimetervågor
\end{tabular}
\caption{ITU-benämningar på radarband mm}
\end{table}

\section{IEEE-benämningar på frekvensband}
\label{sec:IEEE-benamningar}

\subsection{Nuvarande benämningar}

\begin{table}[H]
	\centering
	\begin{tabular}{cr|cr}
		\textbf{Band} & \textbf{Frekvens} & \bf Band & \bf Frekvens \\ \hline
		      A       &        0--250 MHz &    I     &    8--10 GHz \\
		      B       &      250--500 MHz &    J     &   10--20 GHz \\
		      C       &     100--1000 MHz &    K     &   20--45 GHz \\
		      D       &          1--2 GHz &    L     &   40--60 GHz \\
		      E       &          2--3 GHz &    M     &  60--100 GHz \\
		      F       &          3--4 GHz &    N     & 100--200 GHz \\
		      G       &          4--6 GHz &    O     & 100--200 GHz \\
		      H       &          6--8 GHz &          &
	\end{tabular}
	\caption{Moderniserade IEEE-benämningar på frekvensband}
\end{table}

\subsection{Äldre benämningar}

\begin{table}[H]
	\centering
	\begin{tabular}{cr|cr}
		\textbf{Band} & \textbf{Frekvens} &    &                \\ \hline
		      I       &      100--150 MHz & K  & 18,0--25,6 GHz \\
		      G       &      250--500 MHz & Ka &     20--36 GHz \\
		      P       &      225--390 MHz & Q  &     36--46 GHz \\
		      L       &     390--1550 MHz & V  &     46--56 GHz \\
		      S       &    1550--3900 MHz & E  &     60--90 GHz \\
		      C       &    3900--6200 MHz & W  &    56--100 GHz \\
		      X       &   6200--10900 MHz & F  &     90-140 GHz \\
		     Ku       &    10,9--20,0 GHz & D  &   110--170 GHz \\
		              &                   & Y  &   325--500 GHz
	\end{tabular}
	\caption{Äldre IEEE-benämningar som fortfarande används}
\end{table}

\section{Egenskaper olika frekvensband}

För radioamatörer delar man in frekvensbanden i långvåg, mellanvåg, kortvåg,
VHF, UHF och SHF beroende på frekvens, se tabellen under avsnitt
\ref{frekvens-vaglangd}. Dessa har lite olika utbredningsegenskaper.

\subsection{Långvåg}

Markvågsutbredning, relativt höga sändareffekter, tillförlitliga
för\-bind\-el\-ser men i övre delen av frekvensbandet kortare förbindelser
dagtid. På de lägsta frekvenserna erhålls med hög sändareffekt goda förbindelser
på stora avstånd globalt och används även för t.ex. malmprospektering,
kommunikation med ubåtar i undervattensläge.

\subsection{Mellanvåg}

Kombinerar egenskaperna hos angränsande delar av lång- och kortvåg, kan ge
kraftig interferens mellan rymd- och markvåg som då ofta upplevs som markerad
fädning\footnote{Fädning är när signalstyrkan kraftigt varierar pga olika
typer av utbredningsförhållanden, i just detta fall är det för att rymd- och
markvågen kommer att glida i fas mot varandra och i bland samverka, ibland
motverka varandra. Tiden mellan två toppar eller dalar i sådan fädning brukar
vara ungefär 2-5 sekunder.

Det finns även andra typer av fädning som kan uppstå genom reflexer eller
andra utbredningsfenomen men de är mer sällsynta på mellanvåg och ses främst
på VHF/UHF.}.

Särskilt utmärkande för mellanvågen är den i det närmaste avsaknanden av
skipzon eftersom mark- och rymdvåg kompletterar och i regel överlappar
varandra. Jonosfärens D-skikt är heller inte särskilt uttalat i
frekvensområdet, det vill säga, förbindelser via rymdvåg på korta avstånd
mellan 100--300\,km är möjliga även dagtid också under perioder med högre
solaktivitet.

\subsection{Kortvåg}

God rymdvågsutbredning med mycket lång räckvidd redan med låg effekt men
samtidigt starkt avhängigt radiokonditionerna som styrs i väldigt hög grad av
både rymdvädret, tidpunkten på dagen och på året.

Med ökande frekvens blir jonosfärreflektionen allt flackare vilket resulterar
i en alltmer uttalad död zon (skipzon) där signalen endast med svårighet eller
inte alls kan tas mot. Särskilt utmärkande för kortvågen är att den redan med
låg effekt ger under gynnsamma konditioner extremt lång räckvidd via rymdvåg,
inte sällan globalt vilket traditionellt har nyttjats för att sända nyheter
runt hela världen, se exempelvis BBC World Service och liknande.

I dag för kortvågssändning för rundradio en mer tynande tillvaro, Kina sänder
fortfarande ett ganska stort antal kanaler och det finns några sändare kvar
för BBC i UK och även tyskland men det är tveksamt om de lever kvar ens om
några år. I stället har ju Internet tagit över som nyhetsförmedlare.

\subsection{Ultrakortvåg (UKV) eller väldigt hög frekvens (VHF)}

Förbindelser med låg effekt och små antenner, oberoende av jonosfären men då
endast i form av frirumsutbredning, dvs fram till horisonten och under
påverkan av terränghinder mm. Särskilt utmärkande för UKV är att rymdvåg
saknas, markvågsdämpningen till lands är total och kommunikation på högre
frekvenser i princip därför bara sker vid fri sikt mellan sändare och
mottagare.

Eftersom signalerna kan passera jonosfären fungerar det att kommunicera med
satelliter och rymdstationer på frekvenserna över ca 100\,MHz.
Amatörradiobandet på 144--146\,MHz (2-metersbandet) har en avdelning
frekvenser vikta för rymdkommunikaton.

\subsection{Ultrahög frekvens (UHF)}

Bandet är ett av de mest populära för landmobil radio. I dag rymms i detta
band mellan 300--3000\- \,MHz nära nog all landmobil professionell
kommunikation då mycket har lämnat VHF-bandet till förmån för 400\,MHz-bandet.
I skrivande stund ryms även alla mobiltelefoniband inom UHF, det gäller 700,
800, 900, 1800, 2100 och 2600. I framtiden kan det komma frekvensband som
ligger högre, exempelvis 3,5\,GHz eller 5\,GHz.

Bandet är utmärkt för kommunikation mellan fordon, särskilt i urban miljö
fungerar den kortare våglängden bra. Den reflekteras också bättre mellan husen
och frekvenser på främst 400\,MHz har en fantastisk förmåga att leta sig in
fast man befinner sig i radioskugga. Tillsammans med 6-meter, 4-meter och
2-meter i glesbygden fyller 70\,cm-bandet en nisch för landmobil kommunikation
som få andra frekvenser fungerar.

Radioamatörer har också ett av sina större tilldelningar i detta band, mellan
432--438\,MHz. Här finns också en LPD\footnote{Low Power Devices, små
radiostyrningsutrustningar, t.ex. väderstationer, garageportsöppnare med
mera.}-del som vi får samsas med. Främst kör man FM på bandet men det
förekommer SSB och kanske ibland också CW. Det finns ett stort repeaterband
med 2\,MHz shift mellan mottagare och sändare och det är ett praktiskt band
för man kan enkelt bygga antenner med ganska rejält med förstärkning.

Det är också ett populärt band för satellitkommunikation.

\section{Rundradio}

För rundradio på AM-banden, dvs långvåg, mellanvåg och kortvåg används 10
eller 9\,kHz breda kanaler med AM-modulation. Digital rundradio finns på några
håll och använder då ett digitalt sändarsystem som kallas för DRM, Digital
Radio Mondiale.

\subsection{Rundradiobenämningar på frekvensband}
\label{rundradioband}

\begin{table}[H]
\centering
\begin{tabular}{llrl}
	\textbf{Förk.} & \textbf{Namn} & \textbf{Frekvens Rundradio} &     \\ \hline
	LW/LV          & Långvåg       &                  148,5--285 & kHz \\
	MW/MV          & Mellanvåg     &               526,5--1606,5 & kHz \\
	SW/KV          & Kortvåg       &                     4,3--30 & MHz \\
	UKV/VHF        & Ultrakortvåg  &                     88--108 & MHz
\end{tabular}
\caption{Rundradiobanden}
\end{table}

\subsection{Långvåg}

Långvåg används i princip bara inom ITU region 1, det vill säga Europa,
Nordafrika och Mongoliet. Utbredningen är vanligtvis mycket god nattetid och
en långvågsstation kan typiskt täcka en betydligt större yta än en
mellanvågsstation. Långvåg har dock inte rymdvågsutbredning som kortvåg har
utan det är en markvåg som i regel följer jordytans krökning mycket långt.

Långvåg saknas normalt på de flesta radiomottagare i dag.

\subsection{Mellanvåg}
\label{rundradio-frekvenser-mellanvåg}

I dag används mellanvågsbanden mycket sparsamt i Europa. Det finns fortfarande
kvar rundradiostationer på några ställen och i Asien är det ibland väl använt.
Den största användaren annars är faktiskt USA som har ganska många
licensierade radiostationer på mellanvågsbandet.

\begin{table}[h]
\centering
\begin{tabular}{lrrr}
	\bf Område            &  \bf Frekvens & \bf Kanalraster & \bf Kanaler \\ \hline
	Europa, Asien, Afrika & 531--1602 kHz &          9  kHz &         120 \\
	Australien            & 531--1701 kHz &           9 kHz &         131 \\
	Nord- och Sydamerika  & 530--1700 kHz &          10 kHz &         118
\end{tabular}
\end{table}

Mellanvågsbandet har liknande frekvenser i hela världen men start- och
stoppfrekvenserna varierar lite. Australien har ett utökat band med bara
lågeffektstationer ovanför 1610\,kHz.

Mellanvåg finns påfallande ofta på rundradiomottagare i dag då det används
flitigt bland annat i USA.


\subsection{Kortvåg}
\label{rundradio-frekvenser-kortvåg}

Det finns ganska många kortvågsband för rundradio. En del delas med andra
tjänster som t.ex. amatörradio och ibland skiljer sig frekvenserna åt i olika
delar.

\begin{table}[H]
\resizebox{\textwidth}{!}{%
\begin{tabular}{rrl}
	\bf Band &     \bf Frekvens & \bf Noteringar                                                                                          \\
	 \bf [m] &        \bf [kHz] & \bf                                                                                                     \\ \hline
	     120 &       2300--2495 & Mestadels använt i tropikerna, tidsangivelse på 2500 kHz.                                               \\
	      90 &       3200--3400 & Tropikband, begränsad utbredning nattetid. Tidsstation i Kanada på 3330 kHz.                            \\
	      75 &       3900--4000 & Mestadels östra halvklotet mörka timmar. Delas med amerikanska 80 m amatörradiobandet.                  \\
	      60 &       4750--4995 & Tropikband. särskilt använt i Brasilien. Tidsreferenser använder 5000 kHz.                              \\
	      49 &       5900--6200 & Bra åretruntband men sämre mottagning dagtid.                                                           \\
	      41 &       7200--7450 & Mottagning varierar per region. God nattmottagning, särskilt vintertid. Delas med radioamatörer         \\
	      31 &       9400--9900 & Det mest använda bandet. Hyggligt året runt, bra på vintern, tidsreferenser ca 10 MHz.                  \\
	      25 & 11\,600--12\,100 & Bättre på sommaren än på vintern samt vid ''greyline'' precis före och efter solens upp- och nedgång.   \\
	      22 & 13\,570--13\,870 & Tidigare väl använt inom Europa och Asien                                                               \\
	      19 & 15\,100--15\,830 & Varierande mottagning över dygnet, bäst sommartid, Tidsstationer använder 15 MHz.                       \\
	      16 & 17\,480--17\,900 & Dagtid bra, nattetid varierar kraftigt med årstiden, bäst på sommaren.                                  \\
	      15 & 18\,900--19\,020 & Samma egenskaper som 16\,m. Kan bli ett DRM-band. Används ganska sällan just nu.                        \\
	      13 & 21\,450--21\,850 & Sämre mottagning dagtid, kraftigt varierande med ryndvädret både dag och natt. Mycket likt 11\,m.       \\
	      11 & 25\,670--26\,100 & Sällan använt. Dagtid dåligt vid låg solcykelaktivitet men kan fullständigt briljera vid hög aktivitet. \\
	         &                  & Nattlig aktivitet oftast mycket dålig förutom markvågsutbredningen. Ligger nära PR-radiobandet.
\end{tabular}}
\caption{Rundradiofrekvenser på kortvåg.}
\end{table}

Kortvåg saknas oftast på moderna rundradiomottagare.

\subsection{VHF/UKV}

I dag används rundradio på VHF i stort sett bara i band II (FM-radio) och i band III (DAB-radio).

\begin{table}[H]
	\centering
	\begin{tabular}{lrl}
		\textbf{Band} & \textbf{Frekvens} &                                              \\ \hline
		Band I        &        47--68 MHz & Tidigare TV-band också ''6-metersbandet''    \\
		Band II       &   87,5--108,0 MHz & Nuvarande FM-rundradionbandet                \\
		Band III      &      174--230 MHz & Används i dag för DAB                        \\
		Band IV       &      470--582 MHz & I dag refarmerat till främst landmobil radio \\
		Band V        &      582--862 MHz & Marksänd TV
	\end{tabular}
	\caption{IEEE-benämningar på rundradioband}
\end{table}

Alla radiomottagare för rundradio har Band II. De som har DAB har även band
III och ibland även band IV.


\subsection{UHF}

Mycket sällan använt för rundradio.
I föregående tabell ligger band IV och band V inom gränserna för UHF.

\section{Allmänna frekvenser VHF--UHF}

Dessa frekvenser är avsedda för allmänhet eller för de specifika ända\-mål som
anges i PTS undantagsföreskrift. Det innebär att de kan brukas för de ändamål
som anges i PTS för\-fatt\-nings\-sam\-ling\-ar och sammanställning över ej
tillståndspliktiga frekvenser. Observera att du är skyldig att själv
kontrollera bestämmelserna innan en frekvens brukas.

Effekten i tabellen är ustrålad effekt PEP om inte annat anges.

\subsection{Jaktfrekvenser 31 MHz}
% Frekvenser uppdaterade 250428

Frekvenserna på detta band var tidigare till för enbart jakt. I dag är de
öppna för övrig landmobil trafik och kan nyttjas till fritidskommunikation av
annat slag.

Högsta effekt är 5\,W och maximal sändningscykel är 10\% vilket betyder att
under en timme får man sända maximalt 6 minuter.

I oktober 2012 utökades de gamla jaktkanalerna med ett antal nya kanaler
vilket skedde i oktober 2012. De har ingen officiell kanalnumrering eller
egentlig benämning men jag har valt att numrera upp dem efter de traditionella
numren med början på 25.

Kanal 24 har dock tidigare haft en frekvens som inte längre är i bruk, så det
vore förvirrande att använda den -- den saknas därför i listan. Nya kanaler är
markerade i listan med asterisk och har fått nummer från kanal 25 och uppåt
efter frekvens. Detta gör att listan blir en smula oordnad.

\begin{longtable}{rll|rll}   % Uppdaterad 250501
	\caption{Jaktfrekvenser 31\,MHz tabell}\\
	\textbf{Frekvens} & \textbf{Benämning} & \textbf{Tidigare} & \textbf{Frekvens} & \textbf{Benämning} & \textbf{Tidigare} \\ \hline
		\endfirsthead
	\textbf{Frekvens} & \textbf{Benämning} & \textbf{Tidigare} & \textbf{Frekvens} & \textbf{Benämning} & \textbf{Tidigare} \\ \hline
	\endhead
	           30,930 & Jakt 1             &                   &   31,180          &   Jakt 14          &                   \\
	           30,940 & Jakt 25*           &                   &   31,190          &   Jakt 15          &                   \\
	           30,950 & Jakt 26*           &                   &   31,200          &   Jakt 16          &                   \\
	           30,960 & Jakt 27*           &                   &   31,210        &     Jakt 17        &                   \\
	           30,970 & Jakt 28*           &                   &   31,220          &   Jakt 18          &                   \\
	           31,030 & Jakt 29*           &                   &   31,230        &     Jakt 32*       &                   \\
	           31,040 & Jakt 2             &                   &   31,240         &    Jakt 33*        &                   \\
	           31,050 & Jakt 3             & Kanal 1 Eller D   &   31,250          &   Jakt 19          &  Kanal 4 eller E  \\
	           31,060 & Jakt 4             & Kanal 2 Eller A   &   31,260          &   Jakt 20         &   Kanal 5 eller C \\
	           31,070 & Jakt 5             &                   &   31,270          &   Jakt 21          &                   \\
	           31,080 & Jakt 6             &                   &   31,280          &   Jakt 34*         &                   \\
	           31,090 & Jakt 7             &                   &   31,290          &   Jakt 35*         &                   \\
	           31,100 & Jakt 8             &                   &   31,300          &   Jakt 36*         &                   \\
	           31,110 & Jakt 9             &                   &   31,310          &   Jakt 37*         &                   \\
	           31,120 & Jakt 10            &                   &   31,320          &   Jakt 22          & Kanal 6 eller F   \\
	           31,130 & Jakt 30*           &                   &   31,330          &   Jakt 23          &                   \\
	           31,140 & Jakt 11            &                   &   31,340          &   Jakt 38*         &                   \\
	           31,150 & Jakt 12            &                   &   31,350          &   Jakt 39*        &                   \\
	           31,160 & Jakt 13            & Kanal 3 Eller B   &   31,360          &   Jakt 40*         &                   \\
	           31,170 & Jakt 31*           &                   &   31,370          &                    &\\
\end{longtable}

\subsection{Åkeribandet 69 MHz}
\label{69-MHz}

Sedan några år tillbaka finns nu ett nytt band som kan användas för privat
landmobil radio (PMR). Bandet kallas allmänt för 69\,MHz-bandet och har blivit
mycket populärt på sina ställen.

Anledningen är bland annat en stor tillgång på FM-radio för bandet från gamla
åkeriradio som säljs för billiga pengar på diverse begagnatsajter och som
därmed gör det enkelt att komma igång.

Antennstorlekarna är moderata och det är ett ypperligt band för mobilradio där
våglängden är ungefär den dubbla mot 2-metersbandet och fungerar bra i många
sammanhang.

Nackdelen som den delar med 27\,MHz är att många antenner för fordon är
förkortade vilket minskar verkningsgraden på dessa en del men trots detta
fungerar det bra. Antennerna är dock betydligt mindre skrymmande än de för
27\,MHz.

På bandet kör man FM uteslutande och det rekommenderas att man skaffar en
radio med signalstyrkemätare då man på FM inte kan höra lika väl om man är
störd, däremot syns det ju på S-metern om man har störningar. Bandet lider
något av störningar i urbana miljöer men på landsbygden brukar det vara tyst
och fint.

Användningen av bandet regleras i PTS föreskrift Undantag från Tillståndsplikt
och innebär att man får använda max 25\,W ERP (dvs för en dipolantenn), max
10\% sändningscykel (dvs 6 min/timme), en kanalbredd om 25\,kHz och det finns
8 stycken kanaler upplåtna för landmobil radio. I strikt mening är inte
kommunikation bas-bas egentligen tillåten eftersom det är landmobil trafik som
avses i PTS bestämmelser. Kanal 1 får enbart användas för mobil-mobil trafik
inom Västra Götaland och Hallands län.

\begin{table}[ht]
  \centering
\begin{tabular}{rrl}
	\bf Kanal & \bf Frekvens & \bf Noteringar                     \\ \hline
	        1 &      69,0125 & End. mobil i V. Götaland o Halland \\
	        2 &      69,0375 &                                    \\
	        3 &      69,0625 &                                    \\
	        4 &      69,0875 &                                    \\
	        5 &      69,1125 &                                    \\
	        6 &      69,1375 &                                    \\
	        7 &      69,1625 &                                    \\
	        8 &      69,1875 & Anv. som anropskanal
\end{tabular}
\caption{Frekvenser 69 MHz}
\end{table}

\subsection{Jakt och jordbruksfrekvenser 155 MHz}
\label{155-MHz}

Observera att kanalnumren som är traditionella och frekvenserna inte
kommer helt i ordning. Fyra kanaler är markerade med $^R$ och har
särskilda restriktioner på svenskt innanvatten och territorialvatten.

\begin{table}[H]
\centering
\begin{tabular}{rlrl}
	\textbf{Frekvens} & \textbf{Benämning} & \textbf{Effekt} & \textbf{Användningsområde}           \\ \hline
	          155,400 & Jakt K6            &             5 W & Jakt, Jordbruk, Skogsbruk$^R$        \\
	          155,425 & Jakt K1            &             5 W & Jakt, Jordbruk$^R$                   \\
	          155,450 & Jakt K7            &             5 W & Jakt, Jordbruk, Skogsbruk$^R$        \\
	          155,475 & Jakt K2            &             5 W & Jakt, Jordbruk$^R$                   \\
	          155,500 & Jakt K3 VHF-M L1   &             5 W & Jakt, Jordbruk, Skogsbruk, Marin$^M$ \\
	          155,525 & Jakt K4 VHF-M L2   &             5 W & Jakt, Jordbruk, Skogsbruk Marin$^M$  \\
	          156,000 & Jakt K5            &             5 W & Jakt, PMR, Friluftskanal$^P$
\end{tabular}
\caption{Jakt- och jordbruksfrekvenser 155\,MHz}
\end{table}

\footnotesize
\begin{itemize}
	\item[$^M$] Delas med marina VHF-bandet, kanalerna L1 och L2 för fritidsbåtar.
	\item[$^P$] PMR-kanal som kan användas till allmän privatradio.
	\item[$^R$] Dessa kanaler \textit{får ej användas} på svenskt
          territorialvatten eller svenskt inre vatten. Se
          \href{https://pts.se/globalassets/startpage/dokument/legala-dokument/foreskrifter/radio/beslutade_ptsfs-2018-3-undantagsforeskrifter.pdf}{PTSFS2018:3}
          för mer information.
\end{itemize}
\normalsize

\subsection{Öppna PMR-bandet på 446 MHz}
\label{446-MHz}
I nya författningssamlingen står det uttryckligen att
repeateranvändning är förbjuden. De exakta kanalerna har också inte
heller bestämts utan bandet är upplåtet
446,0--446,2\,MHz. Traditionellt används nedanstående kanaler. Max
effekt är 500\,mW och antennen får ej vara av löstagbar
sort. Utrustningen skall vara godkänd för ändamålet.

Sedan sist har ytterligare spektrum tillförts och bandet har nu 16
kanaler. Det medges också digital PMR på alla frekvenserna men
rekommendationen är att använda K1--K8 för analogt och K9--K16 för
digitalt eftersom äldre apparater inte kan gå på de nya kanalerna
medan alla digitala kan det.

Vi vissa numreringar numreras de digitala kanalerna med start på
kanalnummer 1 på K9. I listan står de som D1--D8 där D står för
digitalt.

Endast smalbandig modulation med FM-deviation max 2.5 kHz skall
användas för att inte störa närliggande kanaler. Kanalrastret är
12,5\,kHz så modulationen bör rymmas inom den bandbredden.

\begin{table}[H]
\centering
\begin{tabular}{rll|rll}
	\textbf{Frekvens} & \textbf{Benämning} & \textbf{Rek. Anv.} & \textbf{Frekvens} & \textbf{Benämning} & \textbf{Rek. Anv.} \\ \hline
	        446,00625 & K1                 & PMR                &         446,10625 & K9\ \,\,/D1        & DPMR               \\
	        446,01875 & K2                 & PMR                &         446,11875 & K10/D2             & DPMR               \\
	        446,03125 & K3                 & PMR                &         446,13125 & K11/D3             & DPMR               \\
	        446,04375 & K4                 & PMR                &         446,14375 & K12/D4             & DPMR               \\
	        446,05625 & K5                 & PMR                &         446,15625 & K13/D5             & DPMR               \\
	        446,06875 & K6                 & PMR                &         446,16875 & K14/D6             & DPMR               \\
	        446,08125 & K7                 & PMR                &         446,18125 & K15/D7             & DPMR               \\
	        446,09375 & K8                 & PMR                &         446,19375 & K16/D8             & DPMR
\end{tabular}
\caption{PMR-frekvenser}
\label{tab:pmr-frekvenser}
\end{table}

\subsection{Kortdistansradio (KDR, SRBR) på 444 MHz}
\label{444-MHz}
\label{KDR}
\label{SRBR}

Kallas även SRBR för Short Range Business Radio.  Den traditionella
frekvenslistan ser ut som följer. En ny variant med frekvenser för
12,5\,kHz samt 6,25\,kHz kanaler finns också ute nu och kan ses i
tabell \ref{tab:SRBR-frekvenser}.

\begin{table}[h]
	\centering
\begin{tabular}{rlrl}
	\textbf{Frekvens} & \textbf{Benämning} & \textbf{Effekt} & \textbf{Användningsområde} \\ \hline
	          444,600 & K1                 &             2 W & Short range business radio \\
	          444,625 & K2                 &             2 W & Short range business radio \\
	          444,800 & K3                 &             2 W & Short range business radio \\
	          444,825 & K4                 &             2 W & Short range business radio \\
	          444,850 & K5                 &             2 W & Short range business radio \\
	          444,875 & K6                 &             2 W & Short range business radio \\
	          444,925 & K7                 &             2 W & Short range business radio \\
	          444,975 & K8                 &             2 W & Short range business radio
\end{tabular}
\caption{Frekvenser för SRBR}
\end{table}

SRBR är ett ej tillståndspliktigt frekvenssegment som används för
yrkesmässig radiotrafik.

Rekommendationen är att man skall använda CTCSS eller motsvarande för
att undvika störa och bli störd av andra stationer som delar
frekvenserna.

Från PTSFS2018:3 så har bandet fått nya bärvågsfrekvenser och det har
blivit öppet för att köra med 25, 12,5 eller 6,25\,kHz
Kanalraster. Denna frekvenstabell blir lite mer komplicerad.

% Please add the following required packages to your document preamble:
% \usepackage{multirow}
\begin{table}[h]
	\centering
	\begin{tabular}{|l|l|l|l|l|l|}
		\hline
		\textbf{25 kHz}                                & \textbf{12,5 kHz}                               & \textbf{6,25 kHz}               & \textbf{25 kHz}                               & \textbf{12,5 kHz}                               & \textbf{6,25 kHz}               \\ \hline
		\multicolumn{1}{|c|}{\multirow{4}{*}{444,600}} & \multicolumn{1}{c|}{\multirow{2}{*}{444,59375}} & \multicolumn{1}{c|}{444,590625} & \multicolumn{1}{l|}{\multirow{4}{*}{444,850}} & \multicolumn{1}{l|}{\multirow{2}{*}{444,84375}} & \multicolumn{1}{l|}{444,840625} \\ \cline{3-3}\cline{6-6}
		\multicolumn{1}{|c|}{}                         & \multicolumn{1}{c|}{}                           & \multicolumn{1}{c|}{444,596875} & \multicolumn{1}{l|}{}                         & \multicolumn{1}{l|}{}                           & \multicolumn{1}{l|}{444,846875} \\ \cline{2-3}\cline{5-6}
		\multicolumn{1}{|c|}{}                         & \multicolumn{1}{c|}{\multirow{2}{*}{444,60625}} & \multicolumn{1}{c|}{444,603125} & \multicolumn{1}{l|}{}                         & \multicolumn{1}{l|}{\multirow{2}{*}{444,85625}} & \multicolumn{1}{l|}{444,853125} \\ \cline{3-3}\cline{6-6}
		\multicolumn{1}{|c|}{}                         & \multicolumn{1}{c|}{}                           & \multicolumn{1}{c|}{444,609375} & \multicolumn{1}{l|}{}                         & \multicolumn{1}{l|}{}                           & \multicolumn{1}{l|}{444,859375} \\ \hline
		\multirow{4}{*}{444,650}                       & \multirow{2}{*}{444,64375}                      & 444,640625                      & \multicolumn{1}{l|}{\multirow{4}{*}{444,875}} & \multicolumn{1}{l|}{\multirow{2}{*}{444,86875}} & \multicolumn{1}{l|}{444,865625} \\ \cline{3-3}\cline{6-6}
		                                               &                                                 & 444,646875                      & \multicolumn{1}{l|}{}                         & \multicolumn{1}{l|}{}                           & \multicolumn{1}{l|}{444,871875} \\ \cline{2-3}\cline{5-6}
		                                               & \multirow{2}{*}{444,65625}                      & 444,653125                      & \multicolumn{1}{l|}{}                         & \multicolumn{1}{l|}{\multirow{2}{*}{444,88125}} & \multicolumn{1}{l|}{444,878125} \\ \cline{3-3}\cline{6-6}
		                                               &                                                 & 444,659375                      & \multicolumn{1}{l|}{}                         & \multicolumn{1}{l|}{}                           & \multicolumn{1}{l|}{444,884375} \\ \hline
		\multirow{4}{*}{Saknas}                        & \multirow{2}{*}{444,66875}                      & 444,665625                      & \multicolumn{1}{l|}{\multirow{4}{*}{444,925}} & \multicolumn{1}{l|}{\multirow{2}{*}{444,91875}} & \multicolumn{1}{l|}{444,915625} \\ \cline{3-3}\cline{6-6}
		                                               &                                                 & 444,671875                      & \multicolumn{1}{l|}{}                         & \multicolumn{1}{l|}{}                           & \multicolumn{1}{l|}{444,921875} \\ \cline{2-3}\cline{5-6}
		                                               & \multirow{2}{*}{444,68125}                      & 444,678125                      & \multicolumn{1}{l|}{}                         & \multicolumn{1}{l|}{\multirow{2}{*}{444,93125}} & \multicolumn{1}{l|}{444,928125} \\ \cline{3-3}\cline{6-6}
		                                               &                                                 & 444,684375                      & \multicolumn{1}{l|}{}                         & \multicolumn{1}{l|}{}                           & \multicolumn{1}{l|}{444,934375} \\ \hline
		\multirow{4}{*}{444,800}                       & \multirow{2}{*}{444,79375}                      & 444,790625                      & \multicolumn{1}{l|}{\multirow{4}{*}{444,975}} & \multicolumn{1}{l|}{\multirow{2}{*}{444,91875}} & \multicolumn{1}{l|}{444,915625} \\ \cline{3-3}\cline{6-6}
		                                               &                                                 & 444,796875                      & \multicolumn{1}{l|}{}                         & \multicolumn{1}{l|}{}                           & \multicolumn{1}{l|}{444,921875} \\ \cline{2-3}\cline{5-6}
		                                               & \multirow{2}{*}{444,80625}                      & 444,803125                      & \multicolumn{1}{l|}{}                         & \multicolumn{1}{l|}{\multirow{2}{*}{444,93125}} & \multicolumn{1}{l|}{444,928125} \\ \cline{3-3}\cline{6-6}
		                                               &                                                 & 444,809375                      & \multicolumn{1}{l|}{}                         & \multicolumn{1}{l|}{}                           & \multicolumn{1}{l|}{444,934375} \\ \hline
		\multirow{4}{*}{444,825}                       & \multirow{2}{*}{444,81875}                      & 444,815625                      & \multicolumn{3}{l}{\multirow{4}{*}{}}                                                                                             \\ \cline{3-3}
		                                               &                                                 & 444,821875                      & \multicolumn{3}{l}{}                                                                                                              \\ \cline{2-3}
		                                               & \multirow{2}{*}{444,83125}                      & 444,828125                      & \multicolumn{3}{l}{}                                                                                                              \\ \cline{3-3}
		                                               &                                                 & 444,834375                      & \multicolumn{3}{l}{}                                                                                                              \\ \cline{1-3}
	\end{tabular}
\caption{Nya frekvensindelningen på kortdistansradiobandet}
\label{tab:SRBR-frekvenser}
\end{table}

\clearpage

\subsection{Marina VHF-frekvenser}

Marinbandet på VHF består både av duplex- och simplexkanaler. Simplexkanalerna
används skepp-till-skepp och även ibland mot kustradio. Duplexfrekvenserna
används t.ex. vid telefonsamtal som sätts upp av kuststation till skepp eller
liknande. På dessa arbetskanaler sänder man även ut sjörapporter,
navigationsvarningar och annan information t.ex. säkerhetsvarningar som är
viktiga för sjöfarten.

\subsection{Kanalnummer och frekvens marina VHF-kanaler}

\begin{table}[H]
\centering
\begin{tabular}{rrr|rrr}
\textbf{Kanal} & \textbf{Skepp} & \textbf{Kust} &
\textbf{Kanal} & \textbf{Skepp} & \textbf{Kust} \\ \hline
01 & 156,050 & 160,650 & 60 & 156,025 & 160,625 \\
02 & 156,100 & 160,700 & 61 & 156,075 & 160,675 \\
03 & 156,150 & 160,750 & 62 & 156,125 & 160,725 \\
04 & 156,200 & 160,800 & 63 & 156,175 & 160,775 \\
05 & 156,250 & 160,850 & 64 & 156,225 & 160,825 \\
06 & 156,300 &         & 65 & 156,275 & 160,875 \\
07 & 156,350 & 160,950 & 66 & 156,325 & 160,925 \\
08 & 156,400 &         & 67 & 156,375 & \\
09 & 156,450 &         & 68 & 156,425 & \\
10 & 156,500 &         & 69 & 156,475 & \\
11 & 156,550 &         & 70 & 156,525 & DSC \\
12 & 156,600 &         & 71 & 156,575 & \\
13 & 156,650 &         & 72 & 156,625 & \\
14 & 156,700 &         & 73 & 156,675 & \\
15 & 156,750 &         & 74 & 156,725 & \\
16 & 156,800 & Anrop/Nöd   & 75 & 156,775 & \\
17 & 156,850 &         & 76 & 156,825 & \\
18 & 156,900 & 161,500 & 77 & 156,875 & \\
19 & 156,950 & 161,550 & 78 & 156,925 & 161,525 \\
20 & 157,000 & 161,600 & 79 & 156,975 & 161,575 \\
21 & 157,050 & 161,650 & 80 & 157,025 & 161,625 \\
22 & 157,100 & 161,700 & 81 & 157,075 & 161,675 \\
23 & 157,150 & 161,750 & 82 & 157,125 & 161,725 \\
24 & 157,200 & 161,800 & 83 & 157,175 & 161,775 \\
25 & 157,250 & 161,850 & 84 & 157,225 & 161,825 \\
26 & 157,300 & 161,950 & 85 & 157,325 & 161,925 \\
27 & 157,350 & 161,950 & 86 & 157,325 & 161,925 \\
28 & 157,400 & 162,000 & 87 & 157,375 & \\
   &         &         & 88 & 157,425 & \\
   &         &         &    &         & \\
L1 & 155,500 & Leisure        & F1 & 155,625 &Fishing \\
L2 & 155,525 & Leisure        & F2 & 155,775 &Fishing \\
   &         &         & F3 & 155,825 & Fishing\\
\end{tabular}
\caption{Marin VHF, frekvenslista}
\end{table}

I tabellen listas de kanaler som gäller i svenska farvatten. Andra
länder kan ha andra kanaler eller för olika ändamål. Det krävs en
särskild licens från PTS för att få nyttja dessa frekvenser och
radiooperatören skall ha ett SRC-certifikat (Short Range
Communication).

Anropskanal och nödkanal är kanal 16.

Vid duplextrafik är skiftet -4,6\,MHz.

I tabellen är kanaler som saknar kustfrekvens alltså
simplexkanaler. DSC står för ''Digital Selective Call'' ett sätt att
digitalt anropa skepp eller kuststationer, kanaler vikta för DSC får
inte användas för vanliga samtal.

Kanal 16 är anropsfrekvens om man inte vet motstationen passar en
annan kanal. Den är också nödfrekvens eftersom den passas av de
flesta.

Kanalerna L1--L2 är frekvenser avsedda för fritidsbåtar (Leisure) och
frekvenserna F1--F3 osv är avsedda för yrkesfiske. L1 och L2 delas med
kanal 3 och 4 på jaktradion vilket kan vara bra att känna till.

\subsection{Transponderkanaler}

\begin{table}[h]
\centering
\begin{tabular}{lrl}
	\textbf{Kanal} & \textbf{Skepp} & \textbf{Not}                \\ \hline
	AIS1           &        161,975 & Digital trafik, transponder \\
	AIS2           &        162,025 & Digital trafik, transponder
\end{tabular}
\caption{AIS-frekvenser marin}
\end{table}

\subsection{Stockholm radio}
% Frekvenser kontrollerade 250428
% Ett antal felaktigheter fixade

Radiohorisonten är beräknad i nautiska mil, skeppet lägger till sin egen
radiohorisont för att bestämma om det går att nå kuststationen eller ej.

\subsubsection{Ostkusten}

\begin{table}[h]
\centering
\begin{tabular}{lrr|lrr}
\textbf{Kuststation} & \textbf{Kanal} & \textbf{Horisont} & \textbf{Kuststation} & \textbf{Kanal}& \textbf{Horisont}\\
\hline
Kalix          & 60 & 39 & Luleå         & 61 & 26 \\
Skellefteå     & 23 & 44 & Umeå          & 62 & 54 \\
Örnsköldsvik   & 63 & 42 & Mjällom       & 64 & 43 \\
Kramfors       & 83 & 43 & Härnösand     & 23 & 36 \\
Sundsvall      & 60 & 36 & Hudiksvall    & 61 & 54 \\
Gävle          & 23 & 37 & Östhammar     & 62 & 44 \\
Väddö          & 82 & 32 & Nacka         & 26, 23* & 43 \\
Sv. högarna    & 83 & 15 & Södertälje    & 66 & 30 \\
Torö           & 61 & 26 & Gotska sandön & 65 & 22 \\
Norrköping     & 64 & 43 & Västervik     & 23 & 45 \\
Fårö           & 62 & 25 & Visby         & 63 & 23 \\
Hoburgen       & 61 & 25 & Kalmar        & 60 & 40 \\
Ölands s. udde & 22 & 23 & Karlskrona    & 81 & 24 \\
Karlshamn      & 62 & 48 & Kivik         & 21 & 39\\
\end{tabular}
\caption{Stockholm radio, kanaler och radiohorisont för ostkusten}
\end{table}

*) Sänder ej väder, varningar eller andra listor

\clearpage
\subsubsection{Västkusten}

\begin{table}[h]
\centering
\begin{tabular}{lrr|lrr}
\textbf{Kuststation} & \textbf{Kanal} & \textbf{Horisont} &
\textbf{Kuststation} & \textbf{Kanal} & \textbf{Horisont} \\
\hline
Strömstad   & 22 & 25 & Grebbestad & 62 & 25 \\
Kungshamn   & 23 & 23 & Uddevalla  & 61 & 47 \\
Tjörn       & 81 & 26 & Göteborg   & 60 & 43 \\
Grimeton    & 22 & 35 & Halmstad   & 62 & 52 \\
Helsingborg & 60 & 28 & Malmö      & 65 & 25 \\
\end{tabular}
\caption{Stockholm radio, kanaler och radiohorisont för västkusten}
\end{table}

\subsubsection{Insjöarna}

\begin{table}[h]
\centering
\begin{tabular}{lrr|lrr}
\textbf{Kuststation} & \textbf{Kanal} & \textbf{Horisont} &
\textbf{Kuststation} & \textbf{Kanal} & \textbf{Horisont} \\
\hline
Västerås  & 63 & 40 & Trollhättan & 03 & 32 \\
Bäckefors & 05 & 50 & Kinnekulle  & 01 & 43 \\
Karlstad  & 65 & 36 & Jönköping   & 23 & 49 \\
Motala    & 62 & 47 & Hjälmaren     & $\dagger$  &    \\
\end{tabular}
\caption{Stockholm radio, kanaler och radiohorisont för insjöarna}
\end{table}

$\dagger$) Hjälmaren är permanent tagen ur drift.

\section{Frekvenser amatörradio VHF--UHF}

I denna skrift försöker vi omfatta de viktigaste VHF och UHF-banden
för amatörradio vilket inkluderar 6\,m-bandet, 2\,m-bandet,
70\,cm-bandet och 23\,cm-bandet.

Följande bandplaner gäller amatörradio och bygger på både nationella och
internationella överenskommelse via olika amatörradioföreningar (i Sverige
representerade av SSA) samt delar av IARU mm.

\subsection{Kanalnumrering VHF/UHF}

Denna typ av kanalnumrering är överenskommen inom IARU region 1 för
6\,m, 2\,m och 70\,cm banden på
amatörradiofrekvenser. Kanalnumreringen består av ett prefix som anger
vilket band och här används F--6\,m, V--2\,m, U--70\,cm. Därefter
används 2 siffror på 6m och 2m banden och tre siffror på 70cm bandet
för att ange kanal.

Repeaterfrekvenser anges med tillägget R före kanalnumret och innebär
då normalt duplex med det skift som normalt används för bandet. Vid
repeatrar är det repeaterns utfrekvens som anges, dvs den som
mobilstationen lyssnar på. Exempel: RV48.

\begin{table}[h]
\centering
\begin{tabular}{crrlll}
	\textbf{Band} & \textbf{Startfrekvens} & \textbf{Kanalraster} & \textbf{Duplex} & \textbf{Första kanal} & \textbf{Beräknas}    \\ \hline
	    6\,m      &            51.000\,MHz &            10.0\,kHz & -100\,kHz       & F00                   & $f=51+k\cdot0.01$    \\
	              &                        &                      &                 &                       & $k=(f-51)/0,01$      \\ \hline
	    2\,m      &           145.000\,MHz &            12.5\,kHz & -600\,kHz       & V00                   & $f=145+k\cdot0.0125$ \\
	              &                        &                      &                 &                       & $k=(f-145)/0,0125$   \\ \hline
	   70\,cm     &           430.000\,MHz &            12.5\,kHz & -2000\,kHz      & U000                  & $f=430+k\cdot0.0125$ \\
	                                       &                      &                 &                       & $k=(f-430)/0,0125$   \\ \hline
\end{tabular}
\caption{Kanalnumrering amatörradion VHF/UHF}
\end{table}

Eftersom amatörradiobanden ser lite olika ut i olika länder förekommer det
kanaler i numreringen som inte är tillåtna på vissa ställen. Det är därför
viktig att kontrollera att man fortfarande följer bandplanerna i den region
man är.

\begin{itemize}
\item I 6\,m bandet finns inga FM-kanaler definierade under 51\,MHz. \item För
2\,m-bandet är FM-kanaler endast definierade från 145\,MHz och uppåt. \item I
70\,cm-bandet är inga kanaler definierade i intervallet 432.000--433.000\,MHz.
Observera att startfrekvensen är utanför 70\,cm bandplanen i IARU region 1.
\end{itemize}

\textit{OBS!\\ Information om kanalnumreringen för 23\,cm-bandet tas tacksamt
mot. Maila mig på anders@sikvall.se om du har korrekt information.}

\subsection{Införande av 12,5\,kHz kanalavstånd}

För ett antal år sedan beslutade man sig att gå mot ett smalare kanalraster på
VHF och UHF och införde härmed kanalavstånd på 12,5~kHz i bandplanerna.
Ustrustning med 25~kHz kanalraster är fortsatt tillåten och detta är en
rekommendation. Vid införandet av detta så kom även ett nytt
numreringsalternativ för kanalsystemen baserat på en basfrekvens (som ibland
på svenska band ligger utanför vårt band) och därefter numrerade man i ordning
för respektive 12,5~kHz steg och 10~kHz för kortvåg.

\begin{table}[h]
\centering
\begin{tabular}{rrrr}
	\bf Kod & \bf Basfrekvens & \bf Kanalavstånd & \bf Repeaterskift \\
	        &      \bf  [MHz] &        \bf [kHz] &         \bf [kHz] \\ \hline
	      H &          29,500 &             10,0 &              -100 \\
	      F &          51,000 &             10,0 &              -600 \\
	      V &         145,000 &             12,5 &              -600 \\
	      U &         430,000 &             12,5 &             -2000 \\
	      M &        1240,000 &               25 &             -6000
\end{tabular}
\label{tab:kanalavstand}
\caption{Kanalavstånd och beteckning olika frekvensband}
\end{table}

\subsection{FM-kanaler 6m-bandet}

\begin{longtable}{rrl|rrl}
\caption{FM-kanaler 6-metersbandet}\\
\textbf{Kanal} & \textbf{Tidigare} & \textbf{Anm}
&  \textbf{Kanal} & \textbf{Tidigare} & \textbf{Anm} \\ \hline
	51,500 &      F50 &       & 51,750 &      F75 &  \\
	51,510 &      F51 & Anrop & 51,760 &      F76 &  \\
	51,520 &      F52 &       & 51,770 &      F77 &  \\
	51,530 &      F53 &       & 51,780 &      F78 &  \\
	51,540 &      F54 &       & 51,790 &      F79 &  \\
	51,550 &      F55 &       & 51,800 &      F80 &  \\
	51,560 &      F56 &       & 51,810 &     RF81 &  \\
	51,570 &      F57 &       & 51,820 &     RF82 &  \\
	51,580 &      F58 &       & 51,830 &     RF83 &  \\
	51,590 &      F59 &       & 51,840 &     RF84 &  \\
	51,600 &      F60 &       & 51,850 &     RF85 &  \\
	51,610 &      F61 &       & 51,860 &     RF86 &  \\
	51,620 &      F62 &       & 51,870 &     RF87 &  \\
	51,630 &      F63 &       & 51,880 &     RF88 &  \\
	51,640 &      F64 &       & 51,890 &     RF89 &  \\
	51,650 &      F65 &       & 51,900 &     RF90 &  \\
	51,660 &      F66 &       & 51,910 &     RF91 &  \\
	51,670 &      F67 &       & 51,920 &     RF92 &  \\
	51,680 &      F68 &       & 51,930 &     RF93 &  \\
	51,690 &      F69 &       & 51,940 &     RF94 &  \\
	51,700 &      F70 &       & 51,950 &     RF95 &  \\
	51,710 &      F71 &       & 51,960 &     RF96 &  \\
	51,720 &      F72 &       & 51,970 &     RF97 &  \\
	51,730 &      F73 &       & 51,980 &     RF98 &  \\
	51,740 &      F74 &       & 51,990 &     RF99 &
\end{longtable}

\subsection{FM-kanaler 2 m-bandet}

\begin{longtable}{rrl|rrl}
\caption{FM-kanaler 2\,m-bandet}\\
\textbf{Frekvens} & \textbf{Kanal} & \textbf{Anm} &
\textbf{Frekvens} & \textbf{Kanal} & \textbf{Anm} \\ \hline

145,2125 & V17 &              & 145,5000 & V40  & S20  FM Anrop \\
145,2250 & V18 & S9           & 145,5125 & V41  &               \\
145,2375 & V19 & INET GW      & 145,5250 & V42  & S21           \\
145,2500 & V20 & S10          & 145,5375 & V43  &               \\
145,2625 & V21 &              & 145,5500 & V44  & S22           \\
145,2750 & V22 & S11          & 145,5625 & V45  &               \\
145,2875 & V23 & INET GW      & 145,5750 & V46  & S23           \\
145,3000 & V24 & S12  RTTY    & 145,5875 & V47  &               \\
145,3125 & V25 &              & 145,6000 & RV48 & R0            \\
145,3250 & V26 & S13          & 145,6125 & RV49 & R0X           \\
145,3375 & V27 & INET GW      & 145,6250 & RV50 & R1            \\
145,3500 & V28 & S14          & 145,6375 & RV51 & R1X           \\
145,3625 & V29 &              & 145,6500 & RV52 & R2            \\
145,3750 & V30 & S15 DV Anrop & 145,6625 & RV53 & R2X           \\
145,3875 & V31 &              & 145,6750 & RV54 & R3            \\
145,4000 & V32 & S16          & 145,6875 & RV55 & R3X           \\
145,4125 & V33 &              & 145,7000 & RV56 & R4            \\
145,4250 & V34 & S17 Scout    & 145,7125 & RV57 & R4X           \\
145,4375 & V35 &              & 145,7250 & RV58 & R5            \\
145,4500 & V36 & S18          & 145,7375 & RV59 & R5X           \\
145,4625 & V37 &              & 145,7500 & RV60 & R6            \\
145,4750 & V38 & S19          & 145,7625 & RV61 & R6X           \\
145,4875 & V39 &              & 145,7750 & RV62 & R7            \\
         &     &              & 145,7875 & RV63 & R7X
\end{longtable}

X-kanalerna uppstod när man fick platsbrist och man övergick till en
12.5\,kHz kanaldelning för repeatrar. Först senare övergick man även
till samma kanaldelning på övriga FM-kanaler. De gamla
simplexkanalerna hade inte så stor spridning i Sverige men förekom
rikligt t.ex. i Tyskland med S20 som anropsfrekvens (eller
aktivitetscenter som det numera kallas).


\subsection{FM-kanaler 70\,cm-bandet}

\small

\begin{longtable}{rrl|rrl}
\caption{FM-kanaler amatörradion 70\,cm-bandet}\\
\textbf{Frekvens} & \textbf{Kanal} & \textbf{Anm} &
\textbf{Frekvens} & \textbf{Kanal} & \textbf{Anm} \\ \hline

433,4000 & U272 & SSTV    & 433,7125 & U297 &      \\
433,4125 & U273 &         & 433,7250 & U298 &      \\
433,4250 & U274 &         & 433,7375 & U299 &      \\
433,4375 & U275 &         & 433,7500 & U300 &      \\
433,4500 & U276 & Digital & 433,7625 & U301 &      \\
433,4625 & U277 &         & 433,7750 & U302 &      \\
433,4750 & U278 &         & 433,7875 & U303 &      \\
433,4875 & U279 &         & 433,8000 & U304 & APRS \\
433,5000 & U280 & Anrop   & 433,8125 & U305 &      \\
433,5125 & U281 &         & 433,8250 & U306 &      \\
433,5250 & U282 &         & 433,8375 & U307 &      \\
433,5375 & U283 &         & 433,8500 & U308 &      \\
433,5500 & U284 &         & 433,8625 & U309 &      \\
433,5625 & U285 &         & 433,8750 & U310 &      \\
433,5750 & U286 &         & 433,8875 & U311 &      \\
433,5875 & U287 &         & 433,9000 & U312 &      \\
433,6000 & U288 & RTTY    & 433,9125 & U313 &      \\
433,6125 & U289 &         & 433,9250 & U314 &      \\
433,6250 & U290 &         & 433,9375 & U315 &      \\
433,6375 & U291 &         & 433,9500 & U316 &      \\
433,6500 & U292 &         & 433,9625 & U317 &      \\
433,6625 & U293 &         & 433,9750 & U318 &      \\
433,6750 & U294 &         & 433,9875 & U319 &      \\
433,6875 & U295 &         & 434,0000 & U320 &      \\
433,7000 & U296 & FAX     &          &      &      \\

\end{longtable}

\begin{longtable}{rrl|rrl}
\caption{Amatörradio repeaterkanaler 70 cm-bandet}\\
\textbf{Frekvens} & \textbf{Kanal} & \textbf{Anm}
&  \textbf{Frekvens} & \textbf{Kanal} & \textbf{Anm} \\ \hline

434,6000 & RU368 & RU0  & 434,8000 & RU384 & RU8   \\
434,6125 & RU369 & RU0X & 434,8125 & RU385 & RU8X  \\
434,6250 & RU370 & RU1  & 434,8250 & RU386 & RU9   \\
434,6375 & RU371 & RU1X & 434,8375 & RU387 & RU9X  \\
434,6500 & RU372 & RU2  & 434,8500 & RU388 & RU10  \\
434,6625 & RU373 & RU2X & 434,8625 & RU389 & RU10X \\
434,6750 & RU374 & RU3  & 434,8750 & RU390 & RU11  \\
434,6875 & RU375 & RU3X & 434,8875 & RU391 & RU11X \\
434,7000 & RU376 & RU4  & 434,9000 & RU392 & RU12  \\
434,7125 & RU377 & RU4X & 434,9125 & RU393 & RU12X \\
434,7250 & RU378 & RU5  & 434,9250 & RU394 & RU13  \\
434,7375 & RU379 & RU5X & 434,9375 & RU395 & RU13X \\
434,7500 & RU380 & RU6  & 434,9500 & RU396 & RU14  \\
434,7625 & RU381 & RU6X & 434,9625 & RU397 & RU14X \\
434,7750 & RU382 & RU7  & 434,9750 & RU398 & RU15  \\
434,7875 & RU383 & RU7X & 434,9875 & RU399 & RU15X \\
         &       &      & 435,0000 & RU400 &       \\
\end{longtable}

\normalsize

%RU0X osv är här en efterkonstruktion. Egentligen så användes sällan
%``X-frekvenserna'' på 70cm eftersom man dels hade nästan dubbla
%antalet frekvenser för repeatrar och sedan gammalt ville man
%egentligen inte ha ett smalare kanalraster, i tidernas begynnelse
%körde många amatörer 70cm genom frekvenstrippling från 2m. $144,000
%\cdot 3 = 432,000$\,MHz och $144,025 \cdot 3 = 432,075$\,MHz varför
%man till och med hade bredare kanalraster de-facto.

\clearpage

\begin{landscape}

\subsection{Repeatrar distrikt 0}

\begin{longtable}{llllrrlll}
	\bf Typ           & \bf Mod            & \bf Call & \bf Ort         & \bf Frekvens & \bf Duplex & \bf Access & \bf Lokator & \bf QRV? \\ \hline
	\endhead
Repeater & C4FM               & SKØQO    & Bagarmossen     &     434.5750 &     -2.000 &            & JO99BG      & QRV      \\
	Repeater          & DMR                & SMØWIU/R & Botkyrka        &     434.8750 &     -2.000 & CC 0       & JO89WG      & QRV      \\
	Repeater          & FM                 & SKØRDZ   & Brottby         &     145.6500 &     -0.600 & 1750/77.0  & JO99DN      & QRV      \\
	Repeater          & FM                 & SAØAZT/R & Brottby         &     434.8000 &     -2.000 & 1750/77.0  & JO99BM      & QRV      \\
	Repeater          & DMR                & SMØWIU/R & Dalarö          &     434.8375 &     -2.000 & CC 0       & JO99ED      & QRV      \\
	Repeater          & DMR                & SKØVR    & Djurö           &     434.5875 &     -2.000 & CC 0       & JO99IH      & QRT      \\
	Link              & FM                 & SKØRVF   & Hagsätra        &     434.4250 &      Simpl & 91.5       & JO99AG      & QRV      \\
	Repeater          & FM                 & SKØNN/R  & Haninge         &     434.7750 &     -2.000 & 77.0       & JO99CE      & QRV      \\
	Repeater          & C4FM               & SKØNN    & Haninge         &     434.5375 &     -2.000 &            & JO99CF      & QRV      \\
	Repeater          & DMR                & SKØNN-2  & Haninge         &     434.6375 &     -2.000 & CC 0       & JO99CE      & QRV      \\
	Repeater          & FM                 & SKØQO/R  & Haninge/Brandb. &     145.6875 &     -0.600 & 77.0       & JO99BE      & QRV      \\
	Repeater          & FM                 & SKØQO/R  & Haninge/Brandb. &     434.7500 &     -2.000 & 77.0       & JO99BE      & QRV      \\
	Repeater          & DMR                & SKØQO    & Haninge/Brandb. &     434.5625 &     -2.000 & CC 0       & JO99BE      & QRV      \\
	Repeater          & FM                 & SKØBJ/R  & Huddinge        &     434.6000 &     -2.000 & 123.0      & JO89XF      & QRV      \\
	Repeater          & DMR                & SMØWIU-4 & Högdalen        &     145.5750 &     -0.600 & CC 0       & JO99AF      & QRV      \\
	Repeater          & FM                 & SKØMM/R  & Ingarö          &     145.7750 &     -0.600 & 77.0       & JO99GG      & QRV      \\
	Repeater          & DMR                & SKØNN/1  & Johanneshov     &     434.9250 &     -2.000 & CC 0       & JO99AH      & QRV      \\
	Repeater          & FM                 & SKØCT/R  & Kista           &    1297.0250 &     -6.000 & Carrier    & JO89XJ      & QRV      \\
	Repeater          & FM                 & SKØPQ/R  & Kista           &     145.6750 &     -0.600 & 77.0       & JO89XJ      & QRV      \\
	Repeater          & FM                 & SKØCT/R  & Kista           &     434.6250 &     -2.000 & 77.0       & JO89XJ      & QRV      \\
	Repeater          & DMR                & SKØRYG   & Kista           &     434.9500 &     -2.000 & CC 0       & JO89XJ      & QRV      \\
	Repeater          & DMR                & SKØSX    & Kista           &     434.9875 &     -2.000 & CC 0       & JO89XJ      & QRV      \\
	Repeater          & FM                 & SKØCT/R  & Kista           &     145.7875 &     -0.600 & 77.0       & JO89XJ      & QRV      \\
	Link              & FM                 & SMØUAO   & Kopparmora      &     434.4875 &            & 91.5       & JO99HI      & QRV      \\
	Hotspot           & D-Star             & SEØYOS-C & M/Y Erika       &     434.4500 &     Dupl 0 & DV Carrier & JO99AH      & QRV      \\
	Repeater          & FM/DMR             & SAØAZT   & Norrtälje       &     434.8125 &     -2.000 & 77.0/CC 0  & JO99IS      & QRV      \\
	Repeater          & FM                 & SKØBJ    & Nynäshamn       &     434.9125 &     -2.000 & 123.0      & JO88WT      & Plan     \\
	Repeater          & FM                 & SKØBJ    & Nynäshamn       &     145.7375 &     -0.600 & 123.0      & JO88WT      & QRT      \\
	Repeater          & DMR                & SMØWIU/R & Nynäshamn       &     434.6125 &     -2.000 & CC 0       & JO88XV      & QRV      \\
	Repeater          & FM                 & SKØBJ/R  & Nynäshamn       &     145.7125 &     -0.600 & 123.0      & JO88XV      & QRV      \\
	Repeater          & FM                 & SKØBJ/R  & Nynäshamn       &     434.7125 &     -2.000 & 123.0      & JO88XV      & QRV      \\
	Repeater          & FM                 & SKØBJ/R  & Nynäshamn       &     434.6500 &     -2.000 & 123.0      & JO89XF      & QRV      \\
	Repeater          & DMR                & SGØRPF   & Rimbo           &     434.7875 &     -2.000 & CC 0       & JO99BT      & QRT      \\
	Link              & FM                 & SKØMM    & Sandhamn        &     434.3750 &            & 91.5       & JO99KG      & QRV      \\
	Repeater          & FM/DMR             & SKØRPF   & Sigtuna         &     434.8875 &     -2.000 & 123.0/CC 0 & JO89VP      & QRV      \\
	Repeater          & FM                 & SMØOFV/R & Solna           &     145.7625 &     -0.600 & 123.0      & JO89XI      & QRV      \\
	Repeater          & FM                 & SKØZA/R  & Solna           &     434.8500 &     -2.000 & 123.0      & JO89XI      & QRV      \\
	Hotspot           & D-Star             & SKØAI-B  & Stockholm       &     433.4625 &      Simpl &            & JO89XG      & QRV      \\
	Repeater          & FM                 & SKØNN/R  & Stockholm       &     145.7250 &     -0.600 & 77.0       & JO99AH      & QRV      \\
	Repeater          & FM                 & SKØRIX   & Stockholm c     &     145.6250 &     -0.600 & 77.0/CC 0  & JO99AH      & Plan     \\
	Repeater          & DMR                & SKØRYG   & Stockholm c     &     434.9625 &     -2.000 & CC 0       & JO99AI      & QRV      \\
	Repeater          & FM                 & SMØYIX/R & Söder           &     434.7250 &     -2.000 & 77.0       & JO99BH      & QRV      \\
	Repeater          & DMR                & SKØMG    & Södertälje      &     434.7875 &     -2.000 &            & JO89TE      & Plan     \\
	Repeater          & FM                 & SM5DWC/R & Södertälje      &     434.8250 &     -2.000 & 1750/77.0  & JO89TE      & QRV      \\
	Repeater          & C4FM               & SKØMG/R  & Södertälje      &     145.7000 &     -0.600 & 77.0       & JO89TE      & QRV      \\
	Repeater          & DMR                & SMØWIU-2 & Södertälje      &     434.8750 &     -2.000 & CC 0       & JO89TE      & QRV      \\
	Repeater          & FM                 & SMØMMO/R & Tullinge        &     145.6625 &     -0.600 & 77.0       & JO89XF      & QRT      \\
	Repeater          & DMR                & SKØRMQ   & Tyresö          &     434.5125 &     -2.000 & CC 12      & JO99CH      & QRV      \\
	Repeater          & FM/DMR/C4FM/D-Star & SKØRMT   & Täby            &     434.7375 &     -2.000 & 77.0/CC 0  & JO99AK      & QRV      \\
	Repeater          & DMR                & SKØRYG   & Uppl. Väsby     &     434.7625 &     -2.000 & CC 0       & JO89XM      & QRV      \\
	Repeater          & FM                 & SKØRYG   & Uppl. Väsby     &     434.6750 &     -2.000 & 77.0       & JO89XM      & QRV      \\
	Repeater          & FM                 & SKØYZ/R  & Vallentuna      &     434.8625 &     -2.000 & 77.0       & JO99BM      & QRV      \\
	Repeater          & FM/DMR             & SAØAZT   & Vallentuna      &     434.5500 &     -2.000 & CC 0       & JO99EO      & QRV      \\
	Repeater          & FM                 & SKØMT/R  & Vallentuna      &     434.7000 &     -2.000 & 77.0       & JO99BM      & QRV      \\
	Repeater          & FM/DMR             & SKØVR    & Värmdö          &     434.9750 &     -2.000 & 77.0/CC 0  & JO99FH      & QRV      \\
	Repeater          & FM                 & SLØZS/R  & Västberga       &     145.6000 &     -0.600 & 123.0      & JO89XH      & QRV      \\
	Repeater          & FM                 & SLØZS/R  & Västberga       &     434.9000 &     -2.000 & 123.0      & JO89XH      & QRV      \\
	Repeater          & DMR                & SKØEN    & Älmsta          &     434.6000 &     -2.000 & CC 0       & JO99JX      & QRV      \\
	Repeater          & FM/DMR             & SKØEN    & Älmsta          &     145.7375 &     -0.600 & 77.0/CC 0  & JO99JX      & QRV
\end{longtable}


\clearpage

\subsection{Repeatrar distrikt 1}


\begin{longtable}{llllrrlll}
\bf Typ  & \bf Mod     & \bf Call & \bf Ort     & \bf Frekvens & \bf Duplex & \bf Access & \bf Lokator & \bf QRV? \\ \hline
Repeater   & FM / C4FM & SK1RGU   & Endre        & 145.7750     & -0.600     & 1750 / 218.1 & JO97FO      & QRV      \\
Repeater   & FM        & SL1ZXK/R & Slite        & 434.6000     & -2.000     & 218.1        & JO97JR      & QRV      \\
\end{longtable}



\subsection{Repeatrar distrikt 2}



\begin{longtable}{llllrrlll}
\bf Typ  & \bf Mod     & \bf Call & \bf Ort     & \bf Frekvens & \bf Duplex & \bf Access & \bf Lokator & \bf QRV? \\ \hline \endhead
Repeater   & FM          & SK2AU/R   & Arjeplog           & 145.7000     & -0.600     & 1750       & JP86XC      & QRV      \\
           &             &           & Galtispouda        &              &            &            &             &          \\
Repeater   & FM          & SK2CI     & Boden              & 145.6250     & -0.600     & 1750/107.2 & KP05SS      & QRV      \\
Repeater   & DMR         & SK2CI     & Boden              & 434.8000     & -2.000     & CC 2       & KP05TT      & QRV      \\
Repeater   & FM          & SJ2W      & Burträsk           & 434.9500     & -2.000     & 107.2      & KP04HM      & QRV      \\
Repeater   & FM          & SK2AU/R   & Jörn/Storklinta    & 145.7500     & -0.600     & 1750       & KP05BD      & QRT      \\
Repeater   & DMR         & SK2HG-2   & Kalix              & 434.9875     & -2.000     & CC 2       & KP15OU      & QRV      \\
Repeater   & FM          & SK2HG/R   & Kalix/Raggdynan    & 51.9500      & -0.600     & 107.2      & KP15KW      & QRV      \\
Repeater   & DMR/D-Star  & SK2RJH    & Kalix/Raggdynan    & 434.7500     & -2.000     & CC 2       & KP15KW      & QRV      \\
Repeater   & FM/DMR      & SK2HG/R5  & Kalix/Raggdynan    & 145.7250     & -0.600     & 107.2/CC 2 & KP15KW      & QRV      \\
Repeater   & FM/DMR      & SK2HG/RU5 & Kalix Vattentorn   & 434.7250     & -2.000     & 107.2/CC 2 & KP15NU      & QRV      \\
Repeater   & FM          & SK2RFR    & Kiruna             & 145.6250     & -0.600     & 107.2      & KP07CT      & QRV      \\
Repeater   & FM          & SK2RFR    & Kiruna             & 434.8250     & -2.000     & 107.2      & KP07CT      & QRV      \\
Repeater   & DMR         & SK2RGJ    & Kiruna             & 434.5125     & -2.000     & CC 2       & KP07CT      & QRV      \\
Link       & FM          & SM2YUW    & Kiruna             & 434.4000     & Simplex    & Carrier    & KP07DU      & QRV      \\
Repeater   & FM          & SM2KOT/R  & Kristineb./Viterliden          & 145.6750     & -0.600     & 1750       & JP95HB      & QRT      \\
Repeater   & FM          & SK2DR/R   & Luleå              & 145.6500     & -0.600     & 107.2      & KP15CO      & QRV      \\
Repeater   & DMR/D-Star  & SK2DR     & Luleå              & 434.9000     & -2.000     & CC 2       & KP15CO      & QRV      \\
Repeater   & FM          & SK2LY/R   & Lycksele           & 145.7750     & -0.600     & 107.2      & JP94IO      & QRT      \\
Repeater   & FM          & SK2AZ/R   & Piteå              & 145.6000     & -0.600     & 107.2      & KP05PH      & QRV      \\
Repeater   & FM          & SK2RME    & Piteå              & 434.6000     & -2.000     & 1750/107.2 & KP05RH      & QRV      \\
Repeater   & DMR         & SK2AZ     & Piteå              & 434.8500     & -2.000     & CC 2       & KP05PH      & QRV      \\
Repeater   & FM / DMR    & SK2HG/R3  & Seskarö            & 145.6750     & -0.600     & 107.2/CC 2 & KP15UR      & QRV      \\
Repeater   & FM          & SK2AU/R   & Skellefteå         & 145.7000     & -0.600     & 1750       & KP04LS      & QRV      \\
Repeater   & FM/DMR/C4FM/D-Star      & SK2AU/R   & Skellefteå         & 145.5875     & -0.600     & 107.2      & KP04LS      & QRV      \\
Repeater   & FM          & SJ2W/R    & Skellefteå         & 434.6750     & -2.000     & 1750/107.2 & KP04LS      & QRV      \\
Repeater   & FM          & SK2RMD    & Sorsele            & 145.6000     & -0.600     & 1750       & JP85SM      & QRV      \\
Repeater   & FM          & SK2RMR    & Storuman           & 145.7250     & -0.600     & 1750       & JP85NC      & QRV      \\
Repeater   & FM          & SK2RMR    & Storuman           & 434.7500     & -2.000     & 1750/107.2 & JP85NC      & QRV      \\
Repeater   & FM          & SK2RLF    & Tärnaby            & 145.6250     & -0.600     & 1750       & JP75PR      & QRT      \\
Repeater   & FM          & SK2RLJ    & Umeå/Rödberget     & 145.6500     & -0.600     & 1750       & KP03CU      & QRV      \\
Repeater   & FM          & SK2RYI    & Vindeln/Åsträsk    & 145.6250     & -0.600     & 1750       & KP04DP      & QRV      \\
Repeater   & DMR         & SK2AT     & Vännäs             & 434.9750     & -2.000     & CC 2       & JP93XX      & QRV      \\
Repeater   & FM          & SK2RIU    & Vännäs/Granlundsb. & 145.7250     & -0.600     & 1750       & JP93VU      & QRV      \\
Repeater   & FM          & SK2RIU    & Vännäs/Granlundsb. & 434.7250     & -2.000     & 1750/107.2 & JP93VU      & QRV      \\
Repeater   & FM          & SK2RWJ    & Älvsbyn            & 145.6750     & -0.600     & 107.2      & KP05LQ      & QRV      \\
\end{longtable}



\clearpage

\subsection{Repeatrar distrikt 3}



\begin{longtable}{llllrrlll}
	\bf Typ           & \bf Mod         & \bf Call & \bf Ort             & \bf Frekvens & \bf Duplex & \bf Access         & \bf Lokator & \bf QRV? \\ \hline
	\endhead
Repeater & FM              & SK3RMG   & Bergsjö             &    1297.1000 &     -6.000 & 1750               & JP81MX      & QRV      \\
	Repeater          & FM              & SK3RET   & Bollnäs/Arbrå       &     145.6500 &     -0.600 & 1750/127.3/DTMF *5 & JP81CL      & QRV      \\
	Repeater          & FM              & SK3RIN   & Borgsjö             &     145.7000 &     -0.600 & 1750               & JP72WN      & QRV      \\
	Repeater          & FM              & SK3PH/R  & Delsbo              &      29.6900 &     -0.100 &                    & JP81GT      & Plan     \\
	Repeater          & FM              & SK3YZ/R  & Forsa               &     145.6000 &     -0.600 & 127.3              & JP81KQ      & QRV      \\
	Repeater          & FM              & SK3RQE   & Forsa/Storberget    &     434.6750 &     -2.000 & 1750/127.3         & JP81KQ      & QRV      \\
	Repeater          & FM              & SK3RQE   & Forsa/Storberget    &     145.6750 &     -0.600 & 1750/127.3         & JP81KQ      & QRV      \\
	Repeater          & DMR/C4FM/D-Star & SM3YFX   & Föllinge            &     434.5250 &     -2.000 & CC 3               & JP73HQ      & QRV      \\
	Repeater          & FM              & SK3GW    & Gävle               &     434.8750 &     -2.000 & 1750/127.3/DTMF *  & JP80NP      & QRV      \\
	Repeater          & DMR             & SK3GK    & Gävle               &     434.7000 &     -2.000 & CC 3               & JP80NP      & QRV      \\
	Repeater          & FM              & SK3RMX   & Hoting/Kyrktåsjö    &     145.6000 &     -0.600 & 1750               & JP74XF      & QRV      \\
	Hotspot           & D-Star          & SK3GA-B  & Hudiksvall          &     434.4750 &   Duplex 0 & DV Carrier         & JP81NR      & QRT      \\
	Repeater          & FM              & SK3GA/R  & Hudiksvall          &     145.7750 &     -0.600 & 127.3              & JP81NR      & QRT      \\
	Repeater          & FM/DMR          & SK3RHU   & Hudiksvall          &     145.7125 &     -0.600 & 127.3/CC 3         & JP81NR      & QRV      \\
	Repeater          & DMR             & SK3RHU   & Hudiksvall          &     434.5750 &     -2.000 & CC 3               & JP81NR      & QRV      \\
	Repeater          & FM              & SL3ZB    & Härnösand           &     434.7250 &     -2.000 & 1750               & JP82XP      & QRV      \\
	Repeater          & FM              & SK3WH    & Högakustenbron      &    1297.2750 &     -6.000 & 1750               & JP82XT      & QRV      \\
	Repeater          & FM              & SM3XRJ   & Kramfors            &     434.6000 &     -2.000 & 1750               & JP82VW      & QRV      \\
	Link              & FM              & SM3KDR   & Krokom/Aspås        &     434.9750 &    Simplex & 127.3              & JP73GI      & QRV      \\
	Repeater          & FM              & SK3MF/R  & Nordingrå/Rävsön    &     145.6250 &     -0.600 & 1750               & JP92FW      & QRV      \\
	Repeater          & FM              & SK3MF/R  & Nordingrå/Rävsön    &     434.8500 &     -2.000 & 1750               & JP92FW      & QRV      \\
	Repeater          & FM              & SM3VAC/R & Nyland              &     145.7500 &     -0.600 & 1750               & JP83UA      & QRV      \\
	Repeater          & FM              & SM3VAC/R & Nyland              &     434.9500 &     -1.600 & 1750               & JP83UA      & QRV      \\
	Repeater          & FM              & SK3GK    & Sandviken           &     434.8250 &     -2.000 & 1750/127.3/DTMF *  & JP80FS      & QRV      \\
	Repeater          & FM              & SK3GK/R  & Sandviken           &     145.7000 &     -0.600 & 1750/127.3/DTMF *  & JP80FS      & QRV      \\
	Repeater          & FM              & SK3EK/R  & Sollefteå           &     434.9250 &     -2.000 & 1750/127.3         & JP83DE      & Plan     \\
	Repeater          & FM              & SK3EK/R  & Sollefteå           &     434.6500 &     -1.600 & 1750               & JP83DE      & QRT      \\
	Repeater          & FM              & SK3EK/R  & Sollefteå           &     145.6500 &     -0.600 & 1750               & JP83PD      & QRV      \\
	Repeater          & DMR/C4FM/D-Star & SG9NN    & Sundsvall           &     434.5375 &     -2.000 & CC 3               & JP82OJ      & QRT      \\
	Repeater          & FM              & SK3RFG   & Sundsvall           &     145.7250 &     -0.600 & 1750/127.3         & JP82RJ      & QRV      \\
	Repeater          & D-Star          & SK3RFG-C & Sundsvall/Klissb.   &     145.5875 &     -0.600 & DV Carrier         & JP82OJ      & QRV      \\
	Repeater          & DMR/D-Star      & SK3RFG   & Sundsvall/Klissb.   &     434.8000 &     -2.000 & CC 1               & JP82OJ      & QRV      \\
	Repeater          & DMR             & SK3RFG   & Sundsvall/Nolby     &     434.9875 &     -2.000 & CC 3               & JP82QH      & QRV      \\
	Repeater          & FM              & SK3RYK   & Söderhamn           &     145.7500 &     -0.600 & 1750               & JP81NH      & QRV      \\
	Repeater          & FM              & SK3RYK   & Söderhamn           &     434.7500 &     -1.600 & 1750               & JP81NH      & QRV      \\
	Repeater          & FM              & SK3RQC   & Vemdalen            &     145.6250 &     -0.600 & 1750/74.4          & JP62WK      & QRT      \\
	Repeater          & FM              & SK3RNJ   & Åreskutan           &     145.7250 &     -0.600 &                    & JP63NK      & QRT      \\
	Repeater          & FM              & SM3LEI/R & Årsunda             &     434.6500 &     +1.600 & 1750/88.5/DTMF 1   & JP80IM      & QRV      \\
	Repeater          & FM              & SK3LH/R  & Örnsköldsvik        &     434.8750 &     -2.000 & 1750/127.3/DTMF 3  & JP93IH      & QRV      \\
	Repeater          & DMR             & SK3WH    & Örnsköldsvik        &     145.5750 &     -0.600 & CC 3               & JP93IH      & QRV      \\
	Repeater          & D-Star          & SK3LH-B  & Örnsköldsvik/Malmön &     434.5750 &     -2.000 & DV Carrier         & JP93LF      & QRV      \\
	Repeater          & FM              & SK3RKL   & Örnsköldsvik/Rutb.  &     145.7750 &     -0.600 & 1750               & JP93GJ      & QRV      \\
	Repeater          & FM              & SK3W     & Österfärnebo        &     434.8500 &     -2.000 & 127.3              & JP80JH      & QRV      \\
	Repeater          & FM              & SK3RIA   & Östersund           &     434.7500 &     -2.000 & 127.3              & JP73JE      & QRV      \\
	Repeater          & FM/C4FM         & SK3JR/R  & Östersund           &     145.7500 &     -0.600 & 127.3              & JP73JE      & QRV      \\
	Repeater          & FM/C4FM         & SK3JR/R2 & Östersund/Brattåsen &     145.7875 &     -0.600 & 127.3              & JP73HC      & QRV
\end{longtable}


\clearpage

\subsection{Repeatrar distrikt 4}


\begin{longtable}{llllrrlll}
	\bf Typ               & \bf Mod    & \bf Call & \bf Ort                & \bf Frekvens & \bf Duplex & \bf Access        & \bf Lokator & \bf QRV? \\ \hline
	\endhead
    Repeater & DMR        & SK3JR    & Östersund/Brattåsen    &     434.5625 &     -2.000 & CC 3              & JP73HC      & QRV      \\
	Repeater              & FM         & SK3RQE   &                        &     145.6000 &     -0.600 &                   & JP81NV      & QRV      \\
	Repeater              & DMR        & SA4BNA   & Arvika                 &     434.9750 &     -2.000 & CC 4              & JO69GN      & QRV      \\
	Repeater              & FM         & SK4RWQ   & Arvika/Valfjället      &     434.7750 &     -2.000 & 1750/74.4         & JO69CT      & QRV      \\
	Repeater              & D-Star     & SK4BW-B  & Borlänge               &     434.9000 &     -2.000 & DV Carrier        & JP70RJ      & QRV      \\
	Hotspot               & D-Star     & SG4UZM-B & Borlänge               &     434.5500 &   Duplex 0 & DV Carrier        & JP70RM      & QRV      \\
	Repeater              & FM/C4FM    & SK4RVN   & Borlänge               &     434.8000 &     -2.000 & 85.4              & JP70RJ      & QRV      \\
	Repeater              & DMR        & SK4BW    & Borlänge               &     434.8500 &     -2.000 & CC 12             & JP70RJ      & QRV      \\
	Hotspot               & DMR        & SG4AXV   & Ekshärad               &     433.2000 &    Simplex & DV Carrier        & JP60RE      & QRV      \\
	Repeater              & D-Star     & SG4AXV   & Ekshärad               &     145.6000 &     -0.600 & DV Carrier        & JP60RE      & QRV      \\
	Repeater              & DMR        & SK4RGL   & Falun                  &     434.6250 &     -2.000 & CC 1              & JP70UP      & QRV      \\
	Repeater              & FM         & SK4RGL   & Falun                  &     145.6250 &     -0.600 & 1750/85.4         & JP70UP      & QRV      \\
	Link                  & FM         & SK4AV/R  & Filipstad/Klockarh.    &     145.2000 &            & Carrier           & JO79CR      & QRV      \\
	Repeater              & FM         & SK4BX/R  & Garphyttan/Storstensh. &     145.6500 &     -0.600 & 1750/74.4         & JO79LH      & QRV      \\
	Hotspot               & D-Star     & SG4UOF-C & Glanshammar            &     145.3375 &   Duplex 0 & DV Carrier        & JO79RI      & QRV      \\
	Repeater              & FM         & SK4IL/R  & Grums                  &     434.7250 &     -2.000 & 74.4              & JO69NI      & QRV      \\
	Link                  & FM         &          & Grängesberg            &     145.3500 &            &                   & JP70MB      & QRV      \\
	Repeater              & FM         & SK4HV/R  & Hagfors/Värmullsåsen   &     145.6750 &     -0.600 & 1750/114.8        & JP60VA      & QRV      \\
	Repeater              & FM/DMR     & SK4KR    & Karlskoga              &     434.8000 &     -2.000 & CC 4              & JO79FH      & QRV      \\
	Repeater              & FM/DMR     & SK4RKD   & Karlskoga              &     145.7500 &     -0.600 & 74.4/CC 4         & JO79FJ      & QRV      \\
	Repeater              & FM         & SK4AV    & Karlstad               &     145.5750 &     -0.600 &                   & JO69RK      & QRV      \\
	Repeater              & FM         & SK4EA/R  & Kopparberg             &     145.6000 &     -0.600 & 1750              & JO79MW      & QRV      \\
	Repeater              & FM         & SK4RUV   & Leksand                &     145.7750 &     -0.600 & 1750/85.4         & JP70MQ      & QRV      \\
	Repeater              & FM/DMR     & SM4WIU-3 & Leksand                &     434.6125 &     -2.000 & 85.4/CC4          & JP70MR      & QRV      \\
	Repeater              & FM         & SK4EA/R  & Lindesberg             &     145.6875 &     -0.600 & 1750/74.4         & JO79NP      & QRV      \\
	Link                  & FM         & SK4EA-L  & Lindesberg             &     145.3000 &    Simplex & 136.5             & JO79OO      & QRV      \\
	Repeater              & FM         & SK4DM/R  & Ludvika                &     145.7250 &     -0.600 & 1750              & JP70NC      & QRV      \\
	Repeater              & FM         & SK4DM/R  & Ludvika                &     434.7250 &     -1.600 & 1750/DTMF1        & JP70NC      & QRV      \\
	Hotspot               & DMR        & SA4ATZ   & Malung                 &     144.8375 &    Simplex & CC 4              & JP60UQ      & QRV      \\
	Repeater              & FM         & SM4JDP   & Mora                   &     434.7000 &     -2.000 & Carrier           & JP71GA      & QRV      \\
	Repeater              & D-Star     & SG4TYA   & Mora                   &     145.5750 &     -0.600 & DV Carrier        & JP71GE      & QRV      \\
	Repeater              & DMR        & SK4KO    & Nusnäs                 &     434.9250 &     -2.000 & CC 4              & JP70HW      & Plan     \\
	Link                  & FM         & SM4FBD   & Nybble                 &     145.3000 &    Simplex & Carrier           & JO79BC      & QRV      \\
	Link                  & FM         &          & Nyhammar               &     145.3250 &            &                   & JP70LG      & QRV      \\
	Link                  & FM         & SM4MXN   & Orsa                   &     145.2750 &    Simplex &                   & JP71HC      & QRV      \\
	Repeater              & FM         & SK4RGO   & Orsa/Grönklitt         &     434.7500 &     -2.000 & 1750/85.4         & JP71GF      & QRV      \\
	Repeater              & FM         & SK4RGO   & Orsa/Grönklitt         &     145.7500 &     -0.600 & 1750/85.4         & JP71GF      & QRV      \\
	Repeater              & DMR        & SA4BHE-R & Smedjebacken           &     434.6375 &     -2.000 & CC 4              & JP70GD      & QRV      \\
	Hotspot               & DMR/D-Star & SG4AXQ   & Sunne                  &     432.5000 &   Duplex 0 & CC 1/DV           & JO69NU      & QRV      \\
	Repeater              & FM         & SK4RJJ   & Sunne/Blåbärskullen    &     145.7750 &     -0.600 & 1750/74.4/DTMF *4 & JO69KU      & QRV      \\
	Repeater              & FM         & SK4KO    & Sälen/Lindvallen       &     145.6000 &     -0.600 & 1750/85.4         & JP61OD      & QRV      \\
	Link                  & FM         & SK4RJJ   & Torsby/Hovfjället      &     145.2875 &            & 74.4              & JO69LH      & QRV      \\
	Repeater              & FM         & SK4RPK   & Torsby/Valberget       &     434.6250 &     -2.000 & 1750              & JP60LC      & QRV      \\
	Repeater              & FM         & SK4WV    & Vansbro                &     434.6500 &     -1.600 & 1750/85.4         & JP70AM      & QRT      \\
	Repeater              & FM         & SK4WV    & Vansbro                &     145.6500 &     -0.600 & 1750/85.4         & JP70AM      & QRV      \\
	Repeater              & DMR        & SK4WV    & Vansbro                &     434.6625 &     -2.000 & CC 4              & JP70AM      & QRV      \\
	Repeater              & FM         & SK4RQF   & Årjäng                 &     145.7250 &     -0.600 & 1750              & JO69BJ      & QRV      \\
	Link                  & FM         & SA4THA   & Älvdalen               &     434.5000 &    Simplex & 250.3             & JP71AF      & QRV      \\
	Repeater              & DMR        & SK4TL    & Örebro                 &     434.7250 &     -2.000 & CC 4              & JO79OG      & QRV      \\
	Repeater              & FM         & SK4TL/R  & Örebro/Suttarboda      &      51.9500 &     -0.600 & 1750/DTMF 4       & JO79KH      & Plan     \\
	Repeater              & FM         & SK4TL/R  & Örebro/Suttarboda      &     145.7125 &     -0.600 & 1750/DTMF 4       & JO79KH      & QRV
\end{longtable}



\clearpage

\subsection{Repeatrar distrikt 5}

\begin{longtable}{llllrrlll}
	\bf Typ           & \bf Mod    & \bf Call & \bf Ort              & \bf Frekvens & \bf Duplex & \bf Access        & \bf Lokator & \bf QRV? \\ \hline
	\endhead
Repeater & FM         & SK5BB/R  & Arboga/Kolsva        &     434.8750 &     -2.000 & Carrier           & JP79WO      & QRV      \\
	Repeater          & FM         & SK5BB/R  & Arboga/Kolsva        &     145.6750 &     -0.600 & Carrier           & JP79WO      & QRV      \\
	Repeater          & FM         & SK5LW/R  & Eskilstuna           &      51.8500 &     -0.600 & 82.5              & JO89FJ      & QRV      \\
	Repeater          & FM         & SK5VM/R  & Eskilstuna           &     434.9750 &     -2.000 & 82.5              & JO89GI      & QRV      \\
	Repeater          & FM         & SK5LW/R  & Eskilstuna/Hällby    &     434.8500 &     -2.000 & 82.5              & JO89FJ      & QRV      \\
	Repeater          & FM         & SK5LW/R  & Eskilstuna/Slytan    &     145.6125 &     -0.600 & 82.5              & JO89HF      & QRV      \\
	Repeater          & DMR/D-Star & SK5LW/R  & Eskilstuna/Ärla      &     145.5875 &     -0.600 & CC 5/DV Carrier   & JO89FJ      & QRT      \\
	Repeater          & FM         & SK5BN/R  & Finspång             &     434.9250 &     -2.000 & 107.2             & JO78VR      & QRV      \\
	Repeater          & DMR        & SA5SEL   & Finspång             &     434.7000 &     -2.000 &                   & JO78VQ      & QRV      \\
	Repeater          & D-Star     & SK5UM-B  & Flen                 &     434.5500 &     -2.000 & DV Carrier        & JO89HB      & QRV      \\
	Repeater          & DMR        & SK5UM/R  & Flen                 &     145.6375 &     -0.600 & 82.5/CC 5         & JO89HB      & QRV      \\
	Repeater          & FM         & SK5UM/R  & Flen                 &     145.7625 &     -0.600 & 103.5             & JO89HB      & QRV      \\
	Hotspot           & D-Star     & SG5TAH-C & Flen/Orrhammar       &     145.3375 &   Duplex 0 & DV Carrier        & JO89GB      & QRV      \\
	Repeater          & FM         & SK5UM/R  & Flen/Öja             &     434.7500 &     -2.000 & 1750/91.5         & JO89HB      & Plan     \\
	Repeater          & FM/C4FM    & SK5RCQ   & Kisa                 &     145.7000 &     -0.600 & 82.5              & JO77TX      & QRV      \\
	Repeater          & FM/DMR     & SG5BCG/R & Knivsta              &     434.5250 &     -2.000 & 82.5/CC 5         & JO89VR      & QRV      \\
	Repeater          & FM         & SK5AS/R  & Linköping            &     145.7250 &     -0.600 & 1750              & JO78SJ      & QRV      \\
	Repeater          & FM         & SK5AS/R  & Linköping            &     145.7875 &     -0.600 & 82.5/DTMF 5       & JO78SN      & QRV      \\
	Repeater          & DMR        & SK5RYG   & Linköping            &     434.5125 &     -2.000 & CC 5              & JO78SN      & QRV      \\
	Repeater          & FM         & SK5RYG   & Linköping            &     145.6250 &     -0.600 & 82.5/DTMF 5       & JO78SN      & QRV      \\
	Repeater          & FM/DMR     & SM5DWC/R & Linköping            &     434.8750 &     -2.000 & 82.5/CC 5         & JO78SM      & QRV      \\
	Repeater          & FM         & SM5YMS/R & Linköping            &     434.8000 &     -2.000 & 1750              & JO78SM      & QRV      \\
	Repeater          & FM         & SK5RHT   & Linköping            &      51.9900 &     -0.600 & 82.5/DTMF 5       & JO78SN      & QRV      \\
	Repeater          & FM         & SK5RHT   & Linköping            &      29.6600 &     -0.100 & 82.5              & JO78XH      & QRV      \\
	Repeater          & FM         & SK5LF/R  & Linköping/Majelden   &     434.8250 &     -2.000 & 82.5              & JO78TJ      & QRV      \\
	Repeater          & FM         & SK5WR/R  & Motala               &     145.7375 &     -0.600 & 1750/91.5         & JO78NM      & QRV      \\
	Repeater          & FM/DMR     & SL5ZYT/R & Norrköping           &     434.9500 &     -2.000 & 82.5/CC 5         & JO88DQ      & QRV      \\
	Repeater          & D-Star     & SK5BN-C  & Norrköping           &     145.5750 &     -0.600 & DV Carrier        & JO88BR      & QRV      \\
	Link              & FM         & SM5GXQ-L & Norrköping           &     145.2375 &    Simplex & Carrier/DTMF      & JO88CO      & QRV      \\
	Repeater          & FM/C4FM    & SA5OHR/R & Norrköping           &     434.6625 &     -2.000 &                   & JO88BO      & QRV      \\
	Repeater          & FM         & SK5BN/R  & Norrköping/Kolmården &     145.6000 &     -0.600 & 1750/DTMF 5       & JO88FQ      & QRV      \\
	Repeater          & ATV        & SK5BN/R  & Norrköping/Kolmorden &    1282.0000 &    -30.000 &                   & JO88FQ      & QRV      \\
	Repeater          & FM         & SK5BN/R  & Norrköping/Ö. Eneby  &     434.6000 &     -2.000 & 1750/82.5/DTMF 5  & JO88BO      & QRV      \\
	Repeater          & DMR        & SA5UTR   & Nyköping             &     434.6375 &     -2.000 & CC 5              & JO88MS      & Plan     \\
	Link              & FM         & SM5RVH   & Nyköping             &    1297.5000 &            & Carrier           & JO88LQ      & QRV      \\
	Repeater          & FM         & SM5RYI/R & Sala                 &     145.7125 &     -0.600 & 82.5              & JO89HW      & QRV      \\
	Link              & DMR        & SA5KBE   & Stigtomta            &     145.2875 &    Simplex & CC 5              & JO88JT      & QRV      \\
	Repeater          & FM         & SA5BTT   & Trosa                &     434.8875 &     -2.000 & 82.5              & JO88TV      & QRV      \\
	Repeater          & FM         & SK5DB/R  & Uppsala              &     434.7500 &     -2.000 & 1750/82.5         & JO89VU      & QRT      \\
	Hotspot           & D-Star     & SC5SLU-C & Uppsala              &     145.3250 &   Duplex 0 & DV Carrier        & JO89QW      & QRV      \\
	Hotspot           & D-Star     & SM5EZN-B & Uppsala              &     433.4875 &   Duplex 0 & DV Carrier        & JO89QW      & QRV      \\
	Repeater          & FM         & SK5DB/R  & Uppsala              &     145.7500 &     -0.600 & 1750/82.5         & JO89VU      & QRV      \\
	Hotspot           & DMR        & SA5HAV   & Uppsala              &     434.3750 &    Simplex & CC 5              & JO89VW      & QRV      \\
	Repeater          & FM/DMR     & SG5DV    & Uppsala              &     434.5875 &     -2.000 & 82.5/CC 5         & JO89TU      & QRV      \\
	Repeater          & FM         & SG5DV    & Uppsala              &     145.5875 &     -0.600 & 1750/136.5/DTMF * & JO89TU      & QRV      \\
	Repeater          & DMR        & SA5BJM   & Uppsala/Fjuckby      &     434.5125 &     -2.000 & CC 1              & JO89TX      & QRV      \\
	Link              & FM         & SA5BJM   & Uppsala/Fjuckby      &     144.5750 &    Simplex & Carrier           & JO89TX      & QRV      \\
	Link              & FM         & SA5BJM   & Uppsala/Fjuckby      &     433.4500 &    Simplex & Carrier           & JO89TX      & QRV      \\
	Repeater          & DMR        & SA5HAV/R & Uppsala/Rasbo        &     434.6375 &     -2.000 & CC 5              & JO89VW      & QRV      \\
	Repeater          & FM         & SK5RHQ   & Västerås             &     145.7750 &     -0.600 & 136.5 / 82.5      & JO89GO      & QRV      \\
	Repeater          & FM         & SK5RHQ   & Västerås             &     434.7750 &     -2.000 & 136.5 / 82.5      & JO89GO      & QRV      \\
	Repeater          & FM         & SM5YMS   & Åtvidaberg           &     145.6625 &     -0.600 & 82.5              & JO78XE      & QRV
\end{longtable}


\clearpage

\subsection{Repeatrar distrikt 6}


\begin{longtable}{llllrrlll}
	\bf Typ           & \bf Mod         & \bf Call & \bf Ort               & \bf Frekvens & \bf Duplex & \bf Access        & \bf Lokator & \bf QRV? \\ \hline
	\endhead
Repeater & FM              & SK6RIC   & Alingsås              &     145.6250 &     -0.600 & 1750/114.8        & JO67GW      & QRV      \\
	Repeater          & FM              & SK6RIC   & Alingsås              &     434.6250 &     -2.000 & 1750/114.8        & JO67GW      & QRV      \\
	Repeater          & DMR             & SK6DG    & Alingsås              &     434.5375 &     -2.000 & CC 6              & JO67GV      & QRV      \\
	Repeater          & FM              & SK6RIC   & Alingsås              &    1297.0250 &     -6.000 & Carrier           & JO67GV      & QRV      \\
	Hotspot           & D-Star          & SG6YOW   & Alingsås              &     144.8500 &    Simplex & Duplex +          & JO67GW      & QRV      \\
	Repeater          & FM              & SA6AR/R  & Angered               &     434.9250 &     -2.000 & 1750              & JO67AT      & QRV      \\
	Repeater          & FM/DMR          & SK6RFP   & Bengtsfors            &     145.7000 &     -0.600 & 118.8/CC 6        & JO69CA      & QRV      \\
	Repeater          & FM/DMR          & SL6ZYW/R & Bengtsfors            &     434.6875 &     -2.000 & 114.8/CC 6        & JO69CA      & QRV      \\
	Repeater          & FM              & SK6IF/R  & Bokenäs               &     145.6000 &     -0.600 & 1750/118.8/DTMF 4 & JO58TH      & QRV      \\
	Repeater          & FM              & SK6LK/R  & Borås                 &     434.8000 &     -2.000 & 114.8             & JO67MR      & QRV      \\
	Repeater          & FM              & SK6LK/R  & Borås                 &     145.7750 &     -0.600 & 1750/114.8        & JO67MR      & QRV      \\
	Link              & FM              & SM6FZG   & Borås                 &     144.5125 &    Simplex & 146.2             & JO67MR      & QRV      \\
	Repeater          & DMR             & SM6TKT/R & Borås                 &     434.5500 &     -2.000 & CC 6              & JO67MR      & QRV      \\
	Repeater          & DMR             & SA6ATX/R & Dingle                &     434.8875 &     -2.000 & CC 6              & JO58SM      & Plan     \\
	Repeater          & FM              & SK6JX/R  & Falkenberg            &     145.6250 &     -0.600 & 1750/118.8/DTMF 1 & JO66FV      & QRT      \\
	Link              & FM              & SK6RP    & Floda                 &     433.4750 &            & Carrier           & JO67ET      & QRV      \\
	Repeater          & FM/DMR          & SK6RP/R  & Floda                 &     434.8250 &     -2.000 & 118.8/CC 6        & JO67ET      & QRV      \\
	Repeater          & FM              & SK6RKI   & Guldheden             &    1297.1500 &     -6.000 & 1750              & JO57XQ      & QRT      \\
	Repeater          & D-Star          & SK6SA-B  & Guldheden             &     434.5125 &     -2.000 & DV Carrier        & JO57XQ      & QRV      \\
	Repeater          & FM              & SK6RDG   & Guldheden             &     434.9750 &     -2.000 & 1750/114.8        & JO57XQ      & QRV      \\
	Repeater          & FM              & SK6RFQ   & Guldheden             &      51.8700 &     -0.600 & 1750/114.8/DTMF 6 & JO57XQ      & QRV      \\
	Repeater          & FM              & SK6RFQ   & Guldheden             &      29.6800 &     -0.100 & 1750/114.8/DTMF 6 & JO57XQ      & QRV      \\
	Link              & FM              & SK6AG    & Guldheden             &     144.5750 &    Simplex & 146.2             & JO57XQ      & QRV      \\
	Repeater          & DMR             & SK6AG    & Guldheden             &     434.7875 &     -2.000 & CC 6              & JO57XQ      & QRV      \\
	Repeater          & FM              & SK6RFQ   & Guldheden             &     434.6500 &     -2.000 & 1750/114.8/DTMF 6 & JO57XQ      & QRV      \\
	Repeater          & FM              & SK6RFQ   & Guldheden             &     145.6500 &     -0.600 & 1750/114.8/DTMF 6 & JO57XQ      & QRV      \\
	Link              & FM              & SM6FZG   & Guldheden             &     144.6750 &    Simplex & 146.2             & JO57XQ      & QRV      \\
	Link              & FM              & SM6FZG   & Guldheden             &      51.5500 &    Simplex & 146.2             & JO57XQ      & QRV      \\
	Repeater          & FM/C4FM         & SK6RFQ   & Guldheden             &     434.6000 &     -2.000 & 114.8             & JO57XQ      & QRV      \\
	Hotspot           & D-Star          & SG6JWU-B & Halmstad              &     433.4750 &   Duplex 0 & DV Carrier        & JO66LP      & QRV      \\
	Repeater          & FM/C4FM/D-Star  & SK6RKG   & Halmstad              &     434.9250 &     -2.000 & 114.8             & JO66MS      & QRV      \\
	Repeater          & FM              & SL6BH/R  & Halmstad              &     434.7500 &     -2.000 & 114.8             & JO66KQ      & QRV      \\
	Repeater          & FM              & SK6RKG   & Halmstad              &     145.6750 &     -0.600 & 114.8             & JO66MS      & QRV      \\
	Hotspot           & D-Star          & SK6MA-C  & Hjo                   &     145.2125 &   Duplex 0 & DV Carrier        & JO78DH      & QRV      \\
	Link              & FM              & SM6VAG   & Hjo                   &     145.2375 &    Simplex & Carrier           & JO78AG      & QRV      \\
	Link              & FM              & SM6FZG   & Hönö                  &     144.6250 &    Simplex & 146.2             & JO57TQ      & QRV      \\
	Repeater          & FM/C4FM         & SK6WW/R  & Karlsborg             &     145.7625 &     -0.600 & 94.8              & JO78FM      & QRV      \\
	Repeater          & FM              & SM6CYJ/R & Kinnekulle            &     434.9500 &     -2.000 & Carrier           & JO68QO      & QRV      \\
	Repeater          & FM              & SK6ROY   & Kinnekulle            &     145.6000 &     -0.600 & 1750/114.8        & JO68QO      & QRV      \\
	Link              & FM              & SM6FZG   & Kortedala             &     144.6000 &    Simplex & 146.2             & JO67AS      & QRV      \\
	Repeater          & DMR             & SK6RKI   & Kortedala             &     145.5875 &     -0.600 & CC 6              & JO67AS      & QRV      \\
	Repeater          & FM              & SK6RJW   & Kungsbacka            &     145.7250 &     -0.600 & 114.8             & JO67AL      & QRV      \\
	Repeater          & FM              & SK6RJW   & Kungsbacka            &     434.7250 &     -2.000 & 1750/114.8/DTMF 6 & JO67AL      & QRV      \\
	Link              & FM              & SM6FZG   & Kungsbacka            &     144.6500 &    Simplex & 146.2             & JO67AL      & QRV      \\
	Repeater          & FM/DMR          & SK6IF    & Kungshamn             &     145.6750 &     -0.600 & 94.8/CC 6         & JO58PI      & QRV      \\
	Repeater          & FM              & SK6RPE   & Kungälv               &     145.6125 &     -0.600 & 114.8             & JO57XU      & QRV      \\
	Repeater          & FM              & SK6RPE   & Kungälv               &     434.9000 &     -2.000 & 114.8             & JO57XU      & QRV      \\
	Repeater          & FM              & SA6BXG/R & Kungälv/Romelanda     &     434.7375 &     -2.000 & 114.8             & JO67AX      & QRV      \\
	Repeater          & FM              & SK6IF/R  & Lysekil               &     434.8000 &     -2.000 & 118.8             & JO58RG      & QRV      \\
	Link              & FM              & SM6FZG   & Långedrag             &     144.5250 &    Simplex & 146.2             & JO57WQ      & QRV      \\
	Repeater          & FM              & SK6QW/R  & Mariestad/Katrinefors &     434.9000 &     -2.000 & Carrier           & JO68VQ      & QRV      \\
	Link              & FM              & SM6FZG   & Myggenäs              &     144.6625 &    Simplex & 146.2             & JO58UB      & QRV      \\
	Hotspot           & D-Star          & SK6GB-D  & Mölndal               &     433.7250 &    Simplex & DV Carrier        & JO67AQ      & QRV      \\
	Hotspot           & D-Star          & SK6GB-D  & Mölndal               &     144.8250 &    Simplex & DV Carrier        & JO67AQ      & QRV      \\
	Repeater          & FM              & SM6VBT/R & Mölndal               &     145.7000 &     -0.600 & 118.8             & JO67AP      & QRV      \\
	Repeater          & FM              & SM6VBT/R & Mölndal               &     434.7000 &     -2.000 & 118.8             & JO67AP      & QRV      \\
	Repeater          & FM              & SM6VBT/R & Mölndal               &      29.6900 &     -0.100 & 118.8             & JO67AP      & QRV      \\
	Link              & FM              & SM6FZG   & Mölnlycke             &     144.5875 &    Simplex & 146.2             & JO67BP      & QRV      \\
	Repeater          & FM/DMR/D-Star   & SA6APY   & Skara                 &     434.9875 &     -2.000 & 114.8/CC 6        & JO68RJ      & QRV      \\
	Repeater          & FM/C4FM         & SK6EE/R  & Skara                 &     145.7250 &     -0.600 & 114.8             & JO68RH      & QRV      \\
	Repeater          & FM/C4FM         & SK6EE/R  & Skara                 &     434.5625 &     -2.000 & Carrrier          & JO68RH      & QRV      \\
	Repeater          & FM              & SK6BA/R  & Skene                 &     145.6000 &     -0.600 & 94.8              & JO67HM      & QRV      \\
	Repeater          & FM              & SK6BA/R  & Skene                 &     434.9500 &     -2.000 & 94.8              & JO67HM      & QRV      \\
	Hotspot           & DMR/C4FM/D-Star & SK6BA-B  & Skene                 &     433.5625 &   Duplex 0 & DV Carrier        & JO67HL      & QRV      \\
	Link              & FM              & SM6FZG   & Skårsjön              &     144.5500 &    Simplex & 146.2             & JO67AN      & QRV      \\
	Repeater          & FM/C4FM         & SK6EI/R  & Skövde                &     434.8250 &     -2.000 & 114.8             & JO68VK      & QRV      \\
	Repeater          & FM/C4FM         & SM6THE/R & Skövde                &     145.6875 &     -0.600 & 114.8             & JO68XJ      & QRV      \\
	Repeater          & FM/DMR          & SK6QA/R  & Stenungsund           &     145.7125 &     -0.600 & 114.8/CC 6        & JO58XB      & QRV      \\
	Repeater          & FM/DMR          & SK6IF    & Tanumshede            &     145.5750 &     -0.600 & 118.8/CC 6        & JO58PR      & QRV      \\
	Repeater          & FM              & SK6MA/R  & Tidaholm/Hökensås     &     145.6375 &     -0.600 & 118.8             & JO78AD      & QRV      \\
	Repeater          & FM              & SM6SXJ   & Torup/Galtabo         &     434.8875 &     -2.000 & 1750/114.8/DTMF 1 & JO67LA      & QRV      \\
	Repeater          & D-Star          & SK6DW-B  & Trollhättan           &     434.5250 &     -2.000 & DV Carrier        & JO68DG      & QRV      \\
	Repeater          & FM/DMR          & SK6DW/R  & Trollhättan           &     145.7625 &     -0.600 & 114.8/CC 6        & JO68DG      & QRV      \\
	Repeater          & FM/DMR          & SK6DW/R  & Trollhättan           &     434.8750 &     -2.000 & 118.8/CC 6        & JO68DG      & QRV      \\
	Repeater          & FM              & SM6WSC   & Trollhättan           &     434.7250 &     -2.000 & 1750/CTCSS        & JO68EF      & QRV      \\
	Repeater          & FM/DMR          & SL6ZAQ   & Uddevalla             &     145.7375 &     -0.600 & 114.8/CC 6        & JO58WH      & QRV      \\
	Repeater          & FM              & SM6UDU/R & Uddevalla/Bokenäs     &     434.7750 &     -2.000 & 1750/118.8/DTMF * & JO58UI      & QRV      \\
	Repeater          & FM              & SM6UXW/R & Ulricehamn            &     434.6750 &     -2.000 & 118.8             & JO67RT      & QRV      \\
	Repeater          & FM/C4FM         & SM6UXW/R & Ulricehamn            &     145.6750 &     -0.600 & 118.8             & JO67ST      & QRV      \\
	Repeater          & FM              & SK6DK/R  & Varberg/Veddige       &     434.7000 &     -2.000 & 1750              & JO67EH      & QRV      \\
	Repeater          & FM              & SK6DK/R  & Varberg/Veddige       &     145.7000 &     -0.600 & 1750              & JO67EH      & QRV      \\
	Repeater          & FM              & SG6WAL   & Vedum                 &     145.7875 &     -0.600 & 1750/218.1        & JO68KE      & QRT      \\
	Repeater          & FM              & SM6WNR/R & Vårgårda              &     434.8500 &     -2.000 & 1750/118.8        & JO68JA      & QRV      \\
	Repeater          & FM              & SA6BSN/R & Åmål                  &     434.6000 &     -2.000 & 1750              & JO69IB      & QRV      \\
	Repeater          & FM              & SK6DQ/R  & Älvängen              &     434.7500 &     -2.000 & 114.8             & JO67BW      & QRV      \\
	Repeater          & DMR             & SK6RKI   & Öckerö                &     434.8500 &     -2.000 & CC 6              & JO57TR      & QRV      \\
	Repeater          & FM              & SK6RKI   & Öckerö                &     145.7500 &     -0.600 & 1750/114.8        & JO57TR      & QRV      \\
	Link              & FM              & SM6TZL/L & Örby                  &     145.2375 &    Simplex & Carrier           & JO67IL      & QRV      \\
	Link              & FM              & SM6TZL/L & Örby                  &     434.2250 &    Simplex & Carrier           & JO67IL      & QRV
\end{longtable}


\clearpage

\subsection{Repeatrar distrikt 7}

\begin{longtable}{llllrrlll}
	\bf Typ           & \bf Mod         & \bf Call & \bf Ort              & \bf Frekvens & \bf Duplex & \bf Access        & \bf Lokator & \bf QRV? \\ \hline
	\endhead
Repeater & FM              & SK7RFL   & Algutsrum            &     434.6000 &     -2.000 & 1750/79.7/DTMF 0  & JO86GQ      & QRV      \\
	Repeater          & FM              & SK7RFL   & Algutsrum            &     145.6000 &     -0.600 & 1750/79.7/DTMF 0  & JO86GQ      & QRV      \\
	Repeater          & DMR/D-Star/C4FM & SK7RFL   & Algutsrum            &     434.5500 &     -2.000 & CC 7              & JO86GQ      & QRV      \\
	Repeater          & FM/C4FM         & SK7REZ   & Blentarp/Romeleåsen  &     145.6750 &     -0.600 & 79.7              & JO65TM      & QRT      \\
	Repeater          & FM/C4FM         & SK7EM/R  & Blentarp/Romeleåsen  &     434.8500 &     -2.000 & 79.7              & JO65SN      & QRV      \\
	Repeater          & FM              & SK7RN/R  & Borgholm             &     145.6625 &     -0.600 & 1750/79.7/DTMF *  & JO86HU      & QRV      \\
	Repeater          & FM              & SK7RN/R  & Borgholm             &     434.7750 &     -2.000 & 79.7              & JO86HU      & QRV      \\
	Link              & FM              & SK7RN    & Borgholm             &     145.3125 &    Simplex & 1750/DTMF *       & JO86HV      & QRV      \\
	Repeater          & FM              & SK7GH/R  & Bredaryd             &     145.6875 &     -0.600 & 156.7             & JO67UE      & QRV      \\
	Repeater          & DMR             & SG7BNT   & Bruzaholm            &     434.6000 &     -2.000 & CC 7              & JO77PP      & QRV      \\
	Repeater          & FM              & SK7RN/R  & Böda                 &     145.7500 &     -0.600 & 1750/79.7/DTMF *  & JO87MG      & QRV      \\
	Repeater          & FM/DMR          & SG7WSE   & Ekenässjön           &     145.7125 &     -0.600 & 156.7/CC 7        & JO77ML      & Plan     \\
	Hotspot           & DMR             & SG7WSE   & Ekenässjön           &     144.8500 &    Simplex & CC 7              & JO77ML      & QRV      \\
	Repeater          & DMR             & SK7AF    & Eksjö                &     434.5625 &     -2.000 & CC 7              & JO77MP      & QRV      \\
	Repeater          & FM              & SK7UO/R  & Emmaboda             &     145.7750 &     -0.600 & 1750              & JO76SP      & QRV      \\
	Repeater          & FM/DMR          & SK7DL    & Emmaboda             &     434.7875 &     -2.000 & CC 7              & JO76SP      & QRV      \\
	Hotspot           & D-Star          & SG7WDL-C & Eneryda              &     145.2125 &   Duplex 0 & DV Carrier        & JO76EQ      & QRV      \\
	Repeater          & FM              & SA7BVQ/R & Eslöv                &     434.8125 &     -2.000 & 79.7              & JO65PT      & QRV      \\
	Link              & FM              & SM7FLD   & Everöd               &     145.2375 &    Simplex &                   & JO75BV      & QRV      \\
	Link              & FM              & SM7GXQ   & Färjestaden          &     145.2375 &    Simplex & Carrier/DTMF      & JO86FP      & QRV      \\
	Repeater          & FM              & SK7ROQ   & Gladsax              &     434.8875 &     -2.000 & 79.7              & JO75DN      & QRV      \\
	Repeater          & DMR/D-Star      & SK7RNQ   & Gladsax              &     145.5750 &     -0.600 & CC 7              & JO75DN      & QRV      \\
	Repeater          & FM              & SK7RQX   & Hallandsås           &     145.7875 &     -0.600 & 79.7              & JO66LI      & QRV      \\
	Repeater          & FM              & SK7RSZ   & Hallandsås           &     434.6625 &     -2.000 & 79.7              & JO66LI      & QRV      \\
	Repeater          & FM              & SK7CY    & Helsingborg          &    1297.2000 &     -6.000 & 1750              & JO66IB      & QRT      \\
	Repeater          & DMR             & SK7REE   & Helsingborg          &     434.6000 &     -2.000 & CC 7              & JO66IA      & QRV      \\
	Repeater          & FM              & SK7GH/R  & Hillerstorp          &     434.6000 &     -2.000 & 156.7             & JO67WI      & QRV      \\
	Repeater          & FM              & SK5CN/R  & Hultsfred/Gåskullen  &     145.7625 &     -0.600 & 1750/DTMF 5       & JO77WL      & QRV      \\
	Repeater          & FM              & SK7RGI   & Huskvarna            &     434.7500 &     -2.000 & 1750/156.7/DTMF 6 & JO77DT      & QRV      \\
	Repeater          & FM              & SK7RGI   & Huskvarna            &      29.6800 &     -0.100 & 1750/DTMF 6       & JO77DT      & QRV      \\
	Repeater          & DMR             & SA7BIK   & Höör                 &     434.9125 &     -2.000 & CC 7              & JO65SW      & QRV      \\
	Repeater          & DMR             & SK7RGI   & Jönköping            &     434.9750 &     -2.000 & CC 7              & JO77CS      & QRV      \\
	Repeater          & FM              & SK7RGI   & Jönköping            &     145.7000 &     -0.600 & 1750/156.7/DTMF 6 & JO77CS      & QRV      \\
	Repeater          & FM              & SK7RGI   & Jönköping            &     145.7500 &     -0.600 & 1750/156.7/DTMF 6 & JO77AQ      & QRV      \\
	Repeater          & FM              & SM7JPI/R & Karlshamn            &     434.9250 &     -2.000 & 79.7              & JO76KE      & QRV      \\
	Repeater          & FM              & SK7RFJ   & Karlskrona           &     145.7500 &     -0.600 & 1750/156.7        & JO76TE      & QRV      \\
	Repeater          & FM              & SK7FK/R  & Karlskrona           &     434.7500 &     -2.000 & 1750              & JO76TE      & QRV      \\
	Repeater          & FM/C4FM         & SK7BQ/R  & Kristianstad         &     145.7375 &     -0.600 & 79.7              & JO76AA      & QRV      \\
	Repeater          & DMR             & SK7BQ    & Kristianstad         &     434.5250 &     -2.000 & CC 7              & JO76AA      & QRV      \\
	Repeater          & DMR/D-Star      & SK7RMQ   & Linderöd             &     145.5875 &     -0.600 & CC 14             & JO65VW      & QRV      \\
	Link              & FM              & SA7AUX   & Linneryd             &     145.4000 &    Simplex & Carrier           & JO76NP      & QRT      \\
	Repeater          & FM              & SK7MO/R  & Ljungby              &     145.7250 &     -0.600 & 1750/94.8/DTMF 1  & JO66XV      & QRV      \\
	Repeater          & FM              & SK7RJL/R & Lund                 &     434.7250 &     -2.000 & 79.7              & JO65OR      & QRV      \\
	Repeater          & DMR             & SK7RJL   & Lund                 &     434.5875 &     -2.000 & CC 7              & JO65OR      & QRV      \\
	Repeater          & FM              & SK7REP   & Lund/Harderberga     &     145.7750 &     -0.600 & 79.7              & JO65PQ      & QRV      \\
	Repeater          & DMR/D-Star      & SK7RRV   & Lönsboda             &     434.9500 &     -2.000 & CC 7              & JO76DJ      & Plan     \\
	Hotspot           & D-Star          & SK7RRV-C & Lönsboda             &     144.8875 &   Duplex 0 & DV Carrier        & JO76DJ      & QRV      \\
	Repeater          & D-Star          & SM7XAA   & Malmö                &     434.5250 &     -2.000 & DV Carrier        & JO65MN      & QRV      \\
	Repeater          & DMR             & SK7RJL   & Malmö                &     434.7750 &     -2.000 & CC 7              & JO65LO      & QRV      \\
	Repeater          & D-Star          & SK7RDS   & Malmö                &     145.5625 &     -0.600 & 79.7              & JO65LO      & QRV      \\
	Repeater          & D-Star          & SK7DS    & Malmö                &     434.5125 &     -2.000 & 79.7              & JO65LO      & QRV      \\
	Repeater          & DMR/D-Star      & SK7RPQ   & Malmö                &     434.6125 &     -2.000 & CC 7/XLX/XRF699B  & JO65MN      & QRV      \\
	Repeater          & FM              & SM7HZK/R & Moheda               &     145.6375 &     -0.600 & 1750/225.7/DTMF 1 & JO76HX      & QRV      \\
	Repeater          & FM              & SK7RN/R  & Mörbylånga           &     145.6250 &     -0.600 & 1750/79.7/DTMF *  & JO86FM      & QRV      \\
	Repeater          & FM              & SM7LNT/R & Mörrum               &     434.8250 &     -2.000 & 79.7              & JO76IE      & QRT      \\
	Repeater          & FM              & SL7ZXW/R & Nybro                &     145.6875 &     -0.600 & 1750              & JO76VQ      & QRV      \\
	Repeater          & FM              & SK7RFH   & Nässjö               &     434.8500 &     -2.000 & 1750/DTMF 6       & JO77IP      & QRV      \\
	Repeater          & FM              & SK7RFH   & Nässjö               &     145.6500 &     -0.600 & 1750/156.7        & JO77IP      & QRV      \\
	Repeater          & DMR             & SG7RFH   & Nässjö               &     434.9000 &     -2.000 & CC 7              & JO77IP      & QRV      \\
	Repeater          & DMR             & SG7RFH   & Nässjö               &     145.5875 &     -0.600 & CC 7              & JO77IP      & QRV      \\
	Repeater          & DMR/D-Star      & SK7RGM   & Olofström/Boafallsb. &     434.7125 &     -2.000 & CC 7              & JO76FF      & QRV      \\
	Repeater          & FM/C4FM         & SK7RGM   & Olofström/Boafallsb. &     145.7000 &     -0.600 & 79.7              & JO76FF      & QRV      \\
	Repeater          & FM              & SK7RIH   & Oskarshamn           &     145.7250 &     -0.600 & 1750              & JO87FG      & QRV      \\
	Repeater          & FM              & SK7RIH/R & Oskarshamn           &     434.7250 &     -2.000 & 1750              & JO87EG      & QRV      \\
	Repeater          & FM              & SK7RIH   & Oskarshamn           &      51.9100 &     -0.600 & 1750              & JO87EG      & QRV      \\
	Repeater          & DMR             & SA7CCO   & Sjöbo                &     434.9250 &     -2.000 & CC 7              & JO65UP      & QRV      \\
	Repeater          & FM              & SK7JL    & Spjutsbygd           &     145.7250 &     -0.600 & 79.7              & JO76TH      & QRV      \\
	Repeater          & DMR             & SK7HR    & Sävsjö               &     434.5250 &     -2.000 & CC 7              & JO77HJ      & Plan     \\
	Repeater          & FM              & SK7REE   & Söderåsen/Stenested  &     145.6500 &     -0.600 &                   & JO66NB      & QRV      \\
	Repeater          & FM              & SK7REE   & Söderåsen/Stenested  &      51.8500 &     -0.600 & 79.7              & JO66NB      & QRV      \\
	Repeater          & FM/DMR          & SK7REE   & Söderåsen/Stenested  &     434.6500 &     -2.000 & 79.7/CC 7         & JO66NB      & QRV      \\
	Repeater          & DMR             & SA7BJF/R & Södra Vi             &     434.6625 &     -2.000 & CC 7              & JO77VR      & QRV      \\
	Link              & FM              & SA7BJF   & Södra Vi             &     433.5000 &    Simplex &                   & JO77VR      & QRV      \\
	Hotspot           & D-Star          & SG7HTP-C & Sölvesborg           &     145.2375 &    Simplex & DV Carrier        & JO76GB      & QRV      \\
	Link              & FM              & SM7KUY/R & Sölvesborg           &     434.4000 &    Simplex & 79.7              & JO76HB      & QRV      \\
	Repeater          & FM/DMR          & SA7KSI/R & Tomelilla            &     434.6375 &     -2.000 & 79.7/CC 7         & JO65XN      & Plan     \\
	Repeater          & FM              & SK7IJ/R  & Vetlanda             &     434.6250 &     -2.000 & 156.7             & JO77OL      & QRV      \\
	Repeater          & FM              & SK7IJ/R  & Vetlanda             &     145.6250 &     -0.600 & 1750/156.7        & JO77OL      & QRV      \\
	Repeater          & FM              & SK7RNQ   & Vitaby               &     145.6125 &     -0.600 & 79.7              & JO75BQ      & QRV      \\
	Repeater          & FM              & SK7GH/R  & Värnamo              &     145.6000 &     -0.600 & 1750/156.7        & JO77AE      & QRV      \\
	Repeater          & FM/C4FM         & SK7JD/R  & Västervik            &     145.6750 &     -0.600 & 77.0              & JO87HS      & QRV      \\
	Repeater          & DMR             & SK7JD    & Västervik            &     434.6750 &     -2.000 & CC 7              & JO87HS      & QRV      \\
	Repeater          & DMR             & SK7HW    & Växjö                &     434.7000 &     -2.000 & CC 7              & JO76KU      & QRV      \\
	Repeater          & FM              & SK7HW/R  & Växjö/Hollstorp      &     145.6750 &     -0.600 & 1750/156.7        & JO76KU      & QRV      \\
	Repeater          & FM/DMR          & SK7REE   & Örkelljunga          &     434.9750 &     -2.000 & 79.7/CC 7         & JO66PG      & QRV
\end{longtable}


\end{landscape}


\clearpage

\subsection{Begrepp i amatörradiobandplanerna}

\begin{itemize}
\item QRP: Aktivitetscentrum för låg effekt ($<$5W), svaga signaler
      förekommer, visa hänsyn.
\item QRS: Aktivitetscenter för långsam CW.
\item QRSS: Extremt långsam CW med dator.
\item DV: Digital Voice.
\item Image: Bildmoder exempelvis SSTV och Fax som ryms inom den specificerade
	  maximala bandbredden.
\end{itemize}

\subsection{Trafik- och tumregler}

\begin{itemize}
\item Vid SSB-telefoni används LSB på frekvenser under 10 MHz och USB
      på frekvenser över 10 MHz.
\item Lägsta acceptabla inställda frekvens för LSB är 3 kHz över
      under bandkant!
\item Högsta acceptable inställda frekvens för USB är 3 kHz under
      övre bandkant!
\item IBP är International Beacon Project. Fyrarna sänder med 3 min
      intervaller och används för att studera utbredningen av
      radiosignaler globalt. Fyrarna sänder anrop och fyra 1 s toner.
      Anropet och första tonen sänds med 100W, därefter sänds tonerna
      med 10W, 1W samt 100mW.
\item Vid AM (A3J) skall hänsyn tas så att störningar på annan trafik ej fö\-re\-kom\-mer
      med de sidband som då uppstår, det gäller då både övre och undre
      sidbandet.
\item Ingen som helst sändning är tillåtet inom fyrsegmenten. Detta skall respekteras.
      Lyssna gärna på nödfrekvenserna men används dem icke, om det
      inte är du som svarar på ett nödsamtal! Undvik QSO allt för nära
      dessa också.
\item Var särskilt uppmärksam på satelliters nerlänksfrekvenser på 10\,m-bandet.
      I detta segment skall endast lyssning ske. Ingen sändning är
      tillåten här eller i skyddssegmentet strax ovanför
      satellitsegmentet. Tänk på att satelliters frekvens kan
      dopplerskiftas uppåt en hel del när de rör sig mot mottagaren.
\end{itemize}

\subsection{Bandplan 6 m 50--52 MHz}

\resizebox{\textwidth}{!}{%
\begin{tabular}{rrrll}
	\textbf{Frekvens} &        & \textbf{BW} & \textbf{Trafik} & \textbf{Noteringar}                                          \\ \hline
	           50.000 & 50.100 &      500 Hz & CW              & \textbf{CW anrp. 50.050 och 50.090 (interkont.)}             \\ \hline
	           50.100 & 50.130 &     2.7 kHz & CW, SSB         & Interkontinental DX-trafik. Ej QSO inom Europa               \\ \hline
	           50.100 & 50.200 &     2.7 kHz & CW,SSB          & DX 50.110--50.130, \textbf{50.110 50.150 anrop (interkont.)} \\ \hline
	           50.200 & 50.300 &     2.7 kHz & CW,SSB          & Generell användning. 50.285 för crossband                    \\ \hline
	           50.300 & 50.400 &     2.7 kHz & CW, MGM         & PSK 50.305, EME 50.310 – 50.320, MS 50.350 – 50.380          \\ \hline
	           50.400 & 50.500 &       1 kHz & CW, MGM         & Endast fyrar, 50.401 ±500 Hz WSPR-fyrar                      \\ \hline
	           51.210 & 51.390 &      12 kHz & FM              & Repeater Repeater in, 20/10 kHz kanalavstånd RF81 – RF99     \\ \hline
	           50.500 & 52.000 &      12 kHz & Alla moder      & SSTV 50.510, RTTY 50.600, FM 51.510                          \\ \hline
	           51.810 & 51.990 &      12 kHz & FM Repeater     & Repeater ut, 20/10 kHz kanalavstånd                          \\
	                  &        &             &                 & RF81 – RF99                                                  \\ \hline
\end{tabular}}

\subsection{Bandplan 2 m 144--146 MHz}
\resizebox{\textwidth}{!}{%
\begin{tabular}{rrrll}
\textbf{Frekvens} &  & \textbf{BW} & \textbf{Trafik} & \textbf{Noteringar} \\ \hline
144.0000 & 144.1100  & 500 Hz  & CW, EME      & \textbf{CW anrop 144.050}               \\
         &           &         &              & MS random 144.100                       \\ \hline
144.1100 & 144.1500  & 500 Hz  & CW, MGM      & EME MGM 144.120--144.160                \\
         &           &         &              & PSK31 cent. 144.138                     \\ \hline
144.1500 & 144.1800  & 2.7 kHz & CW, SSB, MGM & EME 144.150--144.160                    \\
         &           &         &              & MGM 144.160--144.180 anrop 144.170      \\ \hline
144.1800 & 144.3600  & 2.7 kHz & CW, SSB, MGM & MS SSB random 144.195--144.205          \\
         &           &         &              & \textbf{SSB anrop 144.300}              \\ \hline
144.3600 & 144.3990  & 2.7 kHz & CW, SSB, MGM & MS MGM random anrop 144.370             \\ \hline
144.4000 & 144.4900  & 500 Hz  & Fyr          & Exklusivt segment fyrar, ej QSO         \\ \hline
144.5000 & 144.7940  & 20 kHz  & All mode     & SSTV, RTTY, FAX, ATV                    \\
         &           &         &              & Linjära transpondrar                    \\ \hline
144.7940 & 144.9625  & 12 kHz  & MGM          & APRS 144.800                            \\ \hline
144.9750 & 145.19350 & 12 kHz  & FM, DV       & Rpt in 144.975--145.1935                \\
         &           &         &              & RV46–-RV63, 12.5 kHz, 600 kHz skift     \\ \hline
145.1940 & 145.2060  & 12 kHz  & FM rymd      & 145.200 för kom. m. bem. rymdfark.      \\ \hline
145.2060 & 145.5625  & 12 kHz  & FM, DV       & FM 145.2125-–145.5875  V17–V47          \\
         &           &         &              & \textbf{FM anrop 145.500}, RTTY 145.300 \\
         &           &         &              & FM simpl. INET GW 145.2375, 2875, 3375  \\
         &           &         &              & DV anrop 145.375                        \\ \hline
145.5750 & 145.7935  & 12 kHz  & FM, DV       & Rpt ut 145.575--145.7875                \\
         &           &         &              & RV46–RV63, 12.5 kHz kanalavstånd        \\ \hline
145.794  & 145.806   & 12 kHz  & FM Rymd      & 145.800, 145.200 dplx m. bem. rymdfark. \\ \hline
145.806  & 146.000   & 12 kHz  & All mode     & Exklusivt satellit                      \\ \hline
\end{tabular}}

\subsection{Bandplan 70 cm 432--438 MHz}
\resizebox{\textwidth}{!}{%
\begin{tabular}{rrrll}
	\textbf{Frekvens} &          & \textbf{BW} & \textbf{Trafik} & \textbf{Anmärkning}                               \\ \hline
	         432.0000 & 432.0250 &      500 Hz & CW              & EME exklusivt.                                    \\ \hline
	         432.0250 & 432.1000 &      500 Hz & CW, PSK31       & CW mellan 432.000--085, \textbf{CW anrop 432.050} \\
	                  &          &             &                 & PSK31 432.088                                     \\ \hline
	         432.1000 & 432.3990 &     2.7 kHz & CW, SSB, MGM    & \textbf{SSB anrop 432.200}                        \\
	                  &          &             &                 & Mikrovåg talkback 432.350, FSK441 432.370         \\ \hline
	         432.4000 & 432.4900 &      500 Hz & Fyr             & Exklusivt segment för fyrar                       \\ \hline
	         432.5000 & 432.5940 &      12 kHz & All mode        & Linjära transpondrar IN 432.500--600              \\ \hline
	         432.5000 & 432.5750 &      12 kHz & All mode        & NRAU Digital rep. in 432.500--575 2 MHz skift     \\ \hline
	         432.5940 & 432.9940 &      12 kHz & All mode        & Linjära transpondrar ut 432.600--800              \\ \hline
	         432.5940 & 432.9940 &      12 kHz & FM              & Rep. in 432.600--975 RU368--398 2 MHz skift       \\ \hline
	         432.9940 & 433.3810 &      12 kHz & FM              & Rep. in 433.000--375 RU368--398 1.6 MHz skift     \\ \hline
	         433.3940 & 433.5810 &      12 kHz & FM              & SSTV (FM/AFSK) 433.400                            \\
	                  &          &             &                 & FM simplex U272--286 \textbf{anrop 433.500}       \\ \hline
	         433.6000 & 434.0000 &      20 kHz & All mode        & RTTY (FM/AFSK) 433.600                            \\
	                  &          &             &                 & FAX 433.700, APRS 433.800                         \\ \hline
	         434.0000 & 434.4940 &      20 kHz & All mode        & NRAU Dig. kanaler 433.450, 434.475                \\ \hline
	         434.5000 & 434.5940 &      20 kHz & All mode        & NRAU Dig. rep. ut 434.500--575, 2 MHz skift       \\ \hline
	         434.5940 & 434.9810 &      12 kHz & FM              & NRAU Rep. ut 434.600--975 RU 368--RU398           \\
	                  &          &             &                 & 12,5 kHz med 2 MHz skift                          \\ \hline
	          435.000 &  438.000 &      20 kHz & All mode        & Exklusivt satellit
\end{tabular}}

\subsection{Bandplan 23 cm 1240--1300 MHz}
\resizebox{\textwidth}{!}{%
\begin{tabular}{rrrll}
	\textbf{Frekvens} &          & \textbf{BW} & \textbf{Trafik} & \textbf{Anmärkning}                           \\ \hline
	         1240.000 & 1243.250 &      20 kHz & Alla moder      & 1240.000 - 1241.000 Digital kommunikation     \\ \hline
	         1243.250 & 1260.000 &      20 kHz & ATV och Data    & Repeater ut 1258.150-1259.350, R20--68        \\ \hline
	         1260.000 & 1270.000 &      12 kHz & Satellit        & Endast för satelliter alla moder              \\ \hline
	         1270.000 & 1272.000 &      20 kHz & Alla moder      & Repeater in, 1270.025-1270.700, RS1--28       \\
	                  &          &             &                 & Packet RS29--50                               \\ \hline
	         1272.000 & 1290.994 &      20 kHz & ATV och Data    & Amatörtelevision ATV                          \\ \hline
	         1290.994 & 1291.481 &      20 kHz & FM och DV       & Repeater in Repeat. in 1291.000--1291.475     \\
	                  &          &             &                 & RM0 – RM19, 25 kHz, 6 MHz skift               \\ \hline
	         1291.494 & 1296.000 &      12 kHz & Alla moder      &                                               \\ \hline
	         1296.000 & 1296.150 &      500 Hz & CW,  MGM        & EME 1296.000--025, \textbf{CW anrop 1296.050} \\
	                  &          &             &                 & PSK31 1296.138 MHz                            \\ \hline
	         1296.150 & 1296.400 &     2.7 kHz & CW, SSB, MGM    & \textbf{SSB anrop 1296.200}                   \\
	                  &          &             &                 & \textbf{FSK441 MS anrop 1296.370}             \\ \hline
	         1296.400 & 1296.600 &     2.7 kHz & CW, SSB, MGM    & Linjära transpondrar infrekvens               \\ \hline
	         1296.600 & 1296.800 &     2.7 kHz & CW, SSB, MGM    & SSTV/FAX 1296.500, MGM/RTTY 1296.600          \\ \hline
	         1296.600 & 1296.800 &     2.7 kHz & CW, SSB, MGM    & Linjära transpondrar utfrekvens               \\
	                  &          &             &                 & 1296.750-.800 lokala fyrar max 10 W           \\ \hline
	         1296.800 & 1296.994 &      500 Hz & Fyrar           & Exklusivt segment för fyrar                   \\ \hline
	         1296.994 & 1297.481 &      20 kHz & FM              & Repeater ut Repeater ut 1297.000--1297.475    \\
	                  &          &             &                 & RM0 – RM19, 25 kHz, 6 MHz skift               \\ \hline
	         1297.494 & 1297.981 &      20 kHz & FM simplex      & Simplex 25 kHz kanaler SM20--39               \\
	                  &          &             &                 & \textbf{FM anrop 1297.500 SM20}               \\ \hline
	         1299.000 & 1299.750 &     150 kHz & Alla moder      & 5 st 150 kHz kanaler för DD,                  \\
	                  &          &             &                 & 1299.075, 225, 375, 525, och 675 $\pm$75 kHz  \\ \hline
	         1299.750 & 1300.000 &      20 kHz & Alla moder      & 8 st FM/DV 25 kHz kan. 1299.775--1299.975
\end{tabular}}

\clearpage

\section{Allmänna Frekvenser HF}

\subsection{PR-bandet 27 MHz}

Detta är det enda bandet som allmänheten kan använda på HF-bandet. Det delar
många egenskaper med 31\,MHz jaktradiobandet men är ett band som är äldre och
mer etablerat.

Maximal uteffekt på bandet är 4W RMS ERP dvs antennvinst överstigande en
1/2-vågs dipol (0 dBd, 2.12 dBi) måste inräknas i effekten efter avdrag för
matningsförlust. Modulationsslag AM, FM och SSB (primärt används USB) är
tillåtet på alla kanaler i dag. Traditionellt används kanal 24 för USB men i
dag får vilken kanal som helst användas.

Kanalerna med A efter är upplåtna för radiostyrning och inte för telefoni.
Undvik därför att använda dessa om du har en sändare som kan använda dessa
frekvenser. De är med i tabellen för den skall vara komplett.

\begin{longtable}{rrl|rrl}
	\textbf{Frekvens}& \textbf{Kanalnr}& \textbf{Övrigt}
	& \textbf{Frekvens} & \textbf{Kanalnr} & \textbf{Övrigt}  \\
	\hline \endhead
	  26,965 &       1 &                &   27,215 &      21 &           \\
	  26,975 &       2 &                &   27,225 &      22 &           \\
	  26,985 &       3 &                &   27,255 &      23 &           \\
	  26,995 &      3A & Radiostyrning  &   27,235 &      24 & SSB (USB) \\
	  27,005 &       4 &                &   27,245 &      25 &           \\
	  27,015 &       5 &                &   27,265 &      26 &           \\
	  27,025 &       6 &                &   27,275 &      27 &           \\
	  27,035 &       7 &                &   27,285 &      28 &           \\
	  27,045 &      7A & Radiostyrning  &   27,295 &      29 &           \\
	  27,055 &       8 &                &   27,305 &      30 &           \\
	  27,065 &       9 &                &   27,315 &      31 &           \\
	  27,075 &      10 &                &   27,325 &      32 &           \\
	  27,085 &      11 &                &   27,335 &      33 &           \\
	  27,095 &     11A & f.d. nödfrekv. &   27,345 &      34 &           \\
	  27,105 &      12 &                &   27,355 &      35 &           \\
	  27,115 &      13 &                &   27,365 &      36 &           \\
	  27,125 &      14 &                &   27,375 &      37 &           \\
	  27,135 &      15 &                &   27,385 &      38 &           \\
	  27,155 &      16 &                &   27,395 &      39 &           \\
	  27,165 &      17 &                &   27,405 &      40 &           \\
	  27,175 &      18 &                &          &         &           \\
	  27,185 &      19 &                &          &         &           \\
	  27,195 &     19A & Radiostyrning  &          &         &           \\
	  27,205 &      20 &                &          &         &
\end{longtable}

Många apparater är endast FM i dag men det finns de som också har SSB. Äldre
apparater hade oftast AM och FM och ibland även SSB. Telegrafi körs i princip
inte på PR-bandet, troligen för att det aldrig varit några krav på det och de
som kör heller inte haft möjlighet förr i tiden att DX-a på bandet.

Innan Televerket släppte upp bestämmelserna var det väldigt hårda bestämmelser
på bandet, i princip var det bara kommunikation inom familjen som tilläts. I dag
kan bandet användas som man vill och det är på sina håll god aktivitet.

Kom ihåg att inte överskrida effektbegränsningarna bara.

\subsection{Marina MF/HF-frekvenser}

De marina HF-banden är uppdelade på ett antal band. Det finns en
generell kanalindelning med 3 kHz per kanal och SSB som
modulationssätt på respektive band. Marina HF-kanaler finns på banden
4, 6, 8, 12, 16, 18, 22 och 25 MHz.

MF även benämnd gränsvåg i marina sammanahang är inte lika ofta
används som den var en gång i tiden. Nedan listas de frekvenser som
används i Sverige.

\subsection{Svenska MF-kanaler}

\begin{longtable}{llrr}
\textbf{Kanal} & \textbf{Placering} & \textbf{Skepp} & \textbf{Kust}  \\ \hline
\endhead

MF1 & Gotland       & 2\,099 & 1\,6874 \\
MF2 & ---           & ---  & ---   \\
MF3 & Gislövshammar & 2\,060 & 1\,797  \\
MF4 & Härnösand     & 2\,216 & 2\,733  \\
MF5 & Bjuröklubb    & 2\,123 & 1\,779  \\
MF6 & Grimeton      & 2\,135 & 1\,710
\end{longtable}

\subsection{Nödfrekvenser}

\begin{longtable}{lrr}
\textbf{Band} & \textbf{Frekvens} & \textbf{DSC Frekvens}\\ \hline \endhead

MF   & 2 182  & 2 187.5  \\
HF4  & 4 125  & 4 207.5  \\
HF6  & 6 215  & 6 312.0  \\
HF8  & 8 291  & 8 414.5  \\
HF12 & 12 290 & 12 577.0 \\
HF16 & 16 429 & 16 804.5 \\
\end{longtable}

\subsection{Primära HF skepp-till-skepp}

\begin{longtable}{lrrrrrrrr}
\textbf{Kanal} & \textbf{HF4} & \textbf{HF6} & \textbf{HF8} &
               \textbf{HF12} & \textbf{HF16} & \textbf{HF18} &
               \textbf{HF22} & \textbf{HF25} \\
\hline
\endhead

A & 4 146 & 6 224 & 8 294 & 12 353 & 16 528 & 18 825 & 22 159 & 25 100 \\
B & 4 149 & 6 227 & 8 297 & 12 356 & 16 531 & 18 828 & 22 162 & 25 103 \\
C &       & 6 230 &       & 12 359 & 16 534 & 18 831 & 22 165 & 25 106 \\
D &       &       &       & 12 362 & 16 537 & 18 834 & 22 168 & 25 109 \\
E &       &       &       & 12 365 & 16 540 & 18 837 & 22 171 & 25 112 \\
F &       &       &       &        & 16 543 & 18 840 & 22 174 & 25 115 \\
G &       &       &       &        & 16 546 & 18 843 & 22 177 & 25 118 \\
\end{longtable}

\section{Amatörradiofrekvenser LF--HF}

Alla frekvenser i kHz, bandbredder i Hz.

\subsection{Bandplan 2.2\,km, 135,7--137,8\,kHz}

\begin{tabular}{rrrll}
\textbf{Frekvens} &  & \textbf{BW} & \textbf{Trafik} & \textbf{Noteringar} \\ \hline
135,7 & 135,8 & 200 & CQ, QRSS, Digi & OBS! Högsta effekt 1W ERP. \\ \hline
\end{tabular}

\subsection{Bandplan 600\,m, 472--479\,kHz}
\begin{tabular}{rrrll}
\multicolumn{2}{c}{\textbf{Frekvens}} & \textbf{BW} & \textbf{Trafik} & \textbf{Noteringar} \\ \hline
472 & 479 & 200 & CW, QRSS, Digi & OBS! Högsta utstrålad effekt 1W EIRP \\ \hline
\end{tabular}

\subsection{Bandplan 160\,m, 1810--2000\,kHz}
\begin{tabular}{rrrll}
\multicolumn{2}{c}{\textbf{Frekvens}} & \textbf{BW} & \textbf{Trafik} & \textbf{Noteringar} \\ \hline
1810 & 1838 & 200  & CW         & Exklusivt för CW. Interkontinental trafik har prio. \\ \hline
1838 & 1840 & 500  & Smalband   & Ej packet på 160m, PSK 1838,150                    \\ \hline
1840 & 1850 & 2700 & Alla moder & Även digimode. SSB QRP 1843 kHz                    \\ \hline
1850 & 1900 & 2700 & Alla moder & OBS! Max 10 W till ant.                             \\ \hline
1900 & 1950 & 2700 & Alla moder & OBS! Max 100 W till ant.                            \\ \hline
1950 & 2000 & 2700 & Alla moder & OBS! Max 10 W till ant.                             \\ \hline
\end{tabular}

\subsection{Bandplan 80\,m, 3500--3800\,kHz}
\begin{tabular}{rrrll}
\multicolumn{2}{c}{\textbf{Frekvens}} & \textbf{BW} & \textbf{Trafik} & \textbf{Noteringar} \\ \hline
3500 & 3510 & 200  & CW             & Exklusivt CW                         \\
      &       &      &                & Interkontinental DX-trafik har prio  \\ \hline
3510 & 3580 & 200  & CW             & Exklusivt CW contest 3510-–560       \\
      &       &      &                & CW QRS 3 555 kHz, CW QRP 3 560       \\ \hline
3580 & 3600 & 500  & Smalband, Digi & PSK 3580,150                        \\
      &       &      &                & Automatiska Digimoder 3590--600     \\ \hline
3600 & 3620 & 2700 & Alla moder     & Digimoder Automatiska Digimoder      \\ \hline
3600 & 3650 & 2700 & Alla moder     & SSB contest 3600--650               \\
      &       &      &                & DV 3630                             \\ \hline
3650 & 3700 & 2700 & Alla moder     & SSB QRP 3690                        \\ \hline
3700 & 3800 & 2700 & Alla moder     & Contest 3700-–800                   \\
      &       &      &                & Image 3775                          \\
      &       &      &                & Region 1 nödfrekvens 3760           \\ \hline
3775 & 3800 & 2700 & Alla moder     & Interkontinental DX-trafik prioritet \\ \hline
\end{tabular}


\subsection{Bandplan 60\,m, 5351\,5--5366,5\,kHz}
\begin{tabular}{rrrll}
\multicolumn{2}{c}{\textbf{Frekvens}} & \textbf{BW} & \textbf{Trafik} & \textbf{Noteringar} \\ \hline
5351\,5 & 5354\,0 &  200 & CW, Digimoder      & OBS! Högsta utstrålad effekt 15 W EIRP\\
        &         &      &                    & i 60\,m-bandet\\ \hline
5354\,0 & 5366\,0 & 2700 & Alla sändningsslag & USB rekommenderas för SSB\\ \hline
5366\,0 & 5366\,5 &   20 & Smalbandsmoder     & För extrema smalbandsmoder,\\
        &         &      &                    & max 20 Hz bandbredd\\ \hline
\end{tabular}

\subsection{Bandplan 40\,m, 7000--7200\,kHz}
\begin{tabular}{rrrll}
\multicolumn{2}{c}{\textbf{Frekvens}} & \textbf{BW} & \textbf{Trafik} & \textbf{Noteringar} \\ \hline
7000 & 7040 & 200  & CW         & Exklusivt CW.                             \\
     &      &      &            & QRP aktivitetscentrum 7030\,kHz           \\ \hline
7040 & 7050 & 500  & Smalband   & Digimoder Automatiska inom 7047–-050\,kHz \\ \hline
7050 & 7060 & 2700 & Alla moder & Digimoder Automatiska inom 7050–-053\,kHz \\ \hline
7060 & 7100 & 2700 & Alla moder & SSB contest i segmentet                   \\
     &      &      &            & DV 7 070 kHz, SSB QRP 7090 kHz            \\ \hline
7100 & 7130 & 2700 & Alla moder & Region 1 nödfrekvens 7110 kHz             \\ \hline
7130 & 7200 & 2700 & Alla moder & SSB contest i segmentet                   \\
     &      &      &            & Image 7165\,kHz                           \\ \hline
7175 & 7200 & 2700 & Alla moder & Interkontinental DX-trafik prio           \\ \hline
\end{tabular}

\subsection{Bandplan 30\,m, 10\,100--10\,150 kHz}
\begin{tabular}{rrrll}
\multicolumn{2}{c}{\textbf{Frekvens}} & \textbf{BW} & \textbf{Trafik} & \textbf{Noteringar} \\ \hline
10\,100 & 10\,140 & 200 & CW       & CW exkl. Max 150 Watt på 30 m           \\
        &         &     &          & CW QRP 10\,116\,kHz                     \\ \hline
10\,140 & 10\,150 & 500 & Smalband & Digimoder PSK 10142,150\,kHz. Ej Packet \\ \hline
\end{tabular}

\subsection{Bandplan 20\,m, 14\,000--14\,350 kHz}
\begin{tabular}{rrrll}
\multicolumn{2}{c}{\textbf{Frekvens}} & \textbf{BW} & \textbf{Trafik} & \textbf{Noteringar} \\ \hline
14\,000 & 14\,070 & 200  & CW         & Exklusivt CW                             \\
        &         &      &            & Conctest 14\,000-–060                    \\
        &         &      &            & CW QRS 14 055, CW QRP 14\,060            \\ \hline
14\,070 & 14\,099 & 500  & Smalband   & PSK 14 070,150                           \\
        &         &      &            & Auto Digimoder 14 089-–099               \\ \hline
14\,099 & 14\,101 & 200  & Fyrar      & Exklusivt IBP, endast fyrar              \\ \hline
14\,101 & 14 \,12 & 2700 & Alla moder & Digitala moder och obevakade Digimoder   \\ \hline
14\,112 & 14\,350 & 2700 & Alla moder & SSB Contest 14 125--300                  \\
        &         &      &            & DV 14 130, DXpedition prio 14\,195$\pm$5 \\ \hline
14\,300 & 14\,350 & 2700 & Alla moder & Image 14\,230, SSB QRP 14\,285           \\
        &         &      &            & Global nödfrekvens 14 300                \\ \hline
\end{tabular}

\subsection{Bandplan 17 m, 18\,068--18\,168 kHz}
\begin{tabular}{rrrll}
\multicolumn{2}{c}{\textbf{Frekvens}} & \textbf{BW} & \textbf{Trafik} & \textbf{Noteringar} \\ \hline

18\,068 & 18\,095 & 200  & CW         & CW exklusivt. QRP 18\,086             \\ \hline
18\,095 & 18\,109 & 500  & Smalband   & Digimoder PSK 18\,100,150             \\
        &         &      &            & Automatiska Digimoder 18\,105--109 \\ \hline
18\,109 & 18\,111 & 200  & Fyrar      & Exklusivt fyrar, IBP fyrnät           \\ \hline
18\,111 & 18\,168 & 2700 & Alla moder & Digi 18\,111--120                   \\
        &         &      &            & SSB QRP 18\,130, DV 18\,150           \\
        &         &      &            & Global nödfrekv. 18\,160              \\ \hline
\end{tabular}

\subsection{Bandplan 15\,m, 21\,000--21\,450 kHz}
\begin{tabular}{rrrll}
\multicolumn{2}{c}{\textbf{Frekvens}} & \textbf{BW} & \textbf{Trafik} & \textbf{Noteringar} \\ \hline

21\,000 & 21\,070 & 200  & CW         & Exklusivt CW, QRS 21\,055, CW QRP 21\,060           \\ \hline
21\,070 & 21\,110 & 500  & Smalband   & PSK 21\,080.150, Automatiska Digimoder 21\,090–-110 \\
21\,110 & 21\,120 & 2700 & Alla moder & Alla moder utom SSB!                                \\
        &         &      &            & Digimoder, och Automatiska Digimoder                \\ \hline
21\,120 & 21\,149 & 500  & Smalband   &                                                     \\ \hline
21\,149 & 21\,151 & 200  & Fyrar      & Exklusivt fyrar. IBP fyrnät                         \\ \hline
21\,151 & 21\,450 & 2700 & Alla moder & DV 21\,180, SSB QRP 21\,285, Image 21\,340          \\
        &         &      &            & Global nödfrekv. 21\,360                            \\ \hline
\end{tabular}

\subsection{Bandplan 12\,m, 24\,890--24\,990 kHz}
\begin{tabular}{rrrll}
\multicolumn{2}{c}{\textbf{Frekvens}} & \textbf{BW} & \textbf{Trafik} & \textbf{Noteringar} \\ \hline
24\,890 & 24\,915 & 200  & CW         & Exklusivt CW, QRP 24\,906                             \\ \hline
24\,915 & 24\,929 & 500  & Smalband   & PSK 24\,920,150, Automatiska Digimoder 24\,925–-929 \\ \hline
24\,929 & 24\,931 & 200  & Fyrar      & Fyrar, IBP fyrnät                                    \\ \hline
24\,931 & 24\,990 & 2700 & Alla moder & Auto Digimoder 24\,931-–940                        \\
       &        &      &            & SSB QRP 24\,950, DV 24,960                            \\ \hline
\end{tabular}

\small
\subsection{Bandplan 10\,m, 28\,000-29\,700 kHz}
\begin{tabular}{rrrll}
\multicolumn{2}{c}{\textbf{Frekvens}} & \textbf{BW} & \textbf{Trafik} & \textbf{Noteringar} \\ \hline
28\,000 & 28\,070 & 200  & CW         & Exklusivt CW, QRS 28\,055, CW QRP 28\,060              \\ \hline
28\,070 & 28\,190 & 500  & Smalband   & PSK 28 120,150, Auto Digimoder inom 28\,120--150       \\ \hline
28\,190 & 28\,199 & 200  & Fyrar IBP  & Regionala fyrar med tidsdelning                        \\ \hline
28\,199 & 28\,201 & 200  & Fyrar IBP  & IBP fyrnät                                             \\ \hline
28\,201 & 28\,225 & 200  & Fyrar IBP  & kontinuerligt sändande fyrar                           \\ \hline
28\,225 & 28\,300 & 2700 & Alla moder & Övriga fyrar                                           \\ \hline
28\,300 & 28\,320 & 2700 & Alla moder & Digimoder och Automatiska Digimoder                    \\ \hline
28\,320 & 29\,100 & 2700 & Alla moder & DV 28\,330 kHz, SSB QRP 28\,360 kHz                    \\
        &         &      &            & Image 28\,680 kHz                                      \\ \hline
29\,100 & 29\,200 & 6000 & Alla moder & FM simplex, 10\,kHz kanaler                            \\
        &         &      &            & Maximalt ±2,5 kHz dev., max 2,5\,kHz mod.frek.         \\ \hline
29\,200 & 29\,300 & 6000 & Alla moder & Digimoder och Automatiska Digimoder                    \\ \hline
29\,300 & 29\,510 & 6000 & Satellit   & Nerlänk fr. satellit. EJ SÄNDNING I SEGMENTET          \\ \hline
29\,510 & 29\,520 & 6000 & Skydd      & Skyddsfrekvens för satelliter. EJ SÄNDNING I SEGMENTET \\ \hline
29\,520 & 29\,590 & 6000 & Alla moder & FM Repeater in RH1--8, 100\,kHz duplex, 2.5\,kHz NBFM  \\ \hline
29\,600 & 29\,620 & 6000 & Alla moder & FM simplex, anrop 29\,600                              \\
        &         &      &            & FM simplex repeater 29\,610                            \\ \hline
29\,620 & 29\,700 & 6000 & Alla moder & FM Repeater ut RH1--8, 100\,kHz duplex                 \\ \hline
\end{tabular}
%\end{landscape}
\normalsize

\clearpage

\begin{landscape}
\section{Amatörradiofyrar}
\subsection{Svenska fyrar VHF/UHF}
\begin{longtable}{rlllrlrrl}
	          \bf Dist & \bf Signal & \bf Mod  & \bf Ort                & \bf Frekvens & \bf Lokator & \bf MASL & \bf MAGL & \bf Riktning \\ \hline
	\endhead
	       0 & SKØCT/B    & CW       & Stockholm              &    5760.9030 & JO99JX      &       60 &       30 & Omni         \\
	                 0 & SKØCT/B    & CW       & Stockholm              &   10368.8400 & JO89XJ      &       50 &       20 & Omni         \\
	                 0 & SKØEN/B    & CW       & Väddö                  &   10368.8470 & JO99JX      &       60 &       30 & 360°         \\
	                 0 & SKØEN/B    & CW       & Väddö                  &    1296.8350 & JO99JX      &       70 &       40 & Omni         \\
	                 1 & SK1UHF     & CW       & Klintehamn             &     432.4050 & JO97CJ      &       65 &       60 & Omni         \\
	                 1 & SK1VHF     & CW       & Klintehamn             &     144.4470 & JO97CJ      &       65 &       60 & Omni         \\
	                 1 & SK1UHG     & CW       & Klintehamn             &    1296.9500 & JO97CJ      &       65 &       60 & Omni         \\
	                 2 & SK2VHF     & CW       & Vindeln/Buberget       &     144.4570 & JP94TF      &      300 &       10 & N+SV         \\
	                 2 & SK2SHF     & CW       & Vännäs                 &    1296.9850 & JP93VU      &      250 &       50 &              \\
	                   &            &          & Granlundsb.            &              &             &          &          &              \\
	                 3 & SM3KDR     & CW       & Krokom/Aspås           &      28.2860 & JP73GI      &      380 &        5 & E-W          \\
	                 3 & SK3UHF     & CW       & Nordingrå              &     432.4550 & JP92FW      &      200 &        8 & Omni         \\
	                 3 & SK3UHG     & CW       & Nordingrå              &    1296.8550 & JP92FW      &      200 &       10 & Omni         \\
	                 3 & SK3SIX     & CW       & Östersund              &      50.4680 & JP73HC      &      480 &        7 & Omni         \\
	                 4 & SK4MPI     & PI4 - CW & Borlänge               &     144.4120 & JP70PI      &      380 &       20 & NV+NO        \\
	                 4 & SK4BX/B    & CW       & Garphyttan/Storstensh. &     432.4600 & JO79LH      &      270 &       10 & N E S W      \\
	                 4 & SK4BX/B    & CW       & Garphyttan/Ånnaboda    &   10368.9600 & JO79LI      &      270 &       10 &              \\
	                 4 & SK4BX/B    & CW       & Garphyttan/Ånnaboda    &    1296.9600 & JO79LI      &      270 &       10 &              \\
	                 6 & SK6YH/B    & CW       & Göteborg               &   10368.8080 & JO57XQ      &      135 &       40 & 184°         \\
	                 6 & SK6MHI     & CW       & Göteborg               &    5760.8000 & JO57XQ      &      135 &       40 & Omni         \\
	                 6 & SK6MHI     & CW       & Göteborg               &   24048.8000 & JO57XQ      &      135 &       40 & Omni         \\
	                 6 & SK6MHI     & CW       & Göteborg               &   10368.8000 & JO57XQ      &      135 &       40 & Omni         \\
	                 6 & SK6MHI     & CW       & Hönö                   &    1296.8000 & JO57TQ      &       40 &       30 & Omni         \\
	                 6 & SK6WW/B    & CW       & Karlsborg/Vaberget     &   10368.8350 & JO78FM      &      240 &       20 & Omni         \\
	                 6 & SK6EI/B    & CW       & Skövde                 &      50.4600 & JO68VJ      &      300 &       30 & South        \\
	                 6 & SK6SHG     & CW       & Tjörn Island           &   24048.8830 & JO57TX      &      118 &        8 & N / S        \\
	                 6 & SK6UHI     & CW       & Tjörn Island           &    1296.8050 & JO57TX      &      128 &       18 & Omni         \\
	                 6 & SK6VHF     & CW       & Tjörn Island           &     144.4060 & JO57TX      &      122 &       12 & Omni         \\
	                 6 & SK6UHF     & CW       & Varberg/Veddige        &     432.4120 & JO67EH      &      175 &       25 & Omni         \\
	                 7 & SM7DTE/B   & CW       & Gärsnäs                &    5760.8410 & JO75DN      &       86 &        8 & Omni         \\
	                 7 & SM7DTE/B   & CW       & Gärsnäs                &   10368.8410 & JO75DN      &       86 &        8 & Omni         \\
	                 7 & SM7DTE/B   & CW       & Gärsnäs                &   24048.8430 & JO75DN      &       86 &        8 & Omni         \\
	                 7 & SK7VHF     & CW       & Sjöbo                  &     144.4610 & JO65UQ      &       25 &       25 & Omni         \\
	                 7 & SK7GH/B    & CW       & Värnamo                &      28.2980 & JO77BF      &      230 &       10 & Omni
\end{longtable}
\end{landscape}

\clearpage

\subsection{IBP -- International Beacon Project}

Det finns flera olika typer av fyrar men för HF är IBP (International Beacon
Project) intressant eftersom det ger operatören möjlighet att utröna hur
utbredningen ser ut för stunden genom att lyssna efter fyrar. Fyrarna har
gemensam hårdvara och synkroniseras mot tidsreferens. Fyrar kan vara offline
av olika skäl, kontrollera mot IBP:s hemsida om du inte hör en fyr du brukar
höra.

Tabellen nedan visar anropssignaler och första sändningssloten som fyren
sänder, dvs SCHED för olika fyrar och frekvenser.

\subsection{Lista över IBP-fyrar}

\begin{table}[H]
\centering
\begin{tabular}{llrrrrr}
\textbf{Signal} & \textbf{QTH} & \textbf{14 100} & \textbf{18 110} &
                \textbf{21 150} & \textbf{24 930} & \textbf{28 200} \\ \hline
4U1UN  & United Nations & 00:00  & 00:10  & 00:20  & 00:30  & 00:40  \\
VE8AT  & Canada         & 00:10  & 00:20  & 00:30  & 00:40  & 00:50  \\
W6WX   & United States  & 00:20  & 00:30  & 00:40  & 00:50  & 01:00  \\
KH6RS  & Hawaii         & 00:30  & 00:40  & 00:50  & 01:00  & 01:10  \\
ZL6B   & New Zealand    & 00:40  & 00:50  & 01:00  & 01:10  & 01:20  \\
VK6RBP & Australia      & 00:50  & 01:00  & 01:10  & 01:20  & 01:30  \\
JA2IGY & Japan          & 01:00  & 01:10  & 01:20  & 01:30  & 01:40  \\
RR9O   & Russia         & 01:10  & 01:20  & 01:30  & 01:40  & 01:50  \\
VR2B   & Hong Kong      & 01:20  & 01:30  & 01:40  & 01:50  & 02:00  \\
4S7B   & Sri Lanka      & 01:30  & 01:40  & 01:50  & 02:00  & 02:10  \\
ZS6DN  & South Africa   & 01:40  & 01:50  & 02:00  & 02:10  & 02:20  \\
5Z4B   & Kenya          & 01:50  & 02:00  & 02:10  & 02:20  & 02:30  \\
4X6TU  & Israel         & 02:00  & 02:10  & 02:20  & 02:30  & 02:40  \\
OH2B   & Finland        & 02:10  & 02:20  & 02:30  & 02:40  & 02:50  \\
CS3B   & Madeira        & 02:20  & 02:30  & 02:40  & 02:50  & 00:00  \\
LU4AA  & Argentina      & 02:30  & 02:40  & 02:50  & 00:00  & 00:10  \\
OA4B   & Peru           & 02:40  & 02:50  & 00:00  & 00:10  & 00:20  \\
YV5B   & Venezuela      & 02:50  & 00:00  & 00:10  & 00:20  & 00:30  \\
\end{tabular}
\caption{IBP-fyrar}
\end{table}

\normalsize

\section{JOTA -- Scouter på amatörradiobanden}

Scouter finns ofta QRV under vissa helger, \textit{Jamboree On The Air, JOTA},
förekommer några gånger per år. Här är en sammanställning av de
standardfrekvenser scouter nyttjar om de inte kör repeatrar eller leta upp
motstationer själva. Scouter kan antingen ha egna signaler, köra under
tillfälliga signaler eller vara second operator åt med någon klubbsignal.

Scouterna har frekvenser på HF likväl som VHF/UHF som de aktiverar vid
särskilda tillfällen ofta i tillsammans med en lokal amatörradioklubb eller
vanliga amtörradioeldsjälar som inte sällan också är scouter. Här kommer en
lista på frekvenser som är vanligt förekommande i scoutsammanhang.

Jotan hålls alltid den 3:e hela (lördag och söndag) helgen i oktober varje år.
Jotan startar officiellt vid invigningen på lördag förmiddag och slutar natten
till måndagen klockan 00:00. Många börjar redan på fredagskvällen och avslutar
på söndagseftermiddagen.

Sändningar under denna tid förekommer från ocertifierade scouter som lånar
klubbsignal, har en tillfällig signal, eller liknande. Ibland lånar
enskilda sändaramatörer ut sina signaler och det finns också ganska många
scouter som har eget amatörradiocertifikat. Många gånger är det också så
kallade tillfälliga "event-signaler" som är i luften.

Sändningarna sker dock alltid ske under direkt överinseende av en radioamatör
men var beredd på att det kommer vara en viss ovana och ske en del misstag.
Strunta i det och ge scouterna en kul radioupplevelse.

\subsection{Nordiska scoutfrekvenser VHF}

\begin{table}[h]
\centering
\begin{tabular}{lrr}
	\textbf{Mode} & \textbf{Frekvens} & \textbf{Kanal} \\ \hline
	FM            &      145.425  MHz &   V34 \\
	SSB           &      144.240  MHz &  \\
	CW            &      144.050  MHz &
\end{tabular}
\caption{Scouters JOTA-frekvenser på 2\,m-bandet}
\end{table}

\subsection{Scoutfrekvenser HF}

\begin{table}[H]
\centering
\begin{tabular}{rrll}
	\textbf{Band} & \textbf{Frekvens} & \textbf{Trafik} & \textbf{Not} \\ \hline
               80 & 3 570  & CW  &             \\
	              & 3 940  & SSB & Ej region 2 \\
	              & 3 690  & SSB &             \\ \hline
	           40 & 7 030  & CW  &             \\
	              & 7 190  & SSB &             \\
	              & 7 090  & SSB &             \\ \hline
	           20 & 14 060 & CW  &             \\
	              & 14 290 & SSB &             \\ \hline
	           17 & 18 080 & CW  &             \\
	              & 18 140 & SSB &             \\ \hline
	           15 & 21 140 & CW  &             \\
	              & 21 360 & SSB &             \\ \hline
	           12 & 24 910 & CW  &             \\
	              & 24 960 & SSB &             \\ \hline
	           10 & 28 180 & CW  &             \\
	              & 28 390 & SSB &             \\ \hline
	            6 & 10 160 & CW  &             \\
	              & 50 160 & SSB &             \\ \hline
\end{tabular}
\caption{Scouters JOTA-frekvenser på HF}
\end{table}

Normalt aktiveras dessa frekvenser tredje veckoslutet i oktober varje år,
fredag till söndag. Då kan det vara många klubbar som finns på frekvenserna
och det är också vanligt att man hör dem på helt andra frekvenser. De
frekvenser som listas här är inte på något vis de enda frekvenser som scouter
använder.
