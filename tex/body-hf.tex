\clearpage

\section{Trafik- och tumregler HF}
\subsection{Kort sammanfattning av reglemente}

OBS! Detta är inte fullständigt radioreglemente naturligtvis utan
endast sammanfattning av några viktiga punkter.

\subsubsection{Begrepp i bandplanerna}

\begin{itemize}
\item QRP: Aktivitetscentrum för låg effekt ($<$5W), svaga signaler förekommer, visa hänsyn.
\item QRS: Aktivitetscenter för långsam CW.
\item QRSS: Extremt långsam CW med dator.
\item DV: Digital Voice.
\item Image: Bildmoder exempelvis SSTV och Fax som ryms inom den specificerade maximala bandbredden.
\end{itemize}

\subsubsection{Trafikregler och tumregler}

\begin{itemize}
\item Vid SSB-telefoni används LSB på frekvenser under 10 MHz och USB
      på frekvenser över 10 MHz.
\item Lägsta acceptabla inställda frekvens för LSB är 3 kHz över
      under bandkant!
\item Högsta acceptable inställda frekvens för USB är 3 kHz under
      övre bandkant!
\item IBP är International Beacon Project. Fyrarna sänder med 3 min
      intervaller och används för att studera utbredningen av
      radiosignaler globalt. Fyrarna sänder anrop och fyra 1 s toner.
      Anropet och första tonen sänds med 100W, därefter sänds tonerna
      med 10W, 1W samt 100mW.
\item Vid AM (A3J) skall hänsyn tas så att störningar på annan trafik ej fö\-re\-kom\-mer
      med de sidband som då uppstår, det gäller då både övre och undre
      sidbandet.
\item Ingen som helst sändning är tillåtet inom fyrsegmenten. Detta skall respekteras.
      Lyssna gärna på nödfrekvenserna men används dem icke, om det
      inte är du som svarar på ett nödsamtal! Undvik QSO allt för nära
      dessa också.
\item Var särskilt uppmärksam på satelliters nerlänksfrekvenser på 10\,m-bandet.
      I detta segment skall endast lyssning ske. Ingen sändning är
      tillåten här eller i skyddssegmentet strax ovanför
      satellitsegmentet. Tänk på att satelliters frekvens kan
      dopplerskiftas uppåt en hel del när de rör sig mot mottagaren.
\end{itemize}

\section{Frekvenser HF}

\input{tex/pr-bandet-27mhz.tex}

\subsection{Amatörradiofrekvenser HF}

\todo{Beskriva bandegenskaperna}

\subsubsection{Bandens olika egenskaper}

Även om man kan säga att alla kortvågsbanden har liknande egenskaper så är
skillnaderna bland dem ofta påtagliga och eftersom det spänner över ett relativt
stor område då 160 till 6 meter brukar räknas som HF även om det strikt räknat
är VHF liksom 160 metersbandet som egentligen faller inom mellanvågsdelen eller
den så kallade gränsvågen så är de band som oftast förekommer på radiotapparater
för kortvåg så därför är de med här.

\subsubsection{160 \& 80 metersbanden}

\todo{- 160 och 80 meter}

\subsubsection{40 metersbandet}

\todo{- 40 meter}

\subsubsection{20 \& 17 metersbanden}

\todo{- 20 och 17 meter}

\subsubsection{15 \& 12 metersbanden}

\todo{- 15 och 12 meter}

\subsubsection{6 metersbandet}

\todo{- 6 meter}

\subsection{JOTA---Jamboree on the air, scoutfrekvenser}

Scouterna har frekvenser på HF likväl som VHF/UHF som de aktiverar vid särskilda
tillfällen ofta i tillsammans med en lokal amatörradioklubb eller vanliga
amtörradioeldsjälar som inte sällan också är scouter. Här kommer en lista på
frekvenser som är vanligt förekommande i scoutsammanhang.

\begin{table}[H]
\centering
\begin{tabular}{rrll}
	\textbf{Band} & \textbf{Frekvens} & \textbf{Trafik} & \textbf{Not} \\ \hline

               80 & 3 570  & CW  &             \\
	              & 3 940  & SSB & Ej region 2 \\
	              & 3 690  & SSB &             \\ \hline
	           40 & 7 030  & CW  &             \\
	              & 7 190  & SSB &             \\
	              & 7 090  & SSB &             \\ \hline
	           20 & 14 060 & CW  &             \\
	              & 14 290 & SSB &             \\ \hline
	           17 & 18 080 & CW  &             \\
	              & 18 140 & SSB &             \\ \hline
	           15 & 21 140 & CW  &             \\
	              & 21 360 & SSB &             \\ \hline
	           12 & 24 910 & CW  &             \\
	              & 24 960 & SSB &             \\ \hline
	           10 & 28 180 & CW  &             \\
	              & 28 390 & SSB &             \\ \hline
	            6 & 10 160 & CW  &             \\
	              & 50 160 & SSB &             \\ \hline
\end{tabular}
\caption{Scouters JOTA-frekvenser på HF}
\end{table}

Normalt aktiveras dessa frekvenser tredje veckoslutet i oktober varje
år, fredag till söndag. Då kan det vara många klubbar som finns på
frekvenserna och det är också vanligt att man hör dem på helt andra
frekvenser. De frekvenser som listas här är inte på något vis de enda
frekvenser som scouter använder.

\subsection{Marina MF/HF-frekvenser}

De marina HF-banden är uppdelade på ett antal band. Det finns en
generell kanalindelning med 3 kHz per kanal och SSB som
modulationssätt på respektive band. Marina HF-kanaler finns på banden
4, 6, 8, 12, 16, 18, 22 och 25 MHz.

MF även benämnd gränsvåg i marina sammanahang är inte lika ofta
används som den var en gång i tiden. Nedan listas de frekvenser som
används i Sverige.

\subsubsection{Svenska MF-kanaler}

\begin{longtable}{llrr}
\textbf{Kanal} & \textbf{Placering} & \textbf{Skepp} & \textbf{Kust}  \\ \hline
\endhead

MF1 & Gotland       & 2 099 & 1 6874 \\
MF2 & ---           & ---  & ---   \\
MF3 & Gislövshammar & 2 060 & 1 797  \\
MF4 & Härnösand     & 2 216 & 2 733  \\
MF5 & Bjuröklubb    & 2 123 & 1 779  \\
MF6 & Grimeton      & 2 135 & 1 710
\end{longtable}

\subsubsection{Nödfrekvenser}

\begin{longtable}{lrr}
\textbf{Band} & \textbf{Frekvens} & \textbf{DSC Frekvens}\\ \hline \endhead

MF   & 2 182  & 2 187.5  \\
HF4  & 4 125  & 4 207.5  \\
HF6  & 6 215  & 6 312.0  \\
HF8  & 8 291  & 8 414.5  \\
HF12 & 12 290 & 12 577.0 \\
HF16 & 16 429 & 16 804.5 \\
\end{longtable}

\subsubsection{Primära HF skepp-till-skepp}

\begin{longtable}{lrrrrrrrr}
\textbf{Kanal} & \textbf{HF4} & \textbf{HF6} & \textbf{HF8} &
               \textbf{HF12} & \textbf{HF16} & \textbf{HF18} &
               \textbf{HF22} & \textbf{HF25} \\
\hline
\endhead

A & 4 146 & 6 224 & 8 294 & 12 353 & 16 528 & 18 825 & 22 159 & 25 100 \\
B & 4 149 & 6 227 & 8 297 & 12 356 & 16 531 & 18 828 & 22 162 & 25 103 \\
C &       & 6 230 &       & 12 359 & 16 534 & 18 831 & 22 165 & 25 106 \\
D &       &       &       & 12 362 & 16 537 & 18 834 & 22 168 & 25 109 \\
E &       &       &       & 12 365 & 16 540 & 18 837 & 22 171 & 25 112 \\
F &       &       &       &        & 16 543 & 18 840 & 22 174 & 25 115 \\
G &       &       &       &        & 16 546 & 18 843 & 22 177 & 25 118 \\
\end{longtable}

\clearpage

\subsection{Fyrar}

\subsubsection{IBP -- International Beacon Project}

Det finns flera olika typer av fyrar men för HF är IBP (International
Beacon Project) intressant eftersom det ger operatören möjlighet att
utröna hur utbredningen ser ut för stunden genom att lyssna efter
fyrar. Fyrarna har gemensam hårdvara och synkroniseras mot
tidsreferens. Fyrar kan vara offline av olika skäl, kontrollera mot
IBP:s hemsida om du inte hör en fyr du brukar höra.

Tabellen nedan visar anropssignaler och första sändningsslotten som
fyren sänder, dvs SCHED för olika fyrar och frekvenser.

\subsubsection{Lista över IBP-fyrar}
\begin{table}[H]
\centering
\begin{tabular}{llrrrrr}
\textbf{Signal} & \textbf{QTH} & \textbf{14 100} & \textbf{18 110} &
                \textbf{21 150} & \textbf{24 930} & \textbf{28 200} \\ \hline

4U1UN  & United Nations & 00:00  & 00:10  & 00:20  & 00:30  & 00:40  \\
VE8AT  & Canada         & 00:10  & 00:20  & 00:30  & 00:40  & 00:50  \\
W6WX   & United States  & 00:20  & 00:30  & 00:40  & 00:50  & 01:00  \\
KH6RS  & Hawaii         & 00:30  & 00:40  & 00:50  & 01:00  & 01:10  \\
ZL6B   & New Zealand    & 00:40  & 00:50  & 01:00  & 01:10  & 01:20  \\
VK6RBP & Australia      & 00:50  & 01:00  & 01:10  & 01:20  & 01:30  \\
JA2IGY & Japan          & 01:00  & 01:10  & 01:20  & 01:30  & 01:40  \\
RR9O   & Russia         & 01:10  & 01:20  & 01:30  & 01:40  & 01:50  \\
VR2B   & Hong Kong      & 01:20  & 01:30  & 01:40  & 01:50  & 02:00  \\
4S7B   & Sri Lanka      & 01:30  & 01:40  & 01:50  & 02:00  & 02:10  \\
ZS6DN  & South Africa   & 01:40  & 01:50  & 02:00  & 02:10  & 02:20  \\
5Z4B   & Kenya          & 01:50  & 02:00  & 02:10  & 02:20  & 02:30  \\
4X6TU  & Israel         & 02:00  & 02:10  & 02:20  & 02:30  & 02:40  \\
OH2B   & Finland        & 02:10  & 02:20  & 02:30  & 02:40  & 02:50  \\
CS3B   & Madeira        & 02:20  & 02:30  & 02:40  & 02:50  & 00:00  \\
LU4AA  & Argentina      & 02:30  & 02:40  & 02:50  & 00:00  & 00:10  \\
OA4B   & Peru           & 02:40  & 02:50  & 00:00  & 00:10  & 00:20  \\
YV5B   & Venezuela      & 02:50  & 00:00  & 00:10  & 00:20  & 00:30  \\
\end{tabular}
\caption{IBP-fyrar}
\end{table}



\normalsize

\input{tex/bandplan-hf}
