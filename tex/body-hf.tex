\clearpage

\section{Trafik- och tumregler HF}
\subsection{Kort sammanfattning av reglemente}

OBS! Detta är inte fullständigt radioreglemente naturligtvis utan
endast sammanfattning av några viktiga punkter.

\subsubsection{Begrepp i bandplanerna}

\begin{itemize}
\item QRP: Aktivitetscentrum för låg effekt ($<$5W), svaga signaler förekommer, visa hänsyn.
\item QRS: Aktivitetscenter för långsam CW.
\item QRSS: Extremt långsam CW med dator.
\item DV: Digital Voice.
\item Image: Bildmoder exempelvis SSTV och Fax som ryms inom den specificerade maximala bandbredden.
\end{itemize}

\subsubsection{Trafikregler och tumregler}

\begin{itemize}
\item Vid SSB-telefoni används LSB på frekvenser under 10 MHz och USB
      på frekvenser över 10 MHz.
\item Lägsta acceptabla inställda frekvens för LSB är 3 kHz över
      under bandkant!
\item Högsta acceptable inställda frekvens för USB är 3 kHz under
      övre bandkant!
\item IBP är International Beacon Project. Fyrarna sänder med 3 min
      intervaller och används för att studera utbredningen av
      radiosignaler globalt. Fyrarna sänder anrop och fyra 1 s toner.
      Anropet och första tonen sänds med 100W, därefter sänds tonerna
      med 10W, 1W samt 100mW.
\item Vid AM (A3J) skall hänsyn tas så att störningar på annan trafik ej fö\-re\-kom\-mer
      med de sidband som då uppstår, det gäller då både övre och undre
      sidbandet.
\item Ingen som helst sändning är tillåtet inom fyrsegmenten. Detta skall respekteras.
      Lyssna gärna på nödfrekvenserna men används dem icke, om det
      inte är du som svarar på ett nödsamtal! Undvik QSO allt för nära
      dessa också.
\item Var särskilt uppmärksam på satelliters nerlänksfrekvenser på 10\,m-bandet.
      I detta segment skall endast lyssning ske. Ingen sändning är
      tillåten här eller i skyddssegmentet strax ovanför
      satellitsegmentet. Tänk på att satelliters frekvens kan
      dopplerskiftas uppåt en hel del när de rör sig mot mottagaren.
\end{itemize}

\section{Frekvenser HF}

\subsection{PR-bandet 27\,MHz}

Detta är det enda bandet som allmänheten kan använda på HF-bandet. Det delar många egenskaper med 31\,MHz jaktradiobandet men är ett band som är äldre och mer etablerat.

Maximal uteffekt på bandet är 4W RMS ERP dvs antennvinst överstigande en 1/2-vågs dipol (0 dBd, 2.12 dBi) måste inräknas i effekten efter avdrag för matningsförlust. Modulationsslag AM, FM och SSB (primärt används USB) är tillåtet på alla kanaler i dag. Traditionellt används kanal 24 för USB men i dag får vilken kanal som helst användas.

Kanalerna med A efter är upplåtna för radiostyrning och inte för telefoni. Undvik därför att använda dessa om du har en sändare som kan använda dessa frekvenser. De är med i tabellen för den skall vara komplett. 

\clearpage

\begin{longtable}{rrl|rrl}
	\textbf{Frekvens}& \textbf{Kanalnr}& \textbf{Övrigt}        
	& \textbf{Frekvens} & \textbf{Kanalnr} & \textbf{Övrigt}  \\
	\hline \endhead
	  26,965 &       1 &                &   27,215 &      21 &          \\
	  26,975 &       2 &                &   27,225 &      22 &          \\
	  26,985 &       3 &                &   27,255 &      23 &          \\
	  26,995 &      3A & Radiostyrning  &   27,235 &      24 & SSB      \\
	  27,005 &       4 &                &   27,245 &      25 &          \\
	  27,015 &       5 &                &   27,265 &      26 &          \\
	  27,025 &       6 &                &   27,275 &      27 &          \\
	  27,035 &       7 &                &   27,285 &      28 &          \\
	  27,045 &      7A & Radiostyrning  &   27,295 &      29 &          \\
	  27,055 &       8 &                &   27,305 &      30 &          \\
	  27,065 &       9 &                &   27,315 &      31 &          \\
	  27,075 &      10 &                &   27,325 &      32 &          \\
	  27,085 &      11 &                &   27,335 &      33 &          \\
	  27,095 &     11A & Tid. nödfrekv. &   27,345 &      34 &          \\
	  27,105 &      12 &                &   27,355 &      35 &          \\
	  27,115 &      13 &                &   27,365 &      36 &          \\
	  27,125 &      14 &                &   27,375 &      37 &          \\
	  27,135 &      15 &                &   27,385 &      38 &          \\
	  27,155 &      16 &                &   27,395 &      39 &          \\
	  27,165 &      17 &                &   27,405 &      40 &          \\
	  27,175 &      18 &                &          &         &          \\
	  27,185 &      19 &                &          &         &          \\
	  27,195 &     19A & Radiostyrning  &          &         &          \\
	  27,205 &      20 &                &          &         &        
\end{longtable}

Många apparater är endast FM i dag men det finns de som också har SSB. Äldre apparater hade oftast AM och FM och ibland även SSB. Telegrafi körs i princip inte på PR-bandet, troligen för att det aldrig varit några krav på det och de som kör heller inte haft möjlighet förr i tiden att DX-a på bandet. 

Innan Televerket släppte upp bestämmelserna var det väldigt hårda bestämmelser på bandet, i princip var det bara kommunikation inom familjen som tilläts. I dag kan bandet användas som man vill och det är på sina håll god aktivitet. 

Kom ihåg att inte överskrida effektbegränsningarna bara.

\subsection{Amatörradiofrekvenser HF}

\todo{Beskriva bandegenskaperna}

\subsubsection{Bandens olika egenskaper}

Även om man kan säga att alla kortvågsbanden har liknande egenskaper så är
skillnaderna bland dem ofta påtagliga och eftersom det spänner över ett relativt
stor område då 160 till 6 meter brukar räknas som HF även om det strikt räknat
är VHF liksom 160 metersbandet som egentligen faller inom mellanvågsdelen eller
den så kallade gränsvågen så är de band som oftast förekommer på radiotapparater
för kortvåg så därför är de med här.

\subsubsection{160 \& 80 metersbanden}

\todo{- 160 och 80 meter}

\subsubsection{40 metersbandet}

\todo{- 40 meter}

\subsubsection{20 \& 17 metersbanden}

\todo{- 20 och 17 meter}

\subsubsection{15 \& 12 metersbanden}

\todo{- 15 och 12 meter}

\subsubsection{6 metersbandet}

\todo{- 6 meter}

\subsection{JOTA---Jamboree on the air, scoutfrekvenser}

Scouterna har frekvenser på HF likväl som VHF/UHF som de aktiverar vid särskilda
tillfällen ofta i tillsammans med en lokal amatörradioklubb eller vanliga
amtörradioeldsjälar som inte sällan också är scouter. Här kommer en lista på
frekvenser som är vanligt förekommande i scoutsammanhang.

\begin{table}[H]
\centering
\begin{tabular}{rrll}
	\textbf{Band} & \textbf{Frekvens} & \textbf{Trafik} & \textbf{Not} \\ \hline

               80 & 3 570  & CW  &             \\
	              & 3 940  & SSB & Ej region 2 \\
	              & 3 690  & SSB &             \\ \hline
	           40 & 7 030  & CW  &             \\
	              & 7 190  & SSB &             \\
	              & 7 090  & SSB &             \\ \hline
	           20 & 14 060 & CW  &             \\
	              & 14 290 & SSB &             \\ \hline
	           17 & 18 080 & CW  &             \\
	              & 18 140 & SSB &             \\ \hline
	           15 & 21 140 & CW  &             \\
	              & 21 360 & SSB &             \\ \hline
	           12 & 24 910 & CW  &             \\
	              & 24 960 & SSB &             \\ \hline
	           10 & 28 180 & CW  &             \\
	              & 28 390 & SSB &             \\ \hline
	            6 & 10 160 & CW  &             \\
	              & 50 160 & SSB &             \\ \hline
\end{tabular}
\caption{Scouters JOTA-frekvenser på HF}
\end{table}

Normalt aktiveras dessa frekvenser tredje veckoslutet i oktober varje
år, fredag till söndag. Då kan det vara många klubbar som finns på
frekvenserna och det är också vanligt att man hör dem på helt andra
frekvenser. De frekvenser som listas här är inte på något vis de enda
frekvenser som scouter använder.

\subsection{Marina MF/HF-frekvenser}

De marina HF-banden är uppdelade på ett antal band. Det finns en
generell kanalindelning med 3 kHz per kanal och SSB som
modulationssätt på respektive band. Marina HF-kanaler finns på banden
4, 6, 8, 12, 16, 18, 22 och 25 MHz.

MF även benämnd gränsvåg i marina sammanahang är inte lika ofta
används som den var en gång i tiden. Nedan listas de frekvenser som
används i Sverige.

\subsubsection{Svenska MF-kanaler}

\begin{longtable}{llrr}
\textbf{Kanal} & \textbf{Placering} & \textbf{Skepp} & \textbf{Kust}  \\ \hline
\endhead

MF1 & Gotland       & 2 099 & 1 6874 \\
MF2 & ---           & ---  & ---   \\
MF3 & Gislövshammar & 2 060 & 1 797  \\
MF4 & Härnösand     & 2 216 & 2 733  \\
MF5 & Bjuröklubb    & 2 123 & 1 779  \\
MF6 & Grimeton      & 2 135 & 1 710
\end{longtable}

\subsubsection{Nödfrekvenser}

\begin{longtable}{lrr}
\textbf{Band} & \textbf{Frekvens} & \textbf{DSC Frekvens}\\ \hline \endhead

MF   & 2 182  & 2 187.5  \\
HF4  & 4 125  & 4 207.5  \\
HF6  & 6 215  & 6 312.0  \\
HF8  & 8 291  & 8 414.5  \\
HF12 & 12 290 & 12 577.0 \\
HF16 & 16 429 & 16 804.5 \\
\end{longtable}

\subsubsection{Primära HF skepp-till-skepp}

\begin{longtable}{lrrrrrrrr}
\textbf{Kanal} & \textbf{HF4} & \textbf{HF6} & \textbf{HF8} &
               \textbf{HF12} & \textbf{HF16} & \textbf{HF18} &
               \textbf{HF22} & \textbf{HF25} \\
\hline
\endhead

A & 4 146 & 6 224 & 8 294 & 12 353 & 16 528 & 18 825 & 22 159 & 25 100 \\
B & 4 149 & 6 227 & 8 297 & 12 356 & 16 531 & 18 828 & 22 162 & 25 103 \\
C &       & 6 230 &       & 12 359 & 16 534 & 18 831 & 22 165 & 25 106 \\
D &       &       &       & 12 362 & 16 537 & 18 834 & 22 168 & 25 109 \\
E &       &       &       & 12 365 & 16 540 & 18 837 & 22 171 & 25 112 \\
F &       &       &       &        & 16 543 & 18 840 & 22 174 & 25 115 \\
G &       &       &       &        & 16 546 & 18 843 & 22 177 & 25 118 \\
\end{longtable}

\clearpage

\subsection{Fyrar}

\subsubsection{IBP -- International Beacon Project}

Det finns flera olika typer av fyrar men för HF är IBP (International
Beacon Project) intressant eftersom det ger operatören möjlighet att
utröna hur utbredningen ser ut för stunden genom att lyssna efter
fyrar. Fyrarna har gemensam hårdvara och synkroniseras mot
tidsreferens. Fyrar kan vara offline av olika skäl, kontrollera mot
IBP:s hemsida om du inte hör en fyr du brukar höra.

Tabellen nedan visar anropssignaler och första sändningsslotten som
fyren sänder, dvs SCHED för olika fyrar och frekvenser.

\subsubsection{Lista över IBP-fyrar}
\begin{table}[H]
\centering
\begin{tabular}{llrrrrr}
\textbf{Signal} & \textbf{QTH} & \textbf{14 100} & \textbf{18 110} &
                \textbf{21 150} & \textbf{24 930} & \textbf{28 200} \\ \hline

4U1UN  & United Nations & 00:00  & 00:10  & 00:20  & 00:30  & 00:40  \\
VE8AT  & Canada         & 00:10  & 00:20  & 00:30  & 00:40  & 00:50  \\
W6WX   & United States  & 00:20  & 00:30  & 00:40  & 00:50  & 01:00  \\
KH6RS  & Hawaii         & 00:30  & 00:40  & 00:50  & 01:00  & 01:10  \\
ZL6B   & New Zealand    & 00:40  & 00:50  & 01:00  & 01:10  & 01:20  \\
VK6RBP & Australia      & 00:50  & 01:00  & 01:10  & 01:20  & 01:30  \\
JA2IGY & Japan          & 01:00  & 01:10  & 01:20  & 01:30  & 01:40  \\
RR9O   & Russia         & 01:10  & 01:20  & 01:30  & 01:40  & 01:50  \\
VR2B   & Hong Kong      & 01:20  & 01:30  & 01:40  & 01:50  & 02:00  \\
4S7B   & Sri Lanka      & 01:30  & 01:40  & 01:50  & 02:00  & 02:10  \\
ZS6DN  & South Africa   & 01:40  & 01:50  & 02:00  & 02:10  & 02:20  \\
5Z4B   & Kenya          & 01:50  & 02:00  & 02:10  & 02:20  & 02:30  \\
4X6TU  & Israel         & 02:00  & 02:10  & 02:20  & 02:30  & 02:40  \\
OH2B   & Finland        & 02:10  & 02:20  & 02:30  & 02:40  & 02:50  \\
CS3B   & Madeira        & 02:20  & 02:30  & 02:40  & 02:50  & 00:00  \\
LU4AA  & Argentina      & 02:30  & 02:40  & 02:50  & 00:00  & 00:10  \\
OA4B   & Peru           & 02:40  & 02:50  & 00:00  & 00:10  & 00:20  \\
YV5B   & Venezuela      & 02:50  & 00:00  & 00:10  & 00:20  & 00:30  \\
\end{tabular}
\caption{IBP-fyrar}
\end{table}



\normalsize

\begin{landscape}
\section{Frekvenser Amatörradio LF/MF/HF}
\subsection{Bandplaner LF/MF/HF}
Alla frekvenser i kHz, bandbredder i Hz.

\subsubsection{Bandplan 2.2\,km, 135,7--137,8\,kHz}
\begin{tabular}{rrrll}
\textbf{Frekvens} &  & \textbf{BW} & \textbf{Trafik} & \textbf{Noteringar} \\ \hline
135,7 & 135,8 & 200 & CQ, QRSS, Digi & OBS! Högsta effekt 1W ERP. \\ \hline
\end{tabular}

\subsubsection{Bandplan 600\,m, 472--479\,kHz}
\begin{tabular}{rrrll}
\multicolumn{2}{c}{\textbf{Frekvens}} & \textbf{BW} & \textbf{Trafik} & \textbf{Noteringar} \\ \hline
472 & 479 & 200 & CW, QRSS, Digi & OBS! Högsta utstrålad effekt 1W EIRP \\ \hline
\end{tabular}

\subsubsection{Bandplan 160\,m, 1810--2000\,kHz}
\begin{tabular}{rrrll}
\multicolumn{2}{c}{\textbf{Frekvens}} & \textbf{BW} & \textbf{Trafik} & \textbf{Noteringar} \\ \hline
1 810 & 1 838 & 200  & CW         & Exklusivt för CW. Interkontinental trafik har prio. \\ \hline
1 838 & 1 840 & 500  & Smalband   & Ej packet på 160m, PSK 1 838,150                    \\ \hline
1 840 & 1 850 & 2700 & Alla moder & Även digimode. SSB QRP 1 843 kHz                    \\ \hline
1 850 & 1 900 & 2700 & Alla moder & OBS! Max 10 W till ant.                             \\ \hline
1 900 & 1 950 & 2700 & Alla moder & OBS! Max 100 W till ant.                            \\ \hline
1 950 & 2 000 & 2700 & Alla moder & OBS! Max 10 W till ant.                             \\ \hline
\end{tabular}

\subsubsection{Bandplan 80\,m, 3500--3800\,kHz}
\begin{tabular}{rrrll}
\multicolumn{2}{c}{\textbf{Frekvens}} & \textbf{BW} & \textbf{Trafik} & \textbf{Noteringar} \\ \hline
3 500 & 3 510 & 200  & CW             & Exklusivt CW                         \\ 
      &       &      &                & Interkontinental DX-trafik har prio  \\ \hline
3 510 & 3 580 & 200  & CW             & Exklusivt CW contest 3510-–560       \\ 
      &       &      &                & CW QRS 3 555 kHz, CW QRP 3 560       \\ \hline
3 580 & 3 600 & 500  & Smalband, Digi & PSK 3 580,150                        \\
      &       &      &                & Automatiska Digimoder 3 590--600     \\ \hline
3 600 & 3 620 & 2700 & Alla moder     & Digimoder Automatiska Digimoder      \\ \hline
3 600 & 3 650 & 2700 & Alla moder     & SSB contest 3 600--650               \\
      &       &      &                & DV 3 630                             \\ \hline
3 650 & 3 700 & 2700 & Alla moder     & SSB QRP 3 690                        \\ \hline
3 700 & 3 800 & 2700 & Alla moder     & Contest 3 700-–800                   \\
      &       &      &                & Image 3 775                          \\
      &       &      &                & Region 1 nödfrekvens 3 760           \\ \hline
3 775 & 3 800 & 2700 & Alla moder     & Interkontinental DX-trafik prioritet \\ \hline
\end{tabular}

\subsubsection{Bandplan 40\,m, 7000--7200\,kHz}
\begin{tabular}{rrrll}
\multicolumn{2}{c}{\textbf{Frekvens}} & \textbf{BW} & \textbf{Trafik} & \textbf{Noteringar} \\ \hline
7\,000 & 7\,040 & 200  & CW         & Exklusivt CW.                             \\
      &       &      &            & QRP aktivitetscentrum 7\,030\,kHz           \\ \hline
7\,040 & 7\,050 & 500  & Smalband   & Digimoder Automatiska inom 7\,047–-050\,kHz \\ \hline
7\,050 & 7\,060 & 2700 & Alla moder & Digimoder Automatiska inom 7\,050–-053\,kHz \\ \hline
7\,060 & 7\,100 & 2700 & Alla moder & SSB contest i segmentet                   \\
      &       &      &            & DV 7 070 kHz, SSB QRP 7\,090 kHz           \\ \hline
7\,100 & 7\,130 & 2700 & Alla moder & Region 1 nödfrekvens 7\,110 kHz            \\ \hline
7\,130 & 7\,200 & 2700 & Alla moder & SSB contest i segmentet                   \\
      &       &      &            & Image 7\,165\,kHz                           \\ \hline
7\,175 & 7\,200 & 2700 & Alla moder & Interkontinental DX-trafik prio           \\ \hline
\end{tabular}

\subsubsection{Bandplan 30 m, 10100--10150 kHz}
\begin{tabular}{rrrll}
\multicolumn{2}{c}{\textbf{Frekvens}} & \textbf{BW} & \textbf{Trafik} & \textbf{Noteringar} \\ \hline
10\,100 & 10\,140 & 200 & CW       & CW exkl. Max 150 Watt på 30 m    \\
       &        &     &          & CW QRP 10\,116\,kHz                     \\ \hline
10\,140 & 10\,150 & 500 & Smalband & Digimoder PSK 10142,150\,kHz. Ej Packet \\ \hline
\end{tabular}

\subsubsection{Bandplan 20 m, 14000--14350 kHz}
\begin{tabular}{rrrll}
\multicolumn{2}{c}{\textbf{Frekvens}} & \textbf{BW} & \textbf{Trafik} & \textbf{Noteringar} \\ \hline
14\,000 & 14\,070 & 200  & CW         & Exklusivt CW                            \\
       &        &      &            & Conctest 14\,000-–060                     \\
       &        &      &            & CW QRS 14 055, CW QRP 14\,060            \\ \hline
14\,070 & 14\,099 & 500  & Smalband   & PSK 14 070,150                          \\
       &        &      &            & Auto Digimoder 14 089-–099              \\ \hline
14\,099 & 14\,101 & 200  & Fyrar      & Exklusivt IBP, endast fyrar             \\ \hline
14\,101 & 14 \,12 & 2700 & Alla moder & Digitala moder och obevakade Digimoder  \\ \hline
14\,112 & 14\,350 & 2700 & Alla moder & SSB Contest 14 125--300                 \\
       &        &      &            & DV 14 130, DXpedition prio 14\,195$\pm$5 \\ \hline
14\,300 & 14\,350 & 2700 & Alla moder & Image 14\,230, SSB QRP 14\,285            \\
       &        &      &            & Global nödfrekvens 14 300               \\ \hline
\end{tabular}

\subsubsection{Bandplan 17 m, 18068--18168 kHz}
\begin{tabular}{rrrll}
\multicolumn{2}{c}{\textbf{Frekvens}} & \textbf{BW} & \textbf{Trafik} & \textbf{Noteringar} \\ \hline
18 068 & 18 095 & 200  & CW         & CW exklusivt. QRP 18 086             \\ \hline
18 095 & 18 109 & 500  & Smalband   & Digimoder PSK 18 100,150             \\
       &        &      &            & Automatiska Digimoder 18 105-–18 109 \\ \hline
18 109 & 18 111 & 200  & Fyrar      & Exklusivt fyrar, IBP fyrnät          \\ \hline
18 111 & 18 168 & 2700 & Alla moder & Digi 18 111–-18 120                  \\
       &        &      &            & SSB QRP 18 130, DV 18 150            \\
       &        &      &            & Global nödfrekv. 18 160\\ \hline
\end{tabular}

\subsubsection{Bandplan 15 m, 21000--21450 kHz}
\begin{tabular}{rrrll}
\multicolumn{2}{c}{\textbf{Frekvens}} & \textbf{BW} & \textbf{Trafik} & \textbf{Noteringar} \\ \hline
21 000 & 21 070 & 200  & CW         & Exklusivt CW, QRS 21 055, CW QRP 21 060          \\ \hline
21 070 & 21 110 & 500  & Smalband   & PSK 21080.150, Automatiska Digimoder 21 090–-110 \\
21 110 & 21 120 & 2700 & Alla moder & Alla moder utom SSB!                             \\
       &        &      &            & Digimoder, och Automatiska Digimoder             \\ \hline
21 120 & 21 149 & 500  & Smalband   &                                                  \\ \hline
21 149 & 21 151 & 200  & Fyrar      & Exklusivt fyrar. IBP fyrnät                      \\ \hline
21 151 & 21 450 & 2700 & Alla moder & DV 21 180, SSB QRP 21 285, Image 21 340          \\
       &        &      &            & Global nödfrekv. 21 360                          \\ \hline
\end{tabular}

\subsubsection{Bandplan 12 m, 24890--24990 kHz}
\begin{tabular}{rrrll}
\multicolumn{2}{c}{\textbf{Frekvens}} & \textbf{BW} & \textbf{Trafik} & \textbf{Noteringar} \\ \hline
24 890 & 24 915 & 200  & CW         & Exklusivt CW, QRP 24 906                             \\ \hline
24 915 & 24 929 & 500  & Smalband   & PSK 24 920.150, Automatiska Digimoder 24 925–-24 929 \\ \hline
24 929 & 24 931 & 200  & Fyrar      & Fyrar, IBP fyrnät                                    \\ \hline
24 931 & 24 990 & 2700 & Alla moder & Auto Digimoder 24 931-–24 940                        \\
       &        &      &            & SSB QRP 24 950, DV 24 960                            \\ \hline
\end{tabular}

\subsubsection{Bandplan 10 m, 28000-29700 kHz}
\begin{tabular}{rrrll}
\multicolumn{2}{c}{\textbf{Frekvens}} & \textbf{BW} & \textbf{Trafik} & \textbf{Noteringar} \\ \hline
28 000 & 28 070 & 200  & CW         & Exklusivt CW, QRS 28 055, CW QRP 28 060                \\ \hline
28 070 & 28 190 & 500  & Smalband   & PSK 28 120.150, Auto Digimoder inom 28 120--150        \\ \hline
28 190 & 28 199 & 200  & Fyrar IBP  & Regionala fyrar med tidsdelning                        \\ \hline
28 199 & 28 201 & 200  & Fyrar IBP  & IBP fyrnät                                             \\ \hline
28 201 & 28 225 & 200  & Fyrar IBP  & kontinuerligt sändande fyrar                           \\ \hline
28 225 & 28 300 & 2700 & Alla moder & Övriga fyrar                                           \\ \hline
28 300 & 28 320 & 2700 & Alla moder & Digimoder och Automatiska Digimoder                    \\ \hline
28 320 & 29 100 & 2700 & Alla moder & DV 28 330 kHz, SSB QRP 28 360 kHz                      \\
       &        &      &            & Image 28 680 kHz                                       \\ \hline
29 100 & 29 200 & 6000 & Alla moder & FM simplex, 10 kHz kanaler                             \\
       &        &      &            & Maximalt ±2.5 kHz dev., max 2.5 kHz mod.frek.          \\ \hline
29 200 & 29 300 & 6000 & Alla moder & Digimoder och Automatiska Digimoder                    \\ \hline
29 300 & 29 510 & 6000 & Satellit   & Nerlänk fr. satellit. EJ SÄNDNING I SEGMENTET          \\ \hline
29 510 & 29 520 & 6000 & Skydd      & Skyddsfrekvens för satelliter. EJ SÄNDNING I SEGMENTET \\ \hline
29 520 & 29 590 & 6000 & Alla moder & FM Repeater in RH1--8, 100 kHz duplex, 2.5 kHz NBFM    \\ \hline
29 600 & 29 620 & 6000 & Alla moder & FM simplex, anrop 29 600                               \\
       &        &      &            & FM simplex repeater 29 610                             \\ \hline
29 620 & 29 700 & 6000 & Alla moder & FM Repeater ut RH1--8, 100 kHz duplex                  \\ \hline
\end{tabular}
\end{landscape}

\clearpage

