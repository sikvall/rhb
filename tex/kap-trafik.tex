\chapter{Trafik}

I detta kapitel tittar vi närmare på hur radiotrafik bedrivs både för amatörradioband och andra band. Här finns också bokstaveringsalfabeten, både det svenska och det internationella, Q-koder och mycket annat som är radio\-trafik\-relaterat.

\section{Uppträdande på banden}

När vi kör amatörradio, PR-radio eller annan radio finns det ett antal saker
att tänka på som har att göra med hur vi beter oss mot varandra på banden. Se
detta som en guide till hur man bör uppträda på banden.

En radioamatör eller annan radioanvändare måste vara \textbf{tolerant}. Vi
delar frekvenser med många andra personer, en del av dem kommer inte ha samma
uppfattning som du själv har om saker och ting. Här gäller det att vara
tolerant, förstående och framför allt inte bli upprörd över personer som
kanske inte beter sig som du önskade att de betedde sig.

Radiooperatörer är normalt \emph{aldrig ensamma på banden} helt oavsett om
någon svarar på ditt allmänna anrop eller ej så finns det i det närmaste
\textbf{garanterat någon som lyssnar}.

Tänk på vad du säger och att du undviker diskutera ämnen som kan verka
\textbf{upprörande} eller \textbf{stötande}. Ämnen som bör undvikas är
\textbf{religion} och livs\-å\-skå\-d\-ni\-ng, \textbf{politisk} ideologi,
\textbf{ekonomiska} eller \textbf{sociala} frågor m.m. där motparter kan ha
starka åsikter som inte nödvändigtvis stämmer med dina egna. Radion är inte ett
agitationsrum för sådana frågor.

\textbf{Svordomar}, \textbf{könsord} och liknande undviker vi helt. Språket
skall vara vårdat men behöver inte vara strikt. Tänk på att din motpart är inte
den enda som lyssnar utan det finns \textit{andra amatörer som lyssnar},
icke-amatörer som lyssnar, myndigheter som lyssnar och så vidare.

Ha \textbf{förståelse} för att andra kanske inte har dina egna detaljkunskaper,
professionalism med mera. Agera \textbf{ödmjukt} gentemot andra människor på
banden.

Blir du ändå upprörd, undvik att \emph{agera på det} över huvud taget. Sänd
inte över annans sändning, s.k. ''gummitumme'', eller stör på annat vis för du
är upprörd. Avsluta hellre QSO:t, byt frekvens eller återkom lite senare när
du lugnat ned dig. Tänk på att \textit{de flesta konflikter orsakas av
okunskap eller brist på förståelse}. \textbf{Agera vuxet} i sådana situationer
och jobba för att \textbf{de-eskalera} situationen.

En skicklig amatör \textit{lyssnar mycket innan sändning}. Vi anropar på ett
korrekt sätt och avslutar på ett korrekt sätt. Vi försöker uppge våra respektive
signaler på ett \emph{tydligt och läsligt sätt}, i dag finns det en tendens att
sluddra över signalerna framför allt på 2m och 70cm banden, gör inte
det. Tydlighet är en vinning i sig.

När någon ny i ringen inträder, räkna upp de deltagande signalerna så att
personen tydligt får en bild av alla som är med och vem som är på turen före och
efter hen.

Vi pratar inte \textbf{nedvärderande} om personer varesig de är andra amatörer
eller ej, eller en viss grupp av personer. Vi undviker \textbf{sexuella
  anspelningar} och vitsar ''\textbf{under bältet}'' liksom allt för
\textbf{personliga detaljer}. Amatörradion är främst
för \textbf{tekniska diskussioner} av rent \textbf{privat natur} eller
av \textbf{allmänt intresse för hobbyn}, tester och prov med mera.

Undvik väldigt \textbf{långa sändningspass}. Ibland händer det saker hos dina
motstationer som att de får ett viktigt telefonsamtal eller måste springa ut i
köket för katten har rivit ner något, ett barn ramlar eller annat som gör att
man måste kvickt lämna radion. Att \textbf{långprata} i sådana lägen gör det
svårt att tala om ''QRX --- jag måste ta hand om en sak, anropar dig igen om 5
min.''. Enstaka gånger kanske man behöver förklara något lite längre men gör det
till en vana att lämna luckor så ofta som möjligt.

\textbf{Nödtrafik har alltid prioritet} och måste respekteras på alla
frekvenser.

\section{Signaler och anrop}

\subsection{Landsprefix}

Här är inte alla länder med utan de vanligaste som körs från Sverige.

\begin{center}
  \begin{footnotesize}
    \begin{longtable}{lll}
      \caption{Utvalda landsprefix} \\
      \textbf{Land}                 & \textbf{DXCC}  & \textbf{Prefixserier}                             \\ \hline
			Belgien                       & ON             & ONA--OTZ                                          \\
			Canada                        & VE             & CYA--CZZ, VAA--VGZ, VOA--VOZ, VXA--VYZ, XJA--XOZ  \\
			Frankrike                     & F              & FAA--FZZ, HWA--HWZ, THA--THZ, TKA--TKZ, TMA--TMZ, \\
			                              &                & TOA--TQZ, TVA--TXZ                                \\
			Frankr. särsk.                & FG FH FK       &                                                   \\
			                              & FM, FO, FP, FR &                                                   \\
			                              & FS, FT, FW, FY &                                                   \\
			Förenta Staterna              & K              & AAA--ALZ, KAA--KZZ, NAA--NZZ, WAA--WZZ            \\
			Grekland                      & SV             & J4A--J4Z, SVA--SVZ                                \\
			Italien                       & I              & IAA--IZZ                                          \\
			Japan                         & JA             & 7JA--7NZ, 8JA--8NZ, JAA--JSZ                      \\
			Kroatien                      & 9A             & 9AA--9AZ                                          \\
			Nederländerna                 & PA             & PAA--PLZ                                          \\
			Polen                         & SP             & 3ZA--3ZZ, HFA--HFZ, SNA--SRZ                      \\
			Rumänien                      & YO             & YOA--YRZ                                          \\
			Ryssland (Eur.)               & UA1 3 4 5 6 7  & RAA--RZZ, UAA-UIZ                                 \\
			Ryssland (Asi.)               & UA8 9 0        & RAA--RZZ, UAA-UIZ                                 \\
			Schweiz                       & HB             & HBA--HBZ HEA--HEZ                                 \\
			Spanien                       & EA             & AMA--AOZ, EAA--EHZ                                \\
			Storbritt. England            & G, 2E, M       & 2AA--2ZZ, GAA--GZZ, MAA--MZZ, VPA--VQZ,           \\
			                              &                & VSA--VSZ,ZBA--ZJZ, ZNA--ZOZ, ZQA--ZQZ             \\
			Storbritt. Skottland          & GM, 2M, MM     &                                                   \\
			Storbritt. Övrigt             & VP2, VP6, VP8  &                                                   \\
			                              & VP9, VQ9, ZB   &                                                   \\
			Sverige                       & SM             & 7SA--7SZ, 8SA--8SZ, SAA--SMZ                      \\
			Tyskland                      & DL             & DAA--DRZ, Y2A--Y9Z                                \\
			Ukraina                       & UT             & EMA--EOZ, URA--UZA                                \\
			Ungern                        & HA             & HAA--HAZ, HGA--HGZ                                \\
			Österrike                     & OE             & OEA--OEZ\\
		\end{longtable}
	\end{footnotesize}
\end{center}

\subsection{Svenska signaler}

Svenska signaler förekommer inom ett antal prefix. Enligt ITU disponerar Sverige
förljande signalserier: 7SA--7SZ samt 8SA--8SZ och vidare de mer kända SAA--SMZ.
Dessa har används till varierande ändamål, exempelvis har flyget signaler i
serien SE-AAA--ZZZ. Polisen har tidigare använt signaler i serien SHA plus fyra
siffror, detta är nu ersatt med nytt system i.o.m. RAKEL. Räddningstjänsten
använde SDA med fyra siffror. Signaler som 7SA + 4 siffror används för mindre
yrkesbåtar SC+4 siffror för fritidsbåtar.

Amatörradion använder ett antal signaler, de viktigaste är:

\begin{tabular}{ll}
	SM & Amatörradiosignal utdelad av PTS (nya signaler tilldelas ej i
        serien) \\ SA & Amatörradiosignal tilldelad av SSA \\ SK & Klubbsignaler
        (som regel tvåställiga efter distriktsiffran) \\ & numera tilldelas även
        klubbar SA-signaler som är tvåställiga efter distriktssiffran \\ SL &
        Militära signaler (som regel två- eller treställiga efter
        distriktsiffran)
\end{tabular}

Dessa signaler följs av en \textit{distriktsiffra} se särskilt avsnitt och sedan
2-ställiga eller 3-ställiga bokstavskombinationer som är den personliga
signalen. Exempel är SM0UEI som är min egen signal, distriktsiffran är 0 dvs
hemmavarande i Stockholms län. Ett annat exempel kan vara SK5JV tidigare
Fagersta amatörradioklubb.

Repeatrar som tillhör klubbar får ofta signal efter klubben med tillägg /R för
repeater.

Det finns numera även ett stort antal signaler som är tillfälliga eller knutna
till särskilda event, exempelvis scoutverksamhet som ibland sänder amatörradio
och särskilda forskningsfartyg, flyg- och rymdfart mm.

Som suffix används följande:

\begin{tabular}{ll}
	/M  & Mobil (rörlig) sändaramatör, även portabel \\
	/MM & Mobil till sjöss (mobil maritime)          \\
	/AM & Mobil i luften (aeromobile)                \\
	/P  & Portabel (för stunden uppsatt station)     \\
	/R  & Repeaterstation
\end{tabular}

\subsection{Svenska distrikten}

Sverige delas in i följande distrikt efter sina län:

\begin{table}[h]
	\centering
\begin{tabular}{cl}
	\textbf{Distrikt} & \textbf{Län}                                     \\ \hline %\endhead
	      0        & Stockholm                                        \\
	      1        & Gotland                                          \\
	      2        & Västerbotten, Norrbotten                         \\
	      3        & Gävleborg, Jämtland, Västernorrland              \\
	      4        & Örebro, Värmland, Dalarna                        \\
	      5        & Östergötland, Södermanland, Västmanland, Uppsala \\
	      6        & Halland, Västra götaland                         \\
	      7        & Skåne, Blekinge, Kronoberg, Jönköping, Kalmar    \\
	      8        & Speciella stationer utanför landets gränser
\end{tabular}
\caption{Distriktssiffor i Sverige}
\end{table}

\subsubsection{Karta över svenska amatörradiodistrikt}

\begin{figure}
	\centering
	\includegraphics[width=12cm]{pic/sm-distrikt-stor}
	\label{fig:sm-distrikt}
	\caption{Svenska distrikt, karta med tillstånd från
          \href{https://SSA.SE}{ssa.se}}
\end{figure}

Distrikten förekommer som siffra i utdelade anropssignaler. Radioamatörer byter
inte distriktsiffra under resa i annat distrikt, i stället används suffix
(tillägg efter ordinarie signal) som t.ex. /M för mobil. Ofta uppger man "SM0UEI
mobilt i SM3-land" (SM0UEI/3/M) ibland (SM0UEI/3M) för att påvisa att man
befinner sig utanför ordinarie distrikt.

En radioamatör kan byta sin distriktsiffra om den sänder från ett annat distrikt
än sitt hemmavarande. Man kan också göra ett tillägg med /n där n är den siffra
för det distrikt man befinner sig i. En stockholmsamatör som befinner sig i
Gävleborgs län kan alltså antingen använda SM3UEI eller SM0UEI/3 även med
tillägget M för mobil och P för portabel om man så önskar.

Det unika för en radioamatörs signal är alltså prefixet + suffixet, som exempel
är identifieraren för SM0UEI prefixet SM och suffixet UEI eftersom
distriktsiffran kan ändra sig.

\section{Terminologi och trafik}

\subsection{Bokstaveringsalfabetet (Svenska)}

\begin{table}[H]
	\centering
\begin{longtable}{cl|cl|cl }
	A & Adam   & O & Olof    & 1 & Ett        \\
	B & Bertil & P & Petter  & 2 & Tvåa       \\
	C & Cesar  & Q & Qvintus & 3 & Trea       \\
	D & David  & R & Rudolf  & 4 & Fyra       \\
	E & Erik   & S & Sigurd  & 5 & Femma      \\
	F & Filip  & T & Tore    & 6 & Sexa       \\
	G & Gustav & U & Urban   & 7 & Sju        \\
	H & Helge  & V & Viktor  & 8 & Åtta       \\
	I & Ivar   & W & Wilhelm & 9 & Nia        \\
	J & Johan  & X & Xerxes  & 0 & Nolla      \\
	K & Kalle  & Y & Yngve   & . & Punkt      \\
	L & Ludvig & Z & Zäta    & , & Komma      \\
	M & Martin & Å & Åke     & - & Minus      \\
	N & Niklas & Ä & Ärlig   & + & Plus       \\
	  &        & Ö & Östen   &   & Mellanslag \\
\end{longtable}
\caption{Svenska bokstaveringsalfabetet}
\end{table}

\subsection{Bokstaveringsalfabetet (Internationella)}
\begin{table}[H]
\centering
\begin{tabular}{cl|cl|cl}
	A & Alfa     &  P   & Papa       & 0 & Zero    \\
	B & Bravo    &  Q   & Quebec     & 1 & One     \\
	C & Charlie  &  R   & Romeo      & 2 & Two     \\
	D & Delta    &  S   & Sierra     & 3 & Tree    \\
	E & Echo     &  T   & Tango      & 4 & Fower   \\
	F & Foxtrot  &  U   & Uniform    & 5 & Fife    \\
	G & Golf     &  V   & Victor     & 6 & Six     \\
	H & Hotel    &  W   & Whiskey    & 7 & Seven   \\
	I & India    &  X   & X-ray      & 8 & Ait     \\
	J & Juliet   &  Y   & Yankee     & 9 & Niner   \\
	K & Kilo     &  Z   & Zulu       & . & Stop    \\
	L & Lima     & Å/AA & Alfa-Alfa  & , & Decimal \\
	M & Mike     & Ä/AE & Alfa-Echo  & - & Minus   \\
	N & November & Ö/OE & Oscar-Echo & + & Plus    \\
	O & Oscar    &      &            &   & Space   \\
\end{tabular}
\caption{Internationella bokstaveringsalfabetet (ITU-alfabetet)}
\end{table}

\subsection{Q-koder}
I tabellen listas några av de vanligast förekommande Q-koderna på amatörradiobanden.
Det finns förstås många fler koder men detta anses som de vanligaste.

\begin{longtable}{ll}
	\textbf{Kod} & \textbf{Fråga / Svar}                                                         \\ \hline
	\endhead
	\caption{Q-koder}\\
\endlastfoot
	QRA & Vad heter er station?                                                \\
	    & Vår station heter ...                                                \\ \hline
	QRB & Hur långt bort från min station befinner ni er?                      \\
	    & Avståndet mellan oss är ungefär ...                                  \\ \hline
	QRG & Kan ni ange min exakta frekvens?                                     \\
	    & Er exakta frekvens är ... (MHz/kHz)                                  \\ \hline
	QRH & Varierar min frekvens/våglängd?                                      \\
	    & Er frekvens/våglängd varierar.                                       \\ \hline
	QRI & Hur är min sändningston (CW)?                                        \\
	    & Er sändningston är 1--God, 2--Varierande, 3--Dålig                   \\ \hline
	QRK & Vilken uppfattbarhet har mina signaler?                              \\
	    & Uppfattbarheten hos dina signaler är:                                \\
	    & 1--Dålig, 2--Bristfällig, 3--Ganska god, 4--God, 5--Utmärkt          \\ \hline
	QRL & Är ni upptagen?                                                      \\
	    & Jag är upptagen med ... (namn/signal) stör ej.                       \\ \hline
	QRM & Är ni störd av annan station?                                        \\
	    & Störningarna är:                                                     \\
	    & 1--Obef., 2--Svaga, 3--Måttliga, 4--Starka, 5--Mycket starka         \\ \hline
	QRN & Besväras ni av atmosfäriska störningar?                              \\
	    & Störningarna är:                                                     \\
	    & 1--Obef., 2--Svaga, 3--Måttliga, 4--Starka, 5--Mycket starka         \\ \hline
	QRO & Kan jag (ska jag) öka sändareffekten?                                \\
	    & Öka sändareffekten.                                                  \\ \hline
	QRP & Kan jag (ska jag) minska sändareffekten?                             \\
	    & Minska sändareffekten.                                               \\ \hline
	QRQ & Kan jag (får jag) öka sändningshastigheten?                          \\
	    & Öka sändningshastigheten.                                            \\ \hline
	QRS & Kan jag (skall jag) sända långsammare?                               \\
	    & Sänd långsammare.                                                    \\ \hline
	QRT & Skall jag avbryta sändningen?                                        \\
	    & Avbryt sändningen                                                    \\ \hline
	QRU & Har ni något till mig?                                               \\
	    & Jag har inget till er. Se även QTC.                                  \\ \hline
	QRV & Är ni redo?                                                          \\
	    & Jag är redo.                                                         \\ \hline
	QRX & När anropar ni mig härnäst?                                          \\
	    & Jag anropar er kl ... (på ... MHz/kHz)                               \\ \hline
	QRZ & Vem anropar mig?                                                     \\
	    & Ni anropas av ... (på ... MHz/kHz).                                  \\ \hline
	QSA & Vilken styrka har mina signaler?                                     \\
	    & Era signaler är:                                                     \\
	    & 1--Ej uppf., 2--Svaga, 3--Ganska starka, 4--Starka, 5--Mycket starka \\ \hline
	QSB & Svajar styrkan på mina signaler?                                     \\
	    & Styrkan på era signaler svajar.                                      \\ \hline
	QSK & Kan du höra mig mellan dina tecken och får jag avbryta dig?          \\
	    & Jag kan höra dig mellan mina tecken och du får avbryta.              \\ \hline
	QSL & Kan ni ge mig kvittens?                                              \\
	    & Jag kvitterar.                                                       \\ \hline
	QSO & Ha ni förbindelse med ... eller ... (förmedlat)?                     \\
	    & Jag har förbindelse med ... (via ...)                                \\ \hline
	QST & Har tidigare använts som allmänt anrop men ersatts av CQ             \\ \hline
	QSY & Skall jag övergå till att sända på annan frekvens?                   \\
	    & Gå över till att sända på annan frekvens (eller ... kHz/MHz).        \\ \hline
	QTC & Hur många telegram har ni att sända?                                 \\
	    & Jag har ... telegram till dig (eller ...).                           \\ \hline
	QTH & Vilken är er geografiska position?                                   \\
	    & Min geografiska position är ...                                      \\ \hline
	QTR & Kan ni ge mig rätt tid?                                              \\
	    & Rätt tid är ...                                                      \\
\end{longtable}

\subsection{Lokator}

Lokator (Maidenhead locator) är ett praktiskt sätt att tala om sin ungefärliga
position genom att ange endast sex stycken tecken. En lokator kan t.ex. se ut
som JO89VK vilket täcker in nordvästa Järfälla. Det finns många verktyg för att
räkna på lokator där ute, det är bra att känna sin egen. Det finns appar för
detta till telefonerna som både kan räkna på bäring, distans mellan två rutor
och dessutom via telefonens GPS bestämma vilken lokator du för närvarande
befinner dig i.

Första paret dela in jorden i 18x18 fält, dvs 20 grader per fält longitud och 10
grader per fält latitud. Varje sådant fält delas sedan in i 10x10 rutor som
numreras 0-9 på vardera axeln. Dessa i sin tur delas sedan in i 24x24 smårutor
som då får storleksordningen 2.5 grader latitud och 5 grader long. vardera.



\section{Repeatrar}

Repeatrars syfte är främst att förlänga kommunikationen från mobila och portabla
amatörsändare. Samtal mellan fasta stationer förekommer men om ni hör varandra
på direkten, övergå gärna till en simplex-frekvens i stället för att belägga
repeatern.

Lämna luckor mellan er när ni växlar station som sänder. Gör det möjligt för
andra att ''breaka-in'' särskilt om ert QSO fortsätter under längre tid. Ta
hänsyn till att andra kanske vill använda repeatern för att nå personer som de
inte kan nå annars. Hänsyn åt båda hållen förutsätts här.

Repeatern är en begränsad resurs. Det är inte okay att lägga beslag på den under
långa perioder när andra kanske behöver den, var ödmjuk inför att någon driver
repeatern och har satt upp den i första hand för att supporta mobila stationer.

Nödtrafik har alltid prioritet.

\section{QSO}

Konsten att genomföra ett radiosamtal (QSO) i olika sammanhang. Ofta blir folk
nervösa i början för hur detta går till. Man säger sin signal och motstationens
i fel ordning eller liknande.

Man börjar alltid med motstationens signal. Det bör fallas naturligt att ropa så
och man avslutar anropet med sin egen signal så att motstationen dels vet vem
som anropar men också andra hör. Kanske vill en annan station ha ett utbyte med
dig om du inte får svar från den tilltänkta.

Ett radiosamtal består som regel av tre delar. Först sker ett anrop, när kontakt
etablerats utväxlas ett antal meddelande (dialog) och när man är klarar avslutas
samtalet. Dessa tre delar är ganska standard. Man följer detta ganska strikt
t.ex. på kortvågen där telefoni oftast innebär SSB. Anledningen är enkel, det
går inte höra när någon släpper sändtangenten eller bara är tyst och tänker.

När man kör FM över repeatrar på VHF/UHF är det inte lika vanligt att man både
öppnar och avslutar varje sändning med motparten och sin egen signal. Men man
skall regelbundet upprepa signalerna och i praktiken är det lämpligt att göra
kanske var femte minut eller oftare.

\subsection{Anropet}

Ett anrop kan se ut ungefär såhär:

\begin{tabular}{ll}
	Station & Meddelande                            \\ \hline
	SMØUEI  & SAØMAD från SMØUEI, SAØMAD kom.       \\
	SAØMAD  & SMØUEI från SAØMAD, jag lyssnar, kom.
\end{tabular}

Därefter övergår radiosamtalet i dialog eller meddelandesändning.

\subsection{Allmänt anrop}

Används när man inte ropar på någon särskild motstation utan önskar samtal med
vem som helst. På svenska använder man ofta just orden ''allmänt anrop'' medan
på engelska är det vanligare att man uttalar CQ (seek you). Ett allmänt androp
kan se ut såhär:

-- Allmänt anrop, allmänt anrop, allmänt anrop från SM0UEI SM0UEI SM0UEI kallar
allmänt anrop och lyssnar.

Eller på engelska:

-- CQ CQ CQ this is SMØUEI calling CQ CQ CQ and standing by.

\subsection{Meddelandesändning}

-- SA0MAD från SM0UEI, tack för svaret. Din signal är 59 hos mig, mitt QTH är
JO89WA och namnet är Anders. SA0MAD från SM0UEI kom.

-- SM0UEI från SA0MAD, tack för rapporten. Din signal är 57 hos mig, jag
befinner mig i JO89VK men kommer under kvällen byta QTH. Jag kommer då vara QRV
på 3663 kHz. QSL? SM0UEI från SA0MAD.

-- SA0MAD från SM0UEI, QSL på det, QRX 19.30 på frekvens 3663 kHz.

\subsection{Avslutning}

-- SA0MAD från SM0UEI, tack för rapport och vi hörs senare, 73, slut kom

-- SM0UEI från SA0MAD, 73 tillbaka, klart slut.

\subsection{Pile-up och tävling}

Ibland kan det bli väldigt många motstationer samtidigt som ropar. Nu
gäller det att spetsa öronen! Först gäller det att sålla. Rara
signaler från långtbortistan ger mer poäng i en contest som regel
eller från länder du inte kört osv beroende på regler. Försök att
sålla med ''du som sänder från Florida'' eller ''VK7 kom igen'' osv
till det är en station kvar. Kör den snabbt, ropa CQ igen och börja
sålla igen. Stationer du hör signalen på kör du direkt.

Direkt när det uppstår en pile-up är det effektivt att köra split. Dvs
du lyssnar 5--10\,kHz upp eller ned från den frekvens du sänder
på. Det gör det lättare för dig att behålla kommandot under
pile-up. Ligger du och sänder i ett frekvensområde som är särskilt
ägnat för DX är det smart att lägga Rx-frekvensen strax utanför. Det
undviker att man stökar ned i DX-bandet.

Kör du split skall du säga det efter varje sändning. "CQ CQ CQ de Sierra Mike
Zero Uniform Echo India listening 5 up" exempelvis. På CW bör en split vara
minst 2 kHz och på SSB bör den vara minst 5 kHz ännu hellre 10 kHz. Tänk på att
när du startar din split måste du kolla så att båda frekvenserna är ok. Låt inte
din pile-up sprida ut sig för mycket även om det är kanske enklare för dig så är
risken stor att den stör någon annan.

Kör korta QSO. Utbyt snabbt den information som behövs och ta sedan nästa. Ha
förståelse för att det kan bli krockar i en pile-up. När du hör en partiell
signal eller station du vill prata med håll fast vid den. Om du har svårt att
läsa den be den repetera tills ni är klara. Genom att du är auktoriteten på
frekvensen kommer pile-up:en att lugna ned sig och vänta på sin tur. Om du
''hattar omkring'' är risken att all radiodisciplin far ut genom fönstret.

Ofta är det 1 kHz upp som gäller vid CW och digitala trafiksätt.  Du vill
försöka få tag i Södra Shetlandsöarna, ett mycket ovanligt DX, som har en stor
pile up och ropar på 14.195 MHz

\begin{tabular}{lrl}
	Station & Frekvens & Meddelande                     \\ \hline
	VP8SSI  &   14.195 & QRZ VP8SSI  5 to 15 UP         \\
	SMØUEI  &   14.203 & SM0UEI                         \\
	VP8SSI  &   14.195 & SM0UEI 59                      \\
	SMØUEI  &   14.203 & 59 thank you                   \\
	VP8SSI  &   14.195 & Thanks. QRZ VP8SSI 5 to 15 up.
\end{tabular}

Du kommer troligtvis att behöva upprepa din anropssignal flera gånger, men
lyssna efter varje gång du ropat så att du hör vem han svarar. Svarar han inte
dig så får du vänta tills han ropar något som indikerar att han avslutat
kontakten. Exempelvis: Thanks, VP8SSI eller VP8SSI 5 to 15 up.

Du deltar i en tävling där man skall ange singnalrapport och löpnumret från
start av tävlingenpå den kontakt du har. Du har hitintills kontaktat 30
stationer i tävlingen. Du har hittat en ledig frekvens och ropar CQ.

\begin{tabular}{ll}
	Station & Meddelande                       \\ \hline
	SMØUEI  & CQ contest SM0UEI SM0UEI contest \\
	ON3XYZ  & ON3XYZ                           \\
	SMØUEI  & ON3XYZ you are 59 031            \\
	ON3XYZ  & Thanks 031 you are 59 044        \\
	SMØUEI  & 44 Thanks. QRZ SM0UEI
\end{tabular}

Notera att både när man jagar DX och när man deltar i tävlingar så ger man och
får man signalrapporten 59 om man kör foni och 599 telegrafi åtminstone i
internationella tävlingar för att minska risken för fel. Får du en annan
signalrapport än 59 eller 599 i en tävling så är det viktigt du anger korrekt i
din logg. I mer lokala tester som NAC och SAC förekommer det att man ger
korrekta signalrapporter, alltså efter hur väl man hörs.

Om du försöker nå en motstation med pile-up var uppmärksam på dennes sändningar
och vänta på din tur. Tala gärna om signal och var du sänder från men släpp
sedan fram andra. Tänk på hur du själv skulle vilja att en pile-up på din egen
station skulle vilja agera. Den gyllene regeln är också alltid lyssna först ---
sänd sedan!

\clearpage



