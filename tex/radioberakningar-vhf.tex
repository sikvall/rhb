% Radioberäkningar för VHF och UHF

\subsection{Radioberäkningar för VHF och UHF}

\subsubsection{Beräkning av radiohorisonten}

Radiohorisonten är den sträcka som markvågen kan nå utan särskilda hjälpmedel och i frånvaro av andra effekter som särskilda kondisioner (tropo eller duktning) och liknande. Avståndet kan beräknas med hjälp av en enkel formel. Radiohorisonten gäller egentligen bara när inget annat är i vägen men kan ge en ledning till den längsta utbredning man kan förvänta sig med markvåg givet en viss höjd.

För skepp på havet stämmer radiohorisonten ganska väl så man hittar denna formel ofta i utbildningsmaterial för marin VHF men då med distansen i nautiska mil i stället för km. För att få detta byter man konstanten 3,57 till 2,2 i stället.

\begin{equation*}
	r = 3,57 \left(\sqrt{h_1}+\sqrt{h_2}\right)
\end{equation*}

Där $r$ är avståndet till radiohorisonten givet i kilometer, $h_1$ är den ena stationens antennhöjd över marken givet i meter och $h_2$ är den andra stationens antennhöjd över marken också givet i meter.

\subsubsection{Sträckdämpning}

Sträckdämpningen beror på flera olika faktorer, inte minst terrängen och det som finns mellan sändaren och mottagaren. I den fria rymden följer den en enkel geometrisk utbredning men närmare marken behöver man stoppa in en del kompensationsfaktorer.

\begin{equation*}
	PL_0 = 20 \cdot \log(f) + 20 \cdot \log(d) - 27,55
\end{equation*}

Där $PL_{0}$ är sträckdämpningen i decibel(dB) (Eng: Path Loss) mellan två sändare givet avståndet $d$ i meter och frekvensen $f$ i MHz. Om man anger $d$ i kilometer i stället adderar man 60 till konstanten och får då 32,45.

För sträckdämpning vid mark får man mäta eller skatta en utbredningsdämpning som en konstant $k$ som man använder för att modifiera formeln med och får då följande variant:


\begin{equation*}
	PL_m = 20 \cdot \log(f) + (20+k) \cdot \log(d) - 27,55
\end{equation*}

Där $PL_m$ är sträckdämpningen vid marken. Faktorn $k$ kan uppskattas enligt följande tabell:

\begin{table}[h]
	\begin{centering}
		\begin{tabular}{r|l}
			\textbf{k} & \textbf{Beskrivning} \\ \hline
			0 & Över öppen terräng med högre frekvenser och fri sikt\\
			5 & Lättare terräng, mindre kullar, gräs och få träd \\
			10 & Tuffare terräng med mer höjdvariation, klippblock, tätare skog \\
			15 & Urban miljö, större hus, höghus \\
			20 & Extremt urband miljö (tänk Manhattan)\\
		\end{tabular}
	\end{centering}
	\label{tab:frirum-faktor}
	\caption{Tabell över korrigeringsfaktor för frirumsutbredning vid marken}
\end{table}

I vanlig svensk terräng är det nog vanligast man hamnar i storleksordningen 5--10.

