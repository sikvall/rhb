\section{Amatörradiocertifikat}

Amatörradio har en lång historia och sträcker sig tillbaka till radions
barndom. Rent formellt så var det i USA i slutet av 1890-talet som
radioamatörer började sända telegram till varandra med den teknik som fanns
till buds då. Det blev mycket populärt bland elingenjörer och andra
teknikintresserade och runt 1910 började man få problem med interferens och
störningar ochg beslutade sig för att formalisera det hela. Olika
restriktioner infördes men i och med detta regelverk fick vi också vissa krav
på kunskaper för att operera radiosändare.

\subsection{Amatörradiocertifikat HAREC}

Om det fulla amatörradiocertifikatet är ett certifikat som är utformat efter
de principer som anges i HAREC T/R 61-02, Harmonised Amateur Radio Examination
Certificate, Vilnius 2004, uppdaterad 2014 och 2016.

Detta är ett ganska omfattande dokument och i Sverige så har vi t.ex.
KonCEPT-boken som används för ubildning av nya radioamatörer, den kan laddas
ned som PDF från ssa.se eller beställas som papperbok i deras webshop. För att
bli radioamatör behöver man svara tillräckligt många rätt på två stycken
delprov, ett teknikprov som avhandlar elektriska kretsar, radioteknik, sändare
och mottagare, vågutbredning, eletromagnetiska fält, grundläggande matematik
och fysik som är tillämplig, viss komponentlära, förståelse för störningar och
att bli störd, filter och antenner mm. Det andra provet är ett prov över
reglementet som tillämpas, både lagar och författningar som reglerar
amatörradio men även saker som mer praktiska som exempelvis
bokstaveringsalfabetet och Q-koder mm.

Ett godkänt sådant prov ger möjlighet att operera som radioamatör på en stor
mängd olika frekvensband, alla de som i Sverige är utmärkta som
amatörradioband.

\subsection{Amatörradiocertifikat insteg}
\label{sec:instegscertifikat}

Instegscertifikatet är ett förenklat teknikprov men ungefär samma prov vad
gäller reglementen osv. Det förenklade teknikprovet innebär dock en del
begränsningar eftersom instegsamatören inte kan ges riktigt samma förtroende
att sända på alla frekvenser. Exempelvis har man därmed begränsat de
frekvensband som får användas till sådana som är exklusiva för amatörradio i
Sverige. 

\subsubsection{Amatörradioband för instegscertifikat}

\begin{table}[H]
	\centering
	\begin{tabular}{rrcl}
		   \textbf{Frekvens} & \textbf{Benämning} & \textbf{Insteg?} & \textbf{Notiser}               \\ \hline
		      7000--7200 kHz &           40 meter &       JA!        &                                \\
		14\,000--14\,350 kHz &           20 meter &       JA!        & Populärt DX-band!              \\
		21\,000--21\,450 kHz &           15 meter &       JA!        &                                \\
		18\,000--29\,700 kHz &           20 meter &       JA!        & Liknande utbredning som 27 MHz \\
		  50,000--51,990 MHz &            6 meter &       JA!        &                                \\
		144,000--146,000 MHz &            2 meter &       JA!        &
	\end{tabular}
	\caption{Amatörradioband i sverige upp till 70 cm}
\end{table}

Myndigheterna har gett instegscertare tillträde till amatörradioband som är
exklusiva för amatörradio vilket innebär att ett antal populära band inte
finns med. Arbete fortsätter på att försöka tillföra även 70 cm bandet
(432--438 MHz) till de band som är tillåtna för instegsamatörer men det har
inte kommit med i denna utgåva av PTS undantagsföreskrift. 
