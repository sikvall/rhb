\section*{Licens}

Denna handbok är författad av Täpp Anders Sikvall, SM5UEI. Den är släppt under
en creative commons licens som kallas för CC BY-NC-SA.

I korthet innebär det att du får distribuera och trycka upp materialet
hur du vill, dela med dig av det men inte för kommersiella ändamål. Du
får dessutom använda materialet i dina egna alster men om du gör det
ska du släppa dessa under samma licens. Mitt namn ska stå kvar i
originalet och om du använder det i ett annat verk ska det vara
tydligt att det är ett derivat.

För fullständig information kan du besöka \\
\url{https://creativecommons.org/licenses/by-nc-sa/4.0/deed.sv}

Licensen gäller alla versioner av radiohandboken, alla utgåvor och
alla utdrag ur den.

\section*{Om författaren}

Jag heter Täpp Anders Sikvall och har haft ett radiointresse sedan barnsben. När
jag gick i gymnasiet tog jag amatörradiolicens efter dåtidens regler och har
sedan haft amatörradio som en hobby.

Jag jobbar också professionellt med radiosystem och på fritiden tycker jag om
att gå i skog och mark (och köra portabelt). Jag utbildar också nya
radioamatörer och är även en av SSA:s provförrättare.

\section*{Förord}

Detta är den uppdaterade utgåvan för november 2023. Det var ett tag sedan som
jag ägnade mig åt det här alstret och det har kommit in lite påpekanden om
rättningar och att repeaterlistor med mera varit ganska utdaterade. Det är
förstås helt riktigt och jag har i skrivande stund faktiskt försökt få in den
färskaste informationen som finns från SSA.

Mycket har hänt sedan föregående information. Jag har flyttat från Stockholm och
har numera blivit SM5UEI men hänger fortfarande på Kvarnbergets amatörradioklubb
på torsdagskvällarna så där är det öppet hus dessa dagar från kl. 19:00 när ni
alla är välkomna att kika förbi.

Bidrag till handboken tas tacksamt mot men jag kommer bedöma ifall materialet
är lämpligt att ta med.  Hela materialet finns numera också på Github om man
vill gräva i det och fixa-dona, komma med förbättrings\-förslag och annat kul
så finns den på följande länk:

\href{https://github.com/sikvall/rhb/}{https://github.com/sikvall/rhb}

Ett tack till alla som har varit med och påtalat fel och förbättringar i boken
genom att skicka in sådana påpekanden på GitHub, via mail eller på annat sätt.
Stort tack för ni hjälper till att påtala fel och ordna förbättringar.

Har ni en massa bra förslag på grejer som ni skulle vilja ha med i boken så låt
mig veta det där, det går också om ni vill arbeta in ändringar direkt och skicka
mig en s.k. "pull request" så kan jag kika på det. Det går också att logga
issues om man hittar fel.

Det går också hitta en massa annat radiorelaterat på min hemsida om ni är
intresserade och där kan ni också normalt hitta radiohandboken som senaste
version i PDF att ladda ner direkt.

\href{https://sikvall.se}{https://sikvall.se/}

Och kör radio där ute. Alltid med stil.

\vspace{4mm}

Karlholm, \DokumentDatum\\
\textit{Täpp-Anders Sikvall\\
	SM5UEI}

\clearpage

\section*{Nyheter i denna utgåva}

\begin{itemize}
  \item Licensvillkoren för verket är uppdaterade
  \item Om författaren tillagt
  \item Mer om kommersiell radio och myndighetsradio tillagt liksom information om
  \item undantag från tillståndsplikt
\end{itemize}
\todo{Fyll i vad som ändrats sedan sist}

\clearpage


