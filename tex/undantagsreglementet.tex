% Undantagsföreskriften

\section{Undantag från tillståndsplikt}

I sverige regleras all radioanvändning med sändare som inte kräver att man söker tillstånd för det i en skrift som ges ut av Post- och Telestyrelsen (PTS) och som bär namnet "Untandag från tillståndsplikt". Denna skrift uppdateras lite då och då när det finns behov av det och här regleras all radiotrafik som inte kräver särskilt tillstånd. 

Radiotrafik som kräver tillstånd regleras i andra dokument eller av respektive myndighet som svarar för examinering och tillståndsgivning exempelvis luftfartsverket för luftfartsradio och så vidare. Försvarsmakten svarar också för användning av militär radio och har långt gående möjligheter att sända på frekvenser som i fredstid används för annat.

Enklast finner man det gällande dokumentet genom att söka efter det på PTS hemsida pts.se.

\subsection{Tillståndsfri radio}

Det är i undantagsföreskriften du hittar alla villkor som hänger samman med tillståndsfri radiosändning. Här står effekter, modulationstyper och användningsområden upptagna för respektive frekvens samt andra villkor.

Det är också i denna skrift som användningen av Amatörradio regleras rent juridiskt. Sedan finns inte alla rekommendationer och liknande med i denna skrift, här får man själv hitta det reglemente som amatörradioorganisationer mm kommit överens om hur man ska nyttja de olika bandet.

Det innebär att radioanvändning för diverse ändamål regleras i denna skrift så som:

\begin{itemize}
	\item PR-radio på 27 MHz
	\item 69 MHz åkeriradio som också får användas för andra ändamål
	\item Personlig radio på 446 och 444 MHz
	\item Radio för jakt, fiske och jordbruk på 155 MHz
	\item Radio för jakt på 31 MHz
	\item ... och mycket mer.
\end{itemize}

En sak som är intressant är att amatörradio behandlas också i samma skrift trots det kräver ett godkänt avlagt prov som leder till ett amatörradiocertifikat. Tidigare reglerades amatörradio separat på Televerkets tid i en författningssamling som kallades för Televerket B:90. Denna har dock helt ersatts av bestämmelserna i undantagsföreskriften.

Däremot krävs det numera inget särskilt tillstånd som det gjorde tidigare för att få använda amatörradio, annat än att den som opererar radioutrustningen skall ha, eller stå under uppsikt av, en person med giltigt amatörradiocertifikat.

