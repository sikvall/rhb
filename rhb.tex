\documentclass[12pt,swedish,a4paper]{article}
\usepackage[utf8]{inputenc}
\usepackage[swedish]{babel}
\usepackage{longtable}
\usepackage[a5paper,textwidth=12cm,textheight=17cm]{geometry}
\usepackage{amsmath}

\begin{document}
\title{Handbok för Radiooperatörer\\
VHF/UHF}
\author{SM0UEI}
\maketitle

\begin{abstract}
Handbok för radiooperatörer innehåller diverse matnyttig information för amatörradiooperatörer i Sverige. Tanken är att samla information som t.ex. är bra att ha på resande fot eller annars som man saknar när man behöver den och inte har tillgång till Internet.

Skriv ut den på A4 med två sidor på samma papper och vik ihop till ett häfte. Ladda ner den i läsplattan eller surfplattan. Papperstypen är vald som A5 för att underlätta läsning på platta och mobiltelefon till och med.

Idéer tankar, korrigeringar och annat som behöver vara med är välkommet att du skickar till anders@sikvall.se så kan jag se till att få med det i nästa utgåva. Denna bok uppdateras sporadiskt när det finns skäl och tid till detta.

Till en början är det VHF och UHF men kommer så småningom utökas med fler och fler band.
\end{abstract}

\clearpage
\tableofcontents
\clearpage

\setlength{\parskip}{0.5em}
\setlength{\parindent}{0pt}

% % % % % % % % % % % % % % % % % % % % % % %

\section{Distrikt}

Sverige delas in i följande distrikt efter sina län:

\begin{tabular}{cl}
\textbf{Dist.} & \textbf{Län}\\
\hline
0 & Stockholm\\
1 & Gotland\\
2 & Västerbotten, Norrbotten\\
3 & Gävleborg, Jämtland, Västernorrland\\
4 & Örebro, Värmland, Dalarna\\
5 & Östergötland, Södermanland, Västmanland, Uppsala\\
6 & Halland, Västra götaland\\
7 & Skåne, Blekinge, Kronoberg, Jönköping, Kalmar\\
8 & Speciella stationer utanför landets gränser, t.ex. militära skepp\\
\end{tabular}

\section{Repeaterlistan}

Repeaterlistan sorteras på geografiskt namn efter var de befinner sig.

\begin{tabular}{lllrrl}
	\textbf{ID} & \textbf{Kommun} & \textbf{Ort} & \textbf{Frekvens} & \textbf{Skift} & \textbf{Access} \\
	\hline
	SK7XY       & Malmö           & Malmö        &           145.625 &       -0.600 & DTMF 0
\end{tabular}

\subsection{VHF 2m}

\subsection{UHF 70cm}

\section{Signaler}

\subsection{Klubbar}

\begin{tabular}{ll|ll}
	Signal & Klubb               & Signal & Klubb                   \\ \hline
	SK5JV  & Fagersta radioklubb & SK0SSA & Sveriges sändaramatörer
\end{tabular}

\subsection{Militära stationer (FRO)}

\begin{tabular}{ll}
	Signal & Beskrivning                \\ \hline
	SL5AB  & Enköpings ledningsbataljon
\end{tabular}

\section{Terminologi och trafik}

\subsection{Q-koder}

\subsection{Bokstaveringsalfabetet (Svenska)}

\begin{tabular}{cl|cl|cl }
	A & Adam   & O & Olof    & 1 & Ett \\
	B & Bertil & P & Petter  & 2 & Tvåa \\
	C & Cesar  & Q & Qvintus & 3 & Trea \\
	D & David  & R & Rudolf  & 4 & Fyra \\
	E & Erik   & S & Sigurd  & 5 & Femma \\
	F & Filip  & T & Tore    & 6 & Sexa \\
	G & Gustav & U & Urban   & 7 & Sju \\
	H & Helge  & V & Viktor  & 8 & Åtta \\
	I & Ivar   & W & Wilhelm & 9 & Nia \\
	J & Johan  & X & Xerxes  & 0 & Nolla \\
	K & Kalle  & Y & Yngve   &   &  \\
	L & Ludvig & Z & Zäta    &   &  \\
	M & Martin & Å & Åke     &   &  \\
	N & Niklas & Ä & Ärlig   &   &  \\
	  &        & Ö & Östen   &   &
\end{tabular}

\subsection{Bokstaveringsalfabetet (Internationella)}

--- Mer om detta här.

\section{Frekvenser}

\subsection{Tillståndsfria frekvenser, ej amatörradio}

Dessa frekvenser är avsedda för allmänhet eller för specifika ändamål som anges. Det innebär att de kan brukas för de ändamål som anges i PTS författningssamlingar och sammanställning över ej tillståndspliktiga frekvenser. Observera att du är skyldig att själv kontrollera bestämmelserna innan en frekvens brukas.

Effekten i tabellen är ustrålad effekt PEP om inte annat anges.\\

\begin{tabular}{rlrl}
	Frekvens & Benämning & Effekt & Användningsområde   \\ \hline
	 155,425 & Jakt K1   &    5 W & Jakt, Jordbruk [M]  \\
	 155,475 & Jakt K2   &    5 W & Jakt, Jordbruk [M]  \\
	 155,475 & Jakt K3   &    5 W & Jakt, Jordbruk [M]  \\
	 155,525 & Jakt K4   &    5 W & Jakt, Jordbruk [M]  \\
	 156,000 & Jakt K5   &    5 W & Jakt, Jordbruk, PMR \\
	 155,400 & Jakt K6   &    5 W & Jakt, Jordbruk [M]  \\
	 155,450 & Jakt K7   &    5 W & Jakt, Jordbruk [M]
\end{tabular}

Jakt K5 är öppen att användas för andra ändamål och sammanfaller ej med marina VHF-bandet vilket de andra gör.

\begin{tabular}{rlrl}
	 Frekvens & Benämning & Effekt & Användningsområde \\ \hline
	446,00625 & PMR446 K1 & 500 mW & PMR [N]              \\
	446,01875 & PMR446 K2 & 500 mW & PMR [N]              \\
	446,03125 & PMR446 K3 & 500 mW & PMR [N]              \\
	446,04375 & PMR446 K4 & 500 mW & PMR [N]              \\
	446,05625 & PMR446 K5 & 500 mW & PMR [N]              \\
	446,06875 & PMR446 K6 & 500 mW & PMR [N]              \\
	446,08125 & PMR446 K7 & 500 mW & PMR [N]              \\
	446,09375 & PMR446 K8 & 500 mW & PMR [N]
\end{tabular}

[M] Delas med marin VHF-radio. Får därför inte användas till sjöss inom landets gränser eller i svenska territorialvatten.

[N] Smalbandig FM-modulation skall användas pga tätt liggande kanaler.

På PMR446 används ibland subtoner för att skapa virtuella grupper och sub-kanaler. De som används är följande toner och frekvenser:

\begin{tabular}{rr|rr|rr}
	 1 &  67,0 & 14 & 107,2 & 27 & 167,9 \\
	 2 &  69,3 & 15 & 110,9 & 28 & 173,8 \\
	 3 &  74,4 & 16 & 114,8 & 29 & 179,9 \\
	 4 &  77,0 & 17 & 118,8 & 30 & 186,2 \\
	 5 &  79,7 & 18 & 123,0 & 31 & 192,8 \\
	 6 &  82,5 & 19 & 127,3 & 32 & 203,5 \\
	 7 &  85,4 & 20 & 131,8 & 33 & 210,7 \\
	 8 &  88,5 & 21 & 136,5 & 34 & 218,1 \\
	 9 &  91,5 & 22 & 141,3 & 35 & 225,7 \\
	10 &  94,8 & 23 & 146,2 & 36 & 233,6 \\
	11 &  97,4 & 24 & 151,4 & 37 & 241,8 \\
	12 & 100,0 & 25 & 156,7 & 38 & 250,3 \\
	13 & 103,5 & 26 & 162,2 &    &
\end{tabular}

\begin{tabular}{rlrl}
	Frekvens & Benämning & Effekt & Användningsområde          \\ \hline
	 444,600 & SRBR K1   &    2 W & Short range business radio \\
	 444,625 & SRBR K2   &    2 W & Short range business radio \\
	 444,800 & SRBR K3   &    2 W & Short range business radio \\
	 444,825 & SRBR K4   &    2 W & Short range business radio \\
	 444,850 & SRBR K5   &    2 W & Short range business radio \\
	 444,875 & SRBR K6   &    2 W & Short range business radio \\
	 444,925 & SRBR K7   &    2 W & Short range business radio \\
	 444,975 & SRBR K8   &    2 W & Short range business radio
\end{tabular}

SRBR är ett ej tillståndspliktigt frekvenssegment som används för yrkesmässig radiotrafik.

\section{Telegrafi}

\section{Formler och matematik}

\subsection{Ohms Lag och effektlagen}

\begin{align}
U &= R\cdot I\\
R &= U/I\\
I &= U/R
\end{align}

\begin{align}
P &= U \cdot I\\
U &= P/I\\
I &= P/U
\end{align}

\begin{align}
P &= U^2/R\\
P &= R \cdot I^2
\end{align}

\subsection{Frirumsutbredning}

En radiovåg i fria rymden utbreder sig med ljusets hastighet $c$ och effektdensiteten kan beskrivas som arean hos en expanderande sfär från en punktformig strålkälla.

Med frekvensen $f$ i MHz och avståndet $d$ i kilometer kan man beräkna sträck\-dämp\-ning\-en\footnote{Eng.: Pathloss} genom följande formel som också kallas \emph{Friis formel för frirumsutbredning} och som tecknas:

\begin{equation}
L=20\cdot \log(f) + 20\cdot \log(d) +32.45
\end{equation}

\subsection{Kaskadkopplade bruskällor}

Kaskadkopplade förstärkarsteg med sina brusfaktorer och respektive gain kända kan beräknas teoretiskt med följande samband där $N_n$ är det $n$:e stegets brusfaktor samt $G_n$ är det $n$:e blockets förstärkningsfaktor (gain). Observera att detta är i faktorer, ej dB. För att räkna om från dB till linjär faktor använder du $F=10^{\mathrm{dB}/10}$ så blir det korrekt.

\begin{equation}
N'=N_1+\frac{N_2-1}{G_1} + \frac{N_3-1}{G_1G_2} + ... + \frac{N_n-1}{G_1G_2...G_{n-1}}
\end{equation}

\subsubsection{Brusfaktor vanliga komponenter}

En bra LNA\footnote{Low Noise Amplifier, preamp} kan ha en brusfaktor under 1 dB. Mellanförstärkare väl matchade brukar ligga runt ca 2 dB. Slutsteg kan ha ganska hög brusfaktor. Det är alltid första steget i kedjan som bestämmer mest hur det skall bli.

Passiva komponenter har samma brusfaktor som deras dämpning i dB.

\end{document}
