% !TeX encoding = UTF-8
% !TeX spellcheck = en_US
\documentclass[10pt,swedish,a4paper]{article}
\usepackage[a4paper,bindingoffset=1cm]{geometry}
\usepackage[utf8]{inputenc}
\usepackage[T1]{fontenc}
\usepackage[swedish]{babel}
\usepackage[colorlinks,linkcolor=black,urlcolor=blue]{hyperref}
\usepackage{enumitem}
\usepackage{multirow}
\usepackage{amsmath}
%\usepackage{wasysym}
\usepackage{tabularx}
%\usepackage{textcomp}
%\usepackage{newcent}
%\usepackage{eulervm}
%\usepackage{fourier}
\usepackage{graphicx}
\usepackage[table,x11names]{xcolor}
\usepackage{fancyhdr}
\usepackage[yyyymmdd]{datetime} \renewcommand{\dateseparator}{-}
\usepackage{lastpage}
\usepackage{newpxtext,newpxmath}
\usepackage{pdfpages}
\usepackage{icomma}
\usepackage{pdflscape}
\usepackage{wrapfig}
\usepackage{float}
\usepackage[bottom]{footmisc}
\usepackage{longtable}
%\usepackage{tikz}
%\usetikzlibrary{arrows,chains,positioning,patterns,calc}
\newif\iftitlefoot
\raggedbottom
\setlist{nosep}
\usepackage{todonotes}
%\usepackage{draftwatermark}
\raggedbottom


%\SetWatermarkText{GRANSKAS}
%\SetWatermarkScale{.7}
%\SetWatermarkLightness{.7}

%%%%%%%%%%%%%%%%%%%%%%%%%%%%%%%%%%%%%%%%%%%%%%%%%%%%%%%%%%%%%%
%%%%% Definiera diverse saker här som används i dokumentet
\newcommand{\TitleText}{Radiohandbok}
\newcommand{\SubtitleText}{För Sändaramatörer\\ och Privatradioanvändare}
\newcommand{\Forfattare}{Täpp-Anders Sikvall}
\newcommand{\Initialer}{SMØUEI}
\newcommand{\DokYear}{19}
\newcommand{\DokVersion}{2.0.0}
\newcommand{\DokumentRevision}{x.y.z}
\newcommand{\DokumentDatum}{\today}
%%%%%%%%%%%%%%%%%%%%%%%%%%%%%%%%%%%%%%%%%%%%%%%%%%%%%%%%%%%%%%%%%%


% Tables are getting a little squashed without this
\renewcommand{\arraystretch}{1.15}

%\titlefootfalse % Använd denna för ren förstasida
\titlefoottrue % Använd denna för en footer på första sidan
%\newcommand{\titlefootcontent}{%
%	\begin{tabularx}{.9\textwidth}{X X X X}	%
%		\textbf{Intressegrupp} & \textbf{Författare} & \textbf{Datum}  & \textbf{Utgåva} \\		%
%		\Mottagare       & \Forfattare          & \DokumentDatum & \DokumentNummer   \\		%            
%	\end{tabularx}%
%}

%\addtolength{\headsep}{3mm}
%\addtolength{\textheight}{15mm}

\begin{document}
	
	
	%%%%%%%%%%%%%%%%%%%%%%%%%%%%%%%%%%%
	%%% Bygger förstasidan här
	
	\newgeometry{left=2cm,right=2cm,bottom=1cm,top=1cm}
	\pagestyle{empty}
	\vfill
	%	\begin{flushright}
	%		\includegraphics[width=0.1\textwidth]{logo/logo}
	%	\end{flushright}
	\vspace*{4cm}
	\centerline{\includegraphics[width=\paperwidth]{logo/rubrikbild}}
	\begin{flushright}
		\Huge{\bfseries{\TitleText}} \\[3mm]
		\Large{\bfseries{\SubtitleText}}
	\end{flushright}
	
	\vfill
	
	%	\small
	%	\begin{tabular}{llll}
	%		\textbf{Rev} & \textbf{Date} & \textbf{Responsible} & \textbf{Description}       \\ \hline
	%		B            & 2019-03-16    & ANSI                 & Uppdateringar med mer info \\
	%		A            & 2019-03-15    & ANSI                 & Initial version
	%	\end{tabular}
	%	\normalsize 
	%	\vfill
	
	%	\iftitlefoot
	%	\scriptsize
	%	\vspace*{-1em}
	%	\hrule
	%	\begin{center}
	%		\begin{tabularx}{.9\textwidth}{X X X X}
	%			\titlefootcontent &
	%		\end{tabularx}
	%	\end{center}
	%	\normalsize
	%	\fi
	\newpage
	
	%\restoregeometry
	
	\newgeometry{top=3cm}
	
	\pagestyle{fancy}
	%\setlength{\headheight}{53pt} 
	
	%	\lhead{\includegraphics[height=10pt]{logo/logo}}
	\lhead{\leftmark}	
	\rhead{
		\scriptsize
		\begin{tabular}{ll}
			\textbf{Version} & \textbf{Datum}\\
			\DokVersion & \DokumentDatum\\
		\end{tabular}
	}
	
	\chead{}
	
	\lfoot{
		\scriptsize
		www.sm0uei.se
	}
	
	\cfoot{\scriptsize \thepage\ / \pageref{LastPage}}
	
	\rfoot{\scriptsize
		anders@sikvall.se
	}
	
	\renewcommand{\footrulewidth}{0.2pt}
	
	\widowpenalty=9999
	\clubpenalty=9999
	
	%	\setlength{\headsep}{1em}2
	
	
	
	%%%%%%%%%%%%%%%%%%%%%%%%%%%%%%%%%%%%%%%%%%%%%%%%%%%%%
	%%% Här börjar dokumentet som skall redigeras
	%
	
	\cleardoublepage
	%\newgeometry{left=3.2cm,right=3.2cm,bottom=2.5cm,top=2.5cm}
	
	%%% Innehållsförteckning
	\tableofcontents
	
	\newpage
	
	% Lista bilagor här
	
	%\listoffigures
	
	%\listoftables
	
	%\listoftodos
	
	
	%%% Justera här om du föredrar indrag som i löpande text. Teknisk dokumentation
	%%% tycker jag fungerar bättre med styckenmellanrum utan indrag. \parskip
	%%% justerar styckemellanrum och \parindent är indrag.
	
	
	
	\setlength{\parskip}{0.5em}
	\setlength{\parindent}{0pt}
	
	\input{init-vhf}
	\section{Signaler och anrop}
\subsection{Landsprefix}

Här är inte alla länder med utan de vanligaste som körs från Sverige.

\begin{center}
	\begin{footnotesize}
		\begin{longtable}{lll}
			\caption{Utvalda landsprefix} \\
			\textbf{Land}                 & \textbf{DXCC}  & \textbf{Prefixserier}                             \\ \hline
			Belgien                       & ON             & ONA--OTZ                                          \\
			Canada                        & VE             & CYA--CZZ, VAA--VGZ, VOA--VOZ, VXA--VYZ, XJA--XOZ  \\
			Frankrike                     & F              & FAA--FZZ, HWA--HWZ, THA--THZ, TKA--TKZ, TMA--TMZ, \\
			                              &                & TOA--TQZ, TVA--TXZ                                \\
			Frankr. särsk.                & FG FH FK       &                                                   \\
			                              & FM, FO, FP, FR &                                                   \\
			                              & FS, FT, FW, FY &                                                   \\
			Förenta Staterna              & K              & AAA--ALZ, KAA--KZZ, NAA--NZZ, WAA--WZZ            \\
			Grekland                      & SV             & J4A--J4Z, SVA--SVZ                                \\
			Italien                       & I              & IAA--IZZ                                          \\
			Japan                         & JA             & 7JA--7NZ, 8JA--8NZ, JAA--JSZ                      \\
			Kroatien                      & 9A             & 9AA--9AZ                                          \\
			Nederländerna                 & PA             & PAA--PLZ                                          \\
			Polen                         & SP             & 3ZA--3ZZ, HFA--HFZ, SNA--SRZ                      \\
			Rumänien                      & YO             & YOA--YRZ                                          \\
			Ryssland (Eur.)               & UA1 3 4 5 6 7  & RAA--RZZ, UAA-UIZ                                 \\
			Ryssland (Asi.)               & UA8 9 0        & RAA--RZZ, UAA-UIZ                                 \\
			Schweiz                       & HB             & HBA--HBZ HEA--HEZ                                 \\
			Spanien                       & EA             & AMA--AOZ, EAA--EHZ                                \\
			Storbritt. England            & G, 2E, M       & 2AA--2ZZ, GAA--GZZ, MAA--MZZ, VPA--VQZ,           \\
			                              &                & VSA--VSZ,ZBA--ZJZ, ZNA--ZOZ, ZQA--ZQZ             \\
			Storbritt. Skottland          & GM, 2M, MM     &                                                   \\
			Storbritt. Övrigt             & VP2, VP6, VP8  &                                                   \\
			                              & VP9, VQ9, ZB   &                                                   \\
			Sverige                       & SM             & 7SA--7SZ, 8SA--8SZ, SAA--SMZ                      \\
			Tyskland                      & DL             & DAA--DRZ, Y2A--Y9Z                                \\
			Ukraina                       & UT             & EMA--EOZ, URA--UZA                                \\
			Ungern                        & HA             & HAA--HAZ, HGA--HGZ                                \\
			Österrike                     & OE             & OEA--OEZ\\
		\end{longtable}
	\end{footnotesize}
\end{center}

\subsection{Svenska signaler}

Svenska signaler förekommer inom ett antal prefix. Enligt ITU disponerar Sverige förljande signalserier: 7SA--7SZ samt 8SA--8SZ och vidare de mer kända SAA--SMZ. Dessa har används till varierande ändamål, exempelvis har flyget signaler i serien SE-AAA--ZZZ. Polisen har tidigare använt signaler i serien SHA plus fyra siffror, detta är nu ersatt med nytt system i.o.m. RAKEL. Räddningstjänsten använde SDA med fyra siffror. Signaler som 7SA + 4 siffror används för mindre yrkesbåtar SC + 4 siffror för fritidsbåtar.

Amatörradion använder ett antal signaler, de viktigaste är:

\begin{tabular}{ll}
	SM & Amatörradiosignal utdelad av PTS (nya signaler tilldelas ej i serien)               \\
	SA & Amatörradiosignal tilldelad av SSA                                                  \\
	SK & Klubbsignaler (som regel tvåställiga efter distriktsiffran)                         \\
	   & numera tilldelas även klubbar SA-signaler som är tvåställiga efter distriktssiffran \\
	SL & Militära signaler (som regel två- eller treställiga efter distriktsiffran)
\end{tabular}

Dessa signaler följs av en \textit{distriktsiffra} se särskilt avsnitt och sedan 2-ställiga eller 3-ställiga bokstavskombinationer som är den personliga signalen. Exempel är SM0UEI som är min egen signal, distriktsiffran är 0 dvs hemmavarande i Stockholms län. Ett annat exempel kan vara SK5JV tidigare Fagersta amatörradioklubb.

Repeatrar som tillhör klubbar får ofta signal efter klubben med tillägg /R för repeater.

Det finns numera även ett stort antal signaler som är tillfälliga eller knutna till särskilda event, exempelvis scoutverksamhet som ibland sänder amatörradio och särskilda forskningsfartyg, flyg- och rymdfart mm.

Som suffix används följande:

\begin{tabular}{ll}
	/M  & Mobil (rörlig) sändaramatör, även portabel \\
	/MM & Mobil till sjöss (mobil maritime)          \\
	/AM & Mobil i luften (aeromobile)                \\
	/P  & Portabel (för stunden uppsatt station)     \\
	/R  & Repeaterstation
\end{tabular}

\subsection{Svenska distrikten}

Sverige delas in i följande distrikt efter sina län:

\begin{table}[h]
	\centering
\begin{tabular}{cl}
	\textbf{Distrikt} & \textbf{Län}                                     \\ \hline %\endhead
	      0        & Stockholm                                        \\
	      1        & Gotland                                          \\
	      2        & Västerbotten, Norrbotten                         \\
	      3        & Gävleborg, Jämtland, Västernorrland              \\
	      4        & Örebro, Värmland, Dalarna                        \\
	      5        & Östergötland, Södermanland, Västmanland, Uppsala \\
	      6        & Halland, Västra götaland                         \\
	      7        & Skåne, Blekinge, Kronoberg, Jönköping, Kalmar    \\
	      8        & Speciella stationer utanför landets gränser
\end{tabular}
\caption{Distriktssiffor i Sverige}
\end{table}

\subsubsection{Karta över svenska amatörradiodistrikt}

\begin{wrapfigure}{R}{15cm}
	\centering
	\includegraphics[width=15cm]{pic/sm-distrikt-stor}
	\label{fig:sm-distrikt}
	\caption{Svenska distrikt, karta med tillståpnd från \href{https://SSA.SE}{ssa.se}}
\end{wrapfigure}

Distrikten förekommer som siffra i utdelade anropssignaler. Radioamatörer 
byter inte distriktsiffra under resa i annat distrikt, i stället används 
suffix (tillägg efter ordinarie signal) som t.ex. /M för mobil. Ofta uppger 
man "SM0UEI mobilt i SM3-land" (SM0UEI/3/M) ibland (SM0UEI/3M) för att 
påvisa att man befinner sig utanför ordinarie distrikt.

En radioamatör kan byta sin distriktsiffra om den sänder från ett
annat distrikt än sitt hemmavarande. Man kan också göra ett tillägg
med /n där n är den siffra för det distrikt man befinner sig i. En
stockholmsamatör som befinner sig i Gävleborgs län kan alltså antingen
använda SM3UEI eller SM0UEI/3 även med tillägget M för mobil och P för
portabel om man så önskar.

Det unika för en radioamatörs signal är alltså prefixet + suffixet,
som exempel är identifieraren för SM0UEI prefixet SM och suffixet UEI
eftersom distriktsiffran kan ändra sig. 

\clearpage


\section{Terminologi och trafik}

\subsection{Bokstaveringsalfabetet (Svenska)}

\begin{table}[H]
	\centering
\begin{longtable}{cl|cl|cl }
	A & Adam   & O & Olof    & 1 & Ett        \\
	B & Bertil & P & Petter  & 2 & Tvåa       \\
	C & Cesar  & Q & Qvintus & 3 & Trea       \\
	D & David  & R & Rudolf  & 4 & Fyra       \\
	E & Erik   & S & Sigurd  & 5 & Femma      \\
	F & Filip  & T & Tore    & 6 & Sexa       \\
	G & Gustav & U & Urban   & 7 & Sju        \\
	H & Helge  & V & Viktor  & 8 & Åtta       \\
	I & Ivar   & W & Wilhelm & 9 & Nia        \\
	J & Johan  & X & Xerxes  & 0 & Nolla      \\
	K & Kalle  & Y & Yngve   & . & Punkt      \\
	L & Ludvig & Z & Zäta    & , & Komma      \\
	M & Martin & Å & Åke     & - & Minus      \\
	N & Niklas & Ä & Ärlig   & + & Plus       \\
	  &        & Ö & Östen   &   & Mellanslag \\
\end{longtable}
\caption{Svenska bokstaveringsalfabetet}
\end{table}

\subsection{Bokstaveringsalfabetet (Internationella)}
\begin{table}[H]
\centering
\begin{tabular}{cl|cl|cl}
	A & Alfa     &  P   & Papa       & 0 & Zero    \\
	B & Bravo    &  Q   & Quebec     & 1 & One     \\
	C & Charlie  &  R   & Romeo      & 2 & Two     \\
	D & Delta    &  S   & Sierra     & 3 & Tree    \\
	E & Echo     &  T   & Tango      & 4 & Fower   \\
	F & Foxtrot  &  U   & Uniform    & 5 & Fife    \\
	G & Golf     &  V   & Victor     & 6 & Six     \\
	H & Hotel    &  W   & Whiskey    & 7 & Seven   \\
	I & India    &  X   & X-ray      & 8 & Ait     \\
	J & Juliet   &  Y   & Yankee     & 9 & Niner   \\
	K & Kilo     &  Z   & Zulu       & . & Stop    \\
	L & Lima     & Å/AA & Alfa-Alfa  & , & Decimal \\
	M & Mike     & Ä/AE & Alfa-Echo  & - & Minus   \\
	N & November & Ö/OE & Oscar-Echo & + & Plus    \\
	O & Oscar    &      &            &   & Space   \\
\end{tabular}
\caption{Internationella bokstaveringsalfabetet (ITU-alfabetet)}
\end{table}

\subsection{Q-koder}
I tabellen listas några av de vanligast förekommande Q-koderna på amatörradiobanden. 
Det finns förstås många fler koder men detta anses som de vanligaste.

\begin{longtable}{ll}
	\textbf{Kod} & \textbf{Fråga / Svar}                                                         \\ \hline 
	\endhead
	\caption{Q-koder}\\    
\endlastfoot
	QRA & Vad heter er station?                                                \\
	    & Vår station heter ...                                                \\ \hline
	QRB & Hur långt bort från min station befinner ni er?                      \\
	    & Avståndet mellan oss är ungefär ...                                  \\ \hline
	QRG & Kan ni ange min exakta frekvens?                                     \\
	    & Er exakta frekvens är ... (MHz/kHz)                                  \\ \hline
	QRH & Varierar min frekvens/våglängd?                                      \\
	    & Er frekvens/våglängd varierar.                                       \\ \hline
	QRI & Hur är min sändningston (CW)?                                        \\
	    & Er sändningston är 1--God, 2--Varierande, 3--Dålig                   \\ \hline
	QRK & Vilken uppfattbarhet har mina signaler?                              \\
	    & Uppfattbarheten hos dina signaler är:                                \\
	    & 1--Dålig, 2--Bristfällig, 3--Ganska god, 4--God, 5--Utmärkt          \\ \hline
	QRL & Är ni upptagen?                                                      \\
	    & Jag är upptagen med ... (namn/signal) stör ej.                       \\ \hline
	QRM & Är ni störd av annan station?                                        \\
	    & Störningarna är:                                                     \\
	    & 1--Obef., 2--Svaga, 3--Måttliga, 4--Starka, 5--Mycket starka         \\ \hline
	QRN & Besväras ni av atmosfäriska störningar?                              \\
	    & Störningarna är:                                                     \\
	    & 1--Obef., 2--Svaga, 3--Måttliga, 4--Starka, 5--Mycket starka         \\ \hline
	QRO & Kan jag (ska jag) öka sändareffekten?                                \\
	    & Öka sändareffekten.                                                  \\ \hline
	QRP & Kan jag (ska jag) minska sändareffekten?                             \\
	    & Minska sändareffekten.                                               \\ \hline
	QRQ & Kan jag (får jag) öka sändningshastigheten?                          \\
	    & Öka sändningshastigheten.                                            \\ \hline
	QRS & Kan jag (skall jag) sända långsammare?                               \\
	    & Sänd långsammare.                                                    \\ \hline
	QRT & Skall jag avbryta sändningen?                                        \\
	    & Avbryt sändningen                                                    \\ \hline
	QRU & Har ni något till mig?                                               \\
	    & Jag har inget till er. Se även QTC.                                  \\ \hline
	QRV & Är ni redo?                                                          \\
	    & Jag är redo.                                                         \\ \hline
	QRX & När anropar ni mig härnäst?                                          \\
	    & Jag anropar er kl ... (på ... MHz/kHz)                               \\ \hline
	QRZ & Vem anropar mig?                                                     \\
	    & Ni anropas av ... (på ... MHz/kHz).                                  \\ \hline
	QSA & Vilken styrka har mina signaler?                                     \\
	    & Era signaler är:                                                     \\
	    & 1--Ej uppf., 2--Svaga, 3--Ganska starka, 4--Starka, 5--Mycket starka \\ \hline
	QSB & Svajar styrkan på mina signaler?                                     \\
	    & Styrkan på era signaler svajar.                                      \\ \hline
	QSK & Kan du höra mig mellan dina tecken och får jag avbryta dig?          \\
	    & Jag kan höra dig mellan mina tecken och du får avbryta.              \\ \hline
	QSL & Kan ni ge mig kvittens?                                              \\
	    & Jag kvitterar.                                                       \\ \hline
	QSO & Ha ni förbindelse med ... eller ... (förmedlat)?                     \\
	    & Jag har förbindelse med ... (via ...)                                \\ \hline
	QST & Har tidigare använts som allmänt anrop men ersatts av CQ             \\ \hline
	QSY & Skall jag övergå till att sända på annan frekvens?                   \\
	    & Gå över till att sända på annan frekvens (eller ... kHz/MHz).        \\ \hline
	QTC & Hur många telegram har ni att sända?                                 \\
	    & Jag har ... telegram till dig (eller ...).                           \\ \hline
	QTH & Vilken är er geografiska position?                                   \\
	    & Min geografiska position är ...                                      \\ \hline
	QTR & Kan ni ge mig rätt tid?                                              \\
	    & Rätt tid är ...                                                      \\ 
\end{longtable}

\subsection{Lokator}

Lokator (Maidenhead locator) är ett praktiskt sätt att tala om sin ungefärliga position genom att ange endast sex stycken tecken. En lokator kan t.ex. se ut som JO89VK vilket täcker in nordvästa Järfälla. Det finns många verktyg för att räkna på lokator där ute, det är bra att känna sin egen. Det finns appar för detta till telefonerna som både kan räkna på bäring, distans mellan två rutor och dessutom via telefonens GPS bestämma vilken lokator du för närvarande befinner dig i.

Första paret dela in jorden i 18x18 fält, dvs 20 grader per fält longitud och 10 grader per fält latitud. Varje sådant fält delas sedan in i 10x10 rutor som numreras 0-9 på vardera axeln. Dessa i sin tur delas sedan in i 24x24 smårutor som då får storleksordningen 2.5 grader latitud och 5 grader long. vardera.

\subsection{Uppträdande}

När vi kör amatörradio finns det ett antal saker att tänka på som har att göra med hur vi beter oss mot varandra på banden. Se detta som en guide till hur man bör uppträda på banden.

En radioamatör måste vara \textbf{tolerant}. Vi delar frekvenser med många andra personer, en del av dem kommer inte ha samma uppfattning som du själv har om saker och ting. Här gäller det att vara tolerant, förstående och framför allt inte bli upprörd över personer som kanske inte beter sig som du önskade att de betedde sig.

Radioamatörer är \emph{aldrig ensamma på banden} helt oavsett om någon svarar på ditt allmänna anrop eller ej så finns det i det närmaste \textbf{garanterat någon som lyssnar}. 

Tänk på vad du säger och att du undviker diskutera ämnen som kan verka \textbf{upprörande} eller \textbf{stötande}. Ämnen som bör undvikas är \textbf{religion} och livs\-å\-skå\-d\-ni\-ng, \textbf{politisk} ideologi, \textbf{ekonomiska} eller \textbf{sociala} frågor m.m. där motparter kan ha starka åsikter som inte nödvändigtvis stämmer med dina egna. Radion är inte ett agitationsrum för sådana frågor.

\textbf{Svordomar}, \textbf{könsord} och liknande undviker vi helt. Språket skall vara vårdat men behöver inte vara strikt. Tänk på att din motpart är inte den enda som lyssnar utan det finns \textit{andra amatörer som lyssnar}, icke-amatörer som lyssnar, myndigheter som lyssnar och så vidare.

Ha \textbf{förståelse} för att andra kanske inte har dina egna detaljkunskaper, professionalism med mera. Agera \textbf{ödmjukt} gentemot andra människor på banden.

Blir du ändå upprörd, undvik att \emph{agera på det} över huvud taget. Sänd inte över annans sändning, s.k. ''gummitumme'', eller stör på annat vis för du är upprörd. Avsluta hellre QSO:t, byt frekvens eller återkom lite senare när du lugnat ned dig. Tänk på att \textit{de flesta konflikter orsakas av okunskap eller brist på förståelse}. \textbf{Agera vuxet} i sådana situationer och jobba för att \textbf{de-eskalera} situationen.

En skicklig amatör \textit{lyssnar mycket innan sändning}. Vi anropar på ett korrekt sätt och avslutar på ett korrekt sätt. Vi försöker uppge våra respektive signaler på ett \emph{tydligt och läsligt sätt}, i dag finns det en tendens att sluddra över signalerna framför allt på 2m och 70cm banden, gör inte det. Tydlighet är en vinning i sig. 

När någon ny i ringen inträder, räkna upp de deltagande signalerna så att personen tydligt får en bild av alla som är med och vem som är på turen före och efter hen.

Vi pratar inte \textbf{nedvärderande} om personer varesig de är andra amatörer eller ej, eller en viss grupp av personer. Vi undviker \textbf{sexuella anspelningar} och vitsar ''\textbf{under bältet}'' liksom allt för \textbf{personliga detaljer}. Amatörradion är främst för \textbf{tekniska diskussioner} av rent \textbf{privat natur} eller av \textbf{allmänt intresse för hobbyn}, tester och prov med mera.

Undvik väldigt \textbf{långa sändningspass}. Ibland händer det saker hos dina motstationer som att de får ett viktigt telefonsamtal eller måste springa ut i köket för katten har rivit ner något, ett barn ramlar eller annat som gör att man måste kvickt lämna radion. Att \textbf{långprata} i sådana lägen gör det svårt att tala om ''QRX --- jag måste ta hand om en sak, anropar dig igen om 5 min.''. Enstaka gånger kanske man behöver förklara något lite längre men gör det till en vana att lämna luckor så ofta som möjligt.

\textbf{Nödtrafik har alltid prioritet} och måste respekteras på alla
frekvenser.

\subsection{Repeatrar}

Repeatrars syfte är främst att förlänga kommunikationen från mobila och portabla amatörsändare. Samtal mellan fasta stationer förekommer men om ni hör varandra på direkten, övergå gärna till en simplex-frekvens i stället för att belägga repeatern.

Lämna luckor mellan er när ni växlar station som sänder. Gör det möjligt för andra att ''breaka-in'' särskilt om ert QSO fortsätter under längre tid. Ta hänsyn till att andra kanske vill använda repeatern för att nå personer som de inte kan nå annars. Hänsyn åt båda hållen förutsätts här. 

Repeatern är en begränsad resurs. Det är inte okay att lägga beslag på den under långa perioder när andra kanske behöver den, var ödmjuk inför att någon driver repeatern och har satt upp den i första hand för att supporta mobila stationer.

Nödtrafik har alltid prioritet.

\subsection{QSO}

Konsten att genomföra ett radiosamtal (QSO) i olika sammanhang. Ofta
blir folk nervösa i början för hur detta går till. Man säger sin
signal och motstationens i fel ordning eller liknande.

Man börjar alltid med motstationens signal. Det bör fallas naturligt
att ropa så och man avslutar anropet med sin egen signal så att
motstationen dels vet vem som anropar men också andra hör. Kanske vill
en annan station ha ett utbyte med dig om du inte får svar från den
tilltänkta.

Ett radiosamtal består som regel av tre delar. Först sker ett anrop, när kontakt etablerats utväxlas ett antal meddelande (dialog) och när man är klarar avslutas samtalet. Dessa tre delar är ganska standard. Man följer detta ganska strikt t.ex. på kortvågen där telefoni oftast innebär SSB. Anledningen är enkel, det går inte höra när någon släpper sändtangenten eller bara är tyst och tänker.

När man kör FM över repeatrar på VHF/UHF är det inte lika vanligt att man både öppnar och avslutar varje sändning med motparten och sin egen signal. Men man skall regelbundet upprepa signalerna och i praktiken är det lämpligt att göra kanske var femte minut eller oftare.

\subsubsection{Anropet}

Ett anrop kan se ut ungefär såhär:

-- SA0MAD från SM0UEI, SA0MAD kom!

Här är det SA0MAD som anropas av SM0UEI. 

Svaret kan se ut ungefär såhär:

-- SM0UEI från SA0MAD kom!

Därefter övergår radiosamtalet i dialog eller meddelandesändning.

\subsubsection{Allmänt anrop} 

Används när man inte ropar på någon särskild motstation utan önskar samtal med vem som helst. På svenska använder man ofta just orden ''allmänt anrop'' medan på engelska är det vanligare att man uttalar CQ (seek you). Ett allmänt androp kan se ut såhär:

-- Allmänt anrop, allmänt anrop, allmänt anrop från SM0UEI SM0UEI SM0UEI kallar allmänt anrop och lyssnar.

Eller på engelska:

-- CQ CQ CQ this is SM0UEI calling CQ CQ CQ and standing by.

\subsubsection{Meddelandesändning}

-- SA0MAD från SM0UEI, tack för svaret. Din signal är 59 hos mig, mitt QTH är JO89WA och namnet är Anders. SA0MAD från SM0UEI kom.

-- SM0UEI från SA0MAD, tack för rapporten. Din signal är 57 hos mig, jag befinner mi i JO89VK men kommer under kvällen byta QTH. Jag kommer då vara QRV på 3663 kHz. QSL? SM0UEI från SA0MAD.

-- SA0MAD från SM0UEI, QSL på det, QRX 19.30 på frekvens 3663 kHz. 

\subsubsection{Avslutning}

-- SA0MAD från SM0UEI, tack för rapport och vi hörs senare, 73, slut kom

-- SM0UEI från SA0MAD, 73 tillbaka, klart slut.

\subsubsection{Pile-up}

Ibland kan det bli väldigt många motstationer samtidigt som ropar. Nu gäller det att spetsa öronen! Först gäller det att sålla. Rara signaler från långtbortistan ger mer poäng i en contest som regel eller från länder du inte kört osv beroende på regler. Försök att sålla med "du som sänder från Florida" eller "VK7 kom igen" osv till det är en station kvar. Kör den snabbt, ropa CQ igen och börja sålla.

Direkt när det uppstår en pile-up är det effektivt att köra split. Dvs du lyssnar 5-10 kHz upp eller ned från den frekvens du sänder på. Det gör det lättare för dig att behålla kommandot under pile-up. Ligger du och sänder i ett frekvensområde som är särskilt ägnat för DX är det smart att lägga Rx-frekvensen strax utanför. Det undviker att man stökar ned i DX-bandet.

Kör du split skall du säga det efter varje sändning. "CQ CQ CQ de Sierra Mike Zero Uniform Echo India listening 5 up" exempelvis. På CW bör en split vara minst 2 kHz och på SSB bör den vara minst 5 kHz ännu hellre 10 kHz. Tänk på att när du startar din split måste du kolla så att båda frekvenserna är ok. Låt inte din pile-up sprida ut sig för mycket även om det är kanske enklare för dig så är risken stor att den stör någon annan. 

Kör korta QSO. Utbyt snabbt den information som behövs och ta sedan nästa. Ha förståelse för att det kan bli krockar i en pile-up. När du hör en partiell signal eller station du vill prata med håll fast vid den. Om du har svårt att läsa den be den repetera tills ni är klara. Genom att du är auktoriteten på frekvensen kommer pile-up:en att lugna ned sig och vänta på sin tur. Om du ''hattar omkring'' är risken att all radiodisciplin far ut genom fönstret.

Använd ett standardmönster när du kör:

-- SM0UEI CQ CQ CQ de SM0UEI 10 UP

-- SM0UEI de ON3XYZ you are 59 sequence 122, QSL?

-- ON3XYZ SM0UEI QSL, 59 back to you, sequence 312 QSL?

-- QSL. CQ CQ CQ de SM0UEI 10 UP ...

Om du försöker nå en motstation med pile-up var uppmärksam på dennes sändningar och vänta på din tur. Tala gärna om signal och var du sänder från men släpp sedan fram andra. Tänk på hur du själv skulle vilja att en pile-up på din egen station skulle vilja agera. Den gyllene regeln är också alltid lyssna först --- sänd sedan!


\newpage

\section{Teknik}

\subsection{Effekt i dBW och dBm}

Effekter anges i W eller i decibel relaterat till 1 mW (dBm) eller relaterat 1W (dBW). Tabell över effekt och decibelwatt nedan:
\begin{table}[h]
\centering
\begin{tabular}{rrr|rrr|rrr}
	\textbf{Effekt} & \textbf{dBW} & \textbf{dBm} & \textbf{Effekt} & \textbf{dBW} & \textbf{dBm} & \textbf{Effekt} & \textbf{dBW} & \textbf{dBm} \\ \hline
	    1 \textmu W &          -60 &          -30 &             1 W &            0 &           30 &           100 W &           20 &           50 \\
	   10 \textmu W &          -50 &          -20 &             3 W &            5 &           35 &           250 W &           24 &           54 \\
	  100 \textmu W &          -40 &          -10 &             5 W &            7 &           37 &           500 W &           27 &           57 \\
	           1 mW &          -30 &            0 &            10 W &           10 &           40 &            1 kW &           30 &           60 \\
	          10 mW &          -20 &           10 &            20 W &           13 &           43 &          1.5 kW &           32 &           62 \\
	         100 mW &          -10 &           20 &            50 W &           17 &           47 &          2.0 kW &           33 &           63
\end{tabular}
\caption{Tabell över effekt och decibelskalor}
\end{table}

\subsection{S-värden, signalvärde, S-meter}

Signalstyrkan i amatörradio uttrycks oftast som S-värden. Dessa fås i regel genom nivån på AGC hos mottagaren. Därför ser man sälla utslag vid riktigt låga signaler.

Standard kalibrering för S-metern är enligt skalan i tabellen \ref{tab:s-varden}

\begin{table}[h]
\centering
\begin{tabular}{r|rr|rr||r|rr|rr}
      & \multicolumn{2}{c|}{\textbf{$<$ 30 MHz}} &
  \multicolumn{2}{c}{\textbf{$>$ 30 MHz}}       && \multicolumn{2}{c|}{\textbf{$<$ 30 MHz}} &
  \multicolumn{2}{c}{\textbf{$>$ 30 MHz}}\\ \textbf{S} & \textbf{dBm}
  & \textbf{\textmu V} & \textbf{dBm} & \textbf{\textmu V}&   \textbf{S} & \textbf{dBm}
  & \textbf{\textmu V} & \textbf{dBm} & \textbf{\textmu V} \\\hline
          
	   1 & -121 & 0.21  & -141 & 0.02 & 9+10 & -63 & 160  & -83 & 16  \\
	   2 & -115 & 0.40  & -135 & 0.04 & 9+20 & -53 & 500  & -73 & 50  \\
	   3 & -109 & 0.80  & -129 & 0.08 & 9+30 & -43 & 1600 & -63 & 160 \\
	   4 & -103 & 1.60  & -123 & 0.16 & 9+40 & -33 & 5000 & -53 & 500 \\
	   5 & -97  & 3.20  & -117 & 0.32 &      &     &      &     &     \\
	   6 & -91  & 6.30  & -111 & 0.63 &      &     &      &     &     \\
	   7 & -85  & 12.60 & -105 & 1.26 &      &     &      &     &     \\
	   8 & -79  & 25.00 & -99  & 2.50 &      &     &      &     &     \\
	   9 & -73  & 50.00 & -93  & 5.00 &      &     &      &     &     \\
\end{tabular}
\caption{Tabell över S-värden, effekt och spänning}
\label{tab:s-varden}
\end{table}

\subsection{Modulationer}

\subsubsection{Bandbredd olika modulationer}

Olika modulationer upptar olika bandbredd. Detta är mycket viktigt att förstå när man ställer in sin radiostation. Detta gäller särskilt att beakta i närheten av nödfrekvenser eller bandkanten. När vi talar om bandbredder här förstås den bandbredd vari minst 98\% av signalens effekt befinner sig.

\begin{table}[H]
\centering
\begin{tabular}{lrl}
	\textbf{Modulation} & \textbf{Bandbredd} & \textbf{Kommentarer}                  \\ \hline
	CW                  &          $<$500 Hz & Smalbandigt                           \\
	AM                  &              6 kHz & Amplitudmodulering med fullt sidband  \\
	SSB*                &              <3 kHz & Amplitudmodulering med enkelt sidband \\
	NFM                 &           7-12 kHz & Smalbandig FM                         \\
	FM                  &             16 kHz & Normal FM                             \\
	WFM                 &          16-75 kHz & Bredbandig FM (t.ex. rundradio)
\end{tabular}
\caption{Normal bandbredd vid olika modulationsslag}
\end{table}

*) För SSB gäller att USB och LSB fungerar lite olika. När man beräknar den högsta eller lägsta frekvensen utgår man från den inställda frekvensen $f$. För USB gäller då att högsta frekvensen är $f+3$\,kHz. För LSB blir det $f-3$\,kHz. Detta innebär att om du sänder på 80\,m-bandet och du får sända telefoni från 3600--3800\,kHz och vill lägga dig i undre bandkanten och köra LSB skall du ställa in din radio på 3603\,kHz som lägsta frekvens. Använd gärna lite marginal och kör exempelvis 3605\,kHz i stället.

Den egentliga modulationsfrekvensen är dock lite mer komplicerad. Normalt anges den verkliga modulationsfrekvensen som ca 2,7\,kHz och det beror på att man i regel filtrerar bort ljudet under 300\,Hz och det över 3000\,Hz. Detta innebär att det akustiska frekvensomfånget blir 300--3000\,Hz och därmed upptar signalen inte mer än 2,7\,kHz.

Det är vanligt att man märker stationer som kör överdriven bandbredd. Antingen som en följd av att man vill öka sin modulationsvinst, okunskap eller man har skruvat i sin radio. Syftet kan var att få bättre genomslag vid långväga förbindelser.

\subsubsection{Telegrafi, CW}

CW står för continuous waves och innebär en rent omodulerad bärvåg. I mottagaren används en oscillator för att återskapa hörbar signal. Detta används för telegrafi och modulationsslaget är oftast A1A. Ibland sänds telegrafi som modulerad AM-bärvåg också som då moduleras med t.ex. 700\,Hz ton. Det är dock mindre vanligt.

Bandbredden för CW är i teorin mycket smal. I praktiken blir den lite beroende på frekvens från några Hz till något hundratal Hz beroende på frekvensband och sändarens beskaffenhet i form av jitter och frekvensstabilitet.

Bandbredden hos CW består av fasbruset vilket normalt är så undertryckt att det egentligen inte betyder så mycker samt stig- respektive falltiden när man nycklar eller släpper nyckeln. Sker detta mjukt är bandbredden låg, har man skarp in- eller urkoppling av bärvågen nyttjar man mer bandbredd.

\subsubsection{Amplitudmodulering, AM}

Amplitudmodulering finns i flera olika varianter. Vanlig AM består av en bärvåg vars styrka varieras i takt med signalen som skall sändas. Denna förändring av bärvågen producerar sidband och det är i dessa som den egentliga informationen återfinns. Bärvågen i sig får dock lejonparten av signalen varför det är en sändningsklass som nästan aldrig används inom amatörradiobanden.

Bandbredden hos AM-modulerad signal kan beräknas genom att man tar två gånger högsta modulationsfrekvensen. Detta ger t.ex. vid en modulationsfrekvens som går från 300-3000\,Hz en bandbredd som varierar med talet från upp till 6\,kHz.

$$B=2f_m$$

Där $f_m$ är högsta modulationsfrekvensen.

\subsubsection{SSB/ESB -- Enkelt sidband, en AM-variant}

Enkelt sidband används av radioamatörer för att minska på bandbredden samt lägga radioenergin där den behövs mest. Eftersom båda sidbanden innehåller samma information kan man filtrera bort dessa samt bärvågen innan man matar sändarens förstärkarsteg med resultatet. I mottagaren behöver man dock återskapa en referenssignal, en så kallad beat-oscillator gör detta. När man ställer in frekvensen så försöker man därmed matcha den ursprungliga frekvensen. Ligger man för långt från låter det kalle anka, kommer man för nära sidbandet låter det dovt och basigt. 

Enkelt sidband förkortas ESB eller SSB (single side-band) och man kan välja vilket sidband man vill använda sig av. På amatörradiofrekvenser under 10 MHz använder man LSB (lägre/lower sidbandet) och på frekvenser över 10 MHz används USB/ÖSB (upper/övre sidbandet). 

Detta är mycket av tradition. Använder man fel sorts sidband hörs det inget vettigt när man försöker lyssna. Språkrytmerna gör dock att vi uppfattar det som att mänskligt tal förekommer. I dag händer det att amatörer bryter mot regeln och sänder med ``fel'' sidband på fel frekvens.

Bandbredden hos SSB är halva den för normal AM egentligen. Den kan därmed beräknas som för AM och halveras.

$$B=f_m$$

Där $f_m$ är högsta modulationsfrekvensen.

\subsubsection{Frekvensmodulering, FM}

Frekvensmodulering består av att man tar en bärvåg och modulerar den med talet genom att skifta dess frekvens. Om skiftet i frekvens är mycket litet kallas moduleringen för fasmodulation. FM-modulering indelas i lite olika klasser beroende på hur stor deviation som används. På amatörradions VHF- och UHF-band talar vi om FM och NFM (Narrow FM, andra namn förekommer också). Ibland talar man om bred FM, normal FM och smal FM på svenska.

Normal FM innebär att deviationen (hur mycket signalen avviker från grundfrekvensen) är lika stor som den högsta modulationsfrekvensen. Det är vanligt att kommunikationsradio använder sig av 3 kHz som högsta modulationsfrekvens och 5 kHz deviation. Deviationen är då något bredare och ger upphov till en viss modulationsvinst. När man talar om FM-radio på UKV-bandet för rundradio så är deviationen ca 75\,kHz och högsta modulationsfrekvens ca 16\,kHz. Där är alltså svinget betydligt bredare än modulationen och detta är bred FM.

Nu för tiden förordas en minskning av bandbredden för FM-sändningar på amatörbanden, främst är det väl VHF och UHF där FM-sändning är vanligast förekommande och där vill man ha en kanalindelning om 12,5\,kHz i stället för som tidigare 25\,kHz. Om man studerar bandbredden hos olika FM-signaler kan man använda sig av Carsons bandbreddsbegrepp:

$$B=2(f_M+f_D)$$

Där $B$ är bandbredden $f_M$ högsta modulationsfrekvensen och $f_d$ är FM-signalens maximala deviation (även kallat sving). Carsons bandbreddsbegrepp säger att 98\% av energin förekommer inom den stipulerade bandbredden. Det betyder att att grannkanalen kan få ungefär 17\,dB lägre signal under sändning vilket fortfarande inte är enormt bra. Carson var för övrigt den som faktiskt uppfan SSB-modulationen.

\begin{center}
\begin{tabular}{rrrr}
Deviation & Modulation & Bandbredd & Kanaldelning\\ \hline
5 kHz & 3 kHz & 16 kHz & 25 kHz\\
2.5 kHz & 3 kHz & 11 kHz & 12.5 kHz\\
\end{tabular}
\end{center}

\subsection{Termiska brusgolvet}
När man lyssnar i radion på en frekvens där ingen nyttosignal finns hörs ett brus. Detta brus består av olika komponenter men en av de viktigaste är det termiska brusgolvet. Detta sätter en nedre gräns för hur svaga signaler en mottagare kan uppfatta.

Mottagaren har i sig också ett termiskt brus, detta beskrivs vanligen med något som kallas \textit{brusfaktor} och säger hur mycket över det termiska brusgolvet mottagaren bidrar med eget brus. 

Bruset är avhängigt temperaturen som mottagarantennen ''ser'' och vanligtvis inomhus använder man närmevärdet 300\,K när man räknar på detta vilket motsvarar 27\,\textdegree C. När man riktar antennerna mot rymden eller på vintern kan man räkna med en lägre brusfaktor pga den lägre antenntemperaturen.

Brusgolvet kan beräknas med hjälp av Boltzmanns konstant och temperaturen i Kelvin. Detta ger oss formeln:

$$P=k_BT\Delta f$$

Där:

\begin{tabular}{lll}
	$k_B$      & Boltzmanns konstant, $1,38065\cdot 10^{-23}$ & [J/K] \\
	$T$        & Temperaturen                                 & [K]   \\
	$\Delta f$ & Bandbredden i mottagaren                     & [Hz]
\end{tabular}

Om vi vet detta kan vi beräkna det termiska brusgolvet:

$$P = 1,38065\cdot 10^{-23} \cdot 300 \cdot 1 = 4.1495\cdot 10^{-21}$$

Om vi räknar om detta i dBm får vi i stället -173,8\,dBm. Detta avrundas normalt till -174\,dBm och är brusgolvet för 1\,Hz. En mottagare som har en mottagarbandbredd på 25\,kHz kommer därmed att se ett brus som är 20\,000 ggr större. I decibel får vi då $-174 + 44 dB = -130$\,dBm.

För att en signal skall kunna detekteras får vi lägga på mottagarens brusgolv, kanske 3\,dB samt hur mycket signal till brus i förhållande vi behöver, för FM ca 12\,dB. När vi gjort detta får vi mottagarens känslighet när den är helt ostörd som bör ligga runt $-130 + 3 + 12 = -115$\,dBm.

\begin{table}[H]
\centering
\begin{tabular}{rr|rr|rr}
	\textbf{RBW} & \textbf{N$_0$} & \textbf{RBW} & \textbf{N$_0$} & \textbf{RBW} & \textbf{N$_0$} \\ \hline
	         0.5 &           -141 &         6.25 &           -136 &          100 &           -124 \\
	         1.0 &           -144 &        12.50 &           -133 &          200 &           -121 \\
	         3.0 &           -139 &        25.00 &           -130 &         5000 &           -107 \\
	         5.0 &           -137 &        50.00 &           -127 &        10\,000 &           -104
\end{tabular}
\caption{Termiska brusgolvet vid några vanliga bandbredder}
\end{table}

Tabellen ovan visar hur brusgolvet ser ut för olika mottagarbandbretter. RBW är Receive Band Width i kHz. $N_0$ är beteckningen för det termiska brusgolvet. Som ni ser dubblas bruseffekten om man dubblar bandbredden vilket kanske inte är så märkligt. Det gör att smalbandig kommunikation har ett bättre läge pga lägre bruseffekt i mottagaren.


\subsection{Return loss och VSWR}

Return loss och VSWR anger samma sak. VSWR är vanligare inom amatörradio medan man i profesionella sammanhang föredrar att prata om return loss. RL är storleken på den reflekterade signalen i förhållande till den framåtgående signalen. Return loss mäts alltså i dB enligt formeln $10\log(P_F/P_R)$ där $P_F$ är den framåtgående effekten (forward) och  $P_R$ är den reflekterade signalen i retur.

\begin{longtable}{rrr|rrr|rrr}
	\caption{VSWR och return loss}\\
		\textbf{RL} & \textbf{VSWR} & \textbf{\%} & \textbf{RL} & \textbf{VSWR} & \textbf{\%} & \textbf{RL} & \textbf{VSWR} & \textbf{\%} \\ \hline 
	\endfirsthead
	\textbf{RL} & \textbf{VSWR} & \textbf{\%} & \textbf{RL} & \textbf{VSWR} & \textbf{\%} & \textbf{RL} & \textbf{VSWR} & \textbf{\%} \\ \hline 	\endhead
	          1 &         17,39 &       79,43 &           8 &          2,32 &       15,85 &          20 &          1,22 &        1,00 \\
	          2 &          8,72 &       63,10 &          10 &          1,92 &       10,00 &          22 &          1,17 &        0,63 \\
	          3 &          5,85 &       50,12 &          12 &          1,67 &        6,31 &          24 &          1,13 &        0,40 \\
	          4 &          4,42 &       39,81 &          14 &          1,50 &        3,98 &          25 &          1,12 &        0,32 \\
	          5 &          3,57 &       31,62 &          15 &          1,43 &        3,16 &          26 &          1,11 &        0,25 \\
	          6 &          3,01 &       25,12 &          16 &          1,38 &        2,51 &          28 &          1,08 &        0,16 \\
	          7 &          2,61 &       19,95 &          18 &          1,29 &        1,58 &          30 &          1,07 &        0,10
\end{longtable}

Acceptabelt RL är ungefär från 12\,dB, riktigt bra från 20 dB och de bästa komponenterna ligger runt 30\,dB. Många antenntuners som går med automatik startar avstämningen först när VSWR är 1:2 eller sämre som motsvarar ca 10\,dB\,RL.

\subsection{CTCSS subtoner}

Inom amatörradio används ofta pilottoner (subtoner) som CTCSS\footnote{Contnuous Tone-Conded Squelch System} för repeatrar och liknande. På PMR446 används subtoner för att skapa virtuella grupper och sub-kanaler. De som används är följande toner och frekvenser:

\begin{table}[H]
\centering
\begin{tabular}{rr|rr|rr|rr|rr}
	 1 &  67,0 &  2 &  69,3 &  3 &  74,4 &  4 &  77,0 &  5 &  79,7 \\ \hline
	 6 &  82,5 &  7 &  85,4 &  8 &  88,5 &  9 &  91,5 & 10 &  94,8 \\ \hline
	11 &  97,4 & 12 & 100,0 & 13 & 103,5 & 14 & 107,2 & 15 & 110,9 \\ \hline
	16 & 114,8 & 17 & 118,8 & 18 & 123,0 & 19 & 127,3 & 20 & 131,8 \\ \hline
	21 & 136,5 & 22 & 141,3 & 23 & 146,2 & 24 & 151,4 & 25 & 156,7 \\ \hline
	26 & 162,2 & 27 & 167,9 & 28 & 173,8 & 29 & 179,9 & 30 & 186,2 \\ \hline
	31 & 192,8 & 32 & 203,5 & 33 & 210,7 & 34 & 218,1 & 35 & 225,7 \\ \hline
	36 & 233,6 & 37 & 241,8 & 38 & 250,3 &    &       &    &
\end{tabular}
\caption{CTCSS-toner, nummer och frekvens}
\end{table}

\subsection{CTCSS-zoner i Sverige}

Rekommendationer för repeatrar i olika distrikt och län att använda CTCSS för att hindra att störningar uppkommer vid conds mm. Det ger också möjligheten för sändaramatörer att öppna just den repeater man önskar om man har flera på samma frekvens omkring sig.

Generellt för dessa är att sista siffran i CTCSS-frekvensen är samma som distriktsiffran.

\begin{table}[H]
\centering
\begin{tabular}{lcccc}
	\textbf{Område}    & \textbf{Primär} & \textbf{Sek. 1} & \textbf{Sek. 2} & \textbf{Sek. 3} \\ \hline
	Distrikt 0         & 77,0            & 123.0           & 67.0            & 100.0           \\
	Distrikt 1         & 218.1           & 233.6           &                 &                 \\
	Distrikt 2         & 107.2           & 146.2           & 162.2           & 186.2           \\
	Distrikt 3         & 127.3           & 141.3           & 250.3           &                 \\
	D4 Värml. / Örebro & 74.4            & 151.4           &                 &                 \\
	D4 Dalarna         & 85.4            & 151.4           &                 &                 \\
	Distrikt 5         & 82.5            & 91.5            & 103.5           & 203.5           \\
	Distrikt 6         & 114.8           & 118,8           & 94.8            & 131.8           \\
	Distrikt 7         & 79.7            & 156.7           & 210.7           & 
\end{tabular}
\caption{Distrikt och CTCSS-toner}
\end{table}

\newpage

\section{Övergripande frekvensplan}

\subsection{Indelning efter frekvens och våglängd}
\label{frekvens-vaglangd}
\begin{table}[H]
\centering
\begin{tabular}{llrlrl}
	\textbf{Förk.} & \textbf{Benämning}   & \textbf{Frekvens} &     & \textbf{Våglängd} &  \\ \hline
	ELF            & Extremt låg frekvens &             3--30 & Hz  &           10--100 & Mm \\
	SLF            & Superlåg frekvens    &           30--300 & Hz  &             1--10 & Mm \\
	ULF            & Ultralåg frekvens    &         300--3000 & Hz  &         100--1000 & km \\
	VLF            & Väldigt låg frekvens &             3--30 & kHz &           10--100 & km \\
	LF (LV)        & Låg frekvens         &           30--300 & kHz &             1--10 & km \\
	MF (MV)        & Mellanfrekvens       &         300--3000 & kHz &         100--1000 & m  \\
	HF (KV)        & Högfrekvens          &             3--30 & MHz &             10--100 & m  \\
	VHF (UKV)      & Väldigt hög frekvens &           30--300 & MHz &             1--10 & m  \\
	UHF            & Ultrahög frekvens    &         300--3000 & MHz &         100--1000 & mm \\
	SHF            & Superhög frekvens    &             3--30 & GHz &           10--100 & mm \\
	EHF            & Extremt hög frekvens &           30--300 & GHz &             1--10 & mm \\
	THF            & Terahertsfrekvens    &          300-3000 & GHz &         100--1000 & µm
\end{tabular}
\caption{Frekvens och våglängd övergripande}
\end{table}

Benämningarna HF, MF och LF har också andra betydelser. Exempelvis an\-vänd\-s HF som beteckning av den signal en antenn tar mot eller sänder oavsett frekvensband, MF kan vara mellansignalen oavsett frekvens efter omvandling i en superheterodynmottagare och LF, ibland benämnt AF (audiofrekvens) är det hörbara ljudet, dvs den modulation som används på signalen.

På engelska används i stället benämningarna RF för radio frequency, IF för intermediate frequency and AF för audio frequency vilket rekommenderas då sammanblandningsrisk med ITU-benämningarna på spektrum inte föreligger.

Amatörradioband finns inom de flesta av dessa frekvensband utom de högsta och lägsta frekvenserna.

\subsection{Rundradiobenämningar och frekvensband}

\begin{table}[H]
\centering
\begin{tabular}{llrl}
\textbf{Förk.} & \textbf{Namn} & \textbf{Frekvens Rundradio} &     \\ \hline
LW/LV          & Långvåg       & 148,5--285                  & kHz \\
MW/MV          & Mellanvåg     & 526,5--1606,5               & kHz \\
SW/KV          & Kortvåg       & 4,3--30                     & MHz \\
UKV            & Ultrakortvåg  & 88--108                     & MHz \\
\end{tabular}
\caption{Rundradiobanden}
\end{table}

\subsection{Radarband och benämningar enligt ITU}

\begin{table}[H]
\centering

\begin{tabular}{lrl}
	\textbf{Band} & \textbf{Frekvens} & \textbf{Benämning}          \\ \hline
	HF            &   0.003--0.03 GHz & High frequency              \\
	VHF           &     0.03--0.3 GHz & Very hgh frequency          \\
	UHF           &        0.3--1 GHz & Ultra high frequency        \\
	L             &          1--2 GHz & Long wave                   \\
	S             &          2--4 GHz & Short wave                  \\
	C             &          4--8 GHz &  \\
	X             &         8--12 GHz & Anv. under 2:a världskriget \\
	Ku            &        12--18 GHz & ''Kurz under''              \\
	K             &        18--27 GHz & Tyska ''Kurz'' (kortvåg)    \\
	Ka            &        27--40 GHz & Kurz-above (över)           \\
	V             &        40--75 GHz &  \\
	W             &       75--110 GHz &  \\
	mm            &      110--300 GHz & Millimetervågor
\end{tabular}
\caption{ITU-benämningar på radarband mm}
\end{table}

\subsection{IEEE-benämningar på frekvensband}

\subsubsection{Äldre benämmomgar}

\begin{table}[H]
	\centering
	\begin{tabular}{lr}
		\textbf{Band} & \textbf{Frekvens} \\ \hline
		A             &        0--250 MHz \\
		B             &      250--500 MHz \\
		C             &     100--1000 MHz \\
		D             &          1--2 GHz \\
		E             &          2--3 GHz \\
		F             &          3--4 GHz \\
		G             &          4--6 GHz \\
		H             &          6--8 GHz \\
		I             &         8--10 GHz \\
		J             &        10--20 GHz \\
		K             &        20--45 GHz \\
		L             &        40--60 GHz \\
		M             &       60--100 GHz \\
		N             &      100--200 GHz \\
		O             &      100--200 GJz
	\end{tabular}
	\caption{Äldre IEEE-benämningar på frekvensband}
\end{table}

\subsubsection{Nuvarande benämningar}

\begin{table}[H]
	\centering
	\begin{tabular}{lr}
		\textbf{Band} & \textbf{Frekvens} \\ \hline
		I             &      100--150 MHz \\
		G             &      250--500 MHz \\
		P             &      225--390 MHz \\
		L             &     390--1550 MHz \\
		S             &    1550--3900 MHz \\
		C             &    3900--6200 MHz \\
		X             &   6200--10900 MHz \\
		Ku            &    10,9--20,0 GHz \\
		Ka            &        20--36 GHz \\
		Q             &        36--46 GHz \\
		V             &         46-56 GHz \\
		W             &       56--100 GHz
	\end{tabular}
	\caption{Nyare IEEE-benämningar på frekvensband}
\end{table}

\subsubsection{Rundradioband}

\begin{table}[H]
	\centering
	\begin{tabular}{lrl}
		\textbf{Band} & \textbf{Frekvens} &                                                 \\ \hline
		Band I        &        47--68 MHz & Tidigare TV-band också ''6-metersbandet''       \\
		Band II       &   87,5--108,0 MHz & Nuvarande FM-rundradionbandet                   \\
		Band III      &      174--230 MHz & Används i dag för DAB                           \\
		Band IV       &      470--582 MHz & I dag refarmerat till främst landmobil komradio \\
		Band V        &      582--862 MHz & Marksänd TV
	\end{tabular}
	\caption{IEEE-benämningar på rundradioband}
\end{table}


\subsection{Egenskaper olika frekvensband}

För radioamatörer delar man in frekvensbanden i långvåg, mellanvåg, kortvåg, VHF, UHF och SHF beroende på frekvens, se tabellen under avsnitt \ref{frekvens-vaglangd}. Dessa har lite olika utbredningsegenskaper.

\subsubsection{Långvåg}

Markvågsutbredning, relativt höga sändareffekter, tillförlitliga för\-bind\-el\-ser men i övre delen av frekvensbandet kortare förbindelser dagtid. På de lägsta frekvenserna erhålls med hög sändareffekt goda förbindelser på stora avstånd globalt och används även för t.ex. malmprospektering, kommunikation med ubåtar i undervattensläge.

\subsubsection{Kortvåg}

God rymdvågsutbredning med mycket lång räckvidd redan med låg effekt men samtidigt starkt avhängigt radiokonditionerna. Med ökande frekvens blir jonosfärreflektionen allt flackare vilket resulterar i en alltmer uttalad död zon (skip). Särskilt utmärkande för kortvågen är att den redan med låg effekt ger under gynnsamma konditioner extremt lång räckvidd via rymdvåg, ibland globalt.

\subsubsection{Mellanvåg}

Kombinerar egenskaperna hos angränsande delar av lång- och kortvåg, kan ge kraftig interferens mellan rymd- och markvåg som ofta upplevs som kraftig fädning. Särskilt utmärkande för mellanvågen är den i det närmaste avsaknanden av skipzon eftersom mark- och rymdvåg kompletterar varandra, jonosfärens D-skikt är heller inte särskilt uttalat i frekvensområdet dvs förbindelser via rymdvåg på korta avstånd mellan 100--300\,km är möjliga även dagtid under perioder med kraftig solaktivitet.

\subsubsection{Ultrakortvåg, väldigt hög frekvens (VHF)}

Förbindeleser med låg effekt och små antenner, oberoende av jonosfären men då endast i form av frirumsutbredning, dvs fram till horisonten och under påverkan av terränghinder mm. Särskilt utmärkande för UKV är att rymdvåg saknas, markvågsdämpningen till lands är total och kommunikation på högre frekvenser i princip därför bara sker vid fri sikt mellan sändare och mottagare.

Eftersom signalerna kan passera jonosfären fungerar det att kommunicera med satelliter och rymdstationer på frekvenserna över ca 100\,MHz. Amatörradiobandet på 144--146\,MHz (2-metersbandet) har en avdelning frekvenser vikta för rymdkommunikaton.

\subsubsection{Ultrahög frekvens (UHF)}

Bandet är ett av de mest populära för landmobil radio. I dag rymms i detta band mellan 300--3000\,MHz näranog all landmobil professionell kommunikation då mycket har lämnat VHF-bandet till förmån för 400\,MHz-bandet. I skrivande stund ryms även alla mobiltelefoniband inom UHF, det gäller 700, 800, 900, 1800, 2100 och 2600. I framtiden kan det komma frekvensband som ligger högre, exempelvis 3,5\,GHz eller 5\,GHz.

Bandet är utmärkt för kommunikation mellan fordon, särskilt i urban miljö fungerar den kortare våglängden bra. Den reflekteras också bättre mellan husen och frekvenser på främst 400\,MHz har en fantastisk förmåga att leta sig in fast man befinner sig i radioskugga. Tillsammans med 6-meter, 4-meter och 2-meter i glesbygden fyller 70\,cm-bandet en nisch för landmobil kommunikation som få andra frekvenser fungerar.

Radioamatörer har också ett av sina större tilldelningar i detta band, mellan 432--438\,MHz. Här finns också en LPD\footnote{Low Power Devices, små radiostyrningsutrustningar, t.ex. väderstationer, garageportsöppnare med mera.}-del som vi får samsas med. Främst kör man FM på bandet men det förekommer SSB och kanske ibland också CW. Det finns ett stort repeaterband med 2\,MHz shift mellan mottagare och sändare och det är ett praktiskt band för man kan enkelt bygga antenner med ganska rejält med förstärkning.

Det är också ett populärt band för satellitkommunikation.

	\section{Frekvenser VHF--UHF}

\subsection{Frekvenser ej amatörradio}

Dessa frekvenser är avsedda för allmänhet eller för specifika ända\-mål som anges. Det innebär att de kan brukas för de ändamål som anges i PTS för\-fatt\-nings\-sam\-ling\-ar och sammanställning över ej tillståndspliktiga frekvenser. Observera att du är skyldig att själv kontrollera bestämmelserna
innan en frekvens brukas.

Effekten i tabellen är ustrålad effekt PEP om inte annat anges.

\subsubsection{Jakt och jordbruksfrekvenser 155 MHz}
\begin{longtable}{rlrl}
\textbf{Frekvens} & \textbf{Benämning} & \textbf{Effekt} & \textbf{Användningsområde}   \\ \hline \endhead
155,425           & Jakt K1            & 5 W             & Jakt, Jordbruk               \\
155,475           & Jakt K2            & 5 W             & Jakt, Jordbruk               \\
155,500           & Jakt K3            & 5 W             & Jakt, Jordbruk$^M$           \\
155,525           & Jakt K4            & 5 W             & Jakt, Jordbruk$^M$           \\
156,000           & Jakt K5            & 5 W             & Generell landmobil radio$^P$ \\
155,400           & Jakt K6            & 5 W             & Jakt, Jordbruk               \\
155,450           & Jakt K7            & 5 W             & Jakt, Jordbruk
\end{longtable}

\footnotesize
\begin{itemize}
	\item[$^M$] Delas med marina VHF-bandet, kanalerna L1 och L2 för fritidsbåtar.
	\item[$^P$] PMR-kanal som kan användas till vad som helst så länge det är landmobil radio och effekten ej överskrids.
\end{itemize}
\normalsize

\subsubsection{Öppna PMR-bandet på 446 MHz}
\begin{longtable}{rlrl}
\textbf{Frekvens} & \textbf{Benämning} & \textbf{Effekt} & \textbf{Användningsområde} \\ \hline \endhead
446,00625         & PMR446 K1          & 500 mW          & PMR$^N$                    \\
446,01875         & PMR446 K2          & 500 mW          & PMR$^N$                    \\
446,03125         & PMR446 K3          & 500 mW          & PMR$^N$                    \\
446,04375         & PMR446 K4          & 500 mW          & PMR$^N$                    \\
446,05625         & PMR446 K5          & 500 mW          & PMR$^N$                    \\
446,06875         & PMR446 K6          & 500 mW          & PMR$^N$                    \\
446,08125         & PMR446 K7          & 500 mW          & PMR$^N$                    \\
446,09375         & PMR446 K8          & 500 mW          & PMR$^N$
\end{longtable}

\footnotesize
\begin{itemize}
	\item[$^N$] Smalbandig FM-modulation skall användas pga tätt liggande kanaler.
\end{itemize}
\normalsize

\subsubsection{Kortdistansradio (KDR)}

Kallas även SRBR för Short Range Business Radio.

\begin{longtable}{rlrl}
	\textbf{Frekvens} & \textbf{Benämning} & \textbf{Effekt} & \textbf{Användningsområde} \\ \hline \endhead
	          444,600 & SRBR K1            & 2 W             & Short range business radio \\
	          444,625 & SRBR K2            & 2 W             & Short range business radio \\
	          444,800 & SRBR K3            & 2 W             & Short range business radio \\
	          444,825 & SRBR K4            & 2 W             & Short range business radio \\
	          444,850 & SRBR K5            & 2 W             & Short range business radio \\
	          444,875 & SRBR K6            & 2 W             & Short range business radio \\
	          444,925 & SRBR K7            & 2 W             & Short range business radio \\
	          444,975 & SRBR K8            & 2 W             & Short range business radio
\end{longtable}

SRBR är ett ej tillståndspliktigt frekvenssegment som används för yrkesmässig radiotrafik.

Rekommendationen är att man skall använda CTCSS eller motsvarande för att undvika störa och bli störd av andra stationer som delar frekvenserna.


\subsection{Frekvenser amatörradio VHF--UHF}

I denna skrift försöker vi omfatta alla VHF och UHF-band vilket inkluderar 6m-bandet, 2m-bandet, 70cm-bandet och 23cm-bandet.

\subsubsection{Kanalnumrering VHF/UHF}

Denna typ av kanalnumrering är överenskommen inom IARU region 1 för 6m, 2m och 70cm banden på amatörradiofrekvenser. Kanalnumreringen består av ett prefix som anger vilket band och här används F--6m, V--2m, U--70cm. Därefter används 2 siffror på 6m och 2m banden och tre siffror på 70cm bandet för
att ange kanal.

Repeaterfrekvenser anges med tillägget R före kanalnumret och innebär då normalt duplex med det skift som normalt används för bandet. Vid repeatrar är det repeaterns utfrekvens som anges, dvs den som mobilstationen lyssnar på. Exempel: RV48.

\begin{tabular}{crrll}
	\textbf{Band} & \textbf{Startfrekvens} & \textbf{Kanalraster} & \textbf{Första kanal} & \textbf{Beräknas}    \\ \hline
	     6 m      & 51.000 MHz             & 10.0 kHz             & F00                   & $f=51+k\cdot0.01$    \\
                      &                        &                      &                       & $k=(f-51)/0,01$      \\ \hline
	     2 m      & 145.000 MHz            & 12.5 kHz             & V00                   & $f=145+k\cdot0.0125$ \\
                      &                        &                      &                       & $k=(f-145)/0,0125$   \\ \hline
	    70 cm     & 430.000 MHz            & 12.5 kHz             & U000                  & $f=430+k\cdot0.0125$ \\
                      &                        &                      &                       & $k=(f-430)/0,0125$   \\ \hline
\end{tabular}

Eftersom amatörradiobanden ser lite olika ut i olika länder förekommer det kanaler i numreringen som inte är tillåtna på vissa ställen. Det är därför viktig att kontrollera att man fortfarande följer bandplanerna i den region man är.

\begin{itemize}
\item I 6 m bandet finns inga FM-kanaler definierade under 51 MHz.
\item För 2m-bandet är FM-kanaler endast definierade från 145 MHz och uppåt.
\item I 70 cm-bandet är inga kanaler definierade i intervallet 432.000--433.000 MHz. Observera att startfrekvensen är utanför 70cm bandplanen i IARU region 1.
\end{itemize}

OBS!\\ Information om kanalnumreringen för 23cm-bandet tas tacksamt mot. Maila mig på anders@sikvall.se om du har korrekt information.

\clearpage
\subsubsection{FM-kanaler 6m-bandet}

\begin{longtable}{rrl|rrl}
\textbf{Kanal} & \textbf{Tidigare} & \textbf{Anm}   
&  \textbf{Kanal} & \textbf{Tidigare} & \textbf{Anm} \\ \hline
	51,500 &      F50 &       & 51,750 &      F75 &  \\
	51,510 &      F51 & Anrop & 51,760 &      F76 &  \\
	51,520 &      F52 &       & 51,770 &      F77 &  \\
	51,530 &      F53 &       & 51,780 &      F78 &  \\
	51,540 &      F54 &       & 51,790 &      F79 &  \\
	51,550 &      F55 &       & 51,800 &      F80 &  \\
	51,560 &      F56 &       & 51,810 &     RF81 &  \\
	51,570 &      F57 &       & 51,820 &     RF82 &  \\
	51,580 &      F58 &       & 51,830 &     RF83 &  \\
	51,590 &      F59 &       & 51,840 &     RF84 &  \\
	51,600 &      F60 &       & 51,850 &     RF85 &  \\
	51,610 &      F61 &       & 51,860 &     RF86 &  \\
	51,620 &      F62 &       & 51,870 &     RF87 &  \\
	51,630 &      F63 &       & 51,880 &     RF88 &  \\
	51,640 &      F64 &       & 51,890 &     RF89 &  \\
	51,650 &      F65 &       & 51,900 &     RF90 &  \\
	51,660 &      F66 &       & 51,910 &     RF91 &  \\
	51,670 &      F67 &       & 51,920 &     RF92 &  \\
	51,680 &      F68 &       & 51,930 &     RF93 &  \\
	51,690 &      F69 &       & 51,940 &     RF94 &  \\
	51,700 &      F70 &       & 51,950 &     RF95 &  \\
	51,710 &      F71 &       & 51,960 &     RF96 &  \\
	51,720 &      F72 &       & 51,970 &     RF97 &  \\
	51,730 &      F73 &       & 51,980 &     RF98 &  \\
	51,740 &      F74 &       & 51,990 &     RF99 &
\end{longtable}

\clearpage
\subsubsection{FM-kanaler 2m-bandet}

\begin{longtable}{rrl|rrl}

\textbf{Frekvens} & \textbf{Kanal} & \textbf{Anm} & 
\textbf{Frekvens} & \textbf{Kanal} & \textbf{Anm} \\ \hline

145,2125 & V17 &              & 145,5000 & V40  & S20  FM Anrop \\
145,2250 & V18 & S9           & 145,5125 & V41  &               \\
145,2375 & V19 & INET GW      & 145,5250 & V42  & S21           \\
145,2500 & V20 & S10          & 145,5375 & V43  &               \\
145,2625 & V21 &              & 145,5500 & V44  & S22           \\
145,2750 & V22 & S11          & 145,5625 & V45  &               \\
145,2875 & V23 & INET GW      & 145,5750 & V46  & S23           \\
145,3000 & V24 & S12  RTTY    & 145,5875 & V47  &               \\
145,3125 & V25 &              & 145,6000 & RV48 & R0            \\
145,3250 & V26 & S13          & 145,6125 & RV49 & R0X           \\
145,3375 & V27 & INET GW      & 145,6250 & RV50 & R1            \\
145,3500 & V28 & S14          & 145,6375 & RV51 & R1X           \\
145,3625 & V29 &              & 145,6500 & RV52 & R2            \\
145,3750 & V30 & S15 DV Anrop & 145,6625 & RV53 & R2X           \\
145,3875 & V31 &              & 145,6750 & RV54 & R3            \\
145,4000 & V32 & S16          & 145,6875 & RV55 & R3X           \\
145,4125 & V33 &              & 145,7000 & RV56 & R4            \\
145,4250 & V34 & S17 Scout    & 145,7125 & RV57 & R4X           \\
145,4375 & V35 &              & 145,7250 & RV58 & R5            \\
145,4500 & V36 & S18          & 145,7375 & RV59 & R5X           \\
145,4625 & V37 &              & 145,7500 & RV60 & R6            \\
145,4750 & V38 & S19          & 145,7625 & RV61 & R6X           \\
145,4875 & V39 &              & 145,7750 & RV62 & R7            \\
         &     &              & 145,7875 & RV63 & R7X
\end{longtable}

X-kanalerna uppstod när man fick platsbrist och man övergick till en 12.5~kHz kanaldelning för repeatrar. Först senare övergick man även till samma kanaldelning på övriga FM-kanaler. De gamla simplexkanalerna hade inte så stor spridning i Sverige men förekom rikligt t.ex. i Tyskland med S20 som anropsfrekvens (eller aktivitetscenter som det numera kallas).

\clearpage
\subsubsection{FM-kanaler 70cm-bandet}

\begin{longtable}{rrl|rrl}
\textbf{Frekvens} & \textbf{Kanal} & \textbf{Anm} &  
\textbf{Frekvens} & \textbf{Kanal} & \textbf{Anm} \\ \hline

433,4000 & U272 & SSTV    & 433,7125 & U297 &      \\
433,4125 & U273 &         & 433,7250 & U298 &      \\
433,4250 & U274 &         & 433,7375 & U299 &      \\
433,4375 & U275 &         & 433,7500 & U300 &      \\
433,4500 & U276 & Digital & 433,7625 & U301 &      \\
433,4625 & U277 &         & 433,7750 & U302 &      \\
433,4750 & U278 &         & 433,7875 & U303 &      \\
433,4875 & U279 &         & 433,8000 & U304 & APRS \\
433,5000 & U280 & Anrop   & 433,8125 & U305 &      \\
433,5125 & U281 &         & 433,8250 & U306 &      \\
433,5250 & U282 &         & 433,8375 & U307 &      \\
433,5375 & U283 &         & 433,8500 & U308 &      \\
433,5500 & U284 &         & 433,8625 & U309 &      \\
433,5625 & U285 &         & 433,8750 & U310 &      \\
433,5750 & U286 &         & 433,8875 & U311 &      \\
433,5875 & U287 &         & 433,9000 & U312 &      \\
433,6000 & U288 & RTTY    & 433,9125 & U313 &      \\
433,6125 & U289 &         & 433,9250 & U314 &      \\
433,6250 & U290 &         & 433,9375 & U315 &      \\
433,6375 & U291 &         & 433,9500 & U316 &      \\
433,6500 & U292 &         & 433,9625 & U317 &      \\
433,6625 & U293 &         & 433,9750 & U318 &      \\
433,6750 & U294 &         & 433,9875 & U319 &      \\
433,6875 & U295 &         & 434,0000 & U320 &      \\
433,7000 & U296 & FAX     &          &      &      \\

\end{longtable}

\clearpage
\begin{longtable}{rrl|rrl}
\textbf{Frekvens} & \textbf{Kanal} & \textbf{Anm}   
&  \textbf{Frekvens} & \textbf{Kanal} & \textbf{Anm} \\ \hline

434,6000 & RU368 & RU0  & 434,8000 & RU384 & RU8   \\
434,6125 & RU369 & RU0X & 434,8125 & RU385 & RU8X  \\
434,6250 & RU370 & RU1  & 434,8250 & RU386 & RU9   \\
434,6375 & RU371 & RU1X & 434,8375 & RU387 & RU9X  \\
434,6500 & RU372 & RU2  & 434,8500 & RU388 & RU10  \\
434,6625 & RU373 & RU2X & 434,8625 & RU389 & RU10X \\
434,6750 & RU374 & RU3  & 434,8750 & RU390 & RU11  \\
434,6875 & RU375 & RU3X & 434,8875 & RU391 & RU11X \\
434,7000 & RU376 & RU4  & 434,9000 & RU392 & RU12  \\
434,7125 & RU377 & RU4X & 434,9125 & RU393 & RU12X \\
434,7250 & RU378 & RU5  & 434,9250 & RU394 & RU13  \\
434,7375 & RU379 & RU5X & 434,9375 & RU395 & RU13X \\
434,7500 & RU380 & RU6  & 434,9500 & RU396 & RU14  \\
434,7625 & RU381 & RU6X & 434,9625 & RU397 & RU14X \\
434,7750 & RU382 & RU7  & 434,9750 & RU398 & RU15  \\
434,7875 & RU383 & RU7X & 434,9875 & RU399 & RU15X \\
         &       &      & 435,0000 & RU400 &       \\

\end{longtable}

RU0X osv är här en efterkonstruktion. Egentligen så användes sällan ``X-frekvenserna'' på 70cm eftersom man dels hade nästan dubbla antalet frekvenser för repeatrar och sedan gammalt ville man egentligen inte ha ett smalare kanalraster, i tidernas begynnelse körde många amatörer 70cm genom frekvenstrippling från 2m. $144,000 \cdot 3 = 432,000$~MHz och $144,025 \cdot 3 = 432,075$~MHz varför man till och med hade bredare kanalraster de-facto.

\clearpage

\subsection{JOTA, Scouters frekvenser}

Scouter finns ofta QRV under vissa helger, \textit{Jamboree On The Air, JOTA}, förekommer några gånger per år. Här är en sammanställning av de standardfrekvenser scouter nyttjar om de inte kör repeatrar eller leta upp motstationer själva. Scouter kan antingen ha egna signaler, köra under
tillfälliga signaler eller vara second operator åt med någon klubbsignal.

\subsection{Nordiska scoutfrekvenser VHF}

\begin{center}
\begin{tabular}{lrr}
	\textbf{Mode} & \textbf{Frekvens} & \textbf{Kanal} \\ \hline
	FM            &      145.425  MHz &   V34 \\
	SSB           &      144.240  MHz &  \\
	CW            &      144.050  MHz &
\end{tabular}
\end{center}

Jotan hålls alltid den 3:e hela (lördag och söndag) helgen i oktober varje år. Jotan startar officiellt vid invigningen på lördag förmiddag och slutar natten till måndagen klockan 00:00. Många börjar redan på fredagskvällen och avslutar på söndagseftermiddagen.

Sändningar under denna tid förekommer från ocertifierade scouter som lånar klubbsignal, har en tillfällig signal utdelad, ibland lånar enskilda sändaramatörer ut sina signaler. Sändningarna skall dock alltid ske under direkt överinseende av en radioamatör men var beredd på att det kommer vara en
viss ovana och ske en del misstag. Strunta i det och ge scouterna en kul radioupplevelse.

\scriptsize
\subsection{Repeatrar, länkar och fyrar VHF/UHF}
\subsubsection{Svenska fyrar VHF/UHF}
\begin{longtable}{llrlrrrlrll}
	Signal   & Placering           &   Frekvens & Loc    &    P & MASL & MAGL & Dir     &  Band & Mode   & Dist \\ \hline
	SKØCT/B  & Stockholm           &  5760.9030 & JO99JX &   80 &   60 &   30 & Omni    &   6cm & CW     & 0    \\
	SKØEN/B  & Väddö               & 10368.8470 & JO99JX & 1000 &   60 &   30 & Omni    &  23cm & CW     & 0    \\
	SKØEN/B  & Väddö               &  1296.8350 & JO99JX &    4 &   70 &   40 & Omni    &  23cm & CW     & 0    \\
	SKØCT/B  & Stockholm           & 10368.8400 & JO89XJ &  0.1 &   50 &   20 & Omni    &   3cm & CW     & 0    \\
	SK1UHF   & Klintehamn          &   432.4050 & JO97CJ &   30 &   65 &   60 & Omni    &  70cm & CW     & 1    \\
	SK1VHF   & Klintehamn          &   144.4470 & JO97CJ &   10 &   65 &   60 & Omni    &    2m & CW     & 1    \\
	SK1UHG   & Klintehamn          &  1296.9500 & JO97CJ &   30 &   65 &   60 & Omni    &  23cm & CW     & 1    \\
	SK1SHH   & Klintehamn          & 10368.8500 & JO97CJ &    3 &   52 &   52 & Omni    &   3cm & CW     & 1    \\
	SK2VHF   & Vindeln/Buberget    &   144.4570 & JP94TF &   80 &  300 &   10 & N+SV    &    2m & CW     & 2    \\
	SK2CP/B  & Kiruna/Esrange      &    50.0520 & KP07MU &   30 &  630 &      & Omni    &    6m & CW     & 2    \\
	SK2SHF   & Vännäs/Granl.b.     &  1296.9850 & JP93VU &   10 &  250 &   50 &         &  23cm & CW     & 2    \\
	SK2SHF   & Vännäs/Granl.b.     &  2320.9850 & JP93VU &   10 &  250 &   50 &         &  13cm & CW     & 2    \\
	SK2DR/B  & Råneå               &  1296.9370 & KP15EU &   14 &      &      & South   &  23cm & CW     & 2    \\
	SK2DR/B  & Råneå               & 10368.8200 & KP15EU &    4 &      &      & South   &   3cm & CW     & 2    \\
	SK3UHH   & Nordingrå/Rävsön    &  2320.9000 & JP92FW &      &  200 &    5 & 220°    &  13cm & CW     & 3    \\
	SK3UHF   & Nordingrå/Rävsön    &   432.4550 & JP92FW &   50 &  200 &    8 & Omni    &  70cm & CW     & 3    \\
	SK3UHG   & Nordingrå/Rävsön    &  1296.8550 & JP92FW &   30 &  200 &   10 & Omni    &  23cm & CW     & 3    \\
	SK3SIX   & Östersund           &    50.4680 & JP73HC &   15 &  480 &    7 & Omni    &    6m & CW     & 3    \\
	SK3VHF   & Östersund           &   144.4210 & JP73HC &   50 &  480 &    7 & 180°    &    2m & CW     & 3    \\
	SM3KDR   & Krokom/Aspås        &    28.2860 & JP73GI &    1 &  380 &    5 & E-W     &   10m & CW     & 3    \\
	SK4BX/B  & Garphyttan/Ånnaboda & 10368.9600 & JO79LI &      &  270 &   10 &         &   3cm & CW     & 4    \\
	SK4MPI   & Borlänge            &   144.4120 & JP70PI &  200 &  380 &   20 & NV+NO   &   2cm & PI4/CW & 4    \\
	SK4BX/B  & Garphyttan/Storst.  &   432.4600 & JO79LH &   50 &  270 &   10 & N E S W &  70cm & CW     & 4    \\
	SK4BX/B  & Garphyttan/Ånnab.   &  1296.9600 & JO79LI &      &  270 &   10 &         &  23cm & CW     & 4    \\
	SK6YH/B  & Göteborg            & 10368.8080 & JO57XQ & 1000 &  135 &   40 & 184°    &   3cm & CW     & 6    \\
	SK6MHI   & Hönö                &  1296.8000 & JO57TQ &   30 &   40 &   30 & Omni    &  23cm & CW     & 6    \\
	SK6MHI   & Göteborg            &  5760.8000 & JO57XQ &   10 &  135 &   40 & Omni    &   6cm & CW     & 6    \\
	SK6UHF   & Varberg/Veddige     &   432.4120 & JO67EH &   10 &  175 &   25 & Omni    &  70cm & CW     & 6    \\
	SK6SHG   & Tjörn Island        & 24048.8830 & JO57TX & 2x1W &  118 &    8 & N/S     & 1.5cm & CW     & 6    \\
	SK6MHI   & Göteborg            & 24048.8000 & JO57XQ &   10 &  135 &   40 & Omni    & 1.5cm & CW     & 6    \\
	SK6UHI   & Tjörn Island        &  1296.8050 & JO57TX &   30 &  128 &   18 & Omni    &  23cm & CW     & 6    \\
	SK6VHF   & Tjörn Island        &   144.4060 & JO57TX &   10 &  122 &   12 & Omni    &    2m & CW     & 6    \\
	SK6WW/B  & Karlsborg/Vaberget  & 10368.8350 & JO78FM &    7 &  240 &   20 & Omni    &   3cm & CW     & 6    \\
	SK6EI/B  & Skövde              &    50.4600 & JO68VJ &   10 &  300 &   30 & South   &    6m & CW     & 6    \\
	SM7DTE/B & Gärsnäs             &  5760.8410 & JO75DN &   40 &   86 &    8 & Omni    &   6cm & CW     & 7    \\
	SM7DTE/B & Gärsnäs             & 10368.8410 & JO75DN &   40 &   86 &    8 & Omni    &   3cm & CW     & 7    \\
	SM7DTE/B & Gärsnäs             & 24048.8430 & JO75DN &   70 &   86 &    8 & Omni    & 1.5cm & CW     & 7    \\
	SK7GH/B  & Värnamo             &    28.2980 & JO77BF &    5 &  230 &   10 & Omni    &   10m & CW     & 7    \\
	SK7VHF   & Sjöbo               &   144.4610 & JO65UQ &   10 &   25 &   25 & Omni    &    2m & CW     & 7    \\
	SK7GH/B  & Värnamo             &  1296.8250 & JO77AE &   10 &  230 &   10 & Omni    &  23cm & CW     & 7
\end{longtable}

\clearpage

\subsubsection{Repeatrar distrikt 0}
\begin{longtable}{llllrrl}
	Typ      & Modulation & Signal   & Ort             & Utfrekvens &   Duplex & Loc    \\ \hline
	Hotspot  & D-Star     & SKØAI-B  & Stockholm       &   433.4625 &  Simplex & JO89XG \\
	Hotspot  & D-Star     & SEØYOS-C & M/Y Erika       &   434.4500 & Duplex 0 & JO99AH \\
	Link     & FM         & SKØMM    & Sandhamn        &   434.3750 &  Simplex & JO99KG \\
	Link     & FM         & SKØMM/L  & Ingarö          &   145.2250 &  Simplex & JO99GG \\
	Link     & FM         & SMØUAO   & Kopparmora      &   434.4875 &  Simplex & JO99HI \\
	Link     & FM         & SKØRVF   & Hagsätra        &   434.4250 &  Simplex & JO99AG \\
	Repeater & FM         & SKØNN/R  & Haninge         &   434.7750 &   -2.000 & JO99BE \\
	Repeater & FM         & SKØCT/R  & Kista           &  1297.0250 &   -6.000 & JO89XJ \\
	Repeater & FM         & SLØZS/R  & Västberga       &   145.6000 &   -0.600 & JO89XH \\
	Repeater & FM         & SLØZS/R  & Västberga       &   434.9000 &   -2.000 & JO89XH \\
	Repeater & FM         & SKØPQ/R  & Kista           &   145.6750 &   -0.600 & JO89XJ \\
	Repeater & FM         & SMØOFV/R & Solna           &   145.7625 &   -0.600 & JO89XI \\
	Repeater & FM         & SKØZA/R  & Solna           &   434.8500 &   -2.000 & JO89XI \\
	Repeater & FM         & SKØRDZ   & Brottby         &   145.6500 &   -0.600 & JO99DN \\
	Repeater & FM         & SAØAZT/R & Brottby         &   434.8000 &   -2.000 & JO99BM \\
	Repeater & FM         & SM5DWC/R & Södertälje      &   434.8250 &   -2.000 & JO89TE \\
	Repeater & FM         & SMØMMO/R & Tullinge        &   145.6625 &   -0.600 & JO89XF \\
	Repeater & FM         & SKØCT/R  & Kista           &   434.6250 &   -2.000 & JO89XJ \\
	Repeater & FM         & SMØYIX/R & Söder           &   434.7250 &   -2.000 & JO99BH \\
	Repeater & FM         & SKØYZ/R  & Vallentuna      &   434.8625 &   -2.000 & JO99BM \\
	Repeater & FM         & SKØCT/R  & Kista           &   434.6625 &   -2.000 & JO89XJ \\
	Repeater & FM         & SKØQO/R  & Haninge         &   145.6875 &   -0.600 & JO99BE \\
	Repeater & FM         & SKØQO/R  & Haninge         &   434.7500 &   -2.000 & JO99BE \\
	Repeater & FM         & SKØRMT   & Täby            &   434.7375 &   -2.000 & JO99AK \\
	Repeater & DMR        & SKØRMT   & Täby            &   434.7375 &   -2.000 & JO99AK \\
	Repeater & C4FM       & SKØRMT   & Täby            &   434.7375 &   -2.000 & JO99AK \\
	Repeater & D-Star     & SKØRMT   & Täby            &   434.7375 &   -2.000 & JO99AK \\
	Repeater & DMR        & SKØRMT   & Täby            &   434.7375 &   -2.000 & JO99AK \\
	Repeater & C4FM       & SKØRMT   & Täby            &   434.7375 &   -2.000 & JO99AK \\
	Repeater & D-Star     & SKØRMT   & Täby            &   434.7375 &   -2.000 & JO99AK \\
	Repeater & DMR        & SKØRYG   & Kista           &   434.9500 &   -2.000 & JO89XJ \\
	Repeater & DMR        & SKØRYG   & Sthlm city      &   434.9625 &   -2.000 & JO99AI \\
	Repeater & DMR        & SMØWIU/R & Nynäshamn       &   434.6125 &   -2.000 & JO88XV \\
	Repeater & DMR        & SMØWIU/R & Botkyrka        &   434.8750 &   -2.000 & JO89WG \\
	Repeater & C4FM       & SKØNN    & Haninge         &   434.5375 &   -2.000 & JO99CF \\
	Repeater & DMR        & SKØSX    & Kista           &   434.9875 &   -2.000 & JO89XJ \\
	Repeater & DMR        & SKØRMQ   & Tyresö          &   434.5125 &   -2.000 & JO99CH \\
	Repeater & DMR        & SMØWIU-4 & Högdalen        &   145.5750 &   -0.600 & JO99AF \\
	Repeater & FM         & SKØMG/R  & Skarpnäck       &   145.7000 &   -0.600 & JO89TE \\
	Repeater & FM/DMR     & SKØRIX   & Sthlm city      &   145.6250 &   -0.600 & JO99AH \\
	Repeater & DMR        & SGØRPF   & Rimbo           &   434.7875 &   -2.000 & JO99BT \\
	Repeater & DMR        & SKØRYG   & Upplands Väsby  &   434.7625 &   -2.000 & JO89XM \\
	Repeater & FM/DMR     & SKØRPF   & Sigtuna         &   434.8875 &   -2.000 & JO89VP \\
	Repeater & C4FM       & SKØQO    & Bagarmossen     &   434.5750 &   -2.000 & JO99BG \\
	Repeater & DMR        & SKØNN/1  & Johanneshov     &   434.9250 &   -2.000 & JO99AH \\
	Repeater & DMR        & SKØVR    & Djurö           &   434.5875 &   -2.000 & JO99IH \\
	Repeater & DMR        & SMØWIU/R & Dalarö          &   434.8375 &   -2.000 & JO99ED \\
	Repeater & FM         & SKØRYG   & Stockholm Norr  &   145.7875 &   -0.600 & JO99DL \\
	Repeater & FM         & SKØRYG   & Upplands Väsby  &   434.6750 &   -2.000 & JO89XM \\
	Repeater & DMR        & SKØEN    & Älmsta          &   434.6000 &   -2.000 & JO99JX \\
	Repeater & FM         & SKØBJ/R  & Nynäshamn       &   145.7125 &   -0.600 & JO88XV \\
	Repeater & C4FM       & SKØMG    & Haninge/Gålö    &   434.6875 &   -2.000 & JO99CC \\
	Repeater & DMR        & SKØQO    & Haninge/Brandb. &   434.5625 &   -2.000 & JO99BE \\
	Repeater & FM/DMR     & SKØVR    & Värmdö          &   434.9750 &   -2.000 & JO99FH \\
	Repeater & FM/DMR     & SKØEN    & Älmsta          &   145.7375 &   -0.600 & JO99JX \\
	Repeater & FM/DMR     & SAØAZT   & Norrtälje       &   434.8125 &   -2.000 & JO99IS \\
	Repeater & FM         & SKØMM/R  & Ingarö          &   145.7750 &   -0.600 & JO99GG \\
	Repeater & DMR        & SKØMG    & Södertälje      &   434.7875 &   -2.000 & JO89TE \\
	Repeater & FM         & SKØBJ    & Nynäshamn       &   145.7375 &   -0.600 & JO88WT \\
	Repeater & FM         & SKØBJ/R  & Nynäshamn       &   434.7125 &   -2.000 & JO88XV \\
	Repeater & FM         & SKØBJ/R  & Nynäshamn       &   434.6500 &   -2.000 & JO89XF \\
	Repeater & FM         & SKØBJ    & Nynäshamn       &   434.9125 &   -2.000 & JO88WT \\
	Repeater & FM/DMR     & SAØAZT   & Vallentuna      &   434.5500 &   -2.000 & JO99EO \\
	Repeater & FM         & SKØBJ/R  & Huddinge        &   434.6000 &   -2.000 & JO89XF \\
	Repeater & DMR        & SMØWIU-2 & Södertälje      &   434.8750 &   -2.000 & JO89TE \\
	Repeater & FM         & SKØMT/R  & Vallentuna      &   434.7000 &   -2.000 & JO99BM \\
	Repeater & C4FM       & SKØMG/R  & Sthlm/Söderort  &   434.6375 &   -2.000 & JO99AH
\end{longtable}

\clearpage

\subsubsection{Repeatrar distrikt 1}

\begin{longtable}{llllrrlcl}
	Typ      & Modulation & Signal   & Ort   & Utfrekvens &  Duplex & Loc    &  \\ \hline
	Repeater & FM         & SL1ZXK/R & Slite &   434.6000 &  -2.000 & JO97JR &     &  \\
	Repeater & FM/C4FM    & SK1RGU   & Endre &   145.7750 &  -0.600 & JO97FO &     &  \\
	Repeater & FM/C4FM    & SK1BL/R  & Endre &   145.7750 & -600kHz & 1750   & QRV & JO97FO
\end{longtable}

\subsubsection{Repeatrar distrikt 2}

\begin{longtable}{llllrrlcl}
	Typ      & Modulation         & Signal    & Ort                     & Utfrekvens &  Duplex & Loc    &  &  \\ \hline
	Link     & FM                 & SM2YUW    & Kiruna                  &   434.4000 & Simplex & KP07DU &  &  \\
	Repeater & FM                 & SK2AU/R   & Arjeplog/Galtispouda    &   145.7000 &  -0.600 & JP86XC &  &  \\
	Repeater & FM                 & SK2AU/R   & Skellefteå              &   145.7000 &  -0.600 & KP04LS &  &  \\
	Repeater & FM                 & SK2RIU    & Vännäs/Granlundsberget  &   145.7250 &  -0.600 & JP93VU &  &  \\
	Repeater & FM                 & SK2RIU    & Vännäs/Granlundsberget  &   434.7250 &  -2.000 & JP93VU &  &  \\
	Repeater & FM                 & SK2RLF    & Tärnaby                 &   145.6250 &  -0.600 & JP75PR &  &  \\
	Repeater & FM                 & SK2RLJ    & Umeå/Rödberget          &   145.6500 &  -0.600 & KP03CU &  &  \\
	Repeater & FM                 & SK2RMD    & Sorsele                 &   145.6000 &  -0.600 & JP85SM &  &  \\
	Repeater & FM                 & SK2RMR    & Storuman                &   145.7250 &  -0.600 & JP85NC &  &  \\
	Repeater & FM                 & SK2RYI    & Vindeln/Åsträsk         &   145.6250 &  -0.600 & KP04DP &  &  \\
	Repeater & FM                 & SK2AU/R   & Jörn/Storklinta         &   145.7500 &  -0.600 & KP05BD &  &  \\
	Repeater & FM                 & SK2LY/R   & Lycksele                &   145.7750 &  -0.600 & JP94IO &  &  \\
	Repeater & FM                 & SM2KOT/R  & Kristineberg/Viterliden &   145.6750 &  -0.600 & JP95HB &  &  \\
	Repeater & FM                 & SK2RFR    & Kiruna                  &   145.6250 &  -0.600 & KP07DU &  &  \\
	Repeater & FM                 & SK2RFR    & Kiruna C                &   434.8250 &  -2.000 & KP07DU &  &  \\
	Repeater & FM                 & SK2DR/R   & Luleå                   &   145.6500 &  -0.600 & KP15CO &  &  \\
	Repeater & FM                 & SK2AZ/R   & Piteå                   &   145.6000 &  -0.600 & KP05PH &  &  \\
	Repeater & FM                 & SK2RWJ    & Älvsbyn                 &   145.6750 &  -0.600 & KP05LQ &  &  \\
	Repeater & FM                 & SK2HG/R   & Kalix/Raggdynan         &    51.9500 &  -0.600 & KP15KW &  &  \\
	Repeater & FM                 & SM2KXX    & Lycksele                &   434.7750 &  -1.600 & JP94HO &  &  \\
	Repeater & FM                 & SK2RMR    & Storuman                &   434.7500 &  -2.000 & JP85NC &  &  \\
	Repeater & FM                 & SK2RME    & Piteå                   &   434.6000 &  -2.000 & KP05RH &  &  \\
	Repeater & DMR                & SK2RGJ    & Kiruna                  &   434.5125 &  -2.000 & KP07CT &  &  \\
	Repeater & DMR/D-Star         & SK2DR     & Luleå                   &   434.9000 &  -2.000 & KP15CO &  &  \\
	Repeater & DMR/D-Star         & SK2RJH    & Kalix/Raggdynan         &   434.7500 &  -2.000 & KP15KW &  &  \\
	Repeater & FM/DMR             & SK2HG/R3  & Seskarö                 &   145.6750 &  -0.600 & KP15UR &  &  \\
	Repeater & FM/DMR             & SK2HG/R5  & Kalix/Raggdynan         &   145.7250 &  -0.600 & KP15KW &  &  \\
	Repeater & FM/DMR             & SK2HG/RU5 & Kalix-Vattentorn        &   434.7250 &  -2.000 & KP15NU &  &  \\
	Repeater & DMR                & SK2AT     & Vännäs                  &   434.9750 &  -2.000 & JP93XX &  &  \\
	Repeater & FM                 & SK2CI     & Boden                   &   145.6250 &  -0.600 & KP05SS &  &  \\
	Repeater & DMR                & SK2AZ     & Piteå                   &   434.8500 &  -2.000 & KP05PH &  &  \\
	Repeater & DMR                & SK2CI     & Boden                   &   434.8000 &  -2.000 & KP05TT &  &  \\
	Repeater & DMR                & SK2HG-2   & Kalix                   &   434.9875 &  -2.000 & KP15OU &  &  \\
	Repeater & FM/DMR/D-Star/C4FM & SK2AU/R   & Skellefteå              &   145.5875 &  -0.600 & KP04LS &  &  \\
	Repeater & FM                 & SJ2W/R    & Skellefteå              &   434.6750 &  -2.000 & KP04LS &  &  \\
	Repeater & FM                 & SJ2W      & Burträsk                &   434.9500 &  -2.000 & KP04HM &  &
\end{longtable}

\subsubsection{Repeatrar distrikt 3}

\begin{longtable}{llllrrlcl}
	Typ      & Modulation      & Signal   & Ort                    & Utfrekvens &   Duplex & Loc    &  &  \\ \hline
	Hotspot  & D-Star          & SK3GA-B  & Hudiksvall             &   434.4750 & Duplex 0 & JP81NR &  &  \\
	Link     & FM              & SM3KDR   & Krokom/Aspås           &   434.9750 &  Simplex & JP73GI &  &  \\
	Repeater & FM              & SK3EK/R  & Sollefteå              &   434.6500 &   -1.600 & JP83DE &  &  \\
	Repeater & FM              & SK3MF/R  & Nordingrå/Rävsön       &   145.6250 &   -0.600 & JP92FW &  &  \\
	Repeater & FM              & SK3MF/R  & Nordingrå/Rävsön       &   434.8500 &   -2.000 & JP92FW &  &  \\
	Repeater & FM              & SK3RFG   & Sundsvall              &   145.7250 &   -0.600 & JP82RJ &  &  \\
	Repeater & FM              & SK3RIA   & Östersund              &   434.7500 &   -2.000 & JP73JE &  &  \\
	Repeater & FM              & SK3RIN   & Borgsjö                &   145.7000 &   -0.600 & JP72WN &  &  \\
	Repeater & FM              & SK3RKL   & Örnsköldsvik/Rutberget &   145.7750 &   -0.600 & JP93GJ &  &  \\
	Repeater & FM              & SK3RMG   & Bergsjö                &  1297.1000 &   -6.000 & JP81MX &  &  \\
	Repeater & FM              & SK3RMX   & Hoting/Kyrktåsjö       &   145.6000 &   -0.600 & JP74XF &  &  \\
	Repeater & FM              & SK3RYK   & Söderhamn              &   145.7500 &   -0.600 & JP81NH &  &  \\
	Repeater & FM              & SK3RYK   & Söderhamn              &   434.7500 &   -1.600 & JP81NH &  &  \\
	Repeater & FM              & SK3WH    & Högakustenbron         &  1297.2750 &   -6.000 & JP82XT &  &  \\
	Repeater & FM              & SK3LH/R  & Örnsköldsvik           &   434.8750 &   -2.000 & JP93IH &  &  \\
	Repeater & FM              & SK3RNJ   & Åre/Åreskutan          &   145.7250 &   -0.600 & JP63NK &  &  \\
	Repeater & FM              & SM3XRJ   & Kramfors               &   434.6000 &   -2.000 & JP82VW &  &  \\
	Repeater & D-Star          & SK3LH-B  & Örnsköldsvik/Malmön    &   434.5750 &   -2.000 & JP93LF &  &  \\
	Repeater & FM              & SL3ZB    & Härnösand              &   434.7250 &   -2.000 & JP82XP &  &  \\
	Repeater & FM              & SK3EK/R  & Sollefteå              &   145.6500 &   -0.600 & JP83PD &  &  \\
	Repeater & D-Star          & SK3RFG-C & Sundsvall/Klissberget  &   145.5875 &   -0.600 & JP82OJ &  &  \\
	Repeater & FM/C4FM         & SK3JR/R  & Östersund              &   145.7500 &   -0.600 & JP73JE &  &  \\
	Repeater & FM              & SK3GK/R  & Sandviken/Kungsberget  &   145.7000 &   -0.600 & JP80FS &  &  \\
	Repeater & FM              & SM3VAC/R & Nyland                 &   145.7500 &   -0.600 & JP83UA &  &  \\
	Repeater & FM              & SM3VAC/R & Nyland                 &   434.9500 &   -1.600 & JP83UA &  &  \\
	Repeater & FM              & SK3RQE   & Forsa/Storberget       &   434.6750 &   -2.000 & JP81KQ &  &  \\
	Repeater & FM              & SA3EJX/R & Forsa/Storberget       &   145.6750 &   -0.600 & JP81KQ &  &  \\
	Repeater & FM              & SK3GW    & Gävle                  &   434.8750 &   -2.000 & JP80NP &  &  \\
	Repeater & FM              & SK3GK    & Sandviken              &   434.8250 &   -2.000 & JP80FS &  &  \\
	Repeater & FM              & SK3RQC   & Vemdalen               &   145.6250 &   -0.600 & JP62WK &  &  \\
	Repeater & FM              & SM3LEI/R & Årsunda                &   434.6500 &   +1.600 & JP80IM &  &  \\
	Repeater & DMR             & SK3WH    & Örnsköldsvik           &   145.5750 &   -0.600 & JP93IH &  &  \\
	Repeater & DMR             & SK3GK    & Gävle                  &   434.7000 &   -2.000 & JP80NP &  &  \\
	Repeater & DMR/D-Star      & SK3RFG   & Sundsvall/Klissberget  &   434.8000 &   -2.000 & JP82OJ &  &  \\
	Repeater & DMR/D-Star/C4FM & SM3YFX   & Föllinge               &   434.5250 &   -2.000 & JP73HQ &  &  \\
	Repeater & FM              & SK3GA/R  & Hudiksvall             &   145.7750 &   -0.600 & JP81NR &  &  \\
	Repeater & FM/DMR          & SK3RHU   & Hudiksvall             &   145.7125 &   -0.600 & JP81NR &  &  \\
	Repeater & DMR             & SK3RHU   & Hudiksvall             &   434.5750 &   -2.000 & JP81NR &  &  \\
	Repeater & FM/C4FM         & SK3JR/R2 & Östersund/Brattåsen    &   145.7875 &   -0.600 & JP73HC &  &  \\
	Repeater & DMR/D-Star/C4FM & SG9NN    & Sundsvall              &   434.5375 &   -2.000 & JP82OJ &  &  \\
	Repeater & FM              & SK3RET   & Bollnäs/Arbrå          &   145.6500 &   -0.600 & JP81CL &  &  \\
	Repeater & DMR             & SK3JR    & Östersund/Brattåsen    &   434.5625 &   -2.000 & JP73HC &  &  \\
	Repeater & DMR             & SK3RFG   & Sundsvall/Nolby        &   434.9875 &   -2.000 & JP82QH &  &  \\
	Repeater & FM              & SK3YZ/R  & Forsa                  &   145.6125 &   -0.600 & JP81KQ &  &  \\
	Repeater & FM              & SK3PH/R  & Delsbo                 &    29.6900 &   -0.100 & JP81GT &  &  \\
	Repeater & FM              & SK3EK/R  & Sollefteå              &   434.9250 &   -2.000 & JP83DE &  &  \\
	Repeater & FM              & SK3RQE   &                        &   145.6000 &   -0.600 & JP81NV &  &  \\
	Repeater & FM              & SK3W     & Österfärnebo           &   434.8500 &   -2.000 & JP80JH &  &
\end{longtable}

\subsubsection{Repeatrar distrikt 4}

\begin{longtable}{llllrrlcl}
	Typ      & Modulation & Signal   & Ort                        & Utfrekvens &   Duplex & Loc    &  &  \\ \hline
	Hotspot  & D-Star     & SG4UOF-C & Glanshammar                &   145.3375 & Duplex 0 & JO79RI &  &  \\
	Hotspot  & D-Star     & SG4UZM-B & Borlänge                   &   434.5500 & Duplex 0 & JP70RM &  &  \\
	Hotspot  & DMR        & SG4AXV   & Ekshärad                   &   433.2000 &  Simplex & JP60RE &  &  \\
	Hotspot  & DMR/D-Star & SG4AXQ   & Sunne                      &   432.5000 & Duplex 0 & JO69NU &  &  \\
	Hotspot  & DMR        & SA4ATZ   & Malung                     &   144.8375 &  Simplex & JP60UQ &  &  \\
	Link     & FM         & SK4AV/R  & Filipstad/Klockarhöjden    &   145.2000 &  Simplex & JO79CR &  &  \\
	Link     & FM         &          & Nyhammar                   &   145.3250 &  Simplex & JP70LG &  &  \\
	Link     & FM         &          & Grängesberg                &   145.3500 &  Simplex & JP70MB &  &  \\
	Link     & FM         & SK4RJJ   & Torsby/Hovfjället          &   145.2875 &  Simplex & JO69LH &  &  \\
	Link     & FM         & SA4THA   & Älvdalen                   &   434.5000 &  Simplex & JP71AF &  &  \\
	Link     & FM         & SM4FBD   & Nybble                     &   145.3000 &  Simplex & JO79BC &  &  \\
	Link     & FM         & SK4EA-L  & Lindesberg                 &   145.3000 &  Simplex & JO79OO &  &  \\
	Link     & FM         & SM4MXN   & Orsa                       &   145.2750 &  Simplex & JP71HC &  &  \\
	Repeater & FM         & SK4DM/R  & Ludvika                    &   145.7250 &   -0.600 & JP70NC &  &  \\
	Repeater & FM         & SK4DM/R  & Ludvika                    &   434.7250 &   -1.600 & JP70NC &  &  \\
	Repeater & FM         & SK4RGO   & Orsa/Grönklitt             &   434.7500 &   -1.600 & JP71GF &  &  \\
	Repeater & FM         & SK4RPK   & Torsby/Valberget           &   434.6250 &   -2.000 & JP60LC &  &  \\
	Repeater & FM         & SK4RQF   & Årjäng                     &   145.7250 &   -0.600 & JO69BJ &  &  \\
	Repeater & FM         & SM4JDP   & Mora                       &   434.7000 &   -2.000 & JP71GA &  &  \\
	Repeater & D-Star     & SG4TYA   & Mora                       &   145.5750 &   -0.600 & JP71GE &  &  \\
	Repeater & FM         & SK4IL/R  & Grums                      &   434.7250 &   -2.000 & JO69NI &  &  \\
	Repeater & FM         & SK4WV    & Vansbro                    &   145.6500 &   -0.600 & JP70AM &  &  \\
	Repeater & FM         & SK4WV    & Vansbro                    &   434.6500 &   -1.600 & JP70AM &  &  \\
	Repeater & FM         & SK4TL/R  & Örebro/Suttarboda          &   145.7125 &   -0.600 & JO79KH &  &  \\
	Repeater & FM         & SK4RGO   & Orsa/Grönklitt             &   145.7500 &   -0.600 & JP71GF &  &  \\
	Repeater & D-Star     & SK4BW-B  & Borlänge                   &   434.9000 &   -2.000 & JP70RJ &  &  \\
	Repeater & FM/C4FM    & SK4RVN   & Borlänge                   &   434.8000 &   -2.000 & JP70RJ &  &  \\
	Repeater & FM         & SK4HV/R  & Hagfors/Värmullsåsen       &   145.6750 &   -0.600 & JP60VA &  &  \\
	Repeater & FM         & SK4EA/R  & Lindesberg                 &   145.6875 &   -0.600 & JO79NP &  &  \\
	Repeater & FM         & SK4RWQ   & Arvika/Valfjället          &   434.7750 &   -2.000 & JO69CT &  &  \\
	Repeater & FM         & SK4RJJ   & Sunne/Blåbärskullen        &   145.7750 &   -0.600 & JO69KU &  &  \\
	Repeater & FM         & SK4BX/R  & Garphyttan/Storstenshöjden &   145.6500 &   -0.600 & JO79LH &  &  \\
	Repeater & FM         & SK4RUV   & Leksand                    &   145.7750 &   -0.600 & JP70MQ &  &  \\
	Repeater & DMR        & SK4BW    & Borlänge                   &   434.8500 &   -2.000 & JP70RJ &  &  \\
	Repeater & DMR        & SK4WV    & Vansbro                    &   434.6625 &   -2.000 & JP70AM &  &  \\
	Repeater & FM         & SK4EA/R  & Kopparberg                 &   145.6000 &   -0.600 & JO79MW &  &  \\
	Repeater & DMR        & SA4BNA   & Arvika                     &   434.9750 &   -2.000 & JO69GN &  &  \\
	Repeater & FM/DMR     & SK4KR    & Karlskoga                  &   434.8000 &   -2.000 & JO79FH &  &  \\
	Repeater & DMR        & SK4RGL   & Falun                      &   434.6250 &   -2.000 & JP70UP &  &  \\
	Repeater & FM         & SK4RGL   & Falun                      &   145.6250 &   -0.600 & JP70UP &  &  \\
	Repeater & FM         & SK4TL/R  & Örebro/Suttarboda          &    51.9500 &   -0.600 & JO79KH &  &  \\
	Repeater & FM/DMR     & SK4RKD   & Karlskoga                  &   145.7500 &   -0.600 & JO79FJ &  &  \\
	Repeater & DMR        & SK4KO    & Nusnäs                     &   434.9250 &   -2.000 & JP70HW &  &  \\
	Repeater & FM/DMR     & SM4WIU-3 & Leksand                    &   434.6125 &   -2.000 & JP70MR &  &  \\
	Repeater & DMR        & SK4TL    & Örebro                     &   434.7250 &   -2.000 & JO79OG &  &  \\
	Repeater & D-Star     & SG4AXV   & Ekshärad                   &   145.6000 &   -0.600 & JP60RE &  &  \\
	Repeater & FM         & SK4KO    & Sälen/Lindvallen           &   145.6000 &   -0.600 & JP61OD &  &  \\
	Repeater & DMR        & SA4BHE-R & Smedjebacken               &   434.6375 &   -2.000 & JP70GD &  &
\end{longtable}



\subsubsection{Repeatrar distrikt 5}

\begin{longtable}{llllrrlcl}
	Typ      & Modulation & Signal   & Ort                    & Utfrekvens &   Duplex & Loc    &  &  \\ \hline
	Hotspot  & D-Star     & SC5SLU-C & Uppsala                &   145.3250 & Duplex 0 & JO89QW &  &  \\
	Hotspot  & D-Star     & SM5EZN-B & Uppsala                &   433.4875 & Duplex 0 & JO89QW &  &  \\
	Hotspot  & D-Star     & SG5TAH-C & Flen/Orrhammar         &   145.3375 & Duplex 0 & JO89GB &  &  \\
	Hotspot  & DMR        & SA5HAV   & Uppsala                &   434.3750 &  Simplex & JO89VW &  &  \\
	Link     & FM         & SM5RVH   & Nyköping               &   145.4750 &  Simplex & JO88LQ &  &  \\
	Link     & FM         & SM5RVH   & Nyköping               &    51.4700 &  Simplex & JO88LQ &  &  \\
	Link     & FM         & SM5RVH   & Nyköping               &    29.1700 &  Simplex & JO88LQ &  &  \\
	Link     & FM         & SM5RVH   & Nyköping               &  1297.5000 &  Simplex & JO88LQ &  &  \\
	Link     & FM         & SM5GXQ-L & Norrköping             &   145.2375 &  Simplex & JO88CO &  &  \\
	Link     & DMR        & SA5KBE   & Stigtomta              &   145.2875 &  Simplex & JO88JT &  &  \\
	Link     & FM         & SA5BJM   & Uppsala/Fjuckby        &   144.5750 &  Simplex & JO89TX &  &  \\
	Link     & FM         & SA5BJM   & Uppsala/Fjuckby        &   433.4500 &  Simplex & JO89TX &  &  \\
	Repeater & FM         & SK5AS/R  & Linköping              &   145.7250 &   -0.600 & JO78SJ &  &  \\
	Repeater & FM         & SK5BN/R  & Finspång               &   434.9250 &   -2.000 & JO78VR &  &  \\
	Repeater & FM/D-Star  & SK5RHQ   & Västerås               &   434.7000 &   -2.000 & JO89GO &  &  \\
	Repeater & FM/C4FM    & SK5RCQ   & Kisa                   &   145.7000 &   -0.600 & JO77TX &  &  \\
	Repeater & FM         & SK5LW/R  & Eskilstuna/Hällby      &   434.8500 &   -2.000 & JO89FJ &  &  \\
	Repeater & FM         & SA5BTT   & Trosa                  &   434.8875 &   -2.000 & JO88TV &  &  \\
	Repeater & FM         & SK5BN/R  & Norrköping/Kolmården   &   145.6000 &   -0.600 & JO88FQ &  &  \\
	Repeater & FM         & SK5BN/R  & Norrköping/Östra Eneby &   434.6000 &   -2.000 & JO88BO &  &  \\
	Repeater & FM         & SK5LF/R  & Linköping/Majelden     &   434.8250 &   -2.000 & JO78TJ &  &  \\
	Repeater & DMR        & SA5BJM   & Uppsala/Fjuckby        &   434.5125 &   -2.000 & JO89TX &  &  \\
	Repeater & FM         & SK5DB/R  & Uppsala                &   145.7500 &   -0.600 & JO89VU &  &  \\
	Repeater & FM         & SK5DB/R  & Uppsala                &   434.7500 &   -2.000 & JO89VU &  &  \\
	Repeater & FM         & SK5RHQ   & Västerås               &   145.7750 &   -0.600 & JO89GO &  &  \\
	Repeater & FM         & SK5RHQ   & Västerås               &   434.7750 &   -2.000 & JO89GO &  &  \\
	Repeater & ATV        & SK5BN/R  & Norrköping/Kolmården   &  1282.0000 &  -30.000 & JO88FQ &  &  \\
	Repeater & FM         & SK5AS/R  & Linköping              &   145.7875 &   -0.600 & JO78SN &  &  \\
	Repeater & FM         & SM5RYI/R & Sala                   &   145.7125 &   -0.600 & JO89HW &  &  \\
	Repeater & DMR        & SK5RYG   & Linköping              &   434.5125 &   -2.000 & JO78SN &  &  \\
	Repeater & FM         & SK5RYG   & Linköping              &   145.6250 &   -0.600 & JO78SN &  &  \\
	Repeater & FM/DMR     & SL5ZYT/R & Norrköping             &   434.9500 &   -2.000 & JO88DQ &  &  \\
	Repeater & FM/DMR     & SG5BCG/R & Knivsta                &   434.5250 &   -2.000 & JO89VR &  &  \\
	Repeater & FM/DMR     & SM5DWC/R & Linköping              &   434.8750 &   -2.000 & JO78SM &  &  \\
	Repeater & FM         & SK5BB/R  & Arboga/Kolsva          &   434.8750 &   -2.000 & JP79WO &  &  \\
	Repeater & FM         & SK5BB/R  & Arboga/Kolsva          &   145.6750 &   -0.600 & JP79WO &  &  \\
	Repeater & D-Star     & SK5BN-C  & Norrköping             &   145.5750 &   -0.600 & JO88BR &  &  \\
	Repeater & FM/DMR     & SG5DV    & Uppsala                &   434.5875 &   -2.000 & JO89TU &  &  \\
	Repeater & FM         & SG5DV    & Uppsala                &   145.5875 &   -0.600 & JO89TU &  &  \\
	Repeater & DMR/D-Star & SK5LW/R  & Eskilstuna/Ärla        &   145.5875 &   -0.600 & JO89FJ &  &  \\
	Repeater & FM         & SK5LW/R  & Eskilstuna             &    51.8500 &   -0.600 & JO89FJ &  &  \\
	Repeater & FM         & SK5VM/R  & Eskilstuna             &   434.9750 &   -2.000 & JO89GI &  &  \\
	Repeater & FM         & SK5LW/R  & Eskilstuna/Slytan      &   145.6125 &   -0.600 & JO89HF &  &  \\
	Repeater & D-Star     & SK5UM-B  & Flen                   &   434.5500 &   -2.000 & JO89HB &  &  \\
	Repeater & FM         & SK5UM/R  & Flen/Öja               &   434.7500 &   -2.000 & JO89HB &  &  \\
	Repeater & DMR        & SK5UM/R  & Flen                   &   145.6375 &   -0.600 & JO89HB &  &  \\
	Repeater & FM         & SM5YMS   & Åtvidaberg             &   145.6625 &   -0.600 & JO78XE &  &  \\
	Repeater & FM         & SM5YMS/R & Linköping              &   434.8000 &   -2.000 & JO78SM &  &  \\
	Repeater & DMR        & SA5HAV/R & Uppsala/Rasbo          &   434.6375 &   -2.000 & JO89VW &  &  \\
	Repeater & DMR        & SL5ZO    & Finspång               &   434.8125 &   -2.000 & JO78VQ &  &  \\
	Repeater & DMR        & SA5UTR   & Nyköping               &   434.6375 &   -2.000 & JO88MS &  &  \\
	Repeater & FM/C4FM    & SA5OHR/R & Norrköping             &   434.6625 &   -2.000 & JO88BO &  &  \\
	Repeater & FM         & SK5RHT   & Linköping              &    51.9900 &   -0.600 & JO78SN &  &  \\
	Repeater & FM         & SK5UM/R  & Flen                   &   145.7625 &   -0.600 & JO89HB &  &  \\
	Repeater & FM         & SK5WR/R  & Motala                 &   145.7375 &   -0.600 & JO78NM &  &  \\
	Repeater & FM         & SK5RHT   & Linköping              &    29.6600 &   -0.100 & JO78XH &  &
\end{longtable}

\subsubsection{Repeatrar distrikt 6}

\begin{longtable}{llllrrlcl}
	Typ      & Modulation      & Signal   & Ort                   & Utfrekvens &   Duplex & Loc    &  &  \\ \hline
	Hotspot  & D-Star          & SK6GB-D  & Mölndal               &   433.7250 &  Simplex & JO67AQ &  &  \\
	Hotspot  & D-Star          & SK6GB-D  & Mölndal               &   144.8250 &  Simplex & JO67AQ &  &  \\
	Hotspot  & D-Star          & SK6MA-C  & Hjo                   &   145.2125 & Duplex 0 & JO78DH &  &  \\
	Hotspot  & D-Star          & SG6JWU-B & Halmstad              &   433.4750 & Duplex 0 & JO66LP &  &  \\
	Hotspot  & DMR/D-Star/C4FM & SK6BA-B  & Skene                 &   433.5625 & Duplex 0 & JO67HL &  &  \\
	Hotspot  & D-Star          & SG6YOW   & Alingsås              &   144.8500 &  Simplex & JO67GW &  &  \\
	Link     & FM              & SA6RP    & Floda                 &   433.4750 &  Simplex & JO67ET &  &  \\
	Link     & FM              & SM6FZG   & Skårsjön              &   144.5500 &  Simplex & JO67AN &  &  \\
	Link     & FM              & SM6FZG   & Kortedala             &   144.6000 &  Simplex & JO67AS &  &  \\
	Link     & FM              & SM6FZG   & Långedrag             &   144.5250 &  Simplex & JO57WQ &  &  \\
	Link     & FM              & SM6FZG   & Hönö                  &   144.6250 &  Simplex & JO57TQ &  &  \\
	Link     & FM              & SK6AG    & Guldheden             &   144.5750 &  Simplex & JO57XQ &  &  \\
	Link     & FM              & SM6FZG   & Mölnlycke             &   144.5875 &  Simplex & JO67BP &  &  \\
	Link     & FM              & SM6FZG   & Borås                 &   144.5125 &  Simplex & JO67MR &  &  \\
	Link     & FM              & SM6YRB   & Lidköping/Kållandsö   &   145.3000 &  Simplex & JO68NP &  &  \\
	Link     & FM              & SM6FZG   & Kungsbacka            &   144.6500 &  Simplex & JO67AL &  &  \\
	Link     & FM              & SM6FZG   & Myggenäs              &   144.6625 &  Simplex & JO58UB &  &  \\
	Link     & FM              & SM6FZG   & Guldheden             &   144.6750 &  Simplex & JO57XQ &  &  \\
	Link     & FM              & SM6FZG   & Guldheden             &    51.5500 &  Simplex & JO57XQ &  &  \\
	Link     & FM              & SM6VAG   & Hjo                   &   145.2375 &  Simplex & JO78AG &  &  \\
	Link     & FM              & SA6EAL   & Hajom                 &   145.4000 &  Simplex & JO67GM &  &  \\
	Link     & FM              & SA6GDS   & Istorp                &   145.2875 &  Simplex & JO67FI &  &  \\
	Link     & FM              & SM6TZL   & Örby                  &   145.2375 &  Simplex & JO67IL &  &  \\
	Repeater & FM              & SA6AR/R  & Angered               &   434.9250 &   -2.000 & JO67AT &  &  \\
	Repeater & FM              & SK6QW/R  & Mariestad/Katrinefors &   434.9000 &   -2.000 & JO68VQ &  &  \\
	Repeater & FM              & SK6DK/R  & Varberg/Veddige       &   434.7000 &   -1.600 & JO67EH &  &  \\
	Repeater & FM              & SK6DK/R  & Varberg/Veddige       &   145.7000 &   -0.600 & JO67EH &  &  \\
	Repeater & FM              & SA6BSN/R & Åmål                  &   434.6000 &   -2.000 & JO69IB &  &  \\
	Repeater & D-Star          & SK6DW-B  & Trollhättan           &   434.5250 &   -2.000 & JO68DG &  &  \\
	Repeater & FM              & SA6BXG/R & Kungälv/Romelanda     &   434.7375 &   -2.000 & JO67AX &  &  \\
	Repeater & FM              & SK6RPE   & Kungälv               &   145.6125 &   -0.600 & JO57XU &  &  \\
	Repeater & FM              & SM6CYJ/R & Kinnekulle            &   434.9500 &   -2.000 & JO68QO &  &  \\
	Repeater & FM              & SK6DQ/R  & Älvängen              &   434.7500 &   -2.000 & JO67BW &  &  \\
	Repeater & FM              & SK6MA/R  & Tidaholm/Hökensås     &   145.6375 &   -0.600 & JO78AD &  &  \\
	Repeater & FM              & SM6UXW/R & Ulricehamn            &   434.6750 &   -2.000 & JO67RT &  &  \\
	Repeater & D-Star          & SK6SA-B  & Guldheden             &   434.5125 &   -2.000 & JO57XQ &  &  \\
	Repeater & FM/C4FM/D-Star  & SK6RKG   & Halmstad              &   434.9250 &   -2.000 & JO66MS &  &  \\
	Repeater & FM              & SK6RPE   & Kungälv               &   434.9000 &   -2.000 & JO57XU &  &  \\
	Repeater & FM              & SM6VBT/R & Mölndal               &   145.7000 &   -0.600 & JO67AP &  &  \\
	Repeater & FM              & SM6VBT/R & Mölndal               &   434.7000 &   -2.000 & JO67AP &  &  \\
	Repeater & FM/C4FM         & SK6EI/R  & Skövde                &   434.8250 &   -2.000 & JO68VK &  &  \\
	Repeater & FM/C4FM         & SK6LK/R  & Borås                 &   434.8000 &   -2.000 & JO67MR &  &  \\
	Repeater & FM/C4FM         & SM6THE/R & Skövde                &   145.6875 &   -0.600 & JO68XJ &  &  \\
	Repeater & FM/C4FM         & SM6UXW/R & Ulricehamn            &   145.6750 &   -0.600 & JO67ST &  &  \\
	Repeater & FM/DMR          & SK6DW/R  & Trollhättan           &   145.7625 &   -0.600 & JO68DG &  &  \\
	Repeater & FM/C4FM         & SK6AG    & Guldheden             &   434.6750 &   -2.000 & JO57XQ &  &  \\
	Repeater & FM              & SL6BH/R  & Halmstad              &   434.7500 &   -2.000 & JO66KQ &  &  \\
	Repeater & FM              & SK6GO/R  & Lunden                &   145.7875 &   -0.600 & JO67AR &  &  \\
	Repeater & FM              & SK6RDG   & Guldheden             &   434.9750 &   -2.000 & JO57XQ &  &  \\
	Repeater & FM              & SK6ROY   & Kinnekulle            &   145.6000 &   -0.600 & JO68QO &  &  \\
	Repeater & FM              & SK6LK/R  & Borås                 &   145.7750 &   -0.600 & JO67MR &  &  \\
	Repeater & FM              & SK6RIC   & Alingsås              &   145.6250 &   -0.600 & JO67GW &  &  \\
	Repeater & FM              & SK6RIC   & Alingsås              &   434.6250 &   -2.000 & JO67GW &  &  \\
	Repeater & FM              & SK6RFQ   & Guldheden             &    51.8700 &   -0.600 & JO57XQ &  &  \\
	Repeater & FM              & SK6RJW   & Kungsbacka            &   145.7250 &   -0.600 & JO67AL &  &  \\
	Repeater & FM              & SK6RFQ   & Guldheden             &    29.6800 &   -0.100 & JO57XQ &  &  \\
	Repeater & FM              & SM6VBT/R & Mölndal               &    29.6900 &   -0.100 & JO67AP &  &  \\
	Repeater & FM/DMR          & SK6RFP   & Bengtsfors            &   145.7000 &   -0.600 & JO69CA &  &  \\
	Repeater & FM/DMR          & SL6ZYW/R & Bengtsfors            &   434.6875 &   -2.000 & JO69CA &  &  \\
	Repeater & FM              & SK6RKI   & Guldheden             &  1297.1500 &   -6.000 & JO57XQ &  &  \\
	Repeater & FM              & SK6IF/R  & Bokenäs               &   145.6000 &   -0.600 & JO58TH &  &  \\
	Repeater & FM              & SK6IF/R  & Lysekil               &   434.8000 &   -2.000 & JO58RG &  &  \\
	Repeater & FM/DMR/D-Star   & SA6APY   & Skara                 &   434.9875 &   -2.000 & JO68RJ &  &  \\
	Repeater & DMR             & SM6TKT/R & Borås                 &   434.5500 &   -2.000 & JO67MR &  &  \\
	Repeater & DMR             & SK6DG    & Alingsås              &   434.5375 &   -2.000 & JO67GV &  &  \\
	Repeater & DMR             & SK6AG    & Guldheden             &   434.7875 &   -2.000 & JO57XQ &  &  \\
	Repeater & FM/DMR          & SA6RP/R  & Floda                 &   434.8250 &   -2.000 & JO67ET &  &  \\
	Repeater & FM/DMR          & SK6IF    & Tanumshede            &   145.5750 &   -0.600 & JO58PR &  &  \\
	Repeater & FM              & SK6RKG   & Halmstad              &   145.6750 &   -0.600 & JO66MS &  &  \\
	Repeater & FM              & SK6JX/R  & Falkenberg            &   145.6250 &   -0.600 & JO66FV &  &  \\
	Repeater & FM              & SK6BA/R  & Skene                 &   145.6000 &   -0.600 & JO67HM &  &  \\
	Repeater & FM              & SK6BA/R  & Skene                 &   434.9500 &   -2.000 & JO67HM &  &  \\
	Repeater & DMR             & SK6RKI   & Kortedala             &   145.5875 &   -0.600 & JO67AS &  &  \\
	Repeater & FM              & SK6RJW   & Kungsbacka            &   434.7250 &   -2.000 & JO67AL &  &  \\
	Repeater & FM/DMR          & SK6QA/R  & Stenungsund           &   145.7125 &   -0.600 & JO58XB &  &  \\
	Repeater & FM/DMR          & SK6DW/R  & Trollhättan           &   434.8750 &   -2.000 & JO68DG &  &  \\
	Repeater & FM              & SK6RFQ   & Guldheden             &   434.6500 &   -2.000 & JO57XQ &  &  \\
	Repeater & FM              & SK6RFQ   & Guldheden             &   145.6500 &   -0.600 & JO57XQ &  &  \\
	Repeater & FM/DMR          & SK6IF    & Kungshamn             &   145.6750 &   -0.600 & JO58PI &  &  \\
	Repeater & DMR             & SK6RKI   & Öckerö                &   434.8500 &   -2.000 & JO57TR &  &  \\
	Repeater & FM              & SK6RKI   & Öckerö                &   145.7500 &   -0.600 & JO57TR &  &  \\
	Repeater & FM/DMR          & SK6QA/R  & Stenungsund           &   434.5625 &   -2.000 & JO58UB &  &  \\
	Repeater & FM              & SG6WAL   & Ytterby               &   145.7875 &   -0.600 & JO57WU &  &  \\
	Repeater & FM              & SM6UDU/R & Uddevalla/Bokenäs     &   434.7750 &   -2.000 & JO58UI &  &  \\
	Repeater & FM/C4FM         & SK6EE/R  & Skara                 &   145.7250 &   -0.600 & JO68RH &  &  \\
	Repeater & FM              & SM6WSC   & Trollhättan           &   434.7250 &   -2.000 & JO68EF &  &  \\
	Repeater & FM/C4FM         & SK6EE/R  & Skara                 &   434.5625 &   -2.000 & JO68RH &  &  \\
	Repeater & FM              & SM6SXJ   & Torup/Galtabo         &   434.8875 &   -2.000 & JO67LA &  &  \\
	Repeater & FM              &          &                       &   434.8625 &   -2.000 & JO67JS &  &  \\
	Repeater & FM              & SK6RIC   & Alingsås              &  1297.0250 &   -6.000 & JO67GV &  &  \\
	Repeater & FM/DMR          & SL6ZAQ   & Uddevalla             &   145.7375 &   -0.600 & JO58WH &  &  \\
	Repeater & FM/C4FM         & SK6WW/R  & Karlsborg             &   145.7625 &   -0.600 & JO78FM &  &
\end{longtable}

\subsubsection{Repeatrar distrikt 7}

\begin{longtable}{llllrrlcl}
	Typ      & Modulation      & Signal   & Ort                     & Utfrekvens &   Duplex & Loc    &  &  \\ \hline
	Hotspot  & D-Star          & SG7WDL-C & Eneryda                 &   145.2125 & Duplex 0 & JO76EQ &  \\
	Hotspot  & D-Star          & SG7HTP-C & Sölvesborg              &   145.2375 &  Simplex & JO76GB &  \\
	Hotspot  & D-Star          & SK7RRV-C & Lönsboda                &   144.8875 & Duplex 0 & JO76DJ &  \\
	Hotspot  & DMR             & SG7WSE   & Ekenässjön              &   144.8500 &  Simplex & JO77ML &  \\
	Link     & FM              & SM7KUY/R & Sölvesborg              &   434.4000 &  Simplex & JO76HB &  \\
	Link     & FM              & SA7AUX   & Linneryd                &   145.4000 &  Simplex & JO76NP &  \\
	Link     & FM              & SM7FLD   & Everöd                  &   145.2375 &  Simplex & JO75BV &  \\
	Link     & FM              & SM5GXQ   & Färjestaden             &   145.2375 &  Simplex & JO86FP &  \\
	Repeater & FM              & SM7GYT/R & Eslöv                   &   434.8125 &   -2.000 & JO65PU &  \\
	Repeater & DMR             & SA7CCO   & Sjöbo                   &   434.9250 &   -2.000 & JO65UP &  \\
	Repeater & D-Star          & SM7XAA   & Malmö                   &   434.5250 &   -2.000 & JO65MN &  \\
	Repeater & FM              & SA7BVQ/R & Eslöv                   &   434.7000 &   -2.000 & JO65PU &  \\
	Repeater & FM              & SK7REP   & Lund/Harderberga        &   145.7750 &   -0.600 & JO65PQ &  \\
	Repeater & FM              & SK7RNQ   & Vitaby                  &   145.6125 &   -0.600 & JO75BQ &  \\
	Repeater & FM              & SK7ROQ   & Gladsax                 &   434.8875 &   -2.000 & JO75DN &  \\
	Repeater & FM              & SK7REE   & Söderåsen/Stenestad     &   145.6500 &   -0.600 & JO66NB &  \\
	Repeater & FM              & SK7REE   & Söderåsen/Stenestad     &    51.8500 &   -0.600 & JO66NB &  \\
	Repeater & FM              & SK7RN/R  & Borgholm                &   145.6625 &   -0.600 & JO86HU &  \\
	Repeater & FM              & SK7RN/R  & Mörbylånga              &   145.6250 &   -0.600 & JO86FM &  \\
	Repeater & FM              & SK7RN/R  & Böda                    &   145.7500 &   -0.600 & JO87MG &  \\
	Repeater & FM              & SK7RFJ   & Karlskrona              &   145.7500 &   -0.600 & JO76TE &  \\
	Repeater & FM              & SK7FK/R  & Karlskrona              &   434.7500 &   -2.000 & JO76TE &  \\
	Repeater & DMR             & SK7HW    & Växjö                   &   434.7000 &   -2.000 & JO76KU &  \\
	Repeater & D-Star          & SK7RGM-B & Asarum                  &   434.7125 &   -2.000 & JO76KF &  \\
	Repeater & DMR/D-Star      & SK7RNQ   & Gladsax                 &   145.5750 &   -0.600 & JO75DN &  \\
	Repeater & FM/C4FM         & SK7BQ/R  & Kristianstad            &   145.7375 &   -0.600 & JO76AA &  \\
	Repeater & FM/C4FM         & SK7REZ   & Blentarp/Romeleåsen     &   145.6750 &   -0.600 & JO65TM &  \\
	Repeater & FM/C4FM         & SK7EM/R  & Blentarp/Romeleåsen     &   434.8500 &   -2.000 & JO65SN &  \\
	Repeater & FM/C4FM         & SK7RGM   & Olofström/Boafallsbacke &   145.7000 &   -0.600 & JO76FF &  \\
	Repeater & DMR/D-Star/C4FM & SK7RQX   & Hallandsås              &   145.7875 &   -0.600 & JO66LI &  \\
	Repeater & FM              & SK7CY    & Helsingborg             &  1297.2000 &   -6.000 & JO66IB &  \\
	Repeater & FM              & SK7IJ/R  & Vetlanda                &   434.6250 &   -2.000 & JO77OL &  \\
	Repeater & FM              & SK7MO/R  & Ljungby                 &   145.7250 &   -0.600 & JO66XV &  \\
	Repeater & FM              & SK7RFH   & Nässjö                  &   434.8500 &   -2.000 & JO77IP &  \\
	Repeater & FM              & SK7RIH   & Oskarshamn              &   145.7250 &   -0.600 & JO87FG &  \\
	Repeater & FM              & SK7RIH/R & Oskarshamn              &   434.7250 &   -2.000 & JO87EG &  \\
	Repeater & FM              & SK7RIH   & Oskarshamn              &    51.9100 &   -0.600 & JO87EG &  \\
	Repeater & FM              & SK7RJL/R & Lund                    &   434.7250 &   -2.000 & JO65OR &  \\
	Repeater & FM              & SK5CN/R  & Hultsfred/Gåskullen     &   145.7625 &   -0.600 & JO77WL &  \\
	Repeater & FM              & SK7RRV   & Lönsboda                &   434.9000 &   -1.600 & JO76DJ &  \\
	Repeater & FM              & SK7RYR   & Gnosjö                  &   145.6875 &   -0.600 & JO67UI &  \\
	Repeater & FM              & SK7UO/R  & Emmaboda                &   145.7750 &   -0.600 & JO76SP &  \\
	Repeater & FM              & SL7ZXW/R & Nybro                   &   145.6875 &   -0.600 & JO76VQ &  \\
	Repeater & FM              & SM7LNT/R & Mörrum                  &   434.8250 &   -2.000 & JO76IE &  \\
	Repeater & FM              & SK7HW/R  & Växjö/Hollstorp         &   145.6750 &   -0.600 & JO76KU &  \\
	Repeater & FM              & SK7IJ/R  & Vetlanda                &   145.6250 &   -0.600 & JO77OL &  \\
	Repeater & FM              & SK7RGI   & Huskvarna               &   434.7500 &   -2.000 & JO77DT &  \\
	Repeater & FM              & SK7RGI   & Jönköping/Taberg        &   145.7500 &   -0.600 & JO77AQ &  \\
	Repeater & FM              & SK7RBK   & Hässleholm/Bjärnum      &   145.7625 &   -0.600 & JO66UG &  \\
	Repeater & FM              & SM7NTJ/R & Aneby                   &   434.7250 &   -2.000 & JO77HU &  \\
	Repeater & FM              & SK7RGI   & Huskvarna               &    29.6800 &   -0.100 & JO77DT &  \\
	Repeater & FM              & SK7RFL   & Algutsrum/Öland         &   434.6000 &   -2.000 & JO86GQ &  \\
	Repeater & FM              & SK7RFH   & Nässjö                  &   145.6500 &   -0.600 & JO77IP &  \\
	Repeater & DMR             & SK7RJL   & Lund                    &   434.5875 &   -2.000 & JO65OR &  \\
	Repeater & DMR             & SG7RFH   & Nässjö                  &   434.9000 &   -2.000 & JO77IP &  \\
	Repeater & DMR             & SG7BNT   & Bruzaholm               &   434.6000 &   -2.000 & JO77PP &  \\
	Repeater & DMR             & SG7RFH   & Nässjö                  &   145.5875 &   -0.600 & JO77IP &  \\
	Repeater & FM/DMR          & SK7REE   & Söderåsen/Stenestad     &   434.6500 &   -2.000 & JO66NB &  \\
	Repeater & FM/DMR          & SK7REE   & Örkelljunga             &   434.9750 &   -2.000 & JO66PG &  \\
	Repeater & FM/D-Star       & SK7JL-B  & Spjutsbygd              &   434.8750 &   -2.000 & JO76TH &  \\
	Repeater & FM              & SK7GH/R  & Värnamo                 &   434.6000 &   -2.000 & JO77AF &  \\
	Repeater & FM              & SM7JPI/R & Svängsta                &   434.9250 &   -2.000 & JO76JE &  \\
	Repeater & DMR             & SK7BQ    & Kristianstad            &   434.5250 &   -2.000 & JO76AA &  \\
	Repeater & DMR             & SA7BIK   & Höör                    &   434.9125 &   -2.000 & JO65SW &  \\
	Repeater & FM              & SM7NTJ/R & Aneby                   &   145.7750 &   -0.600 & JO77HU &  \\
	Repeater & DMR             & SK7REE   & Helsingborg             &   434.6000 &   -2.000 & JO66IA &  \\
	Repeater & DMR             & SK7AF    & Eksjö                   &   434.5625 &   -2.000 & JO77MP &  \\
	Repeater & FM/DMR/D-star   & SK7RBK   & Bjärnum                 &   434.9500 &   -2.000 & JO66UG &  \\
	Repeater & FM/C4FM         & SK7JD/R  & Västervik               &   145.6750 &   -0.600 & JO87HS &  \\
	Repeater & DMR             & SK7RJL   & Malmö                   &   434.7750 &   -2.000 & JO65LO &  \\
	Repeater & FM              & SK7RFL   & Algutsrum/Öland         &   145.6000 &   -0.600 & JO86GQ &  \\
	Repeater & DMR             & SK7RGI   & Jönköping               &   434.9750 &   -2.000 & JO77CS &  \\
	Repeater & DMR             & SK7HR    & Sävsjö                  &   434.5250 &   -2.000 & JO77HJ &  \\
	Repeater & DMR             & SM7NTJ/R & Aneby                   &   434.9250 &   -2.000 & JO77HU &  \\
	Repeater & DMR/D-Star/C4FM & SK7RFL   & Algutsrum/Öland         &   434.5500 &   -2.000 & JO86GQ &  \\
	Repeater & FM              & SK7GH/R  & Värnamo                 &   145.6000 &   -0.600 & JO77AE &  \\
	Repeater & DMR             & SA7BJF/R & Södra Vi                &   434.6625 &   -2.000 & JO77VR &  \\
	Repeater & DMR             & SK7JD    & Västervik               &   434.6750 &   -2.000 & JO87HS &  \\
	Repeater & FM/DMR          & SG7WSE   & Ekenässjön              &   145.7125 &   -0.600 & JO77ML &  \\
	Repeater & FM/DMR          & SA7KSI/R & Tomelilla               &   434.6375 &   -2.000 & JO65XN &  \\
	Repeater & FM/DMR          & SK7DL    & Emmaboda                &   434.7875 &   -2.000 & JO76SP &  \\
	Repeater & FM              & SK7JL    & Spjutsbygd              &   145.7250 &   -0.600 & JO76TH &  \\
	Repeater & D-Star          & SK7RDS   & Malmö                   &   145.5625 &   -0.600 & JO65LO &  \\
	Repeater & D-Star          & SK7DS    & Malmö                   &   434.5125 &   -2.000 & JO65LO &  \\
	Repeater & DMR/D-Star      & SK7RMQ   & Linderöd                &   145.5875 &   -0.600 & JO65VW &  \\
	Repeater & FM              & SM7HZK/R & Moheda                  &   145.6375 &   -0.600 & JO76HX &  \\
	Repeater & DMR/D-Star      & SK7RPQ   & Malmö                   &   434.6125 &   -2.000 & JO65MN &  \\
	Repeater & FM              & SK7RN/R  & Borgholm                &   434.7750 &   -2.000 & JO86HU &
\end{longtable}

\normalsize

\clearpage


%%%%%%%%%% Bandplaner här roterade 90 för enklast läsning %%%%%%%%%%
\begin{landscape}
\subsection{Bandplaner VHF--UHF}
\subsubsection{Bandplan 6m 50--52 MHz}
\begin{tabular}{rrrll}

\textbf{Frekvens} &  & \textbf{BW} & \textbf{Trafik} & \textbf{Noteringar} \\ \hline

50.000 & 50.100 & 500 Hz  & CW          & \textbf{CW anrp. 50.050 och 50.090 (interkont.)}             \\ \hline
50.100 & 50.130 & 2.7 kHz & CW, SSB     & Interkontinental DX-trafik. Ej QSO inom Europa               \\ \hline
50.100 & 50.200 & 2.7 kHz & CW,SSB      & DX 50.110--50.130, \textbf{50.110 50.150 anrop (interkont.)} \\ \hline
50.200 & 50.300 & 2.7 kHz & CW,SSB      & Generell användning. 50.285 för crossband                    \\ \hline
50.300 & 50.400 & 2.7 kHz & CW, MGM     & PSK 50.305, EME 50.310 – 50.320                              \\
       &        &         &             & MS 50.350 – 50.380                                           \\ \hline
50.400 & 50.500 & 1 kHz   & CW, MGM     & Endast fyrar, 50.401 ±500 Hz WSPR-fyrar                      \\ \hline
51.210 & 51.390 & 12 kHz  & FM          & Repeater Repeater in, 20/10 kHz kanalavstånd                 \\
       &        &         &             & RF81 – RF99                                                  \\ \hline
50.500 & 52.000 & 12 kHz  & Alla moder  & SSTV 50.510, RTTY 50.600, FM 51.510                          \\ \hline
51.810 & 51.990 & 12 kHz  & FM Repeater & Repeater ut, 20/10 kHz kanalavstånd                          \\
       &        &         &             & RF81 – RF99                                                  \\ \hline
\end{tabular}

\subsubsection{Bandplan 2m 144--146 MHz}
\begin{tabular}{rrrll}

\textbf{Frekvens} &  & \textbf{BW} & \textbf{Trafik} & \textbf{Noteringar} \\ \hline

144.0000 & 144.1100  & 500 Hz  & CW, EME      & \textbf{CW anrop 144.050}               \\
         &           &         &              & MS random 144.100                       \\ \hline
144.1100 & 144.1500  & 500 Hz  & CW, MGM      & EME MGM 144.120--144.160                \\
         &           &         &              & PSK31 cent. 144.138                     \\ \hline
144.1500 & 144.1800  & 2.7 kHz & CW, SSB, MGM & EME 144.150--144.160                    \\
         &           &         &              & MGM 144.160--144.180 anrop 144.170      \\ \hline
144.1800 & 144.3600  & 2.7 kHz & CW, SSB, MGM & MS SSB random 144.195--144.205          \\
         &           &         &              & \textbf{SSB anrop 144.300}              \\ \hline
144.3600 & 144.3990  & 2.7 kHz & CW, SSB, MGM & MS MGM random anrop 144.370             \\ \hline
144.4000 & 144.4900  & 500 Hz  & Fyr          & Exklusivt segment fyrar, ej QSO         \\ \hline
144.5000 & 144.7940  & 20 kHz  & All mode     & SSTV, RTTY, FAX, ATV                    \\
         &           &         &              & Linjära transpondrar                    \\ \hline
144.7940 & 144.9625  & 12 kHz  & MGM          & APRS 144.800                            \\ \hline
144.9750 & 145.19350 & 12 kHz  & FM, DV       & Rpt in 144.975--145.1935                \\
         &           &         &              & RV46–-RV63, 12.5 kHz, 600 kHz skift     \\ \hline
145.1940 & 145.2060  & 12 kHz  & FM rymd      & 145.200 för kom. m. bem. rymdfark.      \\ \hline
145.2060 & 145.5625  & 12 kHz  & FM, DV       & FM 145.2125-–145.5875  V17–V47          \\
         &           &         &              & \textbf{FM anrop 145.500}, RTTY 145.300 \\
         &           &         &              & FM simpl. INET GW 145.2375, 2875, 3375  \\
         &           &         &              & DV anrop 145.375                        \\ \hline
145.5750 & 145.7935  & 12 kHz  & FM, DV       & Rpt ut 145.575--145.7875                \\
         &           &         &              & RV46–RV63, 12.5 kHz kanalavstånd        \\ \hline
145.794  & 145.806   & 12 kHz  & FM Rymd      & 145.800, 145.200 dplx m. bem. rymdfark. \\ \hline
145.806  & 146.000   & 12 kHz  & All mode     & Exklusivt satellit                      \\ \hline
\end{tabular}

\subsubsection{Bandplan 70cm 432--438 MHz}
\begin{tabular}{rrrll}
	\textbf{Frekvens} &          & \textbf{BW} & \textbf{Trafik} & \textbf{Anmärkning}                               \\ \hline

432.0000 & 432.0250 & 500 Hz  & CW           & EME exklusivt.                                    \\ \hline
432.0250 & 432.1000 & 500 Hz  & CW, PSK31    & CW mellan 432.000--085, \textbf{CW anrop 432.050} \\
         &          &         &              & PSK31 432.088                                     \\ \hline
432.1000 & 432.3990 & 2.7 kHz & CW, SSB, MGM & \textbf{SSB anrop 432.200}                        \\
         &          &         &              & Mikrovåg talkback 432.350, FSK441 432.370         \\ \hline
432.4000 & 432.4900 & 500 Hz  & Fyr          & Exklusivt segment för fyrar                       \\ \hline
432.5000 & 432.5940 & 12 kHz  & All mode     & Linjära transpondrar IN 432.500--600              \\ \hline
432.5000 & 432.5750 & 12 kHz  & All mode     & NRAU Digital rep. in 432.500--575 2 MHz skift     \\ \hline
432.5940 & 432.9940 & 12 kHz  & All mode     & Linjära transpondrar ut 432.600--800              \\ \hline
432.5940 & 432.9940 & 12 kHz  & FM           & Rep. in 432.600--975 RU368--398 2 MHz skift       \\ \hline
432.9940 & 433.3810 & 12 kHz  & FM           & Rep. in 433.000--375 RU368--398 1.6 MHz skift     \\ \hline
433.3940 & 433.5810 & 12 kHz  & FM           & SSTV (FM/AFSK) 433.400                            \\
         &          &         &              & FM simplex U272--286 \textbf{anrop 433.500}       \\ \hline
433.6000 & 434.0000 & 20 kHz  & All mode     & RTTY (FM/AFSK) 433.600                            \\
         &          &         &              & FAX 433.700, APRS 433.800                         \\ \hline
434.0000 & 434.4940 & 20 kHz  & All mode     & NRAU Dig. kanaler 433.450, 434.475                \\ \hline
434.5000 & 434.5940 & 20 kHz  & All mode     & NRAU Dig. rep. ut 434.500--575, 2 MHz skift       \\ \hline
434.5940 & 434.9810 & 12 kHz  & FM           & NRAU Rep. ut 434.600--975 RU 368--RU398           \\
         &          &         &              & 12,5 kHz med 2 MHz skift                          \\ \hline
435.000  & 438.000  & 20 kHz  & All mode     & Exklusivt satellit\\
\end{tabular}

\subsubsection{Bandplan 23cm 1240--1300 MHz}
\begin{tabular}{rrrll}
	\textbf{Frekvens}         &               & \textbf{BW} & \textbf{Trafik} & \textbf{Anmärkning}                                          \\ \hline
	         1240.000         & 1243.250      & 20 kHz      & Alla moder      & 1240.000 - 1241.000 Digital kommunikation                    \\ \hline
	         1243.250         & 1260.000      & 20 kHz      & ATV och Data    & Repeater ut 1258.150-1259.350, R20--68                       \\ \hline
	         1260.000         & 1270.000      & 12 kHz      & Satellit        & Endast för satelliter alla moder                             \\ \hline
	         1270.000         & 1272.000      & 20 kHz      & Alla moder      & Repeater in, 1270.025-1270.700, RS1--28                      \\
                                  &               &             &                 & Packet RS29--50                                              \\ \hline
	         1272.000         & 1290.994      & 20 kHz      & ATV och Data    & Amatörtelevision ATV                                         \\ \hline
	         1290.994         & 1291.481      & 20 kHz      & FM och DV       & Repeater in Repeat. in 1291.000--1291.475                    \\
                                  &               &             &                 & RM0 – RM19, 25 kHz, 6 MHz skift                              \\ \hline
	         1291.494         & 1296.000      & 12 kHz      & Alla moder      &                                                              \\ \hline
	         1296.000         & 1296.150      & 500 Hz      & CW,  MGM        & EME 1296.000--025, \textbf{CW anrop 1296.050}                \\
                                  &               &             &                 & PSK31 1296.138 MHz                                           \\ \hline
	         1296.150         & 1296.400      & 2.7 kHz     & CW, SSB, MGM    & \textbf{SSB anrop 1296.200}                                  \\
                                  &               &             &                 & \textbf{FSK441 MS anrop 1296.370}                            \\ \hline
	         1296.400         & 1296.600      & 2.7 kHz     & CW, SSB, MGM    & Linjära transpondrar infrekvens                              \\ \hline
	         1296.600         & 1296.800      & 2.7 kHz     & CW, SSB, MGM    & SSTV/FAX 1296.500, MGM/RTTY 1296.600                         \\ \hline
	         1296.600         & 1296.800      & 2.7 kHz     & CW, SSB, MGM    & Linjära transpondrar utfrekvens                              \\
                                  &               &             &                 & 1296.750-.800 lokala fyrar max 10 W                          \\ \hline
	         1296.800         & 1296.994      & 500 Hz      & Fyrar           & Exklusivt segment för fyrar                                  \\ \hline
	         1296.994         & 1297.481      & 20 kHz      & FM              & Repeater ut Repeater ut 1297.000--1297.475                   \\
                                  &               &             &                 & RM0 – RM19, 25 kHz, 6 MHz skift                              \\ \hline
	         1297.494         & 1297.981      & 20 kHz      & FM simplex      & Simplex 25 kHz kanaler SM20--39                              \\
                                  &               &             &                 & \textbf{FM anrop 1297.500 SM20}                              \\ \hline
	         1299.000         & 1299.750      & 150 kHz     & Alla moder      & 5 st 150 kHz kanaler för DD,                                 \\
                                  &               &             &                 & 1299.075, 225, 375, 525, och 675 $\pm$75 kHz                 \\ \hline
	         1299.750         & 1300.000      & 20 kHz      & Alla moder      & 8 st FM/DV 25 kHz kan. 1299.775--1299.975
\end{tabular}
\end{landscape}
\clearpage



%\begin{landscape}
%\small
%\subsubsection{PTS Bandplan för VHF}
%\begin{longtable}{llll}
%	\textbf{Frekvens}  & \textbf{Duplex}    & \textbf{Användning}                   & \textbf{Anmärkning}            \\ \hline
%	\endhead
%	30.0--37.5         &                    & Medicinska implantat                  & Undantag f. tillst.pl.         \\
%	                   &                    &                                       & 2006/771/EG 2013/752/EU        \\
%	30.01--68          &                    & Militär användning                    &                                \\
%	30.015--30.025     &                    & Telemetri och Radiostyrning           & Undantag från tillståndsplikt  \\
%	                   &                    &                                       & radiost. trafikljus            \\
%	30,265--30,355     &                    & Telemetri och radiostyrning           & Undantag från tillståndsplikt  \\
%	30,925--31,375     &                    & Landmobil radio                       & Undantag från tillståndsplikt  \\
%	34,995--35,275     &                    & Radiostyrning av modellflygplan       & Undantag från tillståndsplikt  \\
%	37,5--41,015       &                    & Landmobil radio                       &                                \\
%	37,5--38,25        &                    & Radioastronomi                        & Onsala rymd - observatorium    \\
%	39,525--39,55      &                    & Telemetri och radiostyrning           & Undantag från tillståndsplikt  \\
%	40,45--40,575      &                    & Telemetri och radiostyrning           & Undantag från tillståndsplikt  \\
%	40,66--40,7        &                    & Allmän kortdistansradio               & Undantag f. tillst.pl.         \\
%	                   &                    &                                       & 2006/771/EG 2013/752/EU        \\
%	40,66--40,7        &                    & ISM                                   &                                \\
%	40,66--40,8        &                    & Telemetri och radiostyrning           & Undantag från tillståndsplikt  \\
%	41--43,6           &                    & Ljudöverföring                        & Undantag från tillståndsplikt  \\
%	50--52             &                    & Amatörradio                           & Undantag från tillståndsplikt  \\
%	                   &                    &                                       & (Max 200 W)                    \\
%	68--74,8           &                    & Landmobil radio                       &                                \\
%	68--74,8           &                    & PMR                                   &                                \\
%	69,6--69,725       &                    & Landmobil radio                       & Undantag från tillståndsplikt  \\
%	69,6--69,725       &                    & PMR	Undantag från tillståndsplikt   &                                \\
%	73--74,6           &                    & Radioastronomi                        & Onsala rymd - observatorium    \\
%	74,8--75,2         &                    & Radionavigering för luftfart          &                                \\
%	74,8--75,2         &                    & Luftfartsradio (ILS)                  &                                \\
%	75,2--75,625       &                    & Landmobil radio                       &                                \\
%	75,2--75,625       &                    & PMR                                   &                                \\
%	75,625--77,475     & 80.625--82.475     & Landmobil radio                       &                                \\
%	75,625--77,475     & 80.625--82.475     & PMR                                   &                                \\
%	77,475--80,625     &                    & Landmobil radio                       &                                \\
%	77,475--80,625     &                    & PMR                                   &                                \\
%	80,625--82,475     & 75.625--77.475     & Landmobil radio                       &                                \\
%	80,625--82,475     & 75.625--77.475     & PMR                                   &                                \\
%	82,475--87,5       &                    & Landmobil radio                       &                                \\
%	82,475--87,5       &                    & PMR                                   &                                \\
%	87,5--108          &                    & Rundradio	Band II, GE84           &                                \\
%	87,5--108          &                    & Ljudöverföring                        & Undantag från tillståndsplikt  \\
%	                   &                    &                                       & 2006/771/EG 2013/752/EU        \\
%	87,5--108          &                    & FM analog ljudradio	Band II, GE84   &                                \\
%	108--117,975       &                    & Radionavigering för luftfart          &                                \\
%	108--117,975       &                    & Luftfartsradio (ILS)                  &                                \\
%	108--117,975       &                    & Luftfartsradio (VOR)                  &                                \\
%	117,975--137       &                    & Luftfartsradio                        &                                \\
%	121,45--121,55     &                    & Sjöfartsradio (EPIRB)                 & Positionsbestämning;           \\
%	                   &                    &                                       & undantag f. tillst.pl.         \\
%	                   &                    &                                       & COSPAS/SARSAT-sändare          \\
%	121,45--121,55     &                    & PLB                                   & Positionsbestämning;           \\
%	                   &                    &                                       & undantag f. tillst.pl.         \\
%	                   &                    &                                       & COSPAS/SARSAT-sändare          \\
%	137--138           &                    & Satellit                              & MSS nedlänk ERC/DEC/(99)05     \\
%	138--144           &                    & Militär användning                    &                                \\
%	138--143,6         &                    & Landmobil radio                       &                                \\
%	144--146           &                    & Amatörradio                           & Undantag från tillst.pl.       \\
%	146--149,9         &                    & Militär användning                    &                                \\
%	146,05--147        & 150.05--151        & Landmobil radio                       &                                \\
%	146,05--147        & 150.05--151        & PMR                                   &                                \\
%	147--149,9         &                    & Landmobil radio                       &                                \\
%	148--150,05        &                    & Jordstationer (mobila satellitsystem) & MSS upplänk                    \\
%	                   &                    &                                       & Undantag från tillst.pl.       \\
%	                   &                    &                                       & ERC/DEC/(99)05                 \\
%	                   &                    &                                       & ERC/DEC/(99)06                 \\
%	150,05--155,9875   &                    & Militär användning                    &                                \\
%	150,05--151        & 146.05--147        & Landmobil radio                       &                                \\
%	150,05--153        &                    & Radioastronomi                        & Onsala rymd - observatorium    \\
%	150,05--151        & 146.05--147        & PMR                                   &                                \\
%	151--153           &                    & Landmobil radio                       &                                \\
%	151,52--151,555    &                    & Spårning, pejling och datainsamling   & Undantag från tillståndsplikt  \\
%	151,52--152,2675   &                    & Radiopejling av djur                  & Undantag från tillståndsplikt  \\
%	154--155,9875      &                    & Sjöfartsradio                         &                                \\
%	154--155,9875      &                    & Landmobil radio                       &                                \\
%	155,3875--155,5375 &                    & Landmobil radio                       & Undantag från tillståndsplikt; \\
%	                   &                    &                                       & jord- och skogsbruk samt jakt  \\
%	155,9875--156,0125 &                    & Landmobil radio                       & Undantag från tillståndsplikt  \\
%	156,0125--156,3625 & 160.6125--160.9625 & Sjöfartsradio                         & RR AP18                        \\
%	156,3625--156,7625 &                    & Sjöfartsradio                         & RR AP18                        \\
%	156,4875--156,7625 &                    & Landmobil radio                       &                                \\
%	156,5125--156,5375 &                    & Sjöfartsradio (DSC)                   & RR AP18 Nödanrop               \\
%	156,7625--156,8375 &                    & Sjöfartsradio                         & RR AP18, Kanal 16              \\
%	156,8375--156,8875 &                    & Sjöfartsradio                         &                                \\
%	156,8375--158,075  &                    & Landmobil radio                       &                                \\
%	156,8375--158,075  &                    & PMR                                   &                                \\
%	156,8875--157,4375 & 161.4875--162.0375 & Sjöfartsradio                         & RR AP18                        \\
%	158,075--160,6125  & 166.075--168.6125  & Landmobil radio                       &                                \\
%	158,075--160,625   & 166.075--168.625   & Landmobil radio                       &                                \\
%	158,075--160,625   & 166.075--168.625   & PMR                                   &                                \\
%	158,075--160,6125  & 166.075--168.6125  & PMR                                   &                                \\
%	160,6125--160,9625 & 156.0125--156.3625 & Sjöfartsradio                         & RR AP18                        \\
%	160,6125--162,0375 &                    & Landmobil radio                       &                                \\
%	160,6125--162,0375 &                    & PMR                                   &                                \\
%	161,4875--162,0375 & 156.8875--157.4375 & Sjöfartsradio                         & RR AP18                        \\
%	161,85--162,15     &                    & Sjöfartsradio (AIS)                   & RR AP18 ERC/DEC(99)17          \\
%	162,025--163       & 170.025--171       & Landmobil radio                       &                                \\
%	162,025--163       & 170.025--171       & PMR                                   &                                \\
%	163--164           & 171--172           & Militär användning                    &                                \\
%	164--166           & 172--174           & Landmobil radio                       &                                \\
%	164--166           & 172--174           & PMR                                   &                                \\
%	166--166,075       &                    & Landmobil radio                       &                                \\
%	166--166,075       &                    & PMR                                   &                                \\
%	166,075--168,6125  & 158.075 - 160.6125 & Landmobil radio                       &                                \\
%	166,075--168,625   & 158.075 - 160.625  & Landmobil radio                       &                                \\
%	166,075--168,625   & 158.075 - 160.625  & PMR                                   &                                \\
%	166,075--168,6125  & 158.075 - 160.6125 & PMR                                   &                                \\
%	168,625--170,025   &                    & Landmobil radio                       &                                \\
%	168,625--170,025   &                    & Personsökning                         &                                \\
%	168,625--170,025   &                    & PMR                                   &                                \\
%	169,375--169,4     &                    & Allmän kortdistansradio               & Undantag från tillståndsplikt  \\
%	169,4--169,8125    &                    & Allmän kortdistansradio               & Undantag från tillståndsplikt  \\
%	                   &                    & 2006/771/EG 2013/752/EU               &                                \\
%	169,4--169,475     &                    & Hörselhjälpmedel                      & Undantag från tillståndsplikt  \\
%	                   &                    &                                       & 2006/771/EG 2013/752/EU        \\
%	169,4--169,475     &                    & Mätaravläsning                        & Undantag från tillståndsplikt  \\
%	                   &                    &                                       & 2006/771/EG 2013/752/EU        \\
%	169,4875--169,5875 &                    & Hörselhjälpmedel                      & Undantag från tillståndsplikt  \\
%	                   &                    &                                       & 2006/771/EG 2013/752/EU        \\
%	170,025--171       & 162.025--163       & Landmobil radio                       &                                \\
%	170,025--171       & 162.025--163       & PMR                                   &                                \\
%	171--172           & 163--164           & Militär användning                    &                                \\
%	172--174           & 164--166           & Landmobil radio                       &                                \\
%	172--174           & 164--166           & PMR                                   &                                \\
%	174--223           &                    & Rundradio (DVB-T)                     & Äv. DVB-T2 Band III, k. 5--11  \\
%	174--223           &                    & Rundradio (T-DAB)                     & Äv. DVB-T2 Band III, k. 5--11  \\
%	223--230           &                    & Rundradio (DVB-T)                     & Äv. DVB-T2 Band III, k. 12     \\
%	223--230           &                    & Rundradio (T-DAB)                     & Äv. DVB-T2 Band III, k. 12     \\
%	230--240           &                    & Rundradio (T-DAB)                     & WI95revCO07                    \\
%	240--267           &                    & Landmobil radio                       &                                \\
%	242,95--243,05     &                    & SAR (kommunikation)                   & 
%\end{longtable}
%\normalsize
%\end{landscape}


\begin{landscape}
\subsubsection{PTS bandplan för UHF}
\begin{longtable}{llll}
	\textbf{Frekvens}  & \textbf{Duplex}    & \textbf{Användning}     & \textbf{Anmärkning}                   \\ \hline
	\endhead
	328,6--335,4       &                    & Radionavigering         & För luftfart                          \\
	328,6--335,4       &                    & Luftfartsradio (ILS)    &                                       \\
	335,4--386         &                    & Militär användning      &                                       \\
	336--339           & 357--360           & Landmobil radio         &                                       \\
	336--339           & 357--360           & SAP/SAB                 & bärbara ljudlänkar                    \\
	357--360           & 336--339           & Landmobil radio         &                                       \\
	357--360           & 336--339           & SAP/SAB                 & bärbara ljudlänkar                    \\
	370--371           & 386--387           & Fast radio              & Svensk kanalplan Smalbandig           \\
	374--376           & 387--389           & Fast radio              & Svensk kanalplan Smalbandig           \\
	380--385           & 390--395           & Landmobil radio         & Blocktillstånd Rakel                  \\
	380--385           & 390--395           & TETRA                   & Blocktillstånd Rakel                  \\
	386--387           & 370--371           & Fast radio              & Svensk kanalplan Smalbandig           \\
	387--389           & 374--376           & Fast radio              & Svensk kanalplan Smalbandig           \\
	389--390           &                    & Militär användning      &                                       \\
	390--395           & 380--385           & Landmobil radio         & Blocktillstånd Rakel                  \\
	390--395           & 380--385           & TETRA                   & Blocktillstånd Rakel                  \\
	395--399,9         &                    & Militär användning      &                                       \\
	399,9--400,05      &                    & Satellitnavigering      &                                       \\
	400,05--400,15     &                    & Satellit                & Standardfrekvens och tidssignal       \\
	400,15--401        &                    & Satellit                & MSS nedlänk ERC/DEC/(99)05            \\
	400,15--404        &                    & Radiosonder             & Ballongburna sonder                   \\
	400,15--404        &                    & Meteorologi             &                                       \\
	401--406           &                    & Medicinska implantat    & Undantag från tillståndsplikt         \\
	402--403           &                    & Meteorologi             & Via satellit                          \\
	404--406           &                    & Landmobil radio         &                                       \\
	406--406,1         &                    & Sjöfartsradio (EPIRB)   & Undantag från tillståndsplikt         \\
	                   &                    &                         & (COSPAS-SARSAT)                       \\
	406--406,1         &                    & PLB                     & Undantag från tillståndsplikt         \\
	                   &                    &                         & (COSPAS-SARSAT)                       \\
	406,1--410         &                    & Landmobil radio         &                                       \\
	406,1--410         &                    & PMR                     &                                       \\
	410--420           & 420--430           & Landmobil radio         &                                       \\
	410--420           & 420--430           & PMR                     &                                       \\
	410--420           & 420--430           & TETRA                   &                                       \\
	420--430           & 410--420           & Landmobil radio         &                                       \\
	420--430           & 410--420           & PMR                     &                                       \\
	420--430           & 410--420           & TETRA                   &                                       \\
	429,4375--429,4625 &                    & Larmöverföring          & Undantag från tillståndsplikt         \\
	430--432           &                    & Landmobil radio         &                                       \\
	432	438        &                    & Amatörradio             & Undantag från tillståndsplikt         \\
	433,05--434,79     &                    & Allmän kortdist.radio   & Undantag från tillståndsplikt         \\
	                   &                    &                         & 2006/771/EG 2013/752/EU               \\
	438--439,6875      &                    & Telemetri och radiosty. &                                       \\
	438--439,6875      &                    & Kortdistansradio (SRD)  &                                       \\
	439,6875--439,9875 &                    & Telemetri och radiost.  & Undantag från tillståndsplikt         \\
	440--442           &                    & Landmobil radio         &                                       \\
	442--444           & 448--450           & Landmobil radio         &                                       \\
	442--444           & 448--450           & PMR                     &                                       \\
	443,9875--444,4125 &                    & Telemetri och radiost.  & Undantag från tillståndsplikt         \\
	444--444,5875      &                    & Landmobil radio         &                                       \\
	444--444,5875      &                    & Telemetri och radiost.  &                                       \\
	444,5875--444,9875 &                    & Landmobil radio         & Undantag från tillståndsplikt         \\
	444,5875--448      &                    & Landmobil radio         &                                       \\
	444,5875--448      &                    & PMR                     &                                       \\
	446--446,1         &                    & PMR 446                 & Undantag från tillståndsplikt         \\
	446,1--446,2       &                    & Landmobil radio         & Undantag från tillståndsplikt         \\
	                   &                    &                         & Digital PMR ECC/DEC/(05)12            \\
	448--450           & 442--444           & Landmobil radio         &                                       \\
	448--450           & 442--444           & PMR                     &                                       \\
	450--452,5         & 460--462.5         & PMR                     &                                       \\
	452,5--457,5       & 462.5--467.5       & CDMA 450                & Digitala cellulära system             \\
	457,5--460         & 467.5--470         & PMR                     &                                       \\
	457,5125--457,5875 & 467.5125--467.5875 & Kom. ombord fartyg      & 3 kanaler 25 kHz                      \\
	                   &                    &                         & 5 kanaler 12,5 kHz                    \\
	460--462,5         & 450--452.5         & PMR                     &                                       \\
	462,5--467,5       & 452.5--457.5       & CDMA 450                & Digitala cellulära system             \\
	467,5--470         & 457.5--460         & PMR                     &                                       \\
	467,5125--467,5875 & 457.5125--457.5875 & Kom. ombord  fartyg     & 3 kanaler 25 kHz                      \\
	                   &                    &                         & 5 kanaler 12,5 kHz                    \\
	470--790           &                    & Ljudöverföring          & ERC/REC 70-03, Annex 10 e             \\
	470--790           &                    & Rundradio (DVB-T)       & Äv. DVB-T2 Band IV/V, k. 21--60,      \\
	                   &                    &                         & 470-694 MHz tom 31 mars 2020;         \\
	                   &                    &                         & 694-790 MHz tom 31 mars 2017          \\
	790--862           &                    & TRA-ECS	Block     & Undantag från tillståndsplikt         \\
	823--832           &                    & Ljudöverföring          & Undantag från tillståndsplikt         \\
	862--863           &                    & Övrigt                  &                                       \\
	863--865           &                    & Allmän kortdistansradio & Undantag från tillståndsplikt         \\
	863--865           &                    & Ljudöverföring          & Undantag från tillståndsplikt         \\
	865--868           &                    & RFID                    & Undantag från tillståndsplikt         \\
	865--868           &                    & Allmän kortdist.radio   & Undantag från tillståndsplikt         \\
	868--870           &                    & Allmän kortdist.radio   & Undantag från tillståndsplikt         \\
	868,6--869,7       &                    & Larmöverföring          & Undantag från tillståndsplikt         \\
	869,2--869,25      &                    & Trygghetslarm           & Undantag från tillståndsplikt         \\
	870,5375--870,6625 &                    & Telemetri och radiost.  & Undantag från tillståndsplikt         \\
	871--876           &                    & Kortdistansradio (SRD)  &                                       \\
	876--880           & 921--925           & GSM-R                   & MobiSIR telefon till bas              \\
	880--915           & 925--960           & E-GSM                   & Telefon till bas                      \\
	915--916           &                    & Övrigt                  & Skyddband GSM 900                     \\
	916--921           &                    & Kortdistansradio (SRD)  &                                       \\
	921--925           & 876--880           & GSM-R                   & MobiSIR Bas till telefon              \\
	925--960           & 880--915           & E-GSM                   & Bas till telefon                      \\
	960--1164          &                    & Luftfart                & Radionavigering                       \\
	960--1215          &                    & Militär användning      &                                       \\
	960--1215          &                    & Luftfartsradio (DME)    &                                       \\
	1026--1034         & 1086--1094         & Radionavigering         & För luftfart                          \\
	1026--1034         & 1086--1094         & Luftfartsradio (SSR)    &                                       \\
	1086--1094         & 1026--1034         & Radionavigering         & För luftfart                          \\
	1086--1094         & 1026--1034         & Luftfartsradio (SSR)    &                                       \\
	1164--1350         &                    & Radionavigering         & För luftfart                          \\
	1164--1191,795     &                    & Satellitnavigering      & GPS                                   \\
	1164--1215         &                    & Satellitnavigering      & GALILEO                               \\
	1191,795--1215     &                    & Satellitnavigering      & GLONASS                               \\
	1215--1355         &                    & Radiolokalisering       &                                       \\
	1215--1237         &                    & Satellitnavigering      & GPS                                   \\
	1237--1260         &                    & Satellitnavigering      & GLONASS                               \\
	1240--1300         &                    & Amatörradio             & Undantag från tillståndsplikt         \\
	1260--1300         &                    & Satellitnavigering      & GALILEO                               \\
	1300--1375         &                    & Militär användning      &                                       \\
	1330--1427         &                    & Radioastronomi          & Onsala rymd - observatorium           \\
	1375--1400         & 1427--1452         & Fast radio              & Kanalplan T/R 13-01                   \\
	1400--1427         &                    & Satellit                & Jordutforskning via satellit          \\
	1427--1452         & 1375--1400         & Fast radio              & Kanalplan T/R 13-01                   \\
	1452--1492         &                    & MFCN                    &                                       \\
	1492--1544         &                    & Militär användning      &                                       \\
	1518--1525         &                    & Satellit                & MSS nedlänk                           \\
	1525--1559         &                    & Satellit                & MSS nedlänk                           \\
	1544--1545         &                    & Sjöfartsradio           & Nöd- och säkerhetstrafik              \\
	1545--1559         &                    & Militär användning      &                                       \\
	1559--1591         &                    & Satellitnavigering      & GALILEO                               \\
	1563--1587         &                    & Satellitnavigering      & GPS                                   \\
	1591--1610         &                    & Satellitnavigering      & GLONASS                               \\
	1610--1613,8       &                    & Satellit                & Jordstationer (mobila sat.system)     \\
	                   &                    &                         & MSS upplänk                           \\
	                   &                    &                         & Undantag från tillståndsplikt         \\
	1610,6--1613,8     &                    & Radioastronomi          & Onsala rymd - observatorium           \\
	1613,8--1626,5     &                    & Satellit                & Jordstationer (mobila sat.system)     \\
	                   &                    &                         & MSS upplänk                           \\
	                   &                    &                         & Undantag från tillståndsplikt         \\
	1626,5--1645,5     &                    & Satellit                & Jordstationer (mobila sat.system)     \\
	                   &                    &                         & MSS upplänk                           \\
	                   &                    &                         & Undantag från tillståndsplikt         \\
	1645,5--1646,5     &                    & Sjöfartsradio           & Nöd- och säkerhetstrafik              \\
	                   &                    &                         &                                       \\
	1646,5--1660,5     &                    & Satellit                & Jordstationer (mobila sat.system)     \\
	                   &                    &                         & MSS upplänk                           \\
	                   &                    &                         & Undantag från tillståndsplikt         \\
	1660--1670         &                    & Radioastronomi          & Onsala rymd - observatorium           \\
	1668--1670         &                    & Satellit                & Jordstationer (mobila sat.system)     \\
	                   &                    &                         & MSS upplänk                           \\
	1670--1675         &                    & Satellit                & Jordstationer (mobila sat.system)     \\
	                   &                    &                         & MSS upplänk                           \\
	1675--1710         &                    & Satellit                & Meteorologi via satellit              \\
	1710--1785         & 1805--1880         & DCS1800                 & Mobiltel. 1800-bandet mobil till bas  \\
	1718,8--1722,2     &                    & Radioastronomi          & Onsala rymd - observatorium           \\
	1785--1805         &                    & Ljudöverföring          &                                       \\
	1805--1880         & 1710--1785         & DCS1800                 & Mobiltel. 1800-bandet bas till mobil  \\
	1880--1900         &                    & DECT                    & Undantag från tillståndsplikt         \\
	1900--1920         &                    & UMTS                    & Mobiltel. 2100-bandet TDD             \\
	1920--1980         & 2110--2170         & UMTS                    & Mobiltel. 2100-bandet mobil till bas  \\
	1980--2010         &                    & Satellit                & Jordstationer (mobila satellitsystem) \\
	                   &                    &                         & MSS upplänk Blocktillstånd            \\
	                   &                    &                         & Undantag från tillståndsplikt         \\
	2010--2025         &                    & ?                       & Digitala cellulära system             \\
	2025--2110         &                    & Militär användning      &                                       \\
	2025--2110         &                    & Rymdfarkostkontroll     & Upplänk Satellit-satellit             \\
	2110--2170         & 1920--1980         & UMTS                    & Mobiltel. 2100-bandet bas till mobil  \\
	2170--2200         &                    & Satellit                & MSS nedlänk Blocktillstånd            \\
	2200--2300         &                    & Militär användning      &                                       \\
	2200--2290         &                    & Rymdfarkostkontroll     & Nedlänk Satellit-satellit             \\
	2300--2400         &                    & MFCN                    & ECC/DEC/(14)02                        \\
	2400--2483,5       &                    & Militär användning      &                                       \\
	2400--2450         &                    & Amatörradio             & Undantag från tillståndsplikt         \\
	                   &                    &                         & Max 100 mW                            \\
	2400--2483,5       &                    & Landmobil radio         &                                       \\
	2400--2483,5       &                    & Allmän kortdist.radio   & Undantag från tillståndsplikt         \\
	2400--2483,5       &                    & ISM                     & Mikrovågsugnar, bluetooth, wifi...    \\
	2400--2483,5       &                    & Radiobestämning         & Undantag från tillståndsplikt         \\
	2400--2483,5       &                    & Dataöverföring          & WIFI2100-bandet                       \\
	                   &                    &                         & Undantag från tillståndsplikt         \\
	2446--2454         &                    & RFID                    & Undantag från tillståndsplikt         \\
	2483,5--2500       &                    & Medicinska implantat    & Undantag från tillståndsplikt         \\
	2483,5--2500       &                    & Satellit                & MSS nedlänk                           \\
	2500--2690         &                    & LTE-2600                & Mobiltelefoni 2600-bandet             \\
	2655--2690         &                    & Radioastronomi          & Onsala rymd - observatorium           \\
	2690--2700         &                    & Radioastronomi          & Onsala rymd - observatorium           \\
	2700--2900         &                    & Radionavigering         & För luftfart                          \\
	2700--2900         &                    & Radiolokalisering       & 
\end{longtable}
\end{landscape}


\end{document}




	%\subsection{QSO}

Konsten att genomföra ett radiosamtal (QSO) i olika sammanhang. Ofta
blir folk nervösa i början för hur detta går till. Man säger sin
signal och motstationens i fel ordning eller liknande.

Man börjar alltid med motstationens signal. Det bör fallas naturligt
att ropa så och man avslutar anropet med sin egen signal så att
motstationen dels vet vem som anropar men också andra hör. Kanske vill
en annan station ha ett utbyte med dig om du inte får svar från den
tilltänkta.

Ett radiosamtal består som regel av tre delar. Först sker ett anrop, när kontakt etablerats utväxlas ett antal meddelande (dialog) och när man är klarar avslutas samtalet. Dessa tre delar är ganska standard. Man följer detta ganska strikt t.ex. på kortvågen där telefoni oftast innebär SSB. Anledningen är enkel, det går inte höra när någon släpper sändtangenten eller bara är tyst och tänker.

När man kör FM över repeatrar på VHF/UHF är det inte lika vanligt att man både öppnar och avslutar varje sändning med motparten och sin egen signal. Men man skall regelbundet upprepa signalerna och i praktiken är det lämpligt att göra kanske var femte minut eller oftare.

\subsubsection{Anropet}

Ett anrop kan se ut ungefär såhär:

-- SA0MAD från SM0UEI, SA0MAD kom!

Här är det SA0MAD som anropas av SM0UEI. 

Svaret kan se ut ungefär såhär:

-- SM0UEI från SA0MAD kom!

Därefter övergår radiosamtalet i dialog eller meddelandesändning.

\subsubsection{Allmänt anrop} 

Används när man inte ropar på någon särskild motstation utan önskar samtal med vem som helst. På svenska använder man ofta just orden ''allmänt anrop'' medan på engelska är det vanligare att man uttalar CQ (seek you). Ett allmänt androp kan se ut såhär:

-- Allmänt anrop, allmänt anrop, allmänt anrop från SM0UEI SM0UEI SM0UEI kallar allmänt anrop och lyssnar.

Eller på engelska:

-- CQ CQ CQ this is SM0UEI calling CQ CQ CQ and standing by.

\subsubsection{Meddelandesändning}

-- SA0MAD från SM0UEI, tack för svaret. Din signal är 59 hos mig, mitt QTH är JO89WA och namnet är Anders. SA0MAD från SM0UEI kom.

-- SM0UEI från SA0MAD, tack för rapporten. Din signal är 57 hos mig, jag befinner mi i JO89VK men kommer under kvällen byta QTH. Jag kommer då vara QRV på 3663 kHz. QSL? SM0UEI från SA0MAD.

-- SA0MAD från SM0UEI, QSL på det, QRX 19.30 på frekvens 3663 kHz. 

\subsubsection{Avslutning}

-- SA0MAD från SM0UEI, tack för rapport och vi hörs senare, 73, slut kom

-- SM0UEI från SA0MAD, 73 tillbaka, klart slut.

\subsubsection{Pile-up}

Ibland kan det bli väldigt många motstationer samtidigt som ropar. Nu gäller det att spetsa öronen! Först gäller det att sålla. Rara signaler från långtbortistan ger mer poäng i en contest som regel eller från länder du inte kört osv beroende på regler. Försök att sålla med "du som sänder från Florida" eller "VK7 kom igen" osv till det är en station kvar. Kör den snabbt, ropa CQ igen och börja sålla.

Direkt när det uppstår en pile-up är det effektivt att köra split. Dvs du lyssnar 5-10 kHz upp eller ned från den frekvens du sänder på. Det gör det lättare för dig att behålla kommandot under pile-up. Ligger du och sänder i ett frekvensområde som är särskilt ägnat för DX är det smart att lägga Rx-frekvensen strax utanför. Det undviker att man stökar ned i DX-bandet.

Kör du split skall du säga det efter varje sändning. "CQ CQ CQ de Sierra Mike Zero Uniform Echo India listening 5 up" exempelvis. På CW bör en split vara minst 2 kHz och på SSB bör den vara minst 5 kHz ännu hellre 10 kHz. Tänk på att när du startar din split måste du kolla så att båda frekvenserna är ok. Låt inte din pile-up sprida ut sig för mycket även om det är kanske enklare för dig så är risken stor att den stör någon annan. 

Kör korta QSO. Utbyt snabbt den information som behövs och ta sedan nästa. Ha förståelse för att det kan bli krockar i en pile-up. När du hör en partiell signal eller station du vill prata med håll fast vid den. Om du har svårt att läsa den be den repetera tills ni är klara. Genom att du är auktoriteten på frekvensen kommer pile-up:en att lugna ned sig och vänta på sin tur. Om du ''hattar omkring'' är risken att all radiodisciplin far ut genom fönstret.

Använd ett standardmönster när du kör:

-- SM0UEI CQ CQ CQ de SM0UEI 10 UP

-- SM0UEI de ON3XYZ you are 59 sequence 122, QSL?

-- ON3XYZ SM0UEI QSL, 59 back to you, sequence 312 QSL?

-- QSL. CQ CQ CQ de SM0UEI 10 UP ...

Om du försöker nå en motstation med pile-up var uppmärksam på dennes sändningar och vänta på din tur. Tala gärna om signal och var du sänder från men släpp sedan fram andra. Tänk på hur du själv skulle vilja att en pile-up på din egen station skulle vilja agera. Den gyllene regeln är också alltid lyssna först --- sänd sedan!

	%\begin{landscape}
\section{Logbook}
\begin{longtable}{|c|c|c|c|c|c|c|c|c|c|}
	\hline
	Station  & Datum  & \multicolumn{2}{|c|}{Tid UTC} & Frekvens & Mode & Effekt & \multicolumn{2}{|c|}{Rapport} & Noteringar / QSL mm \\
	 Signal  & YYMMDD & HHMM &          HHMM          & MHz/kHz  &      & dBm/W  & RX &            TX            &  \\ \hline
	\endhead &        &      &                        &          &      &        &    &                          &  \\ \hline
	         &        &      &                        &          &      &        &    &                          &  \\ \hline
	         &        &      &                        &          &      &        &    &                          &  \\ \hline
	         &        &      &                        &          &      &        &    &                          &  \\ \hline
	         &        &      &                        &          &      &        &    &                          &  \\ \hline
	         &        &      &                        &          &      &        &    &                          &  \\ \hline
	         &        &      &                        &          &      &        &    &                          &  \\ \hline
	         &        &      &                        &          &      &        &    &                          &  \\ \hline
	         &        &      &                        &          &      &        &    &                          &  \\ \hline
	         &        &      &                        &          &      &        &    &                          &  \\ \hline
	         &        &      &                        &          &      &        &    &                          &  \\ \hline
	         &        &      &                        &          &      &        &    &                          &  \\ \hline
	         &        &      &                        &          &      &        &    &                          &  \\ \hline
	         &        &      &                        &          &      &        &    &                          &  \\ \hline
	         &        &      &                        &          &      &        &    &                          &  \\ \hline
	         &        &      &                        &          &      &        &    &                          &  \\ \hline
	         &        &      &                        &          &      &        &    &                          &  \\ \hline
	         &        &      &                        &          &      &        &    &                          &  \\ \hline
	         &        &      &                        &          &      &        &    &                          &  \\ \hline
	         &        &      &                        &          &      &        &    &                          &  \\ \hline
	         &        &      &                        &          &      &        &    &                          &  \\ \hline
	         &        &      &                        &          &      &        &    &                          &  \\ \hline
	         &        &      &                        &          &      &        &    &                          &  \\ \hline
	         &        &      &                        &          &      &        &    &                          &  \\ \hline
	         &        &      &                        &          &      &        &    &                          &  \\ \hline
	         &        &      &                        &          &      &        &    &                          &  \\ \hline
	         &        &      &                        &          &      &        &    &                          &  \\ \hline
	         &        &      &                        &          &      &        &    &                          &  \\ \hline
	         &        &      &                        &          &      &        &    &                          &  \\ \hline
	         &        &      &                        &          &      &        &    &                          &  \\ \hline
	         &        &      &                        &          &      &        &    &                          &  \\ \hline
	         &        &      &                        &          &      &        &    &                          &  \\ \hline
	         &        &      &                        &          &      &        &    &                          &  \\ \hline
	         &        &      &                        &          &      &        &    &                          &  \\ \hline
	         &        &      &                        &          &      &        &    &                          &  \\ \hline
	         &        &      &                        &          &      &        &    &                          &  \\ \hline
	         &        &      &                        &          &      &        &    &                          &  \\ \hline
	         &        &      &                        &          &      &        &    &                          &  \\ \hline
	         &        &      &                        &          &      &        &    &                          &  \\ \hline
	         &        &      &                        &          &      &        &    &                          &  \\ \hline
	         &        &      &                        &          &      &        &    &                          &  \\ \hline
	         &        &      &                        &          &      &        &    &                          &  \\ \hline
	         &        &      &                        &          &      &        &    &                          &  \\ \hline
	         &        &      &                        &          &      &        &    &                          &  \\ \hline
	         &        &      &                        &          &      &        &    &                          &  \\ \hline
	         &        &      &                        &          &      &        &    &                          &  \\ \hline
	         &        &      &                        &          &      &        &    &                          &  \\ \hline
	         &        &      &                        &          &      &        &    &                          &  \\ \hline
	         &        &      &                        &          &      &        &    &                          &  \\ \hline
	         &        &      &                        &          &      &        &    &                          &  \\ \hline
	         &        &      &                        &          &      &        &    &                          &  \\ \hline
	         &        &      &                        &          &      &        &    &                          &  \\ \hline
	         &        &      &                        &          &      &        &    &                          &  \\ \hline
	         &        &      &                        &          &      &        &    &                          &  \\ \hline
	         &        &      &                        &          &      &        &    &                          &  \\ \hline
	         &        &      &                        &          &      &        &    &                          &  \\ \hline
	         &        &      &                        &          &      &        &    &                          &  \\ \hline
	         &        &      &                        &          &      &        &    &                          &  \\ \hline
	         &        &      &                        &          &      &        &    &                          &  \\ \hline
	         &        &      &                        &          &      &        &    &                          &  \\ \hline
	         &        &      &                        &          &      &        &    &                          &  \\ \hline
	         &        &      &                        &          &      &        &    &                          &  \\ \hline
	         &        &      &                        &          &      &        &    &                          &  \\ \hline
	         &        &      &                        &          &      &        &    &                          &  \\ \hline
	         &        &      &                        &          &      &        &    &                          &  \\ \hline
	         &        &      &                        &          &      &        &    &                          &  \\ \hline
	         &        &      &                        &          &      &        &    &                          &  \\ \hline
	         &        &      &                        &          &      &        &    &                          &  \\ \hline
	         &        &      &                        &          &      &        &    &                          &  \\ \hline
	         &        &      &                        &          &      &        &    &                          &  \\ \hline
	         &        &      &                        &          &      &        &    &                          &  \\ \hline
	         &        &      &                        &          &      &        &    &                          &  \\ \hline
	         &        &      &                        &          &      &        &    &                          &  \\ \hline
	         &        &      &                        &          &      &        &    &                          &  \\ \hline
	         &        &      &                        &          &      &        &    &                          &  \\ \hline
	         &        &      &                        &          &      &        &    &                          &  \\ \hline
	         &        &      &                        &          &      &        &    &                          &  \\ \hline
	         &        &      &                        &          &      &        &    &                          &  \\ \hline
	         &        &      &                        &          &      &        &    &                          &  \\ \hline
	         &        &      &                        &          &      &        &    &                          &  \\ \hline
	         &        &      &                        &          &      &        &    &                          &  \\ \hline
	         &        &      &                        &          &      &        &    &                          &  \\ \hline
	         &        &      &                        &          &      &        &    &                          &  \\ \hline
	         &        &      &                        &          &      &        &    &                          &  \\ \hline
	         &        &      &                        &          &      &        &    &                          &  \\ \hline
	         &        &      &                        &          &      &        &    &                          &  \\ \hline
	         &        &      &                        &          &      &        &    &                          &  \\ \hline
	         &        &      &                        &          &      &        &    &                          &  \\ \hline
	         &        &      &                        &          &      &        &    &                          &  \\ \hline
	         &        &      &                        &          &      &        &    &                          &  \\ \hline
	         &        &      &                        &          &      &        &    &                          &  \\ \hline
	         &        &      &                        &          &      &        &    &                          &  \\ \hline
	         &        &      &                        &          &      &        &    &                          &  \\ \hline
	         &        &      &                        &          &      &        &    &                          &  \\ \hline
	         &        &      &                        &          &      &        &    &                          &  \\ \hline
	         &        &      &                        &          &      &        &    &                          &  \\ \hline
	         &        &      &                        &          &      &        &    &                          &  \\ \hline
\end{longtable}
\end{landscape}
\clearpage


\end{document}
