\section{QSO}

Konsten att genomföra ett radiosamtal (QSO) i olika sammanhang.

\subsection{Radiosamtalets delar}

Ett radiosamtal består som regel av tre delar. Först sker ett anrop, när kontakt etablerats utväxlas ett antal meddelande (dialog) och när man är klarar avslutas samtalet. Dessa tre delar är ganska standard. Man följer detta ganska strikt t.ex. på kortvågen där telefoni oftast innebär SSB. Anledningen är enkel, det går inte höra när någon släpper sändtangenten eller bara är tyst och tänker.

När man kör FM över repeatrar på VHF/UHF är det inte lika vanligt att man både öppnar och avslutar varje sändning med motparten och sin egen signal. Men man skall regelbundet upprepa signalerna och i praktiken är det lämpligt att göra kanske var femte minut eller oftare.

\subsection{Anropet}

Ett anrop kan se ut ungefär såhär:

--- SM0MAD från SM0UEI, SM0MAD kom!

Här är det SM0MAD som anropas av SM0UEI. Svaret kan se ut ungefär såhär:

--- SM0UEI från SM0MAD kom!

Därefter övergår radiosamtalet i dialog eller meddelandesändning.

\subsubsection{Allmänt anrop} 

Används när man inte ropar på någon särskild motstation utan önskar samtal med vem som helst. På svenska använder man ofta just orden ''allmänt anrop'' medan på engelska är det vanligare att man uttalar CQ (seek you). Ett allmänt androp kan se ut såhär:

--- Allmänt anrop, allmänt anrop, allmänt androp från SM0UEI SM0UEI SM0UEI kallar allmänt anrop och lyssnar.

Eller på engelska:

--- CQ CQ CQ this is SM0UEI calling CQ CQ CQ and standing by.

\subsection{Meddelandesändning}

--- SM0MAD från SM0UEI, tack för svaret. Din signal är 59 hos mig, mitt QTH är JO89WA och namnet är Anders. SM0MAD från SM0UEI kom.

--- SM0UEI från SM0MAD, tack för rapporten. Din signal är 57 hos mig, jag befinner mi i JO89VK men kommer under kvällen byta QTH. Jag kommer då vara QRV på 3663 kHz. QSL? SM0UEI från SM0MAD.

--- SM0MAD från SM0UEI, QSL på det, QRX 19.30 på frekvens 3663 kHz. 

\subsection{Avslutning}

--- SM0MAD från SM0UEI, tack för rapport och vi hörs senare, 73, slut kom

--- SM0UEI från SM0MAD, 73 tillbaka, klart slut.

\section{Contest}

Under contest är det vanligtvis så att man kör ett relativt kort QSO för att hinna så många som möjligt. Under constestförhållanden utväxlar man som regel signal, sekvensnummer, signalrapport med varandra. Ofta lägger man till "contest" i sitt anrop exempelvis:

--- CQ Contest CQ Contest CQ Contest SM0UEI SM0UEI CQ Contest

Efter man etablerat kontakt utväxlar man signalrapporter och sekvensnummer

--- SM0UEI from SA0MAD, you are 59 here and sequence 28, SM0UEI from SM0MAD.

--- SA0MAD from SM0UEI, QSL your signals are 58 and my sequence is 112, QSL? SA0MAD from SM0UEI

--- QSL and good luck, SM0UEI from SA0MAD

\subsection{Pile-up}

Ibland kan det bli väldigt många motstationer samtidigt som ropar. Nu gäller det att spetsa öronen! Först gäller det att sålla. Rara signaler från långtbortistan ger mer poäng i en contest som regel eller från länder du inte kört osv beroende på regler. Försök att sålla med "du som sänter från Florida" eller "VK7 kom igen" osv till det är en station kvar. Kör den snabbt, ropa CQ igen och börja sålla.

Om du försöker nå en motstation med pile-up var uppmärksam på dennes sändningar och vänta på din tur. Tala gärna om signal och var du sänder från men släpp sedan fram andra. 