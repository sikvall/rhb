\begin{landscape}
\subsection{Bandplaner VHF--UHF}
\subsubsection{Bandplan 6m 50--52 MHz}
\begin{tabular}{rrrll}

\textbf{Frekvens} &  & \textbf{BW} & \textbf{Trafik} & \textbf{Noteringar} \\ \hline

50.000 & 50.100 & 500 Hz  & CW          & \textbf{CW anrp. 50.050 och 50.090 (interkont.)}             \\ \hline
50.100 & 50.130 & 2.7 kHz & CW, SSB     & Interkontinental DX-trafik. Ej QSO inom Europa               \\ \hline
50.100 & 50.200 & 2.7 kHz & CW,SSB      & DX 50.110--50.130, \textbf{50.110 50.150 anrop (interkont.)} \\ \hline
50.200 & 50.300 & 2.7 kHz & CW,SSB      & Generell användning. 50.285 för crossband                    \\ \hline
50.300 & 50.400 & 2.7 kHz & CW, MGM     & PSK 50.305, EME 50.310 – 50.320                              \\
       &        &         &             & MS 50.350 – 50.380                                           \\ \hline
50.400 & 50.500 & 1 kHz   & CW, MGM     & Endast fyrar, 50.401 ±500 Hz WSPR-fyrar                      \\ \hline
51.210 & 51.390 & 12 kHz  & FM          & Repeater Repeater in, 20/10 kHz kanalavstånd                 \\
       &        &         &             & RF81 – RF99                                                  \\ \hline
50.500 & 52.000 & 12 kHz  & Alla moder  & SSTV 50.510, RTTY 50.600, FM 51.510                          \\ \hline
51.810 & 51.990 & 12 kHz  & FM Repeater & Repeater ut, 20/10 kHz kanalavstånd                          \\
       &        &         &             & RF81 – RF99                                                  \\ \hline
\end{tabular}

\subsubsection{Bandplan 2m 144--146 MHz}
\begin{tabular}{rrrll}

\textbf{Frekvens} &  & \textbf{BW} & \textbf{Trafik} & \textbf{Noteringar} \\ \hline

144.0000 & 144.1100  & 500 Hz  & CW, EME      & \textbf{CW anrop 144.050}               \\
         &           &         &              & MS random 144.100                       \\ \hline
144.1100 & 144.1500  & 500 Hz  & CW, MGM      & EME MGM 144.120--144.160                \\
         &           &         &              & PSK31 cent. 144.138                     \\ \hline
144.1500 & 144.1800  & 2.7 kHz & CW, SSB, MGM & EME 144.150--144.160                    \\
         &           &         &              & MGM 144.160--144.180 anrop 144.170      \\ \hline
144.1800 & 144.3600  & 2.7 kHz & CW, SSB, MGM & MS SSB random 144.195--144.205          \\
         &           &         &              & \textbf{SSB anrop 144.300}              \\ \hline
144.3600 & 144.3990  & 2.7 kHz & CW, SSB, MGM & MS MGM random anrop 144.370             \\ \hline
144.4000 & 144.4900  & 500 Hz  & Fyr          & Exklusivt segment fyrar, ej QSO         \\ \hline
144.5000 & 144.7940  & 20 kHz  & All mode     & SSTV, RTTY, FAX, ATV                    \\
         &           &         &              & Linjära transpondrar                    \\ \hline
144.7940 & 144.9625  & 12 kHz  & MGM          & APRS 144.800                            \\ \hline
144.9750 & 145.19350 & 12 kHz  & FM, DV       & Rpt in 144.975--145.1935                \\
         &           &         &              & RV46–-RV63, 12.5 kHz, 600 kHz skift     \\ \hline
145.1940 & 145.2060  & 12 kHz  & FM rymd      & 145.200 för kom. m. bem. rymdfark.      \\ \hline
145.2060 & 145.5625  & 12 kHz  & FM, DV       & FM 145.2125-–145.5875  V17–V47          \\
         &           &         &              & \textbf{FM anrop 145.500}, RTTY 145.300 \\
         &           &         &              & FM simpl. INET GW 145.2375, 2875, 3375  \\
         &           &         &              & DV anrop 145.375                        \\ \hline
145.5750 & 145.7935  & 12 kHz  & FM, DV       & Rpt ut 145.575--145.7875                \\
         &           &         &              & RV46–RV63, 12.5 kHz kanalavstånd        \\ \hline
145.794  & 145.806   & 12 kHz  & FM Rymd      & 145.800, 145.200 dplx m. bem. rymdfark. \\ \hline
145.806  & 146.000   & 12 kHz  & All mode     & Exklusivt satellit                      \\ \hline
\end{tabular}

\subsubsection{Bandplan 70cm 432--438 MHz}
\begin{tabular}{rrrll}
	\textbf{Frekvens} &          & \textbf{BW} & \textbf{Trafik} & \textbf{Anmärkning}                               \\ \hline

432.0000 & 432.0250 & 500 Hz  & CW           & EME exklusivt.                                    \\ \hline
432.0250 & 432.1000 & 500 Hz  & CW, PSK31    & CW mellan 432.000--085, \textbf{CW anrop 432.050} \\
         &          &         &              & PSK31 432.088                                     \\ \hline
432.1000 & 432.3990 & 2.7 kHz & CW, SSB, MGM & \textbf{SSB anrop 432.200}                        \\
         &          &         &              & Mikrovåg talkback 432.350, FSK441 432.370         \\ \hline
432.4000 & 432.4900 & 500 Hz  & Fyr          & Exklusivt segment för fyrar                       \\ \hline
432.5000 & 432.5940 & 12 kHz  & All mode     & Linjära transpondrar IN 432.500--600              \\ \hline
432.5000 & 432.5750 & 12 kHz  & All mode     & NRAU Digital rep. in 432.500--575 2 MHz skift     \\ \hline
432.5940 & 432.9940 & 12 kHz  & All mode     & Linjära transpondrar ut 432.600--800              \\ \hline
432.5940 & 432.9940 & 12 kHz  & FM           & Rep. in 432.600--975 RU368--398 2 MHz skift       \\ \hline
432.9940 & 433.3810 & 12 kHz  & FM           & Rep. in 433.000--375 RU368--398 1.6 MHz skift     \\ \hline
433.3940 & 433.5810 & 12 kHz  & FM           & SSTV (FM/AFSK) 433.400                            \\
         &          &         &              & FM simplex U272--286 \textbf{anrop 433.500}       \\ \hline
433.6000 & 434.0000 & 20 kHz  & All mode     & RTTY (FM/AFSK) 433.600                            \\
         &          &         &              & FAX 433.700, APRS 433.800                         \\ \hline
434.0000 & 434.4940 & 20 kHz  & All mode     & NRAU Dig. kanaler 433.450, 434.475                \\ \hline
434.5000 & 434.5940 & 20 kHz  & All mode     & NRAU Dig. rep. ut 434.500--575, 2 MHz skift       \\ \hline
434.5940 & 434.9810 & 12 kHz  & FM           & NRAU Rep. ut 434.600--975 RU 368--RU398           \\
         &          &         &              & 12,5 kHz med 2 MHz skift                          \\ \hline
435.000  & 438.000  & 20 kHz  & All mode     & Exklusivt satellit\\
\end{tabular}

\subsubsection{Bandplan 23cm 1240--1300 MHz}
\begin{tabular}{rrrll}
	\textbf{Frekvens}         &               & \textbf{BW} & \textbf{Trafik} & \textbf{Anmärkning}                                          \\ \hline
	         1240.000         & 1243.250      & 20 kHz      & Alla moder      & 1240.000 - 1241.000 Digital kommunikation                    \\ \hline
	         1243.250         & 1260.000      & 20 kHz      & ATV och Data    & Repeater ut 1258.150-1259.350, R20--68                       \\ \hline
	         1260.000         & 1270.000      & 12 kHz      & Satellit        & Endast för satelliter alla moder                             \\ \hline
	         1270.000         & 1272.000      & 20 kHz      & Alla moder      & Repeater in, 1270.025-1270.700, RS1--28                      \\
                                  &               &             &                 & Packet RS29--50                                              \\ \hline
	         1272.000         & 1290.994      & 20 kHz      & ATV och Data    & Amatörtelevision ATV                                         \\ \hline
	         1290.994         & 1291.481      & 20 kHz      & FM och DV       & Repeater in Repeat. in 1291.000--1291.475                    \\
                                  &               &             &                 & RM0 – RM19, 25 kHz, 6 MHz skift                              \\ \hline
	         1291.494         & 1296.000      & 12 kHz      & Alla moder      &                                                              \\ \hline
	         1296.000         & 1296.150      & 500 Hz      & CW,  MGM        & EME 1296.000--025, \textbf{CW anrop 1296.050}                \\
                                  &               &             &                 & PSK31 1296.138 MHz                                           \\ \hline
	         1296.150         & 1296.400      & 2.7 kHz     & CW, SSB, MGM    & \textbf{SSB anrop 1296.200}                                  \\
                                  &               &             &                 & \textbf{FSK441 MS anrop 1296.370}                            \\ \hline
	         1296.400         & 1296.600      & 2.7 kHz     & CW, SSB, MGM    & Linjära transpondrar infrekvens                              \\ \hline
	         1296.600         & 1296.800      & 2.7 kHz     & CW, SSB, MGM    & SSTV/FAX 1296.500, MGM/RTTY 1296.600                         \\ \hline
	         1296.600         & 1296.800      & 2.7 kHz     & CW, SSB, MGM    & Linjära transpondrar utfrekvens                              \\
                                  &               &             &                 & 1296.750-.800 lokala fyrar max 10 W                          \\ \hline
	         1296.800         & 1296.994      & 500 Hz      & Fyrar           & Exklusivt segment för fyrar                                  \\ \hline
	         1296.994         & 1297.481      & 20 kHz      & FM              & Repeater ut Repeater ut 1297.000--1297.475                   \\
                                  &               &             &                 & RM0 – RM19, 25 kHz, 6 MHz skift                              \\ \hline
	         1297.494         & 1297.981      & 20 kHz      & FM simplex      & Simplex 25 kHz kanaler SM20--39                              \\
                                  &               &             &                 & \textbf{FM anrop 1297.500 SM20}                              \\ \hline
	         1299.000         & 1299.750      & 150 kHz     & Alla moder      & 5 st 150 kHz kanaler för DD,                                 \\
                                  &               &             &                 & 1299.075, 225, 375, 525, och 675 $\pm$75 kHz                 \\ \hline
	         1299.750         & 1300.000      & 20 kHz      & Alla moder      & 8 st FM/DV 25 kHz kan. 1299.775--1299.975
\end{tabular}
\end{landscape}
\clearpage

